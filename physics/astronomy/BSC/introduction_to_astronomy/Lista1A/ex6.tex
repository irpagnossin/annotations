%IVAN RAMOS PAGNOSSIN 11/03/2001 irpagnossin@hotmail.com

\begin{quotation}
{\color{blue}{\bf 6.}}
{\it	Usando a unidade astron�mica como unidade de dist�ncia, anos como a unidade de tempo e a massa do Sol como a unidade de massa, a constante
	$k$ da {\it 3$^{\underline{a}}$ Lei de Kepler} vale $1,0$. Nestas unidades, qual � o valor da constante de gravita��o $G$ de Newton?}
\end{quotation}

\vspace{5mm}

	A {\it 3$^{\underline{a}}$ Lei de Kepler} �
	$$
	P^2=ka^3,\;\;\;\;\; k={4\pi^2\over GM}
	$$

	onde $M$ � uma massa qualquer\footnote{Aqui n�o nos interessa ficar definindo qual massa � de qual planeta. Precisamos apenas saber que
$M$ representa uma massa.}. Se 
	$$
	[T]=\mathrm{anos},\;\;\;\; [M]=M_{\bigodot}\;\;\;\; e \;\;\;\;[D]=\mathrm{UA}
	$$  

	ent�o 
	$$
	k=1={4\pi^2\over GM}={4\pi^2\over 1\cdot G}\Rightarrow G=4\pi^2\mathrm{UA^3\over M_{\bigodot}anos^2}
	$$

\vspace{10mm}