\begin{quotation}
{\bf 3.}
{\it	Um operador linear num espa�o de $n$ dimens�es n�o t�m necessariamente $n$ autovetores linearmente independentes. Como exemplo considere um 
operador $\operador A$ num espa�o bidimensional representado em termos da base ortonormal $\{\ket 1,\ket 2\}$:
	$$
	\operador A=4\ket 1\bra 1+\ket 1\bra 2-\ket 2\bra 1+2\ket 2\bra 2
	$$

	{\tt (a)} Escreva a forma matricial deste operador. Esse operador � Hermiteano? {\tt (b)} Escreva e resolva a equa��o caracter�stica. Mostre que
existe apenas um autovalor 3 duplamente degenerado. {\tt (c)} Mostre que existe apenas um autovetor normalizado (a menos de um fator de fase) dado por
$\ket v=(\ket 1-\ket 2)/\sqrt{2}$.
}
\end{quotation}

\vspace{5mm}

	Como no exerc�cio anterior, a forma matricial � clara:
	\[\operador A=\left (\begin{array}{cc}
				4	&	1	\\
				-1	&	2
			\end{array}\right )\]

	E � claro que $\operador A$ � Hermiteano se $\operador A^{\dagger}=\operador A$:
	\[\operador A^{\dagger}=\tilde{\operador{A}}^*=\left (\begin{array}{cc}
								4	&	-1	\\
								1	&	2	
							\end{array}\right )\ne\operador A\]

	Portanto, {\bf $\operador A$ n�o � Hermiteano}. Sua equa��o caracter�stica � dada por $\mathrm{ker}[\operador A-\lambda\operador I]$:
	\[\mathrm{ker}[\operador A-\lambda\operador I]\Rightarrow\left |\begin{array}{cc}
									4-\lambda	&	1	\\
									-1		&	2-\lambda	
								\end{array}\right |=0\Rightarrow(\lambda-3)^2=0 \]

	A solu��o � $a=3$ duplamente degenerado (raiz com multiplicidade alg�brica dois) e, conseq�entemente, o problema de autovalor
$\operador A\ket{\psi}=a\ket{\psi}$ � dado, na forma matricial, por
	\[\left (\begin{array}{cc}
			1	&	1	\\
			-1	&	-1
		\end{array}\right )\cdot\left(\begin{array}{c}
						x	\\
						y
					\end{array}\right)=\left(\begin{array}{c}
									0	\\
									0
							\end{array}\right)\]
	
	lembrando que $\ket{\psi}=x\ket 1+y\ket 2$. Novamente verificamos que as duas linhas da matriz s�o linearmente dependentes, o que nos d� a 
equa��o $x+y=0\Rightarrow x=-y$:
	$$
	\ket{\psi}=x(\ket 1-\ket 2)
	$$

	Normalizando,
	\begin{eqnarray*}
	\iprod{\psi}{\psi}	&=&|x|^2(\bra 1-\bra 2)\cdot(\ket 1-\ket 2)	\\
				&=&|x|^2(\iprod 11-\iprod 12-\iprod 21+\iprod 22)	\\
	1			&=&2|x|^2
	\end{eqnarray*}

	Que d�, obviamente, $|x|={1/\sqrt{2}}e^{\i\phi}$. Ou seja, o autovetor normalizado �
	$$
	\ket{\psi}={\ket 1-\ket 2\over\sqrt{2}}e^{i\phi} 
	$$

	que � exatamente o que o exerc�cio pediu, a menos de uma fase $\phi$.


