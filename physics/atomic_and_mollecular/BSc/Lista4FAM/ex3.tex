\renewcommand{\a}{\hat{a}}
\renewcommand{\aa}[1][]{\hat{a}^{\dagger #1}}

\begin{exercicio}
	Para o estado $\ket n$ do $n$-�simo n�vel de energia do oscilador harm�nico calcule

	{\tt (a)} Os valores esperados $\med{\op X}$, $\med{\op P}$, $\med{\op X^2}$ e
$\med{\op P^2}$.

	{\tt (b)} O produto $\Delta X\Delta P$ e mosre que � igual a $n+1/2)\hbar$. Esse
resultado est� de acordo com o princ�pio da incerteza?

	{\tt (c)} Os valores esperados das energia potencial $\med{\op V}$ e cin�tica 
$\med{\op T}$, e mostre que satisfazem ao teorema do virial: $\med{\op V}=\med{\op T}$.
\end{exercicio}

	{\tt (a)} O valor m�dio de $\op X$ � mais facilmente calculado tomando-se o operador
an�logo adimensional $\op x$, 
	$$
	\hope{n}{\op X}{n}=\sqrt{\hbar\over m\omega}\hope{n}{\hat{x}}{n}
	$$

	Al�m disso, sabemos que $\hat{x}$ pode ser expresso em termos dos operadores de cria��o
$\aa$ e destrui��o $\a$:
	$$
	\hat{x} = {1\over\sqrt{2}}(\a+\aa)
	$$

	Ent�o,
	$$
	\hope{n}{\hat x}{n}={1\over\sqrt{2}}[\hope{n}{\a}{n}+\hope{n}{\aa}{n}]=0
	$$

	pois 
	\begin{eqnarray*}
	\a\ket n 	&=& \sqrt{n}\ket{n-1},				\\
	\aa\ket n	&=& \sqrt{n+1}\ket{n+1}\,\,\mathrm{e}		\\
	\iprod{i}{j}	&=& \delta_{ij}
	\end{eqnarray*}

	Logo $\med{\op X}=0$. O mesmo ocorre com $\op P=\hat p\sqrt{m\omega\hbar}$:
	$$
	\hope{n}{\hat p}{n}={1\over i\sqrt{2}}[\hope{n}{\a}{n}-\hope{n}{\aa}{n}]=0
	$$

	visto que
	$$
	\hat p = {1\over i\sqrt{2}}(\a-\aa)
	$$

	Processo an�logo fazemos para $\op X^2$ e $\op P^2$. Mas note antes que
	\begin{eqnarray*}
	\hat x^2 &=&{1\over2}(\a+\aa)(\a+\aa)={1\over2}(\a^2+\a\aa+\aa\a+\aa[2])\\
		&=&{1\over2}(2\a\aa-1)=\a\aa-{1\over2}				\\
	\hat p^2&=&-{1\over2}(\a-\aa)(\a-\aa)={1\over2}(\a^2-\a\aa-\aa\a+\aa[2])\\
		&=&-{1\over2}(-2\a\aa+1)=\a\aa-{1\over2}		
	\end{eqnarray*}

	Ou seja, $\hat x^2=\hat p^2$. Note que os operadores $\a^2$ e $\aa[2]$ foram simplesmente
exclu�dos das express�es acima pois como vamos aplic�-los num estado $\ket n$ e ent�o efetuar
o produto interno com ele mesmo, necessitamos que haja sempre pares de operadores de cria��o
e aniquila��o, de modo que, se o primeiro eleva o estado para $ket{n+1}$, por exemplo, o segundo 
reduz-lhe a $ket n$ novamente, dando $\hope{n}{\op M}{n}\ne0$, onde $\op M$ � uma combina��o de 
operadores $\a$ e $\aa$.
	\begin{eqnarray*}
	\hope{n}{\op X^2}{n}	&=& {\hbar\over m\omega}\hope{n}{op x^2}{n}			\\
				&=& {\hbar\over m\omega}\hope{n}{\a\aa-{1\over2}}{n}		\\
				&=& {\hbar\over m\omega}[\hope{n}{\a\aa}{n}-\hope{n}{1\over2}{n}]\\
				&=& {\hbar\over m\omega}[\sqrt{n+1}\hope{n}{\a}{n}-{1\over2}]	\\
				&=& {\hbar\over m\omega}[(n+1)\iprod{n}{n}-{1\over2}]		\\
				&=& (n+{1\over2}){\hbar\over m\omega}				\\
	\hope{n}{\op P^2}{n}	&=& {\hbar\over m\omega}\hope{n}{op p^2}{n}			\\
				&=& (n+{1\over2}){\hbar\over m\omega}
	\end{eqnarray*}

	{\tt (b)} A incerteza num autovalor de um operador $\op J$ � dada pela raiz de sua 
vari�ncia:
	$$
	\Delta J=\sqrt{\med{\op J^2}-\med{\op J}^2}
	$$

	Ent�o temos
	\begin{eqnarray*}
	\Delta X &=& \sqrt{\med{\op X^2}-\med{\op X}^2}=\sqrt{n+{1\over2}}\,{\hbar\over m\omega}	\\
	\Delta P &=& \sqrt{\med{\op P^2}-\med{\op P}^2}=\sqrt{n+{1\over2}}\,m\omega\hbar	
	\end{eqnarray*}

	E, portanto, a rela��o de incerteza fica
	$$
	\Delta X\Delta P = \sqrt{n+{1\over2}}\,{\hbar\over m\omega}\cdot
				\sqrt{n+{1\over2}}\,m\omega\hbar=(n+{1\over2})\hbar
	$$

	E esta express�o est� coerente com o que se espera j� que 
	$$
	\Delta X\Delta P \ge {\hbar\over2},\,\,\,\,\,n=0,1,2,...
	$$

	{\tt (c)} O Hamiltoniano do oscilador harm�nico simples unidimensional �
	$$
	\op H = {1\over2m}\op P^2+{1\over2}k\op X^2
	$$

	Portanto, os operadores de energia cin�tica $\op T$ e enmergia potencial $\op V$ s�o
	\begin{eqnarray*}
	\op T = {1\over2m}\op P^2	\\
	\op V = {1\over2}k\op X^2	
	\end{eqnarray*}

	Deduzimos quase de imediato que
	\begin{eqnarray*}
	\hope{n}{\op T}{n}&=&{1\over2m}\hope{n}{\op P^2}{n}={1\over2}(n+{1\over2})\omega\hbar	\\
	\hope{n}{\op V}{n}&=&{1\over2}k\hope{n}{\op X^2}{n}=
				{1\over2}m\omega^2(n+{1\over2}){\hbar\over m\omega}=
						{1\over2}(n+{1\over2})\omega\hbar
	\end{eqnarray*}

	Logo, $\med{\op T}=\med{\op V}$.
