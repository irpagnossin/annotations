\begin{exercicio}[blue]
	O $\el{222}{}{Rn}{}{}$ decai por emiss�o de uma part�cula alfa. Supondo que um n�cleo desse 
elemento esteja inicialmente em repouso, como se distribui a energia cin�tica entre o n�cleo resultante
e a part�cula alfa? Se a energia cin�tica da part�cula alfa emitida f�sse de $5,53\,\mathrm{MeV}$, qual
a energia cin�tica do ``filho do rad�nio'', imediatamente ap�s a desintegra��o?
\end{exercicio}

	A equa��o de desintegra��o do Rad�nio-222 �
	\begin{equation}\label{eq:Rad�nio}
	\el{222}{86}{Rn}{}{}\,\to\,\el{4}{2}{He}{2+}{}+\el{218}{84}{Po}{}{}+Q
	\end{equation}

	O Pol�nio-218 tamb�m � conhecido como $\mathrm{RaA}$. A energia de desintegra��o $Q$ � ent�o 
transformada em energia cin�tica dos produtos da rea��o, principalmente a part�cula alfa, que � muito
leve quando comparada com o Pol�nio.

	De resultados experimentais (veja refer�ncia \cite{serway}) conhecemos as massas dos reagentes:
	\begin{eqnarray*}
	\mathrm{m(Rn)}	&=& 222,017\,571\,\mathrm u	\\
	\mathrm{m(Po)}	&=& 218,008\,965\,\mathrm u	\\
	\mathrm{m(He)}	&=&   4,002\,602\,\mathrm u
	\end{eqnarray*}

	De tal forma que a energia de desintegra��o �
	$$
	{Q\over c^2} = m(Rn)-m(He)-m(Po) = 0,006\,004\,\mathrm{u}\,\Rightarrow \,Q = 5,592\,332\,\mathrm{MeV}
	$$

	Lembrando que $1\,\mathrm{u}=931,434\,320\,\mathrm{MeV/c^2}$. Assim, se a part�cula alfa emitida
tiver uma energia cin�tica de $5,53\,\mathrm{MeV}$, ent�o a energia cin�tica do Pol�nio-218 seria de
$62,332\,\mathrm{keV}$.
