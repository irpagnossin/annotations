\begin{exercicio}[blue]
	Uma fonte de $\el(241,)Am(,)$ (Emissor alfa com energia m�dia de $5,48\unit{MeV}$) est� exposta ao
	ar. Quer-se utiliz�-la para irradiar uma folha de pl�stico colocada a uma certa dist�ncia $d$. Qual
	deve ser essa dist�ncia para que a energia m�dia das part�culas alfa que a atingem seja de 
	$2,0\unit{MeV}$?
\end{exercicio}

	Divida a dist�ncia $d$ da fonte ao alvo em $n$ "fatias" de ar com espessura $\Delta x=d/n$ em cada uma
das quais se pode aplicar a lei do decaimento exponencial. A rela��o entre as energias nas faces $x_i$ e 
$x_{i+1}=x_i+\Delta x$ �, ent�o, 
	$$
	E(x_{i+1}) = E(x_i)e^{-\mu(E(x_i))\,\Delta x}
	$$
	onde $\mu(E)$ � o coeficiente de atenua��o linear.\par
	Por comodidade apenas, vamos renomear $E(x_k)$ para $E_k$ e $\mu(E(x_k))=\mu(E_k)=\mu_k$. Assim 
reescrevemos a lei acima:
	$$  
	E_{i+1} = E_ie^{-\mu_i\Delta x}
	$$

	Aplicando isso de forma recursiva para cada uma das fatias $\Delta x$, desde $x_0=0$ at� $x_n=d$,
	\begin{eqnarray*}
 	E_1 &=& E_0 e^{-\mu_0\Delta x}				\\
	E_2 &=& E_1 e^{-\mu_1\Delta x}				\\
	    &=& E_0 e^{-(\mu_0+\mu_1)\Delta x}			\\
	E_3 &=& E_2 e^{-\mu_2\Delta x}				\\
	    &=& E_0 e^{-(\mu_0+\mu_1+\mu_2)\Delta x}		\\
	    &\vdots&							\\
	E_{n+1} &=& E_0 e^{-\Delta x\sum_{j=0}^\infty\mu_j}
	\end{eqnarray*}

	obtemos a energia medida ap�s uma dist�ncia $d$ no ar. Desenvolvendo um pouco mais conclu�mos que
	$$
	d\bigl ({1\over n}\sum_{j=0}^\infty\mu_j\bigr ) = \ln{E_0\over E_{n+1}}
	$$

	Se notarmos que a grandeza entre par�nteses � a m�dia de $\mu$ no intervalo $E_0$ e $E_{n+1}$:
	$$
	\bar\mu = \bigl ({1\over n}\sum_{j=0}^\infty\mu_j\bigr ) = 
	{1\over E_{n+1}-E_0}\int_{E_0}^{E_{n+1}}\mu(E)\mathrm{dE}
	$$
	concluiremos que a dist�ncia $d$ procurada �
	\begin{equation}\label{eq:d}
	d = {(E_1-E_0)\ln(E_0/E_1)\over\int_{E_0}^{E_1}\mu(E)\mathrm{dE}}
	\end{equation}
	onde $E_1=E_{n+1}$\footnote{Renomeamos novamente por comodidade.}. Desta forma, o problema passa a ser
avaliar a integral acima. Isto pode ser feito tomando os valores experimentais de $\mu_i$ no intervalo de
interesse e ajustando uma {\sl curva} aos pontos para obter uma express�o $\mu(E)$. Utilizamos os dados do 
{\sl Johns}:
\begin{table}[ht]
  \centering
  \begin{tabular}{ccc}
  E(keV)	&     $\mu/\rho$ ($\mathrm{cm^2/g}$)	\\
  \hline
  2	&	0,0445	\\
  3	& 	0,0358	\\
  4	&  	0,0308	\\
  5	&	0,0274	\\
  6	&	0,0250	\\
  \hline
  \end{tabular}
  \caption{\label{tab:ex3}\footnotesize Valores des coeficientes m�ssicos de absor��o como fun��o da energia
	     para o ar. Densidade $1,205\unit{kg/m^3}$, conforme {\sl Johns}.}
\end{table}

	Neste intervalo, conseguimos ajustar uma par�bola com $\chi_{\mathrm{red}}^2=0,9966$:
	$$
	\mu(E) = 0,1231\,E^2-1,5522\,E+7,9385\qquad(\times10^{-3})
	$$

	Imaginando que, apesar da distribui��o de energias do feixe, ele possa ser considerado monoernerg�tico,
aplicando esta express�o na equa��o (\ref{eq:d}), com $E_1=2\unit{MeV}$ e $E_0=5,48\unit{MeV}$, encontramos 
$d=253,29\unit{m}$.
