\begin{quotation}
{\bf 7.}
{\it 	Um planeta tem um n�cleo de 		$1000\,\,\mathrm{km}$ de raio e densidade 
$8,5\,\,\mathrm{g/cm^3}$; Um n�cleo externo de 	$800\,\,\mathrm{km}$ de espessura e 
$5,3\,\,\mathrm{g/cm^3}$; Um manto de 		$1800\,\,\mathrm{km}$ e densidade
$3,1\,\,\mathrm{g/cm^3}$ e uma crosta de 	$20\,\,\mathrm{km}$ e densidade
$2,6\,\,\mathrm{g/cm^3}$. {\tt (a)} Qual � a densidade m�dia do planeta? {\tt (b)} Estime
a press�o na interface entre o n�cleo externo e o manto. {\tt (c)} Qual � a for�a diferencial
gravitacional que uma lua a $80$ raios planet�rios, de massa $50$ vezes menor que a do planeta,
exerce sobre pontos diametralmente opostos do planeta, tanto na linha lua-Planeta como perpendicular
a ela? {\tt (d)} Qual a velocidade orbital da lua? {\tt (e)} Onde se encontra o baricentro do
sistema?}
\end{quotation}

	%*******TABELA COM R_I E \RHO_I*********

	{\tt (a)} A densidade m�dia $\bar{\rho}$ do planeta � simplesmente a massa total dividida
pelo volume deste:
	\begin{eqnarray*}
	\bar{\rho}	&=&{\rho_1V_1+\rho_2V_2+\rho_3V_3+\rho_4V_4\over V}	\\
			&=&{\rho_1V_1+\rho_2V_2+\rho_3V_3\over V}+\rho_4	\\
			&=&{4\pi[(rho_1-rho_2)r_1^3+(\rho_2-\rho_3)r_ 2^3+
				(\rho_3-\rho_4)r_3^3]\over{{4\over3}\pi R^3}}+\rho_4 \\
			&=&(\rho_1-\rho_2)({r_1\over R})^3+(\rho_2-\rho_3)({r_ 2\over R})^3+
				(\rho_3-\rho_4)({r_3\over R})^3+\rho_4		\\
			&\cong&3,4\,\,\mathrm{g/cm^3}
	\end{eqnarray*}

	{\tt (b)} A equa��o de equil�brio hidrodin�mico de um planeta �
	$$
	{dp\over dr}=-\rho(r){Gm(r)\over r^2}
	$$

	O que pode ser resolvido facilmente reescrevendo-a como
	$$
	\int_0^{p_0}dp = -G\int_R^{R_0}{\rho(r)m(r)\over r^2}dr
	$$

	onde o limite inferior da primeira integral denota o fato de que a press�o na 
superf�cie do planeta � muito menor que no interior dele, e $R_0$ � o raio onde se quer
avaliar a press�o (No nosso caso $R_0=r_2$ e $p_0=p_3$). Ent�o
	\begin{eqnarray*}
	p_3	&=& G\int_{r_2}^R{\rho(r)m(r)\over r^2}dr				\\
		&=& G\int_{r_2}^{r_3}{\rho_2^2{{4\over3}\pi(r^3-r_2^3)}\over r^2}dr
			+G\int_{r_3}^R{\rho_3^2{{4\over3}\pi(r^3-r_3^3)}\over r^2}dr	\\
		&=& {4\over3}\pi\rho_2^2G\int_{r_2}^{r_3}{r^3-r_2^3\over r^2}dr+
			{4\over3}\pi\rho_3^2G\int_{r_3}^R{r^3-r_3^3\over r^2}dr		\\
		&=& {4\over3}\pi G[\rho_2^2({r_3^2\over2}+{r_2^3\over r_3}-{3\over2}r_2^2)+
			\rho_3^2({R^2\over2}+{r_3^3\over R}-{3\over2}r_3^2)]		\\
		&\cong& 254000\,\,\mathrm{atm} 
	\end{eqnarray*}

	{\tt (c)} A for�a, por unidade de massa, que a lua exerce sobre um ponto do planeta �
simplesmente 
	$$
	g={F\over \delta m} = G{M\over d^2}
	$$

	onde $\delta m$ � a massa que sofre esta acelera��o, $M$ a massa da lua e $d$ a dist�ncia entre
os dois. No caso, queremos determinar diferen�a de acelera��o dada aos elementos a $d_A=81R$ 
(mais distante) e $d_B=79R$ (mais pr�ximo):
	\begin{eqnarray*}
	g_{B}-g_{A}	&=&{mG\over R^2}({1\over79^2}-{1\over81^2})	\\
			&=&{MG\over 50R^2}({1\over79^2}-{1\over81^2})	\\
			&=&{4\over3}\pi R^3{\bar{\rho}G\over50R^2}({1\over79^2}-{1\over81^2})
									\\
			&=&{4\pi\bar{\rho}GR\over150}({1\over79^2}-{1\over81^2})
									\\
			&=&{4\pi\cdot3400\cdot6,67\times10^{-11}\cdot3,62\times10^6\over150}
				({1\over79^2}-{1\over81^2})		\\
			&\cong&5,4\times10^{-7}
	\end{eqnarray*}

	Logo, a diferencial gravitacional por unidade de massa � $0,54\,\mathrm{\mu N}$.

	{\tt (d)} A velocidade orbital da lua pode ser calculada facilmente utilizando-se a
terceira Lei de Kepler e assumindo que a �rbita seja aproximadamente circular\footnote{Isto �
importante para podermos garantir que a velocidade orbital � constante.}. Ent�o temos
	\begin{eqnarray*}
	T^2	&=&{4\pi^2\over G(m+M)}a^3					\\
		&=&{4\pi^2\over G(M/50+M)}(80R)^3				\\	
		&=&{200\pi^2\over51{4\over3}\pi R^3\bar{\rho}G}(80R)^3		\\
		&=&{1,024\times^8\pi\over214\cdot3,4\times10^3\cdot6,67\times10^{-11}}
										\\
		&=&2,08\times10^{13}\,\,\mathrm{s^2}				\\
	T	&=&52\,\mathrm{d}20\mathrm{h}43\mathrm{m}9\mathrm{s}	
	\end{eqnarray*}

	Logo, a velocidade �
	\begin{eqnarray*}
	v	&=&{2\pi a\over T}={2\pi 80R\over T}={160\pi\cdot3,62\times10^6\over 4567389}	\\
		&=&398,4\,\,\mathrm{m/s}
	\end{eqnarray*}

	{\tt (e)} J� sabemos, desde a lista anterior, que a posi��o de cada um, relativo ao
centro de massa do sistema, relaciona-se com as massas dos componentes por
	$$
	{r_1\over r_2}={m_2\over m_1}
	$$

	Logo, se $r_1$ � a posi��o do planeta ($m_2=M$), $r_2$ a da lua ($m_2=M/50$) e $a=80R$
a dist�ncia entre eles,
	$$
	{r_1\over a-r_2}={M/50\over M} \Rightarrow r_1=5678\,\,\mathrm{km}
	$$

	E, consequentemente, $r_2=283921,6\,\,\mathrm{km}$.
	

	

	

	
	


	


	