\begin{quotation}
{\bf 8.}
{\it	O {\bf raio} do Sol, visto a partir da Terra, � aproximadamente $16$ minutos de arco. O 
fluxo solar medido no topo da atmosfera � $1,3533\times10^6\,\,\mathrm{ergs\cdot cm^2/s}$ (Constante 
solar). Calcule a temperatura efetiva do Sol usando somente os dados acima.}
\end{quotation} 

	O fluxo � a quantidade de energia por �rea e por tempo. Suponha que encontramos o fluxo 
$\Phi_{\bigodot}$ na superf�cie do Sol. Ent�o a energia correspondente � simplesmente
$E_{\bigodot}=4\pi R_{\bigodot}^2\Phi_{\bigodot}$. Por conserva��o de energia, sabemos que,
ao atingir a Terra, o fluxo ali medido ($\Phi_{\bigoplus}$) deve ser tal que a energia total 
contida numa casca esf�rica de raio $d=1\,\,\mathrm{UA}$ -- A dist�ncia Terra-Sol -- deva ser
a mesma. Ent�o,
	$$
	L_{\bigodot}=E_{\bigodot}=4\pi R_{\bigodot}^2\Phi_{\bigodot}=4\pi d^2\Phi_{\bigoplus}
	$$ 
	
	que d� diretamente
	$$
	\Phi_{\bigodot}=\Phi_{\bigoplus}({d\over R_{\bigodot}})^2
	$$

	Al�m disso, por simples trigonometria percebemos que
	$$
	{d\over R_{\bigodot}}={1\over\tan\theta}
	$$

	sendo $\theta$ o �ngulo sob o qual se v� o {\bf raio} do Sol a partir da Terra. Al�m disso,
a Lei de Stefan diz que o fluxo de energia eletromagn�tica de um corpo negro depende
da quarta pot�ncia da temperatura apenas. Ent�o, considerando o Sol como um corpo negro,
	$$
	\Phi_{\bigodot}=\sigma T^4
	$$
	
	com $\sigma=5,67\times10^{-5}\,\,\mathrm{ergs/cm^2\cdot s\cdot K^4}$ a constante de 
Stefan-Boltzmann. Disto tudo fica f�cil concluir que
	\begin{eqnarray*}
	T^4	&=&{\Phi_{\bigoplus}\over\sigma\tan\theta}		\\
		&=&{1,3533\times10^6\over5,67\times10^{-5}\tan(16')}	\\
	T	&=&5761\,\,\mathrm{K}
	\end{eqnarray*}

	






