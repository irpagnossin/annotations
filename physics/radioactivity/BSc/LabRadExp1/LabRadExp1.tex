\documentclass[a4paper,10pt]{article}

%----------------------------------------------------------------------------------%
%-------------- DEFINE A VERS�O DO COMPILADOR EXIGIDA PELO ARQUIVO ----------------%
%----------------------------------------------------------------------------------%
\NeedsTeXFormat{LaTeX2e}[1998/12/01]

%----------------------------------------------------------------------------------%
%------------------------- CARREGA OS PACOTES NECESS�RIOS -------------------------%
%----------------------------------------------------------------------------------%
\usepackage[latin1]{inputenc}
\usepackage[portuges]{babel}
\usepackage[pdftex]{graphics,color}
\usepackage{graphicx}
\usepackage{wrapfig} 
\usepackage{epsfig} 
\usepackage{ae}
\usepackage{algorithm}
\usepackage{alltt}
\usepackage{amsfonts}
\usepackage{amstext}
\usepackage{enumerate}
\usepackage{float}
\usepackage{fancyvrb}
\usepackage{fontenc}
\usepackage{geometry}
\usepackage{hyperref}
\usepackage{ifthen}
\usepackage{indentfirst}
\usepackage{lastpage}
\usepackage{longtable}
\usepackage{lscape}
\usepackage{makeidx}
\usepackage{mathrsfs}
\usepackage{multicol}
\usepackage{pifont}
\usepackage{psfrag}
\usepackage{setspace}
\usepackage{showidx}
\usepackage{subfigure}
\usepackage{texnames}
\usepackage{textcomp}
\usepackage{ulem}
\usepackage{url}
\usepackage{varioref}
\usepackage{version}
\usepackage{wasysym}

\makeindex
\pagestyle{myheadings}
\markboth{}{\rm Medindo radia��es com detectores Geiger-M�ller}

\begin{document}

\title{Medindo radia��es com detectores Geiger-M�ller}%
%\author{Ivan R. Pagnossin \\ irpagnossin@hotmail.com \and
% 	Ricardo \\ yano.ricardo@webemails.com	     \and
%	Wille \\ jwamorin@yahoo.com}% 
\author{Ivan R. Pagnossin \\ irpagnossin@hotmail.com}
\maketitle

\renewcommand{\abstractname}{Abstract}

%%%%%%%%%%%%%%%%%%%%%%%%%%%%%%%%%%%%%%%%%%%%%%%%%%%%%%%%%%%%%%%%%%%%%%%%%
%	Arquivo:	irpackage.tex					%
%	Projeto:	irpackage					%
%	Autor:		Ivan Ramos Pagnossin		 		%
%	Data inicial:	29/04/2001					%
%%%%%%%%%%%%%%%%%%%%%%%%%%%%%%%%%%%%%%%%%%%%%%%%%%%%%%%%%%%%%%%%%%%%%%%%%

%%%%%%%%%%%%%%%%%%%%%%%%%%%%%%%%%%%%%%%%%%%%%%%%%%%%%%%%%%%%%%%%%%%%%%%%%
%	Define um formato para operadores				%
%%%%%%%%%%%%%%%%%%%%%%%%%%%%%%%%%%%%%%%%%%%%%%%%%%%%%%%%%%%%%%%%%%%%%%%%%
\newcommand{\op}[1]{\mathcal{#1}}

%%%%%%%%%%%%%%%%%%%%%%%%%%%%%%%%%%%%%%%%%%%%%%%%%%%%%%%%%%%%%%%%%%%%%%%%%
%	Define os vetores Bra e Ket do espa�o de Hilbert		%
%%%%%%%%%%%%%%%%%%%%%%%%%%%%%%%%%%%%%%%%%%%%%%%%%%%%%%%%%%%%%%%%%%%%%%%%%
\newcommand{\bra}[1]{\ensuremath{\langle#1|}}
\newcommand{\ket}[1]{\ensuremath{|#1\rangle}}

%%%%%%%%%%%%%%%%%%%%%%%%%%%%%%%%%%%%%%%%%%%%%%%%%%%%%%%%%%%%%%%%%%%%%%%%%
%	Define o produto interno de vetores do espa�o de Hilbert	%
%%%%%%%%%%%%%%%%%%%%%%%%%%%%%%%%%%%%%%%%%%%%%%%%%%%%%%%%%%%%%%%%%%%%%%%%%
\newcommand{\iprod}[2]{\ensuremath{\langle#1|#2\rangle}}

%%%%%%%%%%%%%%%%%%%%%%%%%%%%%%%%%%%%%%%%%%%%%%%%%%%%%%%%%%%%%%%%%%%%%%%%%
%	Define o valor esperado de um operador (argumento 2)		%
%%%%%%%%%%%%%%%%%%%%%%%%%%%%%%%%%%%%%%%%%%%%%%%%%%%%%%%%%%%%%%%%%%%%%%%%%
\newcommand{\hope}[3]{\ensuremath{\langle#1|#2|#3\rangle}}

%%%%%%%%%%%%%%%%%%%%%%%%%%%%%%%%%%%%%%%%%%%%%%%%%%%%%%%%%%%%%%%%%%%%%%%%%
%	A m�dia do argumento 1						%
%%%%%%%%%%%%%%%%%%%%%%%%%%%%%%%%%%%%%%%%%%%%%%%%%%%%%%%%%%%%%%%%%%%%%%%%%
\newcommand{\med}[1]{\langle #1 \rangle}

%%%%%%%%%%%%%%%%%%%%%%%%%%%%%%%%%%%%%%%%%%%%%%%%%%%%%%%%%%%%%%%%%%%%%%%%%
%	Integral impr�pria do primeiro tipo				%
%%%%%%%%%%%%%%%%%%%%%%%%%%%%%%%%%%%%%%%%%%%%%%%%%%%%%%%%%%%%%%%%%%%%%%%%%
\newcommand{\intii}{\int_{-\infty}^{\infty}}

%%%%%%%%%%%%%%%%%%%%%%%%%%%%%%%%%%%%%%%%%%%%%%%%%%%%%%%%%%%%%%%%%%%%%%%%%
%	Elemento qu�mico. Em ordem dos argumentos, o n�mero de massa,	%
% o n�mero at�mico, o s�mbolo do elemento, a val�ncia e o n�mero   de	%
% n�utrons.								%
%%%%%%%%%%%%%%%%%%%%%%%%%%%%%%%%%%%%%%%%%%%%%%%%%%%%%%%%%%%%%%%%%%%%%%%%%
%\newcommand{\el}[5]{\ensuremath{^{#1}_{#2}\mathrm{#3}^{#4}_{#5}}}
\def\el(#1,#2)#3(#4,#5){\ensuremath{^{#1}_{#2}\mathrm{#3}^{#4}_{#5}}}

\newcommand{\unit}[1]{\ensuremath{\,\mathrm{#1}}}

%%%%%%%%%%%%%%%%%%%%%%%%%%%%%%%%%%%%%%%%%%%%%%%%%%%%%%%%%%%%%%%%%%%%%%%%%
%	Environment para enunciado de exer�cios 			%
%%%%%%%%%%%%%%%%%%%%%%%%%%%%%%%%%%%%%%%%%%%%%%%%%%%%%%%%%%%%%%%%%%%%%%%%%
\newcounter{exercicio} 
\newenvironment{exercicio}[1][black]
	{\vspace{5mm}
	 \stepcounter{exercicio}
	 \begin{quotation}
	 {\color{#1}{\bf Exerc�cio \arabic{exercicio}.}}
	 \itshape}     
        {\end{quotation}
 	 \vspace{5mm}}



\label{sec:analise}
\subsection{Circuito RC} 

	O primeiro passo em dire��o � compreens�o de circuitos sob regimes estacion�rios de correntes alternadas � encontrar um modelo f�sico e uma 
descri��o matem�tica capazes de explicar os resultados e prever possibilidades futuras. Isto foi desenvolvido no item anterior e agora cabe-nos verificar
se de fato s�o coerentes e satisfat�rias. Para isso, um problema bastante simples: Considere um circuito com duas imped�ncias em s�rie, 
sendo a primeira puramente resistiva ($Z_R=R$) e a segunda puramente capacitiva ($\hat{Z}_c=-j/\omega C$). Neste caso, a {\it Lei de Ohm}\footnote
{A {\it Lei de Ohm} neste caso deve ser aquela desenvolvida na se��o \ref{sec:teoria}, cuja aplica��o extende-se para circuitos de correntes alternadas.} 
aplicada ao resistor d�
	\begin{equation}
	\hat{Z}_R={\hat{V}_R\over\hat{I}}  \Rightarrow  |\hat{Z}_R|={|\hat{V}_R|\over|\hat{I}|}  
	\label{eq:corrente}
	\end{equation} 

	Ou, de forma mais simplificada, $R=V_R/I$. Note que a corrente n�o precisa de �ndice pois � a mesma em todo o circuito. Mas $R$ e $V_R$ s�o
perfeitamente mensur�veis com os instrumentos que temos � m�o. No caso, utilizamos o mult�metro e o oscilosc�pio, respectivamente. Ent�o � claro que
podemos calcular a corrente. Mas conhecendo-se a corrente e aplicando novamente a {\it Lei de Ohm} ao capacitor podemos determinar o valor do 
{\bf m�dulo}\footnote{Medimos o m�dulo simplesmente porque � mais f�cil obter o valor de pico da tens�o e da corrente no capacitor que s�o, respectivamente,
os m�dulos da forma complexa da tens�o e da corrente.} da imped�ncia do capacitor:
	\begin{equation}
	\hat{Z}_C={\hat{V}_C\over\hat{I}}  \Rightarrow  |\hat{Z}_C|={|\hat{V}_C|\over|\hat{I}|} \Rightarrow  Z_C={V_c\over I}
	\label{eq:impedancia}
	\end{equation} 

	Al�m disso, o modelo diz que $Z_C=1/\omega C$. Consequentemente, calculamos a capacit�ncia -- dado que sabemos tamb�m medir a freq��ncia
com o oscilosc�pio -- e, como esta � determinada pelo fabricante, podemos comparar nosso resultado e concluirmos se o modelo � veross�mil ou n�o. A equa��o acima
deve ser v�lida para qualquer freq��ncia. Ent�o, ao inv�s de calcularmos $N$ vezes a capacit�ncia atrav�s das $N$ medidas de imped�ncia e freq��ncia, constru�mos
o gr�fico $(Z_C,\sigma_{Z_C})\times1/\omega$\footnote{Aqui consideramos j� ter sido feita a transfer�ncia de erros de $1/\omega$ para $Z_C$.}, procedemos a regre��o
linear e, do coeficiente angular do ajuste, calculamos a capacit�ncia. Ao coeficiente linear nada se pode associar fisicamente com este modelo, a n�o ser, talvez, 
a uma avalia��o da qualidade de nossas medidas, visto que esperamo-lo ser zero. 
	Se formos capazes de concordar a teoria com a experi�ncia neste caso teremos confirmado a efici�ncia do modelo do capacitor.

	Medimos $(f,\sigma_f)$, $(V_R,\sigma_{V_R})$, $(V_C,\sigma_{V_C})$ e $(\phi,\sigma_{\phi})$, onde $\phi$ � a defasagem entre os sinais da tens�o
e corrente. Isto pode ser medido colocando-se um canal do oscilosc�pio no resistor, o outro no capacitor\footnote{� verdade que em ambos os casos estaremos medindo 
tens�es mas, como o resistor n�o possui parte complexa em sua imped�ncia, o que significa que ele n�o imprime diferen�a de fase entre a tens�o e a corrente 
sobre ele, ent�o a tens�o sobre o resistor est� em fase com a corrente de todo o circuito e, como tudo o que se quer � uma medida de tempo, o m�todo prova-se 
ser perfeitamente aceit�vel.} e medindo-se a diferen�a de fase entre os dois sinais. Quanto � freq��ncia, sua medida foi obtida de forma digital, atrav�s do
menu {\it measure} do oscilosc�pio\footnote{� claro que poder�amos ter medido o per�odo e, em seguida, calculado a freq��ncia. Mas, porque n�o utilizar os
recursos que temos?} e sua incerteza ($\sigma_f$) foi estimada como a soma quadr�tica do erro estat�stico ($\sigma_e$), devido �s flutua��es dos valores 
indicados, e do erro instrumental ($\sigma_i$), igual � unidade do d�gito significativo:
	\begin{equation}
	\sigma_f^2=\sigma_e^2+\sigma_i^2
	\label{eq:incerteza}
	\end{equation}
 
	As incertezas nas tens�es sobre o resistor e o capacitor foram tamb�m obtidas de forma digital, da mesma forma que a freq��ncia. Ou seja, vale
a equa��o (\ref{eq:incerteza}) para a obten��o de $\sigma_{V_R}$ e $\sigma_{V_C}$, efetuando-se, � claro, as devidas mudan�as de �ndices.
	
	Medimos o valor do resistor com o mult�metro e obtemos $100,5\,\Omega$. Mas curto-circuitando as pontas do mult�metro encontramos $0,5\,\Omega$. Ent�o,
o valor verdadeiro do resistor $R$ � $100,0\,\Omega$ -- precisamente o valor nominal -- com uma incerteza de $0,5\,\Omega$, conforme especifica��es do manual
do mult�metro\footnote{Medimos em fundo de escala $200\,\Omega$, cuja precis�o era $0,2\%+3$ d�gitos. Resolu��o de $100\,\mathrm{m}\Omega$. Ent�o, a
incerteza � $0,002\cdot100,0+3\cdot0,1=0,5\,\Omega$.}. 

	O capacitor era de $1\mu\mathrm{F}\,\,(10\%)\,\,250\mathrm{V}$ nominal. O modelo utilizado para o capacitor n�o inclu�a uma resist�ncia em paralelo
pois esperava-se que esta resist�ncia fosse muito grande e de fato isto foi verificado quando medimos a resist�ncia �hmica, com o mult�metro, entre os terminais do
capacitor e encontramos uma resist�ncia maior que $20\,\mathrm{M}\Omega$. N�o podemos precisar o valor pois o maior fundo de escala do mult�metro era de 
$20\,\mathrm{M}\Omega$. Assim, esperamos, atrav�s do roteiro h� pouco comentado, encontrar um valor de capacit�ncia entre $0,9\,\mu\mathrm{F}$ e 
$1,1\,\mu\mathrm{F}$.

	 A primeira grandeza que determinamos � a corrente, cuja rela��o � (\ref{eq:corrente}). Como $V_R$ e $R$ n�o s�o correlacionados, ent�o a incerteza na 
corrente � simplesmente
	\begin{eqnarray} 
	\sigma_I^2	&=&({\partial I\over\partial R}\sigma_R)^2+({\partial I\over\partial V_R}\sigma_{V_R})^2	\nonumber\\
			&=&({V_R\sigma_R\over R^2})^2+({\sigma_{V_R}\over R})^2						\nonumber\\
	\label{eq:erro corrente} 		
			&=&{(I\sigma_R)^2+\sigma_{V_R}^2\over R^2}
	\end{eqnarray}

	J� da rela��o (\ref{eq:impedancia}) podemos tirar a incerteza na imped�ncia do capacitor, com uma ressalva: Aqui, $V_C$ e $I$ s�o 
interdependentes, de modo que a propaga��o de erros fica um pouco diferente:
	\begin{eqnarray*}
	\sigma_{Z_C}^2	&=	&({\partial Z\over\partial V_C}\sigma_{V_C})^2+({\partial Z\over\partial I}\sigma_I)^2
				+2{\partial Z\over\partial V_C}{\partial Z\over\partial I}\sigma_{IV_C}^2 		\\
			&\le	&({\partial Z\over\partial V_C}\sigma_{V_C})^2+({\partial Z\over\partial I}\sigma_I)^2
				+2{\partial Z\over\partial V_C}{\partial Z\over\partial I}\sigma_I\sigma_{V_C} 		\\
	\sigma_{Z_C}^2	&\le	&{\sigma_{V_C}^2+(Z\sigma_I)^2\over I^2}-2{V_C\sigma_{V_C}\sigma_I\over I^3}		
	\end{eqnarray*}

	Note que fizemos uso da rela��o $\sigma_{IV_C}^2\le\sigma_I\sigma_{V_C}$. Mas como a �ltima parcela da �ltima equa��o � sempre negativa\footnote{Lembre-se 
que $V_C$ e $I$ s�o os m�dulos dos vetores complexos $\hat{V}_C$ e $\hat{I}$ respectivamente.}, podemos superestimar o erro cometido em $Z$:
	\begin{equation}
	\sigma_{Z_C}={\sqrt{(Z\sigma_I)^2+\sigma_{V_C}^2}\over I}
	\label{eq:erro impedancia}
	\end{equation} 	


	Assim, acabamos de determinar $(Z_C,\sigma_{Z_C})$. Falta o conjuto $(1/\omega,\sigma_{1/\omega})$, que deve ser obtido da freq��ncia:
	$$
	\omega=2\pi f	\Rightarrow	{1\over\omega}={1\over 2\pi f}
	$$

	e a incerteza � tamb�m simples:
	$$
	\sigma_{1/\omega}=|{\partial\over\partial f}({1\over\omega})\sigma_f|={\sigma_f\over\omega f}
	$$

	Agora podemos plotar o gr�fico $(Z_C,\sigma_{Z_C})\times(1/\omega,\sigma_{1/\omega})$, mas n�o conv�m pois dificulta a lineariza��o dos pontos se
considerarmos os erros em $1/\omega$. O que fazemos � transferir estes erros para a imped�ncia $Z_C$, deste modo: Primeiro ignoramos os erros na abscissa e
plotamos o gr�fico $(Z_C,\sigma_{Z_C})\times(1/\omega)$. Linearizamos e determinamos o coeficiente angular $m$. Da� basta utilizarmos a express�o
	$$
	\sigma_{1Z_C}=\sigma_{Z_C}+m\sigma_{1/\omega}
	$$

	onde $\sigma_{1Z_C}$ � a nova incerteza em $Z_C$ e $1/\omega$ passa a n�o ter erro algum.

\begin{figure}[htbp]
	\begin{minipage}[t]{0.46\textwidth}% 
		\includegraphics[scale=0.25]{capacitanciaRC.jpg} 
		\caption{\footnotesize Determinamos o valor da capacit�ncia pelo coeficiente angular do ajuste de reta aos pontos do gr�fico.}
	\end{minipage}
	\hfill
	\begin{minipage}[t]{0.46\textwidth}%  
		\centering
		\includegraphics[scale=0.25]{defasagemRC.jpg}
		\caption{\footnotesize A defasagem entre a corrente e a tens�o permanece pr�xima de um quarto de per�odo ($\pi/2\mathrm{rad}$), conforme 
			previsto no modelo.} 
	\end{minipage}
	\label{fig:capacitancia}       
\end{figure}

%%%%%%%%%%%%%%%%%%%%%%%%%%%%%%%%%%%%%%%%%%%%%%%%%%%%% ERRO DE REFER�NCIA CRUZADA %%%%%%%%%%%%%%%%%%%%%%%%%%%%%%%%%%%%%%%%%%%%%%%%%%%%%%%%%%%%%%%%%%
							     
	Ao fazermos o ajuste de reta aos pontos do gr�fico da figura 1, escolhemos um ajuste que n�o force a passagem da reta pela origem pois,
como foi comentado no come�o, obtermos um coeficiente linear pequeno pode indicar boa qualidade das medidas. De fato, encontramos $a=2(4)$ e 
$b=957(8)\times10^3$, respectivamente coeficientes linear e angular. Ao compararmos a forma $y=a+bx$ da equa��o ajustada � equa��o $Z_C=1/\omega C$, vemos que 
	$$
	b={1\over C}  \Rightarrow	C={1\over b}	\Rightarrow C=1,044\mu\mathrm{F}
	$$

	al�m disso,
	$$
	\sigma_C=|{\partial C\over\partial b}\sigma_b|=C^2\sigma_b=0,009\mu\mathrm{F}
	$$

	Quanto � defasagem entre a corrente e a tens�o, sabemos da teoria que a corrente est� $\pi/2\,\,\mathrm{rad}$ atrasada com rela��o � tens�o. Considere
$\phi$ a defasagem medida em unidades de tempo e $\Phi$ a defasagem em graus. Se $f=1/T$, onde $T$ � o per�odo em unidades de tempo, as duas se relacionam
da seguinte forma:
	$$
	\Phi=360\cdot{\phi\over T}=360\cdot\phi\cdot f
	$$

	e, consequentemente, a incerteza �
	$$
	\sigma_{\Phi}=\phi\sqrt{({\sigma_f\over f})^2+({\sigma_{\phi}\over\phi})^2}
	$$

%%%%%%%%%%%%%%%%%%%%%%%%%%%%%%%%%%%%%%%%%%%%%%%%%%% ERRO DE REFER�NCIA CRUZADA %%%%%%%%%%%%%%%%%%%%%%%%%%%%%%%%%%%%%%%%%%%%%%%%%%%%%%%%%%%%%%%%%%%
	Daqui fazemos o gr�fico $(f,\sigma_f)\times(\Phi,\sigma_{\Phi})$ (figura 2). 

	
	Isto finaliza nossa verifica��o do modelo utilizado para o capacitor e, como encontramos $C=1,044(9)\,\mu\mathrm{F}$, conclu�mos que de fato o modelo 
descreve bem o componente, ao menos para baixas freq��ncias.
	
\subsection{Circuito RL} 

	Podemos utilizar o mesmo circuito anterior para analisar o modelo dado ao indutor, bastanto para isso substituir o capacitor
pelo indutor. A metodologia � exatamente a mesma: Medimos a tens�o sobre o resistor conhecido (O mesmo que no caso anterior) determinando a corrente no
circuito. Disto, e medindo a tens�o sobre o indutor, encontramos $Z_L$. Mas h� um pequeno detalhe: O indutor, sendo um fio comprido, tem uma resist�ncia
interna consider�vel, e n�o h� como separ�-la do indutor. Ou seja, o Indutor n�o �, de forma alguma, ideal. Ele exige que levemos em conta esta nova 
imped�ncia, que consideramos estar em s�rie:
	\begin{equation}
	\hat{Z}_L=Z_R+X_Le^{j\theta}=R_L+\omega Le^{j\theta}
	\label{eq:reatancia indutiva}
	\end{equation} 

	Este � o modelo do indutor. E lembre-se que na determina��o da imped�ncia do circuito RC dissemos tratar-se do m�dulo dela. Aqui isto se torna 
evidente pois temos
	\begin{equation}
	|\hat{Z}_L|^2=R_L^2+\omega^2L^2
	\label{eq:indutancia e resistencia}
	\end{equation} 

	E a lineariza��o � feita no plano $Z_L^2\times\omega^2$. 
	
	Medimos $(f,\sigma_f)$, $(V_R,\sigma_{V_R})$, $(V_L,\sigma_{V_L})$ e $(\phi,\sigma_{\phi})$. A defasagem foi obtida da mesma forma que no caso 
anterior e as incertezas na freq��ncia e tens�es no resistor e indutor, obtidas exatamente como em (\ref{eq:incerteza}).	
	
	O indutor utilizado � de $40\,\mathrm{mH}$ com incerteza de $3\,\mathrm{mH}$ e resist�ncia interna $R_L=9,94\pm0,09\,\Omega$, $1000$ espiras e $1\,A$
de corrente m�xima suportada. Estas s�o as grandezas que queremos verificar atrav�s do modelo para o indutor e, neste caso, a regress�o dos pontos 
$Z_L^2\times\omega^2$ dar� um coeficiente angular, ao qual associamos o quadrado da indut�ncia, e um coeficiente
linear, que associamos ao quadrado da resist�ncia interna do indutor, Equa��o~\ref{eq:indutancia e resistencia}.

	Os erros na corrente e na imped�ncia do indutor s�o id�nticos a, respectivamente, (\ref{eq:erro corrente}) e (\ref{eq:erro impedancia}), pois s�o
obtidos das mesmas equa��es (\ref{eq:corrente}) e (\ref{eq:impedancia}):

	\begin{minipage}{0.46\textwidth}
	$$ 
	\sigma_I^2={(I\sigma_R)^2+\sigma_{V_R}^2\over R^2}
	$$
	\end{minipage}
	\hfill
	e
	\hfill
	\begin{minipage}{0.46\textwidth}
	$$
	\sigma_{Z_L}^2={(Z\sigma_I)^2+\sigma_{V_L}^2\over I^2}
	$$
	\end{minipage} 

	Temos agora de determinar $(\omega^2,\sigma_{\omega^2})$ e $(Z_L^2,\sigma_{Z_L^2})$\footnote{Note que $\sigma_{Z_L}^2\ne\sigma_{Z_L^2}$.}:

	\begin{minipage}{0.46\textwidth}
	$$
	\omega=2\pi f	\Rightarrow	\omega^2=4\pi^2 f^2
	$$
 	\end{minipage} 
	\hfill
	e 
	\hfill
	\begin{minipage}{0.46\textwidth}
	$$
	\sigma_{\omega^2}=|{\partial\omega^2\over\partial f}\sigma_f|=4\pi\omega\sigma_f
	$$
	\end{minipage} 

	Al�m disso,
	$$
	\sigma_{Z_L^2}=|{\partial\over\partial Z}(Z^2)\sigma_{Z_L}|=2{Z_L}\sigma_{Z_L}
	$$

	O que falta agora � transferir o erro de $\omega^2$ para $Z_L^2$, e isto � feito tamb�m identicamente ao caso anterior, de modo a termos, por fim
$(Z_L^2,\sigma_{Z_L^2})\times\omega^2$. 


\begin{figure}[htbp]
	\begin{minipage}[t]{0.46\textwidth}% 
		\includegraphics[scale=0.25]{indutanciaRL.jpg} 
		\caption{\footnotesize Ao linearizar e ajustar a reta aos pontos, ao coeficiente angular d� o valor da Indut�ncia, enquanto que o
			coeficiente linear d� o valor da resist�ncia interna.}
	\end{minipage}
	\hfill
	\begin{minipage}[t]{0.46\textwidth}%  
		\centering
		\includegraphics[scale=0.25]{defasagemRL.jpg}
		\caption{\footnotesize Uma vez que o Indutor possui uma resist�ncia interna apreci�vel, ele deixa de ser ideal, fazendo com que a diferen�a
			de fase entre tens�o e corrente no circuito seja diferente de $\pi/2\,\mathrm{rad}$, embora caminhe para este valor � medida que
			a freq��ncia aumenta.}
	\end{minipage}
\end{figure}

	Agora ambos os coeficientes angular e linear t�m significado f�sico, $a=84(37)=R_L^2\Rightarrow R_L=9,2\,\Omega$ e 
$b=1705\times10^{-6}=L^2\Rightarrow L=41,29\,\mathrm{mH}$. Ent�o

\vspace{10pt}
	\begin{minipage}{0.46\textwidth}
	$$
	\sigma_{R_L}=|{\partial R_L\over\partial a}\sigma_a|={\sigma_a\over2R_L}=2,0\,\Omega
	$$
	\end{minipage}
	\hfill
	e
	\hfill
	\begin{minipage}{0.46\textwidth}
	$$
	\sigma_L=|{\partial L\over\partial b}\sigma_b|={\sigma_b\over 2L}=0,29\mathrm{mH}
	$$
	\end{minipage}

\vspace{10pt} 
%%%%%%%%%%%%%%%%%%%%%%%%%%%%%%%%%%%%%%%%%%% ERRO DE REFER�NCIA CRUZADA %%%%%%%%%%%%%%%%%%%%%%%%%%%%%%%%%%%%%%%%%%%%%%%%%%%%%%%%%%%%%%%
	J� a defasagem, embora os c�lculos sejam exatamente os mesmos que no caso anterior, o resultado � bem diferente pois a fase varia 
(fig.5).

	
	Assim, como encontramos $L=41,29(29)\,\mathrm{mH}$ e $R_L=9,2(20)\,\Omega$, concluimos que o modelo para o Indutor tamb�m est� bom.

\subsection{Filtro RC}

	Agora que j� verificamos que os modelos utilizados para descrever circuitos de correntes alternadas s�o razo�veis, vamos aplic�-los a circuitos simples
e ver no que d�. O primeiro deles � o Filtro RC, cuja principal caracter�stica verIficada � a de filtrar sinais de acordo com a freq��ncia. Neste caso, sinais
com freq��ncia inferior a $\omega_c$ que entrarem no circuito sair�o como se nada tivesse acontecido. Por�m, se o sinal for de freq��ncia superior a $\omega_c$,
ent�o este ser� atenuado cada vez mais at� seja perdido na sa�da. Este � o que se chama {\it Filtro RC Passa-Baixa}.

	A freq��ncia de corte foi determinada na se��o \ref{sec:teoria} e �
	\begin{equation}
	\omega_c={1\over RC}	\Rightarrow	f_c={1\over 2\pi RC}
	\label{eq:fcorte}
	\end{equation} 

	Escolhemos utilizar uma freq��ncia por volta de $500\,\mathrm{Hz}$ e, com o capacitor que tinhamos, de $1\,\mu\mathrm{F}\pm10\%$, ter�amos de conseguir
um resistor de $318\,\Omega$. Conseguimos um de $333\,\Omega$ (nominal), cujo valor real era $328,0\,\Omega$ com incerteza de $1,3\,\Omega$. Ent�o a freq��ncia de
corte foi avaliada em $485\,\mathrm{Hz}$ com incerteza de $8\,\mathrm{Hz}$. O erro foi obtido por propaga��o atrav�s da equa��o (\ref{eq:fcorte}):
	$$
	\sigma_{f_c}={f_c\over2\pi}\sqrt{({\sigma_R\over R})^2+({\sigma_C\over C})^2}
	$$

	Medimos ent�o $(V_e,\sigma_{V_e})$ e $(V_s,\sigma_{V_s})$ e, como o ganho real ($\mathrm{Re}[\hat{G}]$) � dado simplesmente pela raz�o da tens�o de
sa�da $V_s$ pela tens�o de entrada $V_e$, a incerteza no ganho � avaliada -- como em (\ref{eq:erro impedancia}), em
	$$
	\sigma_G={\sqrt{\sigma_{V_s}^2+G^2\sigma_{V_e}^2}\over V_C}
	$$

	Ent�o o que temos a fazer � simplesmente plotar o gr�fico de $(G,\sigma_G)\times(f,\sigma_f)$\footnote{Perceba que como n�o pretendemos fazer ajuste
algum, n�o h� necessidade de se transferir os erros da abscissa para a ordenada.}.


\begin{figure}[htbp]
	\begin{minipage}[t]{0.46\textwidth}% 
		\includegraphics[scale=0.25]{ganhologlog.jpg} 
		\caption{\footnotesize O ganho real do filtro RC como fun��o da freq��ncia do sinal de entrada. A freq��ncia de corte est� indicada em vermelho.}
	\end{minipage}
	\hfill
	\begin{minipage}[t]{0.46\textwidth}%  
		\centering
		\includegraphics[scale=0.25]{defasagem_filtroRC.jpg}
		\caption{\footnotesize A defasagem do sinal entre os sinais de entrada e sa�da em fun��o da freq��ncia.} 
	\end{minipage}
\end{figure}

\subsection{Circuito Integrador}
	
	Um outro efeito do filtro RC passa-baixa � modificar a forma do sinal de sa�da para freq��ncias muito maiores que $\omega_c$. Essa modifica��o n�o
� aleat�ria nem imprevis�vel. O que ocorre � que o sinal de sa�da passa a ter a forma da integral do sinal de entrada. Pela se��o \ref{sec:teoria}:
	$$
	V_s(t)={1\over RC}\int V_e(t)\mathrm{dt}
	$$

	e o que fizemos foi simplesmente copiar em papel milimetrado -- respeitando as escalas do oscilosc�pio -- a forma dos sinais de entrada e 
sa�da\footnote{Neste circuito utilizamos um sinal quadrado de entrada e escolhemos a freq��ncia $5005\,\mathrm{Hz}\gg f_c$.}.  

	A verifica��o da concord�ncia entre a teoria e a experi�ncia � feita atrav�s de um c�lculo simples, levando em considera��o a figura \ref{fig:integrador}:

\begin{figure}[htb]
	\center
	\includegraphics[scale=0.6]{integrador.jpg}
	\caption{\footnotesize Forma de onda obtida da tela do oscilosc�pio. O sinal quadrado � o de entrada e o dente de serra � a sa�da.}
	\label{fig:integrador} 
\end{figure}  
	
	Tome o meio per�odo da onda, intervalo esse no qual temos $V_e(t)=1V$. Ent�o, o sinal de sa�da deve ser
	$$
	V_s(t)={1\over RC}\int_{t_0}^t 1\cdot\mathrm{dt}={(t-t_0)\over RC}\approx3048,78\cdot(t-t_0)  
	$$

	Escolhendo $t_0=46,25\,\mu\mathrm{s}$, $V_s(t)=3048,78\cdot(t-46,25)$ ($[t]=\mu\mathrm{s}$), ent�o em $t=0$ (Que corresponde � borda de subida do
sinal de entrada) teremos $-141,0\,\mathrm{mV}$, contra $-140\,\mathrm{mV}$ medido. E, para $t=92,5\,\mu\mathrm{s}$, $V_s=141,0\,\mathrm{mV}$, contra os
$140\,\mathrm{mV}$ medido. Vemos que h� uma �tima concord�ncia entre a teoria e a pr�tica. 

	Pode-se pergutar por que utilizamos o valor $t_0=46,25\,\mu\mathrm{s}$. Isto foi feito porque, embora na figura~\ref{fig:integrador} tenhamos desenhado
ambos os sinais centralizados longitudinalmente na refer�ncia, isto de fato n�o ocorre pois o sinal do capacitor � sempre positivo (Ou sempre negativo, dependendo
da refer�ncia que se toma ao medir), enquanto que o sinal de entrada de fato muda de sinal a cada meio ciclo. Ent�o o que ter�amos � um sinal DC a ser somado � 
sa�da do integrador. Para n�o fazer isso, pois ter�amos de modificar os dados experimentais por argumentos te�ricos, preferimos alterar a origem do sinal de 
forma a corresponder com o esperado.

\subsection{Perda de energia por histerese de um indutor}

	Todos os circuitos dissipam energia, principalmente sobre a forma de {\it Efeito Joule}. Contudo, esta n�o � �nica forma que um circuito tem para
perder energia. Se o circuito possuir um transformador\footnote{Que n�o passa de dois indutores arranjados de forma a trocarem energia magn�tica.}, tamb�m 
devemos levar em considera��o a energia passada do circuito para o material ferromagn�tico (e para o
meio exterior) do indutor, na forma de eletromagnetiza��o destes �tomos. 

	A energia perdida por ciclo de oscila��o do sinal de entrada foi estimada na se��o~\ref{sec:teoria} em
	\begin{equation}
	u\approx{R_2C\over R_1}\cdot{N_p\over N_s}\int V_C\mathrm{dV_R}
	\label{eq:histerese}
	\end{equation}

	onde $R_1$ � o resistor limitador de corrente, $R_2$ e $C$ s�o o resistor e o capacitor do circuito integrador acoplado � sa�da do circuito que nos 
interessa (Circuito RL), $N_p/N_s$ � a rela��o do n�mero de espiras entre o terminal principal e secund�rio do indutor e a integral � ao longo de todo um
per�odo da histerese.

	Avaliamos $R_1$ em $1\,013\pm2\,\Omega$ e $R_2$ em $101,6\pm0,2\,\mathrm{k}\Omega$ e o capacitor era de $1\,\mu\mathrm{F}\pm10\%$ (Nominal). Portanto
o circuito integrador tem uma freq��ncia de corte de aproximadamente $1,6\,\mathrm{Hz}$. Como utilizamos a freq��ncia da rede ($60\,\mathrm{Hz}$) no nosso
circuito, ent�o est� claro que de fato o circuito RC acoplado ao terminal secund�rio do transformador est� agindo como integrador.

	O valor de $N_p/N_s$ pode ser obido medindo-se as tens�es de entrada ($44,2(1)\,V$) e sa�da ($6,90(1)\,V$) do transformador:
	$$
	{N_p\over N_s}={V_s\over V_e}\approx6,406
	$$

	O erro � f�cil:
	$$
	\sigma_{N_p/N_s}={\sqrt{\sigma_{V_e}^2+(N_p/N_s)^2\sigma_{V_s}^2}\over V_s}=0,009    
	$$

	Assim, temos quase todos os dados necess�rios para resolver (\ref{eq:histerese}). Falta apenas a integral. Mas esta � a �rea da histerese, em 
$\mathrm{Voltz}^2$, e podemos medi-la copiando-a para um papel milimetrado e calculando a �rea ``no bra�o''. Mas h� uma forma mais inteligente de se
resolver isso -- E tamb�m mais precisa: Determinamos a densidade do papel milimetrado vegetal utilizado medindo-se uma �rea conhecida e pesando-se este
peda�o de papel em uma balan�a anal�tica de alta precis�o. No caso, recortamos um peda�o de papel vegetal de $16\,\mathrm{cm}^2$ e avaliamos sua massa
em $0,1074(1)\,\mathrm{g}$. A incerteza na �rea podemos estimar por volta de $2\,\mathrm{mm}^2$. Ent�o a densidade do papel � 
$\rho=0,006712(10)\mathrm{g/cm^2}$.

	Agora recortamos a figura da histerese impressa no papel milimetrado e medimos sua massa. Fizemos, na realidade, duas medidas de massa pois o tra�o
da histerese na tela do oscilosc�pio n�o era muito bem definido, de modo que acabamos por desenhar uma ``histerese interna'' e uma ``histerese interna'',
sendo que pressupomos que a verdadeira est� contida entre essas duas. Ent�o, medimos a massa da ``histerese externa'' em $m_{he}=0,0603(1)\,\mathrm{g}$ e,
consequentemente, sua �rea � 
	$$
	A_{he}={m_{he}\over\rho}=8,983\,\mathrm{cm}^2
	$$

	com incerteza de $0,013\,\mathrm{cm}^2$. J� para a ``histerese interna'' obtivemos $m_{hi}=0,0412(10)\,\mathrm{g}$, dando uma �rea $6,138\,\mathrm{cm}^2$
com incerteza $0,017\,\mathrm{cm}^2$. Na tela do oscilosc�pio, 1 divis�o equivale a $0,9\,\mathrm{cm}$ e, al�m disso, quando fizemos a medida est�vamos com
o canal 1 em $5,00\,\mathrm{V/div}$ e o canal 2 em $100\,\mathrm{mV/div}$. Ent�o as �reas das histereses interna e externa em divis�es da escala do
oscilosc�pio s�o, respectivamente, $7,578(2)\,\mathrm{div}^2$ e $11,086(16)\,\mathrm{div}^2$. E como a cada $\mathrm{div}^2$ correspondem 
$5,00\cdot100\times10^{-3}=0,5\mathrm{V}^2$, ent�o temos que as �reas interna e externa em Voltz s�o $3,789(1)\,\mathrm{V}^2$ e $5,543(8)\,\mathrm{V}^2$.
	
	Finalmente temos tudo o que precisamos:
	$$ 
	u_{hi}\approx{101,6\times10^3\cdot1\times10^{-6}\over1013}\cdot6,406\cdot3,789\approx2,43\,\mathrm{mJ}
	$$

	e
	$$
	u_{he}\approx{101,6\times10^3\cdot1\times10^{-6}\over1013}\cdot6,406\cdot5,543\approx3,6\,\mathrm{mJ}
	$$

	Ou seja, a energia perdida por ciclo de histerese est� entre $2,43(24)\,\mathrm{mJ}$ e $3,6(3)\,\mathrm{mJ}$. Ou ainda, $u=3,0(8)\,\mathrm{mJ}$. Este �ltimo
foi obtido somando-se quadraticamente as incertezas em cada uma das histereses (``interna'' e ``externa'') e o desvio padr�o desse conjunto de duas medidas.




















\end{document}
 
