\documentclass[a4paper, 12pt]{scrartcl}
	\usepackage[utf8]{inputenc}
	\usepackage{fourier}
	\usepackage[brazil]{babel}
	\usepackage{amsmath}
	\usepackage{paralist}
	\usepackage{tikz}\usetikzlibrary{calc,arrows}
	\tikzset{>=stealth',every on chain/.append style={join},every join/.style={->}}

	\title{A órbita dos planetas}
	\subtitle{Solução do problema de Kepler, ou ``a 1\textordfeminine~lei de Kepler''}
	\author{I. R. Pagnossin}

	\newcommand\earthmass{\ensuremath{M_{\oplus}}}
	\newcommand\sunmass{\ensuremath{M_{\odot}}}

\begin{document}
	\maketitle

	\begin{abstract}
		Este artigo demonstra como obter a equação da órbita da Terra, utilizando os seguintes conceitos físicos:
		\begin{compactitem}
			\item Lei da gravitação universal de Newton
			\item 2\textordfeminine~lei de Newton
			\item Momento angular
			\item Torque
			\item Conservação do momento angular
			\item Cinemática: posição, velocidade e aceleração.
		\end{compactitem}

		Ele também utiliza os seguintes conceitos matemáticos:
		\begin{compactitem}
			\item Álgebra vetorial
			\item Coordenadas polares
			\item Seções cônicas
		\end{compactitem}
	\end{abstract}

	\begin{align*}
		\vec F &= - G \sunmass \earthmass \frac{\hat r}{r^2} &&\leftarrow\text{Lei da gravitação universal de Newton} \\
		       &= \earthmass \vec a = \earthmass \frac{d^2\vec r}{dt^2}         &&\leftarrow\text{2\textordfeminine~lei de Newton}.
	\end{align*}

	Igualando, obtemos a equação da órbita:
	\begin{equation}\label{eq:a}
		\vec a = \frac{d^2\vec r}{dt^2} = - D \frac{\hat r}{r^2},
	\end{equation}
	onde fizemos $D = G\sunmass$.

	Como $\vec F \parallel \vec r$ (movimento descrito no referencial do Sol), o torque exercido pela força gravitacional $\vec F$ sobre a Terra é nulo: $\vec \tau \doteq \vec r \times \vec F = \vec 0$.
	Consequentemente, o momento angular orbital $\vec L$ da Terra é conservado:
	\begin{equation}\label{eq:L}
		\vec L = \earthmass r^2 \dot\theta \hat k,
	\end{equation}
	utilizando coordenadas polares.

	Efetuando o produto vetorial de \eqref{eq:a} por \eqref{eq:L}, obtemos:
	\begin{equation*}
		\vec a \times \vec L = - D \frac{\hat r}{r^2} \times \earthmass r^2 \dot\theta \hat k
		                     = - \earthmass D \dot\theta \hat{r}\times\hat{k}
		                     = D \earthmass \dot\theta \hat\theta,
	\end{equation*}
	lembrando que, para a base polar ortonormal $(\hat{r}, \hat{\theta}, \hat{k})$, $\hat\theta = \hat{k}\times\hat{r}$.

	Usando a identidade $\frac{d\hat{r}}{dt} = \dot\theta \hat\theta$,
	\begin{align}
		\vec a \times \vec L &= D \earthmass \frac{d\hat{r}}{dt} \Rightarrow \nonumber\\
		\Rightarrow \frac{d \vec v}{dt} \times \vec L &= \frac{d}{dt}\left(D\earthmass\hat{r}\right) \Rightarrow \nonumber\\
		\Rightarrow \frac{d}{dt}\left(\vec v \times \vec L\right) &= \frac{d}{dt}\left(D\earthmass\hat{r}\right) \Rightarrow \nonumber\\
		\Rightarrow \vec{v} \times \vec{L} &= D\earthmass \left(\hat r + \vec e\right), \label{eq:vxL}
	\end{align}
	onde $\vec e$ é uma constante de integração.

	Efetuando o produto escalar de $\vec{r}$ por \eqref{eq:vxL},
	\begin{align*}
		\vec{r} \cdot \left(\vec{v} \times \vec{L}\right) &= \vec{r} \cdot D\earthmass \left(\hat{r} + \vec{e}\right) \Rightarrow\\
		\Rightarrow \vec{r} \times \vec{v} \cdot \vec{L} &= D\earthmass \left(\vec{r} \cdot \hat{r} + \vec{r} \cdot \vec{e}\right) \Rightarrow\\
		\Rightarrow \frac{\vec{L}}{\earthmass} \cdot \vec{L} &= D\earthmass \left(r + re\cos\varphi\right) \Rightarrow\\
		\Rightarrow L^2 &= D\earthmass^2\left(1 + e \cos\varphi\right) r.
	\end{align*}

	Se usarmos a orientação definida por $\vec e$ como eixo polar, $\varphi \equiv \theta$ e, consequentemente,
	\begin{equation*}
		r(\theta) = \frac{L^2/D\earthmass^2}{1 + e \cos\theta},
	\end{equation*}
	que é a equação de uma seção cônica com excentricidade $e$, em coordenadas polares.

	Logo, decorre da lei da gravitação universal de Newton e sua segunda lei do movimento que o movimento dos planetas ao redor do Sol deve ser uma seção cônica.
	Entretanto, perceba que essa solução só é válida em primeira aproximação, na qual desconsideramos:
	\begin{compactitem}
		\item O torque exercido pela força gravitacional do Sol sobre a Terra, que devido ao fato dela não ser perfeitamente esférica, não é realmente nulo.
		\item O torque exercido pelos demais astros do sistema solar.
	\end{compactitem}

	\begin{figure}
		\centering
		\begin{tikzpicture}[scale=1]
	\clip (-2,-1.6) rectangle (7.5,3);

	\coordinate (r) at (10:4);
	\coordinate (A) at (10:6);
	\coordinate (B) at ($(r)+(-1.031,-0.688)$);
	\coordinate (vxL) at ($(A)+(-1.031,-0.688)$);
	\coordinate (L) at (2,2.5);

	% \hat{r}
	\draw [->, thick] (0,0) -- (10:1);
	\draw (10:1) node[anchor=south] {$\hat{r}$};

	% \hat{\theta}
	\draw [->, thick] (0,0) -- (100:1);
	\draw (100:1) node[anchor=south] {$\hat{\theta}$};

	% \vec{r}: position vector
	\draw [->] (0,0) -- (r);
	\draw ($2/3*(r)$) node[anchor=south east] {$\vec{r}$};

	% \vec{v}: velocity vector
	\draw [->] (r) -- +(70:2);
	\draw (r) -- +(70:2) node[anchor=south west] {$\vec{v}$};

	% v x L
	\draw [->] (r) -- +(-20:1);
	\draw (r) -- +(-20:1) node[anchor=north west] {$\vec{v} \times \vec{L}$};

	% DM\hat r
	\draw [dashed, ->] (r) -- (A);
	\draw (r) (A) node[anchor=west] {$D\earthmass\hat{r}$};

	% DM\vec e
	\draw [dashed, ->] (r) -- (B);
	\draw (r) (B) node[anchor=north east] {$D\earthmass\vec{e}$};

	\draw [dotted, -] (A) -- (vxL) -- (B);

	% Pola axis
	\draw [dashed] (0,0) -- (-146.28:2);
	\draw (-146.28:2) node[anchor=north west] {\footnotesize $\text{Eixo polar} \parallel \vec e$};

	% \theta and \varphi
	\draw [->] (-146.28:0.25) arc (-146.28:+10:0.25);
	\draw (-73:0.25) node[anchor=north] {$\theta \equiv \varphi$};

	\draw (L) node {$\bigodot$};
	\draw (L) node[anchor=south west] {$\vec{L}$ e $\hat k$};

	% Sun
	\draw [color=gray,->] (160:0.5) -- (160:0.1);
	\draw [color=gray] (160:0.5) node[anchor=east] {\footnotesize Sol};
	\filldraw[fill=black] (0,0) circle (0.05);

	% Earth
	\draw [color=gray,->] (r) +(130:1) -- ($(r) +(130:0.1)$);
	\draw [color=gray] (r) +(130:1) node[anchor=south] {\footnotesize Terra};
	\filldraw[fill=black] (r) circle (0.05);


\end{tikzpicture}
		\caption{vetores que caracterizam a órbita.}
		\label{fig:orbit-vectors}
	\end{figure}

	\section*{Símbolos}
	\begin{compactdesc}
		\item[$\vec F$:] Força gravitacional
		\item[$\vec L$:] Momento angular
		\item[$\vec r$:] Posição orbital da Terra
		\item[$\vec v$:] Velocidade orbital da Terra
		\item[$\vec a$:] Aceleração orbital da Terra
		\item[$G$:] Constante gravitacional
		\item[\earthmass:] Massa da Terra
		\item[\sunmass:] Massa do Sol
		\item[$(\hat r, \hat \theta, \hat k)$:] Base ortonormal polar
	\end{compactdesc}


\end{document}