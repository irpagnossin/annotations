\begin{quotation}
{\bf 1.}
{\it	Sejam $\ket{u}$ e $\ket{v}$ dois vetores de um espa�o complexo com produto interno. Prove a desigualdade triangular
	\begin{equation}
	|\ket{u}+\ket{v}|\le|\ket{u}|+|\ket{v}|
	\label{eq:dt}
	\end{equation}
}
\end{quotation}

\vspace{5mm}

	Consideremos uma base ortonormal $\{\ket{i}\}$ deste espa�o complexo. Ent�o, os vetores $\ket{u}$ e $\ket{v}$ podem ser expressos por
	$$
	\ket{u}=\sum_{i=0}^{\infty}\underbrace{\iprod{i}{u}}_{u_i}\ket{i}
	$$
	$$
	\ket{v}=\sum_{i=0}^{\infty}\underbrace{\iprod{i}{v}}_{v_i}\ket{i}
	$$

	Nestas condi��es, podemos tamb�m escrever a soma destes dois vetores como
	$$
	\ket{u}+\ket{v}=\sum_iu_i\ket{i}+\sum_iv_i\ket{i}=\sum_i(u_i+v_i)\ket{i}
	$$

	Calculemos agora o quadrado do m�dulo da soma destes dois vetores:
	\begin{eqnarray*}
	\mathrm{mod}^2(\ket{u}+\ket{v})	&=&[\sum_i(u_i+v_i)^*\bra{i}]\cdot[\sum_j(u_j+v_j)\ket{j}]	\\
					&=&\sum_i\sum_j(u_i+v_i)^*(u_j+v_j)\iprod{i}{j}			\\
					&=&\sum_i(u_i+v_i)^*(u_i+v_i)					\\
					&=&\sum_i(|u_i|^2+|v_i|^2)+\sum_i(u_i^*v_i+v_i^*u_i)		\\
					&=&\sum_i(|u_i|^2+|v_i|^2)+|\iprod{u}{v}|+|\iprod{v}{u}|	\\
					&=&\sum_i(|u_i|^2+|v_i|^2)+2|\iprod{u}{v}|		
	\end{eqnarray*}

	E vejamos tamb�m o quadrado do m�dulo de $\ket{u}$ e $\ket{v}$:
	\begin{eqnarray*}
 	\mathrm{mod}^2\ket{u}	&=&\iprod uu								\\
				&=&(\sum_iu_i^*\bra{i})\cdot(\sum_ju_j\ket{j})				\\
				&=&\sum_i\sum_ju_i^*u_j\iprod ij					\\
				&=&\sum_i|u_i|^2							\\
	\mathrm{mod}^2\ket{v}	&=&\sum_i|v_i|^2		
	\end{eqnarray*}

	Queremos provar a equa��o (\ref{eq:dt}). Partamos dela mas, ao inv�s de usarmos o sinal $\le$ com significado conhecido para n�s, utilizaremos
o s�mbolo $\prec$ para indicar que existe uma rela��o de ordem entre os dois membros da equa��o mas que n�o a conhecemos ainda:
	\begin{eqnarray}
	\label{eq:dt2}\mathrm{mod}(\ket{u}+\ket{v})		&\prec&\mathrm{mod}\ket{u}+\mathrm{mod}\ket{v}							
													  \\
	\mathrm{mod}^2(\ket{u}+\ket{v})				&\prec&\mathrm{mod}^2\ket{u}+\mathrm{mod}^2\ket{v}+2\cdot\mathrm{mod}\ket{u}\mathrm{mod}\ket{v}	
												\nonumber \\
	\sum_i(|u_i|^2+|v_i|^2)+2|\iprod{u}{v}|			&\prec&\sum_i(|u_i|^2+|v_i|^2)+2\mathrm{mod}\ket{u}\mathrm{mod}\ket{v}				
												\nonumber \\
	|\iprod{u}{v}|						&\prec&|\ket{u}|\cdot|\ket{v}|	\nonumber 
	\end{eqnarray}

	Mas a �ltima express�o � precisamente a desigualdade de Schwartz e, como as manipula��es entre os dois membros n�o influenciaram a rela��o
$\prec$ entre eles, ent�o simplesmente substituimos $\prec$ por $\le$ em (\ref{eq:dt2}) e demonstramos (\ref{eq:dt}).

	