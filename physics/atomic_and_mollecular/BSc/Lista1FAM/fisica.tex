\newcommand{\operador}[1]{\mathcal{#1}}% FORMA MAIS F�CIL PARA EXPRESSAR UM OPERADOR
\newcommand{\bra}[1]{\ensuremath{\langle#1|}}% O VETOR BRA DO ESPA�O DE HILBERT.
\newcommand{\ket}[1]{\ensuremath{|#1\rangle}}% O VETOR KET DO ESPA�O DE HILBERT.
\newcommand{\iprod}[2]{\ensuremath{\langle#1|#2\rangle}}% O PRODUTO INTERNO NO ESPA�O DE HILBERT.
\newcommand{\hope}[3]{\ensuremath{\langle#1|#2|#3\rangle}}% O VALOR ESPERADO DO OPERADOR #2. OU AINDA: A ``ESPERAN�A MATEM�TICA''. 
\newcommand{\egauss}{\ensuremath{\mathbf{\nabla}\cdot\mathbf{D}=\rho}}%A LEI DE GAUSS PARA O CAMPO EL�TRICO.
\newcommand{\mgauss}{\ensuremath{\mathbf{\nabla}\cdot\mathbf{B}=0}}%A LEI DE GAUSS PARA O CAMPO MAGN�TICO.
\newcommand{\faraday}{\ensuremath{\mathbf{\nabla}\times\mathbf{E}=-{\partial\mathbf{B}\over\partial t}}}%A LEI DE FARADAY.
\newcommand{\ampere}{\ensuremath{\mathbf{\nabla}\times\mathbf{H}=\mathbf{J}+{\partial\mathbf{D}\over\partial t}}}%A LEI DE AMP�RE.


%ESTE NOVO COMANDO PRODUZ TODAS AS QUATRO EQUA��ES DE MAXWELL NO V�CUO E NA MAT�RIA NAS FORMAS DIFERENCIAL E INTEGRAL.
%S�O TR�S ARGUMENTOS: O PRIMEIRO � UM N�MERO DE 1 A 4 IDENTIFICANDO A EQUA��O DESEJADA:
%{1} PARA A LEI DE GAUSS PARA O CAMPO EL�TRICO, {2} PARA O CAMPO MAGN�TICO, {3} PARA A LEI DE FARADAY E {4} PARA A LEI DE AMP�RE;
%O SEGUNDO ARGUMENTO INDICA A FORMA DA EQUA��O DESEJADA: {d} PARA A FORMA DIFERENCIAL E {i} PARA A FORMA INTEGRAL;
%O TERCEIRO ARGUMENTO INDICA O MEIO: {v} PARA V�CUO E {m} PARA UM MEIO MATERIAL.
\newcommand{\maxwell}[3]{ 
	\ifthenelse{\equal{#1}{1}}{
		\ifthenelse{\equal{#2}{d}}{
			\ifthenelse{\equal{#3}{v}}{
				\ensuremath{\mathbf{\nabla}\cdot\mathbf{E}={\rho\over\epsilon_0}}
						  }{}
			\ifthenelse{\equal{#3}{m}}{
				\ensuremath{\mathbf{\nabla}\cdot\mathbf{D}={\rho}}
						  }{}
					  }{}
		\ifthenelse{\equal{#2}{i}}{
			\ifthenelse{\equal{#3}{v}}{
				\ensuremath{\oint\limits_{\Sigma}\mathbf{E}\cdot\mathrm{d}\mathbf{\Sigma}={\mathrm{q}\over\epsilon_0}}
						  }{}
			\ifthenelse{\equal{#3}{m}}{
				\ensuremath{\oint\limits_{\Sigma}\mathbf{D}\cdot\mathrm{d}\mathbf{\Sigma}=\mathrm{q_f}} 
						  }{}
				          }{}
				  }{}		
	\ifthenelse{\equal{#1}{2}}{
		\ifthenelse{\equal{#2}{d}}{
			\ifthenelse{\equal{#3}{v}}{
				\ensuremath{\mathbf{\nabla}\cdot\mathbf{B}=0}
						  }{}
			\ifthenelse{\equal{#3}{m}}{
				\ensuremath{\mathbf{\nabla}\cdot\mathbf{B}=0}
						  }{}
					  }{}
		\ifthenelse{\equal{#2}{i}}{
			\ifthenelse{\equal{#3}{v}}{
				\ensuremath{\oint\limits_{\Sigma}\mathbf{B}\cdot\mathrm{d}\mathbf{\Sigma}=0}
						  }{}
			\ifthenelse{\equal{#3}{m}}{
				\ensuremath{\oint\limits_{\Sigma}\mathbf{B}\cdot\mathrm{d}\mathbf{\Sigma}=0} 
						  }{}
				          }{}
				  }{}		
	\ifthenelse{\equal{#1}{3}}{
		\ifthenelse{\equal{#2}{d}}{
			\ifthenelse{\equal{#3}{v}}{
				\ensuremath{\mathbf{\nabla}\times\mathbf{E}=-{\partial\mathbf{B}\over\partial\mathrm{t}}}
						  }{}
			\ifthenelse{\equal{#3}{m}}{
				\ensuremath{\mathbf{\nabla}\times\mathbf{E}=-{\partial\mathbf{B}\over\partial\mathrm{t}}}
						  }{}
					  }{}
		\ifthenelse{\equal{#2}{i}}{
			\ifthenelse{\equal{#3}{v}}{
				\ensuremath{\oint\limits_{\Gamma}\mathbf{E}\cdot\mathrm{d}\mathbf{\Gamma}=
						-\mathrm{d\over dt}\int\limits_{\Sigma}\mathbf{B}\cdot\mathrm{d}\mathbf{\Sigma}}
						  }{}
			\ifthenelse{\equal{#3}{m}}{
				\ensuremath{\oint\limits_{\Gamma}\mathbf{E}\cdot\mathrm{d}\mathbf{\Gamma}=
						-\mathrm{d\over dt}\int\limits_{\Sigma}\mathbf{B}\cdot\mathrm{d}\mathbf{\Sigma}} 
						  }{}
				          }{}
				  }{}		
	\ifthenelse{\equal{#1}{4}}{
		\ifthenelse{\equal{#2}{d}}{
			\ifthenelse{\equal{#3}{v}}{
				\ensuremath{\mathbf{\nabla}\times\mathbf{B}=\mu_0\mathbf{J}+\mu_0\epsilon_0{\partial\mathbf{E}\over\partial\mathrm{t}}}
						  }{}
			\ifthenelse{\equal{#3}{m}}{
				\ensuremath{\mathbf{\nabla}\times\mathbf{H}=\mathbf{J}+{\partial\mathbf{D}\over\partial\mathrm{t}}}
						  }{}
					  }{}
		\ifthenelse{\equal{#2}{i}}{
			\ifthenelse{\equal{#3}{v}}{
				\ensuremath{\oint\limits_{\Gamma}\mathbf{B}\cdot\mathrm{d}\mathbf{\Gamma}=
						\mu_0I+\mu_0\epsilon_0\mathrm{d\over dt}\int\limits_{\Sigma}\mathbf{E}\cdot\mathrm{d}\mathbf{\Sigma}}
						  }{}
			\ifthenelse{\equal{#3}{m}}{
				\ensuremath{\oint\limits_{\Gamma}\mathbf{H}\cdot\mathrm{d}\mathbf{\Gamma}=
						I+\mathrm{d\over dt}\int\limits_{\Sigma}\mathbf{D}\cdot\mathrm{d}\mathbf{\Sigma}}     
						  }{}
				          }{}
				  }{}		

			}


