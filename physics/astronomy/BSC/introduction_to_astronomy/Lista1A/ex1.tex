%IVAN RAMOS PAGNOSSIN 11/03/2001 irpagnossin@hotmail.com

\begin{quotation}
{\color{blue}{\bf 1.}}
{\it Determine literalmente a dist�ncia entre o Sol e Marte por triangula��o tomando a unidade astron�mica como unidade de comprimento.}
\end{quotation}

\begin{figure}[htb]
\begin{center} 
\includegraphics[scale=0.6]{kepler.pdf}
\caption{\footnotesize Esquema de triangula��o utilizado para determinar a dist�ncia Sol-Marte como fun��o da dist�ncia Sol-Terra (=1UA).}
\label{fig:sol-marte} 
\end{center} 
\end{figure}

	Para resolver este problema imaginaremos a situa��o ilustrada na figura~\ref{fig:sol-marte}, que representa a oposi��o da Terra com Marte
alguns instantes antes (Ponto $E$) e alguns instantes depois (Ponto $E'$) de ocorrer. Necessitamos esta proximidade entre os eventos $E$ e $E'$ para 
podermos considerar Marte fixo no ponto $P$. $S$ � o Sol e os �ngulos $\alpha$\footnote{Note que este �ngulo, em graus, � aproximadamente o per�odo em dias 
entre os dois eventos $E$ e $E'$.}, $\beta$ e $\gamma$ s�o facilmente medidos. Desta forma, $\delta$ tamb�m est� definido atrav�s do quadril�tero
$\Box SEPE'$: $\delta=2\pi-(\alpha+\beta+\gamma)$. 
	Pressupostamente conhecemos $D=1\mathrm{UA}$ e queremos determinar $r$. Vejamos. Do tri�ngulo $\bigtriangleup SEE'$ podemos encontrar o valor do segmento
$\bar{EE'}$: 
	\begin{eqnarray*}
	(\bar{EE'})^2	&=&D^2+D^2-2D^2\cos\gamma	\\
			&=&2D^2(1-\cos\gamma) 		\\	
			&=&4D^2\sin^2({\gamma\over2})
	\end{eqnarray*}

	Isto �, ${\bar{EE'}\over D}=2\sin({\gamma\over2})$. Al�m disso, como este tri�ngulo � Is�celes, os �ngulos $\hat{E'ES}$ e $\hat{SE'E}$ s�o 
congruentes (E chamamo-los $\mu$ na figura~\ref{fig:sol-marte}). Isto nos permite determin�-los: $\gamma+2\mu=\pi\Rightarrow\mu=(\pi-\gamma)/2$.
Da� encontramos o �ngulo $\lambda=\hat{EE'P}$ como sendo $\lambda=\alpha-(\pi-\gamma)/2$, o que � necess�rio para determinarmos, pela {\it Lei do Senos}, 
a dist�ncia entre a Terra e Marte no instante $E$ no tri�ngulo $\bigtriangleup PEE'$:
	$$
	{\bar{EE'}\over\sin\delta}={d\over\sin\lambda}\Rightarrow d=\bar{EE'}{\sin\lambda\over\sin\delta}
	$$

	Mas $\sin\delta=-\sin(\alpha+\beta+\gamma)$ e $\sin\lambda=\cos(\alpha+{\gamma\over2})$. Portanto
	$$
	d=\xi(\alpha,\beta,\delta)D=2D\sin({\gamma\over2})[{\cos(\alpha+{\gamma\over2})\over-\sin(\alpha+\beta+\gamma)}]	
	$$

	Por fim, pela {\it Lei dos cossenos} aplicada ao tri�ngulo $\bigtriangleup SEP$, 
	\begin{eqnarray*}
	r^2		&=&D^2+d^2-2Dd\cos\beta			\\
			&=&D^2+D^2\xi^2-2D^2\xi\cos\beta	\\
	{r\over D}	&=&\sqrt{1+[\xi-2\cos\beta]\xi} 	
	\end{eqnarray*}

	com $\xi(\alpha,\beta,\gamma)=-2{\sin({\gamma\over2})\cos(\alpha+{\gamma\over2})\over\sin(\alpha+\beta+\gamma)}$. E o problema est� resolvido
uma vez que expressamos a dist�ncia Sol-Marte como fun��o de par�metros conhecidos, e com boa precis�o!
	    
\vspace{10mm}







