\begin{quotation}
{\bf 2.}
{\it Em que porcentagem diminui o peso de uma pessoa entre o p�lo sul e o equador devido {\bf(a)} 
� rota��o di�ria da Terra somente e {\bf (b)} devido ao achatamento nos p�los (Ignore a rota��o
neste caso)? 
}
\end{quotation} 

	{\tt (a)} A varia��o de peso neste caso deve-se � For�a de Coriolis, que produz uma
acelera��o perpendicular ao vetor velocidade angular da Terra. A componente perpendicular desta � 
superf�cie da Terra naquele ponto op�e-se � For�a gravitacional, acabando por reduzir o peso:
	\begin{eqnarray*}
	p		&=&p_0-m\omega^2R\cos^2\phi		\\
	{p\over p_0}	&=&1-{m\omega^2R\cos^2\phi\over p_0}	\\
			&=&1-{\omega^2R\cos^2\phi\over g}	\\
			&\approx&1-0,0033\cdot\cos^2\phi
	\end{eqnarray*}

	Isto quer dizer que o peso varia conforme $0,0033\cdot\cos^2\phi$. Temos, ent�o, uma
varia��o de at� $0,33\%$ entre os p�los e o equador\footnote{O resultado aqui obtido vale tanto
para o p�lo norte quanto para o p�lo sul. Isto fica evidenciado pelo simples fato de que a 
depend�ncia com a latitude � uma fun��o par ($\cos^2\phi$).}.

	{\tt (b)} Aqui a varia��o deve-se exclusivamente � for�a gravitacional: Uma vez que
o raio da Terra � menor nos p�los, � tamb�m menor a dist�ncia ao centro da Terra (Centro de Massa),
o que aumenta a for�a de atra��o e, consequentemente, o peso:
	
	O raio equatorial da Terra � $R_e=6378,2\,\,\mathrm{km}$, o que d� uma atra��o (Ou ainda
um peso)  
	$$
	p_e=G\cdot{M\,m\over R_e^2}
	$$

	O raio polar, por sua vez, � $R_p=6356,8\,\,\mathrm{km}$,
	$$
	p_p=G\cdot{M\,m\over R_p^2}
	$$

	Dividindo uma pela outra conclu�mos que
	$$
	p_p=({R_e\over R_p})^2p_e\approx1,0067\,p_e
	$$

	Ou seja, o peso de uma pessoa aumenta em aproximadamente $0,67\%$ ao caminhar desde o
equador a um dos p�los. 

	

	
