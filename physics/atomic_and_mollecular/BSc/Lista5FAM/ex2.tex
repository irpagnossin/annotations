\begin{exercicio}[blue]
	
	Uma part�cula encontra-se num estado $\ket{lm}$ que � um autoket dos observ�veis $\opv L^2$ e
$\op L_z$ com autovalotes $\hbar^2 l(l+1)$ e $\hbar m$, respectivamente. Mostre que
	
	{\tt (a)} $\med{\op L_x}=\med{\op L_y}=0$

	{\tt (b)} $\med{\op L_x^2}=\med{\op L_y^2}={1\over 2}\hbar^2[l(l+1)-m^2]$

\end{exercicio}

	Item {\tt (a)}: Sabemos que
	\begin{eqnarray*} 
	\op L_+	&=& \op L_x+i\op L_y	\\
	\op L_- &=& \op L_x-i\op L_y	
	\end{eqnarray*}

	ent�o, podemos facilmente obter
	\begin{eqnarray*}
	\op L_x &=& {1\over 2}(\op L_++\op L_-)	\\
	\op L_y	&=& {1\over 2i}(\op L_+-\op L_-)
	\end{eqnarray*}

	Assim temos
	\begin{eqnarray*}
	\med{\op L_x}	&=& \hope{lm}{\op L_x}{lm}					\\
			&=& {1\over2}\hope{lm}{\op L_++\op L_-}{lm}			\\
			&=& {1\over2}(\hope{lm}{\op L_+}{lm}+\hope{lm}{\op L_-}{lm})	\\
			&=& 0								\\
	\med{\op L_y}	&=& \hope{lm}{\op L_y}{lm}					\\
			&=& {1\over2}\hope{lm}{\op L_+-\op L_-}{lm}			\\
			&=& {1\over2}(\hope{lm}{\op L_+}{lm}-\hope{lm}{\op L_-}{lm})	\\
			&=& 0
	\end{eqnarray*} 

	Item {\tt (b)}: 
	\begin{eqnarray*}
	\op L_x^2 	&=& {1\over4}(\op L_++\op L_-)(\op L_++\op L_-)				\\
			&=& {1\over4}(\op L_+^2+\op L_+\op L_-+\op L_-\op L_++\op L_-^2)	\\
			&=& {1\over4}(\op L_+\op L_-+\op L_-\op L_+)				\\
	\op L_y^2	&=& -{1\over4}(\op L_+-\op L_-)(\op L_+-\op L_-)			\\
			&=& -{1\over4}(\op L_+^2-\op L_+\op L_--\op L_-\op L_++\op L_-^2)	\\
			&=& {1\over4}(\op L_+\op L_-+\op L_-\op L_+)
	\end{eqnarray*}
	
	onde os termos quadr�ticos foram ignorados simplesmente porque tensionamos calcular a m�dia, que
� dada por $\hope{lm}{\op L_x^2}{lm}$, o mesmo valendo para $\op L_y$. Ocorre $\op L_+^2$ eleva o estado
$\ket{lm}$ para $\ket{lm+2}$, fazendo a m�dia resultar $\iprod{lm}{lm+2}=0$. Portanto, essas parcelas n�o
tem interesse para n�s no momento. Por conseguinte conclu�mos que $\op L_x^2=\op L_y^2$\footnote{Note que 
$\op L_x^2=\op L_y^2$ apenas neste tipo de aplica��o, onde podemos
ignorar os termos quadr�ticos. Em outra situa��o n�o podemos assumir a identidade.}. 
	
	Para simplificar ainda mais nossos c�lculos, notemos que
	$$
	\opv L^2=\op L_x^2+\op L_y^2+\op L_z^2=2\op L_x^2+\op L_z^2 \Rightarrow 
			\op L_x^2={1\over2}(\opv L^2-\op L_z^2)
	$$
 
	Assim,
	\begin{eqnarray*} 
	\med{\op L_x^2}=\med{\op L_y^2}	&=&  \hope{lm}{\op L_x^2}{lm}					\\
					&=& {1\over2}\hope{lm}{\opv L^2-\op L_z^2}{lm}			\\
					&=& {1\over2}(\hope{lm}{\opv L^2}{lm}-\hope{lm}{\op L_z^2}{lm})	\\
					&=& {1\over2}[l(l+1)\hbar^2-m^2\hbar^2]				\\
					&=& {1\over2}\hbar^2[l(l+1)-m^2]
	\end{eqnarray*} 

	
  

