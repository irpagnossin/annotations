\begin{exercicio}[blue]
	Um feixe de raios X com f�tons de $10\unit{keV}$ � atenuado por uma l�mina de chumbo de espessura
desconhecida. Obtenha essa espessura sabendo que o feixe transmitido tem uma intensidade $100$ vezes 
menor que o incidente.
\end{exercicio}

	Das tabelas do {\sl Johns} obtemos que o coeficiente de atenua��o m�ssico � de 
$\mu/\rho=0,00488\unit{m^2/kg}$ e a densidade de massa $\rho=11360\unit{kg/m^3}$, donde obtemos o 
coeficiente de atenua��o linear $\mu=55,44\unit{m^{-1}}$.\par
	A partir da lei de atenua��o linear temos que $x=\ln(I_0/I)/\mu$, sendo $x$ a espessura da l�mina,
$I_0$ a intensidade incidente sobre a l�mina e $I$ a intensidade emergente. Como procuramos o valor de $x$
correspondente � rela��o $I_0/I=100$, � claro que
	$$
	x={1\over55,44}\ln100=8,3\unit{cm}
	$$

	Ou seja, uma l�mina de chumbo de $8,3\unit{cm}$ de espessura � capaz de atenuar cerca de $99\%$ 
dos f�tons incidentes. Ou ainda: $99\%$ dos f�tons incidentes interagem com a l�mina.
