\begin{quotation}
{\bf 14.}
{\it	Um feixe de luz branca incide, a partir do v�cuo, num material transparente sob um �ngulo de 
	$10^{\mathrm{o}}$ com respeito � normal. O feixe com $\lambda=6800\,\,\mbox{\AA}$ sofre 
	um desvio de $0,3^{\mathrm{o}}$. J� a radia��o com $\lambda=3300\,\,\mbox{\AA}$ sofre um
	desvio de $0,38^{\mathrm{o}}$.{\tt (a)} Calcule o �ndice de refra��o do meio. {\tt (b)} Calcule
	as constantes $A$ e $B$ da equa��o de Cauchy.}
\end{quotation}

	{\tt (a)} O �ndice de refra��o do v�cuo �, por defini��o, $1$. Ent�o, pela Lei de Snell-Descartes, 
o �ndice de refra��o para um dado comprimento de onda �
	$$
	n(\lambda)={\sin{\hat{i}}\over\sin{\hat{r}(\lambda)}}
	$$

	onde $\hat{i}$ e $\hat{r}$ s�o os �ngulos de incid�ncia e refra��o do feixe de comprimento de onda
$\lambda$. Lembrando que a radia��o eletromagn�tica, indo de um meio menos denso para outro mais denso,
sofre um desvio em dire��o � normal, concluiremos que os �ngulos de refra��o s�o $\hat{r}=9,7^{\mathrm{o}}$
para o feixe com $6800\,\,\mbox{\AA}$ e $\hat{r}=9,62^{\mathrm{o}}$ para o feixe com $3300\,\,\mbox{\AA}$.
Ent�o
	$$
	n(6800\,\,\mbox{\AA})={\sin(10^{\mathrm{o}})\over\sin(9,7^{\mathrm{o}})}\cong 1,031
	$$

	e
	$$
	n(3300\,\,\mbox{\AA})={\sin(10^{\mathrm{o}})\over\sin(9,62^{\mathrm{o}})}\cong 1,039
	$$

	{\tt (b)} A equa��o de Cauchy �
	$$
	n(\lambda)=A+{B\over\lambda_0^2}
	$$
	
	onde $\lambda_0$ � o comprimento de onda incidente e $A$ e $B$ s�o constantes a se determinar.
Do item anterior sabemos que
	\begin{equation} 
	n(6800\,\,\mbox{\AA})=A+{B\over6800^2}=1,0306179
	\label{eq:6800}
	\end{equation} 

	e
	\begin{equation} 
	n(3300\,\,\mbox{\AA})=A+{B\over3300^2}=1,0391069
	\label{eq:3300}
	\end{equation} 

	Subtraindo (\ref{eq:3300}) de (\ref{eq:6800}) encontramos $B$
	\begin{eqnarray*}
	n(6800)-n(3300)	&=& 1,0306179-1,03910687\\
			&=& A+{B\over6800^2}-A-{B\over3300^2}	\\
			&=& {3300^2-6800^2\over(3300\cdot6800)^2}B=0,00848897
	\end{eqnarray*}

	donde $B=120923,709$. Com $B$ fica f�cil determinar $A$ a partir de (\ref{eq:6800}), por exemplo:
	$$
	1,0306179 = A+{120923,709\over6800^2} \Rightarrow A=1,028002
	$$


	
