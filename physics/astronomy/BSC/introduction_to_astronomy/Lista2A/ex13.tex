\begin{quotation}
{\bf 13.}
{\it	Em compara��o com o Sol (Temperatura por volta de $5700\,\,\mathrm{K}$), quanta energia um estrela
	a $20000\,\,\mathrm{K}$ emite?}
\end{quotation}

	O fluxo de energia � relacionado � temperatura pela lei de Stefan:
	$$
	\Phi=\sigma T^4
	$$

	Portanto devemos ter, para o Sol,
	$$
	\Phi_{\bigodot}=\sigma T_{\bigodot}^4
	$$
	
	e, para a estrela em quest�o,
	$$
	\Phi_*=\sigma T_*^4
	$$

	Logo podemos concluir que
	$$
	{\Phi_*\over\Phi_{\bigodot}}=({T_*\over T_{\bigodot}})^4=({20000\over5700})^4\cong 150
	$$

	Logo, uma estrela a $20000\,\,\mathrm{K}$ emite $150$ vezes mais eneria que o Sol no mesmo
intervalo de tempo.
