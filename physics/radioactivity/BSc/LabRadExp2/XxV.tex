\section{Varia��o da exposi��o com a tens�o}
\label{sec:XxV}

	N�o conseguimos uma explica��o razo�vel --- nem un�nime --- sobre o porqu� de a exposi��o variar com
o quadrado da tens�o aplicada ao tubo; tampouco encontramos uma fonte bibliogr�fica que o fizesse. Veja
a se��o~\ref{sec:conclusao} para ver {\bf nossa} explica��o e {\bf nossa} contra-argumenta��o.
  \begin{figure}[htb]
    \centering
    \includegraphics[scale=0.35]{figuras/XxV.pdf}
    \caption{\footnotesize\label{fig:XxV}
      A depend�ncia da exposi��o com o quadrado da tens�o � um fato verificado em v�rias bibliografias.
      Contudo, s�o poucas --- N�o encontramos nenhuma --- que se dedica a explicar concisamente. O
      resultado do ajuste foi $X(V)=-1,42(4)+0,0474(17)V+0,00026(11)V^2$.}
   \end{figure}
 