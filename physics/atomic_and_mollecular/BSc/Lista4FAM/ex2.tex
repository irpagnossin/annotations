\begin{exercicio}
	Mostre que as fun��es de onda normalizadas do oscilador harm�nico unidimensional
satisfazem a rela��o de recorr�ncia
	\begin{equation}\label{eq:mhs recorrencia}
	\sqrt{n}\psi_n(x)=\sqrt{2}\alpha x\psi_{n-1}(x)-\sqrt{n-1}\psi_{n-2}(x)
	\end{equation} 

	para $n=1,2,3,...$ onde $\alpha=\sqrt{m\omega/\hbar}$. Essa rela��o permite calcular
as fun��es de onda de qualquer n�vel excitado partindo da fun��o de onda do n�vel 
fundamental.
\end{exercicio}

	As fun��es de onda para os n�veis $n$, $n-1$ e $n-2$ s�o
	\begin{eqnarray*}
\psi_n(x)&=&{\alpha^{1/2}\over2^{n/2}(n!)^{1/2}\pi^{1/4}}H_n(x)e^{-\alpha^2x^2/2}	\\
\psi_{n-1}(x)&=&{\alpha^{1/2}\over2^{n-1/2}[(n-1)!]^{1/2}\pi^{1/4}}H_{n-1}(x)
				e^{-\alpha^2x^2/2}					\\
\psi_{n-2}(x)&=&{\alpha^{1/2}\over2^{n-2/2}[(n-2)!]^{1/2}\pi^{1/4}}H_{n-2}(x)
				e^{-\alpha^2x^2/2}
	\end{eqnarray*}

	Aplicando na equa��o (\ref{eq:mhs recorrencia}),
	\begin{eqnarray*}
\sqrt{n}{\alpha^{1/2}H_n(x)e^{-\alpha^2x^2/2}\over2^{n/2}(n!)^{1/2}\pi^{1/4}}	&=&
	\sqrt{2}\alpha x{\alpha^{1/2}H_{n-1}(x)e^{-\alpha^2x^2/2}\over2^{n-1/2}
				[(n-1)!]^{1/2}\pi^{1/4}}-
	\sqrt{n-1}{\alpha^{1/2}\over2^{n-2/2}[(n-2)!]^{1/2}\pi^{1/4}}H_{n-2}(x)
				e^{-\alpha^2x^2/2}					\\
	{\sqrt{n}H_n(y)\over2^{n/2}\sqrt{n!}}&=&
		{yH_{n-1}(y)\over2^{n/2-1}\sqrt{(n-1)!}}-
			\sqrt{n-1}{H_{n-2}(y)\over2^{n/2-1}\sqrt{(n-2)!}}		\\
	{1\over2}\sqrt{n}H_n(y)&=&\sqrt{n}yH_{n-1}(y)-
				\sqrt{n-1{n!\over(n-2)!}}H_{n-2}(y)			\\	
	H_n(y) &=& 2yH_{n-1}(y)-2(n-1)H_{n-2}(y)
	\end{eqnarray*}

	Mas esta �ltima express�o j� nos � conhecida como a rela��o de recorr�ncia dos
polin�mios de Hermite, demonstrada por outros meios. Segue, ent�o, que est� verificada a
veracidade de (\ref{eq:mhs recorrencia}).
				