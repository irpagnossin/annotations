\documentclass[a4paper,10pt]{article}

\usepackage[latin1]{inputenc}
\usepackage[portuges]{babel}
\usepackage[pdftex]{graphics,color}
\usepackage[pdftex]{graphicx}
\usepackage{wrapfig} 
\usepackage{epsfig} 
\usepackage{url}
\usepackage{ae,algorithm,alltt}
\usepackage{amsfonts,amstext,enumerate}
\usepackage{float,fancyvrb,fontenc}
\usepackage{geometry,hyperref,ifthen}
\usepackage{indentfirst,lastpage,longtable}
\usepackage{lscape,makeidx,mathrsfs}
\usepackage{multicol,pifont,psfrag}
\usepackage{setspace,showidx,subfigure}
\usepackage{texnames,textcomp,ulem}
\usepackage{url,varioref,version}
\usepackage{wasysym}

\begin{document}

\pagestyle{myheadings}
\markboth{}{\rm Lentes e polariza��o da luz}
\renewcommand{\thefootnote}{\fnsymbol{footnote}}	

\title {Lentes e Polariza��o da luz}  
\author {I.R. Pagnossin \\ irpagnossin@hotmail.com \and R.B. Naca \\ rogerio\_naca@procomp.com.br}
\date {\today}
\maketitle

\begin{abstract}
  	No presente experimento procuramos evidenciar v�rias leis utilizadas na �ptica geom�trica e
�ptica f�sica. Come�amos por verificar a {\it lei dos pontos conjugados} determinando, para isso,
a dist�ncia focal de uma lente convergente. Em seguida comparamos duas formas diferentes de se calcular 
a {\it amplifica��o linear} desta mesma lente: $m=-i/o$ e $m=h_i/h_o$. A lente divergente, que n�o forma 
imagem
projet�vel, tornou-se um bom exerc�cio no que tange as conven��es ``estranhas'' utilizadas pela �ptica
geom�trica durante a determina��o indireta de sua dist�ncia focal. A lupa, outro instrumento bastante
�til e conhecido foi tamb�m avaliado. Mais interessante ainda foi o estudo do ``efeito lupa'' que um orif�cio
pequeno assume. Em seguida, e j� passando para a �ptica f�sica, estudamos a polariza��o de um laser de
He--Ne, o {\it �ngulo de Brewster} e sua rela��o com o {\it �ndice de refra��o}, 
{\it coeficientes de reflex�o} e a {\it Lei de Malus}. Finalmente,
divertimo-nos com o interessante efeito da {\it birrefring�ncia} e {\it fotoelasticidade}.
 
\end{abstract}
\section{Introdu��o te�rica} 
  \label{sec:teoria} 
\subsection{Modelo f�sico do indutor e capacitor}

	Quando sinais el�tricos sobre um circuito eletr�nico deixa de ser cont�nuo (DC) para ser alternado (AC), as equa��es

\vspace{10pt}	
	\begin{minipage}{0.46\textwidth}
	$$
	\mathbf{\nabla}\times\mathbf{E}=\mathbf{0}
	$$
	\end{minipage}
	\hfill
	e
	\hfill
	\begin{minipage}{0.46\textwidth}
	$$
	\mathbf{\nabla}\times\mathbf{H}=\mathbf{J}
	$$
	\end{minipage}

\vspace{10pt} 
	deixam de valer e passam a assumir a sua forma completa ({\it Equa��es de Maxwell})

\vspace{10pt} 	
	\begin{minipage}{0.46\textwidth}
	$$
	\mathbf{\nabla}\times\mathbf{E}=-{\partial\mathbf{B}\over\partial t}
	$$
	\end{minipage}
	\hfill
	e
	\hfill
	\begin{minipage}{0.46\textwidth}
	$$
	\mathbf{\nabla}\times\mathbf{H}=\mathbf{J}+{\partial\mathbf{H}\over\partial t}
	$$
	\end{minipage}

\vspace{10pt} 
	E esta altera��o leva-nos a redefinir as {\it Leis de Kirchhoff}:
	
	\begin{itemize}
	\item	A soma alg�brica das correntes {\it instant�neas} que fluem atrav�s de um n� � igual a zero.
	\item	A soma alg�brica das voltagens {\it instant�neas} aplicadas em uma malha fechada � igual � soma alg�brica das contravoltagens
		{\it instant�neas} na malha.
	\end{itemize}

	Aqui entra uma nova nomenclatura: Designamos {\it Voltagem} aquelas ddp existentes em elementos ativos do circuito, como por exemplo baterias,
fontes de corrente, etc. E {\it Contravoltagem} aquelas sobre elementos passivos, tais como resistores, capacitores e indutores, por exemplo. Isto s�o apenas
nomes e n�o h� problema algum em invert�-los.  

	Al�m disso, os componentes Capacitor e Indutor passam a se comportar de forma bastante diferente, principalmente o indutor, que at� ent�o n�o passava
de um peda�o de fio enrolado. Mas estes elementos n�o obedecem � {\it Lei de Ohm} do resistor pois

\vspace{10pt}
	\begin{minipage}{0.46\textwidth}
	$$
	V_C(t)={1\over C}\int_{t_0}^tI(t)\mathrm{dt}
	$$
	\end{minipage}
	\hfill
	e
	\hfill
	\begin{minipage}{0.46\textwidth}
	$$
	V_L(t)=L\mathrm{dI\over dt}
	$$
	\end{minipage}

\vspace{10pt} 
	Como fazer agora para dimensionarmos um circuito? Considere o problema de resolver um circuito com um resistor, um capacitor e um indutor em s�rie,
alimentados por uma fonte de tens�o alternada $V(t)=V_0\cos(\omega t)$\footnote{Escolhemos esta forma de sinal pois, como sabemos das s�ries de Fourier, sinais
senoidais e cossenoidais podem ser combinados de modo a formar qualquer sinal peri�dico que se queira. Ent�o, escolhendo esta forma para $V(t)$ estaremos ao
mesmo tempo tratando o problema com extrema generalidade e simplificando os c�lculos.}.

	Pela 2$^{\mathrm{a}}$ {\it Lei de Kirchhoff} temos que 
	\begin{equation} 
	V(t)=RI+L{dI\over dt}+{1\over C}\int_{t_0}^tIdt
	\label{eq:circuitoRLC}
	\end{equation}

	Podemos escolher trabalhar com $\hat{V}(t)=V_0e^{j\omega t}$ ao inv�s de $V(t)$, desde que $\mathrm{Re}[\hat{V}(t)]=V(t)$. Usando isto em
(\ref{eq:circuitoRLC}) e desenvolvendo (Refer�ncia \cite{reitz}, cap�tulo 13 -- Correntes que variam lentamente.), chegamos a
	$$
	V_0e^{j\omega t}=[R+j\omega L+{1\over j\omega C}]I_0e^{j\omega t}
	$$

	Chame o fator entre colchetes de $\hat{Z}$ e teremos
	$$
	\hat{V}=\hat{Z}\hat{I}
	$$

	Esta � a {\it Lei de Ohm} para correntes alternadas e $\hat{Z}$ corresponde a caracter�sticas resistivas, mas n�o s� no sentido da resist�ncia � 
passagem de corrente, mas tamb�m a altera��es nesses sinais. � parte complexa dessa imped�ncia d�-se o nome de reat�ncia, que pode ser {\it capacitiva} 
ou {\it indutiva}.

	Todo componente eletr�nico possui caracter�sticas resistivas, capacitivas e indutivas, inclusive os pr�prios resistor, capacitor e indutor. E � a�
que entra o modelo a ser utilizado em cada caso. Para freq��ncias n�o muito altas, como � o nosso caso, o modelo do resistor � simplesmente uma resist�ncia
real, sem reat�ncia. Um capacitor � puramente capacitivo, ou seja, possui apenas reat�ncia capacitiva. O indutor, entretanto, n�o � puramente indutivo.
Ele possui uma caracter�stica resistiva real, que � precisamente a resist�ncia do fio de que � composto. Por este motivo utilizamos para ele um modelo que
associa em s�rie uma resist�ncia real e uma reat�ncia indutiva. Resumindo,
	
\vspace{10pt}
	\begin{minipage}{0.3\textwidth}
	$$
	\hat{Z}_R=R\,\mathrm{,}
	$$
	\end{minipage}
	%\hfill
	\begin{minipage}{0.3\textwidth} 
	$$
	\hat{Z}_C=j{1\over\omega C}
	$$ 
	\end{minipage}
	%\hfill
	e
	\begin{minipage}{0.3\textwidth}
	$$
	\hat{Z}_L=R_L+j\omega L
	$$
	\end{minipage} 

\vspace{10pt} 
	E s�o esses os modelos que pretendemos verificar com as experi�ncias.

\subsection{Filtro RC}

	Considere o circuito RC em s�rie. A tens�o de sa�da (Tens�o sobre o capacitor) � $\hat{V}_s(t)=\hat{Z}_C\hat{I}(t)$ e a tens�o de entrada
� $\hat{V}_e(t)=(\hat{Z}_R+\hat{Z}_C)\hat{I}(t)$. Portanto, o {\it ganho} complexo de tens�o entre a entrada e a sa�da do filtro RC �
	$$
	\hat{G}={\hat{V}_s(t)\over\hat{V}_e(t)}={\hat{Z}_C\hat{I}(t)\over(\hat{Z}_C+\hat{Z}_R)\hat{I}(t)}={1\over1+{\hat{Z}_R\over\hat{Z}_C}}=
			{1\over1+j\omega RC}={1\over1+{\omega\over\omega_c}}
	$$

	com $\omega_c=1/RC$ sendo a freq��ncia de corte do filtro, na qual o ganho � precisamente $1/\sqrt{2}$. O ganho real �, ent�o
	$$
	|\hat{G}|={1\over\sqrt{1+({\omega\over\omega_c})^2}}
	$$

	Al�m disso, $\hat{V}_s(t)=\hat{G}\hat{V}_e(t)$, e como 
	$$
	\hat{G}=|\hat{G}|(1-j\omega/\omega_c)={e^{-j\omega\arctan({\omega\over\omega_c})}\over\sqrt{1+({\omega\over\omega_c})^2}}
	$$

	percebemos que um dos efeitos do ganho complexo � alterar a fase do sinal de entrada. Em outras palavras, a diferen�a de fase entre os sinais
de entrada e sa�da � $-\arctan(\omega/\omega_c)$.

\subsection{Circuito Integrador}

	Acabamos de ver que num circuito RC a tens�o de entrada e de sa�da est�o relacionadas pelo ganho complexo na forma
	$$
	\hat{V}_s(t)={1\over1+j{\omega\over\omega_c}}\cdot\hat{V}_e(t)
	$$

	Mas, se estivermos numa faixa de freq��ncia muito maior que a de corte ($\omega\gg\omega_c$), ent�o poderemos aproximar a equa��o anterior para
	$$
	\hat{V}_s(t)\approx{1\over j{\omega\over\omega_c}}\hat{V}_e(t)
	$$

	Considerando que $\hat{V}_e(t)=V_0e^{j\omega t}$\footnote{Se este n�o for o sinal, ele certamente pode ser escrito em termos de s�ries de Fourier,
que possuem essa forma. Portanto, esta considera��o � verdadeira e gen�rica.}, identificamos que 
	$$
	{\hat{V}_e(t)\over j\omega}=\int\hat{V}_e(t)\mathrm{dt}
	$$

	Logo teremos
	$$
	\hat{V}_s(t)={1\over RC}\int\hat{V}_e(t)\mathrm{dt}
	$$

	Dito de outra forma, para altas freq��ncias, al�m de atenuar o sinal de entrada (Fator $1/RC$), o filtro RC tamb�m o integra.  

\subsection{Perdas de energia por histerese magn�tica}

	A perda de energia por ciclo devido � histerese magn�tica � dada por (Veja refer�ncia~\cite{reitz}, cap�tulo 12 -- Energia magn�tica, se��o
4 -- Perdas por histerese.)
	$$
	u=\int\mathbf{B}\cdot\mathrm{d}\mathbf{H}
	$$

	Fazendo-se algumas estimativas de $\mathbf{H}$ e $\mathbf{B}$ (Consulte a refer�ncia~\cite{apostila}, p�gina 24), podemos aproximar esta 
energia em termos de grandezas mensur�veis:
	$$
	u\approx{R_2 C\over R_1}\cdot{N_p\over N_s}\int V_c\cdot dV_R
	$$

	onde $R_1$ � o resistor limitador de corrente, $R_2$ � o resistor do circuito integrador acoplado ao terminal secund�rio do transformador de
$N_p$ espiras prim�rias e $N_s$ espiras secund�rias. A tens�o $V_C$ � sobre o capacitor do circuito integrador (sa�da deste circuito) e $V_R$ � a tens�o
sobre o resistor limitador de corrente.
	












\section{An�lise de dados} 
  \subsection{Lentes convergentes}\label{sec:lc} 
    	Primeiramente verificamos a lei dos pontos conjugados ao determinar a dist�ncia focal de uma
lente convergente. Al�m disso, tamb�m foi poss�vel verificar a veracidade da equa��o de amplifica��o
linear $m=-i/o$. Para isso posicionamos uma fonte luminosa, uma lente convergente e um anteparo 
opaco tal como demonstrado na figura 5.

%\begin{figure}[htb]
  %\centering
  %\label{fig:lente convergente}
  %\includegraphics[scale=0.25]{figuras/Alc.jpg}
  %\caption{\footnotesize Arranjo experimental que possibilitou a verifica��o da lei dos pontos conjugados
%e a express�o da amplifica��o linear: O objeto, representado por uma fonte luminosa na forma de uma cruz
%de malta de dimens�es $2,35\,\mathrm{cm}\times1,65\,\mathrm{cm}$ � captado pela lente convergente ($C36$)
%que projeta a imagem real sobre o anteparo mais a frente. A posi��o de forma��o da imagem � identificada 
%como aquela em que a imagem forma-se em foco.}
%\end{figure} 

	Note que este procedimento s� � poss�vel por que lentes convergentes formam imagens reais, poss�veis
de se projetar. De fato, veremos mais a frente (sess�o \ref{sec:ld}) que no caso de lentes
divergentes a situa��o complica-se quando queremos determinar sua dist�ncia focal.

	Mantemos, ent�o, a fonte fixa e variamos as posi��es da lente e do anteparo (ou, equivalentemente,
do objeto e da imagem, respectivamente) e medimos estas posi��es mais o tamanho vertical\footnote{Tamb�m 
poder�amos ter medido a 
dimens�o horizontal mas, devido �s posi��es dos objetos sobre a bancada, esta medida mostrou-se ser 
desajeitadamente dif�cil de se conseguir.} da imagem formada. Isto posto � f�cil encontrar os conjuntos
$(i,\sigma_i)$ e $(h_i,\sigma_{h_i})$, cujas incertezas foram avaliadas em $0,5\,\mathrm{mm}$. J� para o
grupo $(o,\sigma_o)$ procedemos da seguinte forma: Procuramos posicionar o anteparo uma vez se aproximando
da lente e outra se afastando. Com isso determinamos as posi��es m�xima e m�nima para a forma��o em foco,
conforme nossa avalia��o fisiol�gica, da imagem. Da� obtemos a m�dia ($o$) e o desvio padr�o; a este 
�ltimo somamos quadraticamente a incerteza sistem�tica $0,5\,\mathrm{mm}$ utilizada para os conjuntos 
anteriores de modo a obter $\sigma_o$. O passo seguinte foi inverter $o$ e $i$ de modo a linearizar a
equa��o de Gauss:
	$$
	{1\over i} = {1\over f} - {1\over o} \Rightarrow y = {1\over f} - x
	$$

	O c�lculo de incertezas mostra que $\sigma_y=y^2\sigma_i$, valendo o mesmo para $x$. Plota-se
o gr�fico $1/i\times 1/o$, faz-se a transfer�ncia de incertezas 
	$$
	\sigma_y^2=\sigma_{y_0}^2+m^2\sigma_x^2
	$$ 

	e refazemos o gr�fico de forma mais precisa: $(y,\sigma_y)\times x$ e ajusta-se a reta. Obtivemos
$y=0,03977(2)-0,999(15)\,x$, dando-nos 
	$$
	f = {1\over 0,03977} = 25,14(13)\,\mathrm{cm\,,}
	$$ 

	sendo a incerteza encontrada simplesmente por
	$$
	\sigma_f = f^2\sigma_a = 25,145^2 \times 0,0002 = 0,13
	$$

	onde $f$ � obviamente a dist�ncia focal (em cent�metros) e $a$ � apenas o coeficiente linear
da curva ajustada. Note que, al�m da incerteza para nos dar uma no��o de acur�cia, o coeficiente
angular comporta-se como o previsto, pois � muito pr�ximo da unidade; �, de fato, indistingu�vel,
dada a sua incerteza.  
	
	Uma forma de avaliar se o resultado obtido est� correto � tentar focalizar um objeto muito
dist�nte. Neste caso, a imagem formar-se-� justamente no foco:
	$$
	\lim_{o\rightarrow\infty} {1\over f} = \lim_{o\rightarrow\infty} {1\over i}+{1\over o} = {1\over i}
		\Rightarrow i=f
	$$

	Utilizamos como objeto distante o Sol e encontramos $f=25,0\pm 2,0\,\mathrm{cm}$. A incerteza aqui �
bastante grande pois a medida era desajeitada de se conseguir. De qualquer modo vemos que de fato h�
coer�ncia entre os resultados\footnote{Apenas como complemento da informa��o, medimos tamb�m a dist�ncia
focal da lente $C5$ pelo mesmo m�todo, obtendo $f=14,8\pm2,0\,\mathrm{cm}$. Obviamente n�o pudemos aplicar
a mesma id�ia � lente divergente $D25$.}. Al�m disso, o pr�prio laborat�rio de demonstra��es encontrou,
conforme tabela fornecida, $f_{C36}=26,40\,\mathrm{cm}$, $f_{C5}=14,40\,\mathrm{cm}$ e 
$f_{D25}=9,40\,\mathrm{cm}$. Confirmam-se as coer�ncias. 
 
%\begin{figure}[htb]
  %\centering
  %\includegraphics[scale=0.5]{figuras/lc.jpg}
  %\caption{\footnotesize $1/i$ relaciona-se linearmente com $1/o$, sendo o coeficiente linear o inverso
	%da dist�ncia focal.} 
%\end{figure}  

	Conv�m comentar que os limites superior e inferior dos dados coletados foram determinados,
respectivamente, pelo suporte da lente, que j� n�o era grande o bastante para focalizar a imagem, e pela 
amplifica��o, que j� n�o cabia no anteparo.


	Com rela��o � amplifica��o linear, verificamos uma concord�ncia muito boa entre as equa��es

\begin{minipage}[c]{0.46\textwidth}
  $$
  m = -{i\over o} 
  $$
\end{minipage} 
e
\begin{minipage}[c]{0.46\textwidth} 
  $$
  m = {h_i\over h_o}\mathrm{,}
  $$ 
\end{minipage}

	onde $h_i$ e $h_o$ s�o as dimens�es verticais da imagem e do objeto, respectivamente. Como n�o
h� nenhum ajuste ou an�lise mais aprofundada a se fazer com rela��o a este ponto conv�m apresentar os
dados obtidos tanto por uma quanto pela outra equa��o:

\begin{table}[htbp]
  \centering
  \label{tab:amplifica��o linear} 
  \begin{tabular}{c|c}
  $m =h_i/h_o $		&	$m = -i/o$	 	\\
  \hline
  1,70			&	1,74			\\
  1,02			&	1,03			\\
  0,72			&	0,65			\\
  0,55			&	0,57			\\
  0,45			&	0,46			\\
  0,38			&	0,39			\\
  0,34			&	0,34			\\
  0,30			&	0,30			\\
  0,28			&	0,27			\\
  0,26			&	0,24			\\
  \hline
  \end{tabular} 
  \caption{\footnotesize Compara��o entre as amplifica��es lineares obtidas por cada uma das equa��es
	$m=-i/o$ e $m=h_i/h_o$ aplicadas aos dados colhidos no experimento.}
\end{table} 

	
    
 







  \subsection{Lentes divergentes}\label{sec:ld} 
    	No caso das lentes divergentes o processo � significativamente mais complicado, uma vez que este
tipo de lente forma imagens virtuais e, portanto, imposs�veis de se projetar. O ``truque'' para resolver
este problema � associar uma lente convergente cuja dist�ncia focal conhecemos. Neste caso temos duas
op��es:

\begin{itemize}
\item Organizamos em seq��ncia o {\bf objeto} (fonte luminosa), uma lente {\bf convergente}, uma lente 
	{\bf divergente} e o {\bf anteparo}, ou
\item Alteramos as posi��es das duas lentes para obter a seq��ncia {\bf objeto}, lente {\bf divergente},
	lente {\bf convergente} e {\bf anteparo}.
\end{itemize} 

%\begin{figure}[htb]
  %\centering
  %\label{fig:op��o 1}
  %\includegraphics[scale=0.5]{figuras/Ald1.jpg}
  %\caption{\footnotesize A primeira op��o para se medir a dist�ncia focal de uma lente divergente consiste
	%em projetar uma imagem real, oriunda da lente convergente, na lente divergente que, por sua vez,
	%produz uma imagem virtual do lado {\bf real}, entretanto. Esta pode, ent�o, ser projetada num
	%anteparo e as medidas pertinentes serem feitas.} 
%\end{figure}

	A id�ia da primeira op��o � fazer da {\bf imagem real} da lente {\it convergente} o 
{\bf objeto virtual} da lente divergente. Como este objeto � virtual, localiza-se no lado real da
configura��o (Perceba a origem dos feixes de luz). Ao projet�-lo sobre a lente divergente, esta forma
uma imagem no mesmo lado do objeto. Ou seja, {\bf real}, e n�o mais virtual. Esta obviamente pode ser
projetada no anteparo. 

	Ent�o, na figura 7, o anteparo deve ser colocado na posi��o da {\it imagem 2}.
Poder-se-ia argumentar: ``Mas se colocarmos o anteparo na posi��o da imagem 2 os feixes de luz da
imagem 1 -- objeto da segunda lente -- seriam bloqueados pelo anteparo!''. Isto de fato n�o ocorre
simplesmente porque o processo de forma��o da imagem n�o acontece em passos, como no desenho. Os feixes
atravessam a lente convergente e, em seguida, a lente divergente, formando diretamente a imagem 2. O
desenho acima � apenas uma representa��o do processo. 

%\begin{figure}[htb]
  %\centering
  %\label{fig:op��o 2}
  %\includegraphics[scale=0.5]{figuras/Ald2.jpg}
  %\caption{\footnotesize Alternativamente fazemos com que a imagem virtual da lente divergente, atingida
	%primeiramente pelos feixes de luz, seja o objeto real da lente convergente que, por sua vez,
	%projeta a imagem real num anteparo.} 
%\end{figure} 

	A segunda op��o � ligeiramente diferente: Procuramos utilizar a {\bf imagem virtual} da lente 
divergente como {\bf objeto real} da lente {\it convergente} que, por sua vez, forma outra imagem, 
{\bf real} e projet�vel. O processo aqui � deveras mais compreens�vel que na op��o anterior.

	Verificou-se experimentalmente que a primeira op��o era ruim pois exigia uma proximidade excessiva
entre as duas lentes e o anteparo\footnote{Esta proximidade era exigida pela amplifica��o e pelo ponto de 
forma��o das imagens, que de forma alguma cabiam no anteparo e no suporte das lentes.}, dificultando as 
medidas e reduzindo enormemente a faixa de valores mensur�veis. Assim esta forma foi deixada de lado e 
preferimos utilizar a segunda op��o. Tamb�m por quest�es pr�ticas escolhemos n�o a lente $C36$, a qual j� 
hav�amos determinado a dist�ncia focal, mas a $C5$, cuja dist�ncia focal, sendo menor, permitia uma gama 
maior de possibilidades de medidas.

	O procedimento ent�o � o seguinte: Fixamos as duas lentes e variamos as posi��es da fonte 
(objeto 1) e do anteparo (imagem 2), medindo a posi��o de cada um desses componentes. Com isso somos 
capazes de medir $(o_1,\sigma_{o_1})$ e $(i_2,\sigma_{i_2})$. A partir de $o_1$, e conhecendo a dist�ncia
focal da lente convergente $C5$ ($14,40\,\mathrm{cm}$), podemos calcular a posi��o do objeto real desta
lente:
	$$
	{1\over o_2} = {1\over f_c}-{1\over i_2},
	$$
	
	que � a imagem da primeira lente (divergente -- $D25$). Na �ltima express�o denotamos a dist�ncia
focal por $f_c$ para identificar que se trata da lente {\bf c}onvergente. Significado an�logo t�m $f_d$.
Al�m disso, $o_2,\,i_2,\,f_c>0$, o que caracteriza a lente como convergente.

	Uma vez que acabamos de encontrar a posi��o da imagem da primeira lente (Objeto da segunda) e medimos
a posi��o do primeiro objeto, podemos determinar a dist�ncia focal da lente divergente.
	$$
	{1\over f_d} = {1\over i_1}+{1\over o_1}
	$$

	A partir daqui o processo � id�ntico ao da lente convergente, pois j� determinamos $(o,\sigma_o)$ e
$(i,\sigma_i)$: $y=1/i$ e $x=1/o$. Ressalva-se, entretanto, que como a lente � divergente, $i,f_d<0$ e $o>0$.
Encontramos $y=-0,111(7)-1,61(3)x$. Ent�o
	$$
	f_d = {1\over-0,111} \Rightarrow f_d = -9,0(6)\,\mathrm{cm}
	$$

%\begin{figure}[htb]
  %\centering
  %\label{fig:lentes divergentes}
  %\includegraphics[scale=0.5]{figuras/ld.jpg}
  %\caption{\footnotesize Resultados obtidos na determina��o da dist�ncia focal de uma lente divergente.}  
%\end{figure} 

	Aqui, entretanto, percebemos que o coeficiente linear est� bem longe de ser $1$. Isto indica um 
erro sistem�tico no experimento, que cresce conforme a dist�ncia entre os componentes do experimento 
diminui. Isto n�o � de espantar pois, embora melhor que a op��o 1, este processo de medida tamb�m
n�o foi o que se poderia dizer ``uma beleza'', principalmente quando o anteparo localizava-se
pr�ximo �s lentes, dificultando o acesso visual. Bastante provavelmente podemos incluir a� tamb�m o
cansa�o visual decorrente da extenuante tarefa de procurar a melhor focaliza��o da imagem.  
 





 
  \subsection{Amplifica��o angular da lupa}\label{sec:lupa}  
    	A amplifica��o angular da lupa pode ser determinada simplesmente visualizando-se uma escala
a certa dist�ncia atrav�s da lupa. Escolhendo a imagem de forma a ficar focalizada no olho, ou seja,
simplesmente escolhendo a melhor imagem para os nossos olhos, automaticamente estamos determinando a 
posi��o da imagem. Percorremos este processo para cada um dos olhos de cada um dos elementos do grupo
(2 no caso). Vejamos um caso: Olho direito do Rog�rio: O Rog�rio posicionou-se e a lupa de forma
a ver uma escala em cent�metros verticalmente fixada na parede atrav�s da lupa. O Ivan, por sua vez,
mediu as dist�ncias da parede � lupa e ao olho direito do Rog�rio, encontrando, respectivamente,
$11,5\,\mathrm{cm}$ e $31,5\,\mathrm{cm}$. O Rog�rio disse que enxergou a refer�ncia $0\,\mathrm{cm}$
na parte inferior do di�metro da lupa e $3,5\,\mathrm{cm}$ na parte superior. Como j� hav�amos medido
o di�metro da lupa e encontrado $\phi = 4,4\,\mathrm{cm}$, isto significa que um objeto de 
$3,5\,\mathrm{cm}$ de
altura passou a ser enxergado pelo olho direito do Rog�rio como tendo $4,4\,\mathrm{cm}$ de altura.

	O �ngulo sob o qual o Rog�rio enxergou o {\bf objeto} � simplesmente a altura do objeto dividida
pela dist�ncia dos olhos at� a parede:
	$$
	\tan\theta_o={h_o\over D_o}={3,5\over31,5}=0,\dot{1} \Rightarrow \theta_o=6,34^{\mathrm{o}}
	$$

	e o �ngulo de visada da {\bf imagem} �
	$$
	\tan\theta_i={h_i\over D_i}={4,4\over31,5-11,5}=0,22 \Rightarrow \theta_i=12,41^{\mathrm{o}} 
	$$

	Portanto a amplifica��o angular $m_\theta$ �
	$$
	m_\theta = {\theta_i\over\theta_o}={12,41\over6,34}=1,96
	$$

	A amplifica��o angular m�xima prevista para o Rog�rio (E tamb�m para o Ivan, j� que ambos
tem ponto pr�ximo de $11\,\mathrm{cm}$) �
	$$
	M_\theta = {11\over f}+1\,\,\,\,\,\mathrm{[f]=cm}
	$$

	onde $f$ � o foco da lupa. Utilizamos a lente $C5$ ($f=14,40\,\mathrm{cm}$) e, portanto
$M=1,76$. Mas este valor � razoavelmente menor que a amplifica��o encontrada. Isto n�o importa muito
pois o procedimento de medida � extremamente incerto, j� que, primeiro, fizemos o experimento de p�;
Segundo, seguramos a lente e a r�gua que utilizamos para encontrar as dist�ncias com as pr�prias m�os; E
por fim, h� muita margem para paralaxes na medida das dist�ncias, isso sem falar que n�o levamos em conta
o di�metro do globo ocular. Se assim o fizermos, aumentando em aproximadamene $8\,\mathrm{cm}$ as
duas dist�ncias, encontraremos $m_\theta=1,76$. Ou seja, apesar de tudo os resultados s�o bons.

	Abaixo apresentamos o resultado para cada um dos componentes do grupo.

\begin{table}[htbp]
\begin{minipage}[b]{0.46\textwidth} 
  \centering
  \begin{tabular}{cc|cc}
  \multicolumn{4}{c}{Ivan} 							\\
  \hline												
  \multicolumn{2}{c}{Olho direito} & \multicolumn{2}{c}{Olho esquerdo} 		\\
  \hline									
  $2,19$	&	$1,91^*$	&	$1,84$	&	$1,68^*$ 	\\
  \hline									
  \end{tabular} 
\end{minipage}
\hfill
\begin{minipage}[b]{0.46\textwidth}
  \centering
  \begin{tabular}{cc|cc} 							
  \multicolumn{4}{c}{Rog�rio} 							\\	
  \hline									
  \multicolumn{2}{c}{Olho direito} & \multicolumn{2}{c}{Olho esquerdo} 		\\
  \hline									
  $1,96$	&	$1,76^*$	&	$1,24$	&	$1,12^*$	\\
  \hline				
  \end{tabular}
\end{minipage} 
  \caption{\footnotesize Amplifica��o angular de uma lupa com dist�ncia focal $14,40\,\mathrm{cm}$ a
	v�rias dist�ncias para cada um dos olhos de cada um dos componentes de grupo. Os valores com
	asterisco (*) levam em considera��o o di�metro do globo ocular, tomado como $8\,\mathrm{cm}$.} 
\end{table} 



 
  \subsection{Efeito lupa num orif�cio}\label{sec:orificio}  
    	Um orif�cio pode ser utilizado como lupa se for suficientemente pequeno quandocomparado com os 
objetos que se pretende observar. O princ�pio � bastante simples e n�o passa de semelhan�a de 
tri�ngulos. Observe a figura 10,

%\begin{figure}[htb]
  %\centering
  %\label{fig:orif�cio}
  %\includegraphics[scale=0.5]{figuras/orificio1.jpg}
  %\caption{\footnotesize Um orif�cio funciona como lupa quando for suficientemente pequeno.} 
%\end{figure} 

	Tiramos imediatamente que
	$$
	{h_o\over D_o}={h_i\over D_i} \Rightarrow h_i = {D_i\over D_o}h_o
	$$

	e, se $D_i>D_o$, ent�o � claro que $h_i>h_o$, significando uma amplia��o linear. Em verdade,
note ainda que a express�o acima pode ser escrita como
	$$
	{h_i\over h_o}={D_i\over D_o}
	$$

	que � exatamente a express�o da amplifica��o linear, a menos do sinal negativo inclu�do 
simplesmente por quest�o de conven��o.

	Outro aspecto interessante de se notar � que analisando cuidadosamente o movimento dos
objetos atrav�s do orif�cio percebemos que a velocidade com que se movem, no sentido contr�rio, � 
aparentemente maior que a velocidade com que se move o orif�cio. Isto se deve ao fato de que
inconscientemente movemo-nos no sentido do orif�cio um pouco mais r�pido, procurando focalizar um
ponto do objeto atr�s do orif�cio. Veja a figura 11,

%\begin{figure}[htb]
  %\centering
  %\label{fig:orif�cio movimento}
  %\includegraphics[scale=0.5]{figuras/orificio2.jpg}
  %\caption{\footnotesize O movimento aparentemente estranho dos objetos quando observados atrav�s de
	%um pequeno orif�cio.} 
%\end{figure} 

	� f�cil ver que
	$$
	{d\over D}={s_0\over s_A} \Rightarrow {d/t\over D/t}={s_0\over s_A} \Rightarrow {v_0\over v_A}=
		{s_0\over s_A} \Rightarrow v_A={s_A\over s_0}v_0
	$$

	e, portanto, se $s_A>s_0$, veremos os objetos atrav�s do orif�cio moverem-se mais rapidamente que
movimentamos o orif�cio. Este efeito � id�ntico �quela situa��o em que, por exemplo, estando pr�ximo
a um poste e nos movimentando procuramos observar um carro, mais r�pido e mais longe do poste, exatamente 
atr�s deste. Muitas vezes ocorre de  n�o conseguirmos ver o carro durante um certo per�odo de tempo pois,
embora estejamos movendo-nos a velocidades diferentes, a linha que nos liga passa sempre pelo poste.




 
  \subsection{Polariza��o do laser He--Ne}\label{sec:HeNe}
    	O laser de He--Ne utilizado na experi�ncia n�o � polarizado. Al�m disso, n�o existe homogeneidade
dos campos el�tricos perpendiculares � dire��o de propaga��o. Isto � suficiente para impedir-nos de
utiliz�-lo como fonte n�o polarizada e/ou polarizada. Mais que isso: Como pretendemos utilizar algum
plano de polariza��o para os experimentos seguintes, decorre a necessidade de um polar�ide. 

	Entretanto, o polar�ide por si s� n�o resolve o problema pois detectamos que mesmo a esperada 
onda polarizada ap�s o polar�ide oscila temporalmente em intensidade, indicando-nos que n�o s� o laser
n�o � polarizado como tamb�m varia com o tempo. Este tempo foi avaliado em $7$ minutos aproximadamente. 

%\begin{figure}[htb]
  %\centering
  %\includegraphics[scale=0.5]{figuras/HeNe.jpg}
  %\caption{\footnotesize Arranjo esquem�tico utilizado para a verifica��o do comportamento ``quase'' 
	%polarizado 
	%do laser. Embora n�o haja polariza��o plana do feixe, os diversos planos de oscila��o do campo 
	%el�trico mudam com o tempo, caracterizando uma luz bastante ruim de se trabalhar quando 
	%necessitamos de luz polarizada.} 
%\end{figure} 

	Surge ent�o a necessidade de procurarmos uma fonte mais est�vel. Por isso o laser foi utilizado
apenas para determinarmos o �ngulo de Brewster, cujas medidas n�o s�o afetadas pela mudan�a dos planos
de oscila��o da onda. Para os outros experimentos -- Lei de Malus (sess�o \ref{sec:malus}) e
coeficientes de reflex�o (sess�o \ref{sec:coeficientes}) -- utilizamos uma l�mpada 
incandescente de tungst�nio, cuja emiss�o � uniforme -- ou quase -- em todas as dire��es.
 
  \subsection{�ngulo de Brewster}\label{sec:brewster} 
    	O �ngulo de Brewster para o lucite foi obtido organizando nossos aparelhos da forma ilustrada
na figura 13:

%\begin{figure}[htb]
  %\centering
  %\label{fig:brewster}
  %\includegraphics[scale=0.5]{figuras/Abrewster.jpg}
  %\caption{\footnotesize Arranjo experimental que permitiu a medi��o do �ngulo de Brewster para o lucite.}
%\end{figure} 

	O que se fez foi variar o �ngulo de incid�ncia do laser sobre o lucite atrav�s do goni�metro e 
procurar o feixe refletido pela primeira superf�cie cuja intensidade fosse zero ou a menor poss�vel. Na 
maioria dos casos encontrava-se uma intensidade n�o nula mas bastante fraca. Isto ocorre devido �s 
imperfei��es na superf�cie do lucite\footnote{Possivelmente devido tamb�m � oleosidade das m�os dos
elementos do grupo que agregava-se durante o manuseio do bloco de lucite.}.

	Deste modo, fizemos o experimento um total de $6$ vezes seguidas, retornando-se sempre antes de
cada medida � posi��o de incid�ncia $0^{\mathrm{o}}$ (normal � superf�cie do lucite). Isto � necess�rio 
pois o goni�metro n�o � um equipamento perfeito: Procur�vamos a posi��o de incid�ncia normal n�o lendo
a escala do goni�metro, mas sim fazendo com que o laser refletido fosse direcionado para a fonte que o
originou, garantindo a incid�ncia normal. Mas, quando isto era feito, notava-se que o goni�metro indicava
uma incid�ncia entre $+1^{\mathrm{o}}$ e $-1^{\mathrm{o}}$, inclusive. Esta flutua��o foi levada em 
considera��o nos c�lculos, adicionando-se ou subtarindo-se este pequeno erro ao valor de �ngulo de
Brewster encontrado. Por exemplo: Numa das medidas encontramos que o feixe refletido tinha sua intensidade
m�nima (�ngulo de Brewster) em $56^{\mathrm{o}}$ e o erro inicial era de $1^{\mathrm{o}}$. Ent�o o
�ngulo efetivamente encontrado foi $55^{\mathrm{o}}$. Este tipo de flutua��o deve-se principalmente ao
afrouxamento existente entre as pe�as m�veis que constitui o goni�metro e � m�scara\footnote{A m�scara
de alinhamento � um pequeno desn�vel existente entre um semi-c�rculo e outro do goni�metro, que permite
posicionarmos a superf�cie do lucite exatamente no centro de rota��o do goni�metro.} de alinhamento do
bloco de lucite.

	Encontramos, ent�o, para o �ngulo de Brewster, $\theta_B=54,25(28)^{\mathrm{o}}$. A incerteza
foi calculada como a incerteza da m�dia das seis medidas efetuadas: $\sigma_{\theta_B}=\sigma/\sqrt{6}$,
com $\sigma$ o desvio padr�o da distribui��o.

	Considerando o �ndice de refra��o do ar igual a unidade, conclu�mos que o �ndice de refra��o $n$ do
lucite e o �ngulo de Brewster relacionam-se aproximadamente por
	$$
	\tan\theta_B\cong n
	$$

	Por conseguinte, $n=1,389(14)$.
   











 
  \subsection{Lei de Malus}\label{sec:malus} 
    	Como foi visto na introdu��o te�rica, a lei de Malus relaciona a intensidade da luz polarizada
com o �ngulo $\theta$ de inclina��o da malha do polar�ide (e, portanto, o plano de polariza��o daquele
polar�ide). Para fazermos esta medida precisamos de
dois polar�ides: O primeiro para polarizar num plano conveniente a luz proveniente da fonte n�o 
polarizada e o segundo para variarmos o plano da malha (fig.14).

%\begin{figure}[htb]
  %\centering
  %\label{fig:malus 1} 
  %\includegraphics[scale=0.5]{figuras/malus1.jpg}
  %\caption{\footnotesize Arranjo experimental utilizado na verifica��o da Lei de Malus. O primeiro
	%polar�ide garante que a luz incidente sobre o segundo � polarizada.}
%\end{figure} 

	As medidas foram feitas desde $\theta=0^{\mathrm{o}}$ at� $\theta=180^{\mathrm{o}}$, tomando o
cuidado de se verificar as intensidades para $\theta=0^{\mathrm{o}}$ e $\theta=90^{\mathrm{o}}$ antes de
cada medida. Este procedimento � necess�rio pois a fonte luminosa de tungst�nio ainda sim n�o �
necessariamente perfeitamente homog�nea em sua emiss�o, e isto pode causar alguma flutua��o na medida,
o que de fato ocorreu, como pode ser visto na figura 15:

%\begin{figure}[htb]
  %\centering
  %\label{fig:malus 2}
  %\includegraphics[scale=0.5]{figuras/malus2.tex}
  %\caption{\footnotesize A intensidade m�xima do feixe emergente do segundo polar�ide flutua um
	%pouco, como se pode verificar. Isto se deve �s imperfei��es de emiss�o da luz incandescente.
	%Os valores em vermelho s�o os esperados teoricamente, considerando a intensidade m�xima como
	%a m�dia da flutua��o encontrada.} 
%\end{figure}

	Uma alternativa para resolver este problema � usar intensidades normalizadas pela intensidade
m�xima. Neste caso o que se faz � tomar uma medida de $I_\theta$, subtair $I_f$ para aquele caso  e 
divid�-la 
por $I_0$. Isto feito chegamos � figura 16:

%\begin{figure}[htb]
  %\centering
  %\label{fig:malus 3}
  %\includegraphics[scale=0.5]{figuras/malus3.jpg}
  %\caption{\footnotesize O mesmo resultado da figura \ref{fig:malus 2} normalizando os valores das 
	%intensidades e excluindo a radia��o de fundo $I_f$.} 
%\end{figure} 

	Pela teoria de Malus, a intensidade $I_\theta$ do feixe polarizado deve variar conforme
	$$
	I_\theta = I_0 \cos^2\theta +I_f,
	$$  

	onde $I_0$ � a intensidade para $\theta=0^{\mathrm{o}}$ e $I_f$ a intensidade para 
$\theta=90^{\mathrm{o}}$ (radia��o de fundo pois nesta situa��o o polar�ide bloqueia totalmente a luz
incidente).

	Podemos ent�o normalizar esta equa��o fazendo $x=\cos^2\theta$ e $y=I_\theta$. Isto posto plotamos
a figura 17:

%\begin{figure}[htb]
  %\centering
  %\label{fig:malus 4}
  %\includegaphics[scale=0.5]{figuras/malus4.jpg}
  %\caption{\footnotesize Lineariza��o da lei de Malus, cujo objetivo � determinarmos experimentalmente os
	%valores de $I_0$ e $I_f$.}
%\end{figure}

	E conclu�mos que
	$$
	I_\theta = 21,4(3)\cos^2\theta+0,11(15)
	$$

	As intensidades na express�o anterior s�o dadas, na realidade, em $\mathrm{mV}$, a tens�o sobre
o resistor de carga do circuito do fotodiodo utilizado no experimento (figura 18).

%\begin{figure}[htb]
  %\centering
  %\label{fig:foto diotod}
  %\includegraphics[scale=0.5]{figuras/fotodiodo.jpg}
  %\caption{\footnotesize Fotodiodo utilizado para medir as intensidades luminosas e sua rela��o entre
	%corrente, em $m\mathrm{A}$ de coletor e intensidade luminosa, em $\mathrm{mW/cm^2}$.}
%\end{figure} 

	 

	
 
  \subsection{Coeficientes de reflex�o}\label{sec:coeficientes}  
    	A intensidade do feixe refletido por uma superf�cie diel�trica depende do �ngulo de incid�ncia.
De fato, vimos anteriormente que para um determinado �ngulo -- o  de Brewster -- a intensidade do
feixe polarizado paralelamente � superf�cie do diel�trico � zero. O comportamento da intensidade de uma
luz n�o polarizada incidente numa tal superf�cie � um pouco dif�cil de medir, mas � bastante mais f�cil
determinar separadamente as intensidades dos feixes polarizados paralela e perpendicularmente � 
superf�cie do diel�trico para, em seguida, calcularmos a raz�o $r=I_\parallel/I_\perp$, que d� uma boa
no��o do comportamento.

	O que fizemos ent�o foi disponibilizar o arranjo da figura \ref{fig:coeficiente} e medirmos
as intensidades para $\phi=0^{\mathrm{o}}$ e $\phi=90^{\mathrm{o}}$\footnote{Note que, diferentemente
do caso da lei de Malus, n�o fazemos uma sele��o pr�via do plano de incid�ncia desejado.}.

%\begin{figure}[htb]
  %\centering
  %\label{fig:coeficiente} 
  %\includegraphics[scale=0.5]{figuras/Acoeficientes.jpg} 
  %\caption{\footnotesize Arranjo experimental utilizado na verifica��o dos coeficientes de refles�o. O
	%filtro � utilizado para bloquear os raios infravermelhos, ignorados pelo polar�ide mas n�o pelo
	%fotodiodo.}
%\end{figure}

	Teoricamente o que se espera �
	$$
	r={R_\parallel\over R_\perp}={{\tan^2(\hat i-\hat t)\over\tan^2(\hat i+\hat t)}\over
					{\sin^2(\hat i-\hat t)\over\sin^2(\hat i+\hat t)}}=
					{\cos^2(\hat i+\hat t)\over\cos^2(\hat i-\hat t)}
	$$

	onde $\hat i$ e $\hat t$ s�o, respectivamente, os �ngulos de incid�ncia e refra��o (transmiss�o). 
Variamos $\hat i$ desde $15^{\mathrm{o}}$, o m�nimo permitido pelo goni�metro, at� $70^{\mathrm{o}}$,
onde o feixe incidente passava a influenciar drasticamente na leitura do fotodiodo. Isto porque a fonte
de luz utilizada era suficientemente extensa para possuir raios que incidiam diretamente sobre o fotodiodo
sem serem refletidos no lucite.

	O resultado obtido, comparado com o toricamente esperado � apresentado na figura 19.

%\begin{figure}[htb]
  %\centering
  %\includegraphics[scale=0.5]{figuras/coeficientes.jpg}
  %\caption{\footnotesize O comportamento da raz�o $r$ entre as intensidades dos feixes polarizados paralela 
	%e perpendicularmente � superf�cie do lucite em compara��o com o esperado teoricamente. Note o
	%�ngulo de Brewster a aproximadamente $\hat i = 54^{\mathrm{o}}$.}
%\end{figure} 

 
  \subsection{Birrefring�ncia e fotoelasticidade}\label{sec:birrefring�ncia}  
    	Nossa experi�ncia neste t�tulo limitou-se a divertirmo-nos observando s�lidos anisotr�picos
que respondiam diferentemente � refra��o face uma press�o ou tens�o sofrida. A figura 20 mostra dois
exemplos de birrefring�ncia parecidos com aqueles que vimos durante o experimento.  
\section{Conclus�es}
  	Todos os experimentos tiveram concord�ncia muito boa com a teoria, exceto talvez o caso da
lente divergente que, como comentamos, apresentou um erro sistem�tico bastante elevado em compara��o
com todos os outros resultados obtidos. 
	Mas existe um ponto que n�o foi sequer inclu�do no seu respectivo local pois apresentou 
resultados absurdos. Trata-se da determina��o do �ndice de refra��o da lente a partir da f�rmula do
fabricante de lentes:
	$$
	{1\over f}=(n-1)[{1\over R_2}-{1\over R_1}]+{(n-1)^2\over n}{t\over R_1R_2}
	$$

	em primeira aproxima��o desconsideramos a segunda parcela e utilizamos
	$$
	{1\over f}=(n-1)[{1\over R_2}-{1\over R_1}]
	$$

	A dist�ncia focal, da lente convergente $C36$ por exemplo, foi determinada na sess�o \ref{sec:lc}:
$f=25,14(3)\,\mathrm{cm}$ e os raios $R_1$ e $R_2$ medidos atrav�s de um pequeno aparelho chamado 
rel�gio, que relacionava sua escala $x$ com o raio de curvatura da lente atrav�s da express�o
	$$
	{1\over R}=1,92\cdot x+0,4\,\,\,\,\,\,\mathrm{[R]=m}
	$$

	Para este caso espec�fico medimos $x=1,75$ e $x=1,875$, que correspondem respectivamente a
$R_1=26,60\,\mathrm{cm}$ e $R_2=25,00\,\mathrm{cm}$. Mas estes valores s�o muito pr�ximos da dist�ncia
focal encontrada. Este � o primeiro ponto estranho. Se insistirmos nestes valores e procurarmos calcular
os �ndices de refra��o, encontramos $n=17,53$, muito alto!\footnote{O valor mais pr�ximo que pudemos obter
atrav�s de \cite[p�g.12-42]{handbook} foi o do Tantalato de pot�ssio ($\mathrm{KTaO_3}$), e subst�ncias
com este �ndice eram muito raras!} Este � o segundo ponto estranho. Se tentarmos
resolver o problema lembrando dos velhos tempos de col�gio onde $R=2f$ e recalcularmos o �ndice, ainda
assim encontramos $n=34,06$, ainda mais absurdo! Segue abaixo a tabela de �ndices de refra��o encontrados
para cada uma das duas lentes convergentes:

\begin{table}[htbp]
  \centering
  \begin{tabular}{cc}
  \hline
  Lente		&	$n$			 	\\
  \hline						
  C36		&	17,53				\\
  C5		&	58,28				\\
  \hline
  \end{tabular}
\end{table}    

	O �ndice de refra��o da lente divergente n�o foi poss�vel medir pois $R_1=R_2=8,22\,\mathrm{cm}$ e,
consequentemente, o primeiro termo anula-se, exigindo a inclus�o do segundo que, por sua vez, exigia a
espessura da lente, que n�o foi medida, por motivos desconhecidos.

	Finalmente repetimos: A despeito deste ponto, todos os outros experimentos foram bem sucedidos.


 
\begin{thebibliography}{9}
  \bibitem{griffiths} D.J.Griffiths, {\it Introduction to eletrodynamics}, third edition, Prentice Hall,
			New Jersey, 1999.
\bibitem{alex}	    A.B. Meyknecht, E.E. Alonso, J.P. Machado, {\it Lentes e
polariza��o da luz}, relat�rio de f�sica experimental IV, IFUSP, 1999. 	
\bibitem{handbook}  D.R. Lide, {\it Handbook of Chemistry and Physics}, $73^{\mathrm{a}}$ed., 1992-1993
 
\end{thebibliography}

\end{document}
 


