\begin{exercicio}[blue]
	A camada semi-redutora de alum�nio para um feixe monocrom�tico de f�tons � de $5\unit{cm}$.
\begin{itemize}
\item[a.]	Quais as porcentagens do feixe incidente que ser�o transmitidas por camadas de
		$1\unit{mm}$, $1\unit{cm}$ e $10\unit{cm}$ desse material?
\item[b.]	Calcule os coeficientes de atenua��o linear, eletr�nico e de massa para essa
		energia de f�tons.
\item[c.]	Qual a energia do feixe incidente?
\end{itemize}
\end{exercicio}

	Se para $5\unit{cm}$ de espessura temos uma atenua��o de $50\,\%$ (Camada semi-redutora), deduzimos
facilmente que 
	$$
	\mu={1\over x}\ln{I_0\over I}={1\over5}\ln2=0,1386\unit{cm^{-1}}
	$$

	Com isso somos perfeitamente capazes de calcular 
	\begin{eqnarray*}
	{I(1\unit{mm})\over I_0}	&=& e^{-0,1386\cdot0,1}=0,9862\;(98,62\%)	\\
	{I(1\unit{cm})\over I_0}	&=& e^{-0,1386\cdot1,0}=0,8706\;(87,06\%)	\\
	{I(10\unit{cm})\over I_0}	&=& e^{-0,1386\cdot10,0}=0,2501\;(25,01\%)	
	\end{eqnarray*}

	Sabendo que a densidade m�ssica do alum�nio � $2699\unit{kg/m^3}$ podemos calcular o coeficiente
de atenua��o m�ssico:
	$$
	{\mu\over\rho}={0,1386\over2699}\cdot100=5,1352\times10^{3}\unit{m^2/kg}
	$$

	Na tabela do {\sl Johns} encontramos
	
	\begin{table}[htbp]
	\centering
	\begin{tabular}{cc}
	$h\nu$ (MeV)	&	$\mu/\rho$ ($\unit{m^2/kg}$)			\\
	\hline
	\rule{1.5ex}{0mm}1,25		&	0,00549				\\
	\rule{1.5ex}{0mm}1,5		&	0,00501				\\			
	\hline
	\end{tabular}
	\end{table}
	

	Interpolando linearmente podemos obter uma boa aproxima��o para a energia correspondente ao
coeficiente de atenua��o m�ssico que acabamos de calcular:
	$$
	{1,5-1,25\over501-549}={1,5-\xi\over501-513,52}\quad\Rightarrow\quad \xi=1,435
	$$

	Portanto a enegia do feixe incidente � de aproximadamente $1,435\unit{MeV}$.\par
	Chamemos $\varrho$ a densidade eletr�nica do alum�nio, que � igual a 
$2,902\times10^{26}\unit{e^-/kg}$ ({\sl Johns}), ou ainda $7,832\times10^{29}\unit{e^-/m^3}$. Ent�o
	$$
	{\mu\over\varrho}={13,86\over7,832}=1,769\times10^{-29}\unit{m^2/e^-}
	$$
	� o coeficiente de atenua��o eletr�nico do alum�nio.
