\documentclass[a4paper,12pt]{article}
\usepackage[utf8]{inputenc}
\usepackage[T1]{fontenc}
\usepackage[brazil]{babel}
\usepackage{amsmath}

\let\vec=\mathbf
\begin{document}

  Considere um operador linear $A$ que, num sistema de coordenadas qualquer, é expresso por:
  \begin{equation*}
  A = \begin{pmatrix}
    1 & 3 \\
    3 & 1
  \end{pmatrix}.
  \end{equation*}
  
  Seus autovalores (ou valores próprios), são números reais (ou complexos) $\lambda$ que satisfazem a equação:
  \begin{equation*}
  A \vec u = \lambda \vec u,
  \end{equation*}
  sendo $\vec u = \begin{pmatrix}x\\y\end{pmatrix}$ um vetor desse espaço.
  Então, a equação anterior fica assim:
  \begin{equation*}
  \begin{pmatrix}
    A_{xx} & A_{xy} \\
    A_{yx} & A_{yy}
  \end{pmatrix}
  \begin{pmatrix}
    x \\
    y
  \end{pmatrix}
  =
  \lambda
  \begin{pmatrix}
    x \\
    y
  \end{pmatrix}
  \Rightarrow
  \begin{pmatrix}
    A_{xx} x + A_{xy} y \\
    A_{yx} x + A_{yy} y
  \end{pmatrix}
  =
  \begin{pmatrix}
    \lambda x \\
    \lambda y
  \end{pmatrix}
  \end{equation*}





\end{document}
