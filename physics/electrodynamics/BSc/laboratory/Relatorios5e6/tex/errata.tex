\documentclass[a4paper,10pt]{article}

\usepackage[latin1]{inputenc}
\usepackage[portuges]{babel}
\usepackage[pdftex]{graphics,color}
\usepackage[pdftex]{graphicx}
\usepackage{wrapfig} 
\usepackage{epsfig} 
\usepackage{url}
\usepackage{ae,algorithm,alltt}
\usepackage{amsfonts,amstext,enumerate}
\usepackage{float,fancyvrb,fontenc}
\usepackage{geometry,hyperref,ifthen}
\usepackage{indentfirst,lastpage,longtable}
\usepackage{lscape,makeidx,mathrsfs}
\usepackage{multicol,pifont,psfrag}
\usepackage{setspace,showidx,subfigure}
\usepackage{texnames,textcomp,ulem}
\usepackage{url,varioref,version}
\usepackage{wasysym}

\begin{document}

\pagestyle{myheadings}
\markboth{}{\rm Lentes e polariza��o da luz}
\renewcommand{\thefootnote}{\fnsymbol{footnote}}	

\title {Lentes e Polariza��o da luz}  
\author {I.R. Pagnossin \\ irpagnossin@hotmail.com \and R.B. Naca \\ rogerio\_naca@procomp.com.br}
\date {\today}
\maketitle

\section{Errata} 

	Ao que nos conta os deuses parecem ter um grande senso de humor: Ontem, cerca de $5$ minutos
ap�s termos entregue o relat�rio da experi�ncia descobrimos o erro ocorrido no processo de determina��o
do �ndice de refra��o das lentes: Ocorreu simplesmente que n�o levamos em considera��o os sinais dos
raios de curvatura. Levando isto em conta e colocando $R_1<0$ e $R_2>0$ sempre, refazemos os c�lculos e em 
primeira aproxima��o\footnote{Ou seja, ignorando o segundo
termo da f�rmula do fabricante de lentes.} encontramos

\begin{table}[htbp]
  \centering
  \begin{tabular}{cc}
  \hline
  Lente		&	$n$			 	\\
  \hline						
  C36		&	1,51				\\
  C5		&	1,52				\\
  \hline
  \end{tabular}
\end{table}    

	Utilizando estes �ndices para calcular um valor mais preciso ($c$), incluindo o segundo termo da
f�rmula do fabricante de lentes, verificamos tratarem-se de valores despres�veis:
	\begin{eqnarray*}
	\mathrm{c(C36)} &=& {(n-1)^2\over n}{t\over R_1R_2}				\\
				&=&{(1,51-1)^2\over1,51}{0,5\over26,60\cdot25,00}\cong0,00013	\\
	\mathrm{c(C5)}  &=&{(1,52-1)^2\over1,52}{0,67\over14,79\cdot15,06}\cong0,00053	\\
	\end{eqnarray*} 

	Como pode ser visto, encontramos $t=0,5\,\mathrm{cm}$ para a lente $C36$ e $t=0,67\,\mathrm{cm}$
para $C5$. Para a lente divergente, encontramos em primeira aproxima��o $n=1,44$. A segunda aproxima��o 
neste caso n�o foi poss�vel pois n�o tinhamos o valor da espessura $t$, isto pois o micr�metro, aparelho
que utilizamos para medir esta espessura, n�o alcan�ava o centro da lente divergente.

	Finalmente conclu�mos com mais tranq�ilidade o relat�rio...

\end{document} 





