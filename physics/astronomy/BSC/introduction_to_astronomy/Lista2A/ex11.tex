\begin{quotation} 
{\bf 11.}
{\it	Como voc� mediria a velocidade de rota��o de uma estrela?}
\end{quotation}

	Para uma estimativa, basta que saibamos a massa e o raio desta. A partir da� podemos determinar,
pela terceira Lei de kepler, a velocidade de rota��o da estrela se esta dependesse apenas de for�as 
gravitacionais, que n�o � o caso, mas serve como uma estimativa de grandezas. De fato, pelo m�todo abaixo
avaliamos a velocidade de rota��o equatorial do Sol em aproximadamente $10\,\,\mathrm{km/s}$, contra os 
$2\,\,\mathrm{km/s}$ que conhecemos hoje.

	O raio pode ser estimado sabendo-se a dist�ncia -- que pode ser obtido por paralaxe ou pelo 
diagrama HR, uma vez conhecidas sua magnitude aparente e classe espectral -- e sua magnitude.

	A massa s� pode ser estimada se houver corpos revolucionando ao seu redor, de modo a aplicarmos,
outra vez, a terceira Lei de Kepler. Se o corpo for um disco de acre��o, por exemplo, compensa medir
a velocidade de rota��o pelo efeito Doppler em dois pontos diferentes do disco (Desde que n�o diamentalmente
opostos); Caso seja uma outra estrela ou planeta, conv�m medir o per�odo diretamente. 

	Qualquer que seja o caso, ser� necess�rio avaliar a dist�ncia at� a estrela. Para isto n�o temos 
muitas alternativas: Sabendo-se a dist�ncia de observa��o e a dist�ncia angular entre os corpos podemos, 
por trigonometria, encontrar esta dist�ncia. Mas isto � razo�vel pois, se n�o formos capazes de distinguir 
dois objetos, voltaremos � situa��o inicial onde vemos apenas uma estrela sem nada ao seu redor e, neste
caso, simplesmente n�o � poss�vel medir a velocidade, mas sim apenas estim�-la por modelos f�sicos.

	 