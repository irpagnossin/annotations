\begin{exercicio}[blue]
	Considere uma part�cula movendo-se na presen�a de um potencial central proporcional a $r^\mu$. Ent�o pode-se mostrar que os valores esperados da energia cin�tica e potencial num estado estacion�rio satisfazem a rela��o $2\med{\op T}=\mu\med{\op V}$, conhecido como {\it Teorema do Virial}.
	
	{\tt (a)} Usando o teorema do virial no caso de um �tomo com um el�tron e carga nuclear $Z$, mostre que
		$$
		\hope{nlm}{\op V}{nlm} = \med{\op
V}_{nlm}=2E_n,\,\,\,\,\,\,\med{\op T}_{nlm}=-E_n
 		$$
		
		onde
		$$
		E_n=-{e^2\over4\pi\epsilon_0a_0}{Z^2\over n^2}
		$$
		
		{\tt (b)} usando o resultado do item {\tt (a)} mostre que o valor esperado do inverso da dist�ncia ao n�cleo � dada por
		$$
		\med{{1\over r}}_{nlm}={Z\over a_0n^2}
		$$
		
\end{exercicio}


	{\tt (a)} No caso do hidrog�nio, sabemos que $\mu=-1$ e, portanto, o teorema do Virial toma a forma $2\med{\op T}=-\med{\op V}$. Mas a energia mec�nica total do �tomo � simplesmente
		$$
		E_n = \med{E_n}= \med{\op T} + \med{\op V} = -{1\over2}\med{\op V} + \med{\op V}
		$$	

	que obviamente resulta em $\med{\op V}=2E_n$ e, por conseguinte, $\med{\op T}=-E_n$.
	
	{\tt (b)} 
	\begin{eqnarray*}
	\med{\op V} &=& \hope{nlm}{\op V}{nlm} = \med{{1\over4\pi\epsilon_0}\cdot{Ze^2\over r}}		\\
	2E_n				&=& {Ze^2\over4\pi\epsilon_0}\med{{1\over r}}_{nlm}
	\end{eqnarray*}
	
	Segue que 
	$$
	\med{{1\over r}}_{nlm} = {4\pi\epsilon_0\over Ze^2}2E_n = {Z\over a_0n^2}
	$$	
