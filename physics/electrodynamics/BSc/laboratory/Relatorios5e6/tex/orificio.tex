	Um orif�cio pode ser utilizado como lupa se for suficientemente pequeno quandocomparado com os 
objetos que se pretende observar. O princ�pio � bastante simples e n�o passa de semelhan�a de 
tri�ngulos. Observe a figura 10,

%\begin{figure}[htb]
  %\centering
  %\label{fig:orif�cio}
  %\includegraphics[scale=0.5]{figuras/orificio1.jpg}
  %\caption{\footnotesize Um orif�cio funciona como lupa quando for suficientemente pequeno.} 
%\end{figure} 

	Tiramos imediatamente que
	$$
	{h_o\over D_o}={h_i\over D_i} \Rightarrow h_i = {D_i\over D_o}h_o
	$$

	e, se $D_i>D_o$, ent�o � claro que $h_i>h_o$, significando uma amplia��o linear. Em verdade,
note ainda que a express�o acima pode ser escrita como
	$$
	{h_i\over h_o}={D_i\over D_o}
	$$

	que � exatamente a express�o da amplifica��o linear, a menos do sinal negativo inclu�do 
simplesmente por quest�o de conven��o.

	Outro aspecto interessante de se notar � que analisando cuidadosamente o movimento dos
objetos atrav�s do orif�cio percebemos que a velocidade com que se movem, no sentido contr�rio, � 
aparentemente maior que a velocidade com que se move o orif�cio. Isto se deve ao fato de que
inconscientemente movemo-nos no sentido do orif�cio um pouco mais r�pido, procurando focalizar um
ponto do objeto atr�s do orif�cio. Veja a figura 11,

%\begin{figure}[htb]
  %\centering
  %\label{fig:orif�cio movimento}
  %\includegraphics[scale=0.5]{figuras/orificio2.jpg}
  %\caption{\footnotesize O movimento aparentemente estranho dos objetos quando observados atrav�s de
	%um pequeno orif�cio.} 
%\end{figure} 

	� f�cil ver que
	$$
	{d\over D}={s_0\over s_A} \Rightarrow {d/t\over D/t}={s_0\over s_A} \Rightarrow {v_0\over v_A}=
		{s_0\over s_A} \Rightarrow v_A={s_A\over s_0}v_0
	$$

	e, portanto, se $s_A>s_0$, veremos os objetos atrav�s do orif�cio moverem-se mais rapidamente que
movimentamos o orif�cio. Este efeito � id�ntico �quela situa��o em que, por exemplo, estando pr�ximo
a um poste e nos movimentando procuramos observar um carro, mais r�pido e mais longe do poste, exatamente 
atr�s deste. Muitas vezes ocorre de  n�o conseguirmos ver o carro durante um certo per�odo de tempo pois,
embora estejamos movendo-nos a velocidades diferentes, a linha que nos liga passa sempre pelo poste.




