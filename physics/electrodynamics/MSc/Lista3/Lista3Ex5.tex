\documentclass[a4paper,10pt]{article}

\usepackage[latin1]{inputenc}
\usepackage[brazil]{babel}
\usepackage[pdftex]{graphics,graphicx,color}
\usepackage{wrapfig} 
\usepackage{hyperref}
\usepackage{geometry}

\begin{document}

  
  %%%%%%%%%%%%%%%%%%%%%%%%%%%%%%%%%%%%%%%%%%%%%%%%%%%%%%%%%%%%%%%%%%%%%%%%%
%	Arquivo:	irpackage.tex					%
%	Projeto:	irpackage					%
%	Autor:		Ivan Ramos Pagnossin		 		%
%	Data inicial:	29/04/2001					%
%%%%%%%%%%%%%%%%%%%%%%%%%%%%%%%%%%%%%%%%%%%%%%%%%%%%%%%%%%%%%%%%%%%%%%%%%

%%%%%%%%%%%%%%%%%%%%%%%%%%%%%%%%%%%%%%%%%%%%%%%%%%%%%%%%%%%%%%%%%%%%%%%%%
%	Define um formato para operadores				%
%%%%%%%%%%%%%%%%%%%%%%%%%%%%%%%%%%%%%%%%%%%%%%%%%%%%%%%%%%%%%%%%%%%%%%%%%
\newcommand{\op}[1]{\mathcal{#1}}

%%%%%%%%%%%%%%%%%%%%%%%%%%%%%%%%%%%%%%%%%%%%%%%%%%%%%%%%%%%%%%%%%%%%%%%%%
%	Define os vetores Bra e Ket do espa�o de Hilbert		%
%%%%%%%%%%%%%%%%%%%%%%%%%%%%%%%%%%%%%%%%%%%%%%%%%%%%%%%%%%%%%%%%%%%%%%%%%
\newcommand{\bra}[1]{\ensuremath{\langle#1|}}
\newcommand{\ket}[1]{\ensuremath{|#1\rangle}}

%%%%%%%%%%%%%%%%%%%%%%%%%%%%%%%%%%%%%%%%%%%%%%%%%%%%%%%%%%%%%%%%%%%%%%%%%
%	Define o produto interno de vetores do espa�o de Hilbert	%
%%%%%%%%%%%%%%%%%%%%%%%%%%%%%%%%%%%%%%%%%%%%%%%%%%%%%%%%%%%%%%%%%%%%%%%%%
\newcommand{\iprod}[2]{\ensuremath{\langle#1|#2\rangle}}

%%%%%%%%%%%%%%%%%%%%%%%%%%%%%%%%%%%%%%%%%%%%%%%%%%%%%%%%%%%%%%%%%%%%%%%%%
%	Define o valor esperado de um operador (argumento 2)		%
%%%%%%%%%%%%%%%%%%%%%%%%%%%%%%%%%%%%%%%%%%%%%%%%%%%%%%%%%%%%%%%%%%%%%%%%%
\newcommand{\hope}[3]{\ensuremath{\langle#1|#2|#3\rangle}}

%%%%%%%%%%%%%%%%%%%%%%%%%%%%%%%%%%%%%%%%%%%%%%%%%%%%%%%%%%%%%%%%%%%%%%%%%
%	A m�dia do argumento 1						%
%%%%%%%%%%%%%%%%%%%%%%%%%%%%%%%%%%%%%%%%%%%%%%%%%%%%%%%%%%%%%%%%%%%%%%%%%
\newcommand{\med}[1]{\langle #1 \rangle}

%%%%%%%%%%%%%%%%%%%%%%%%%%%%%%%%%%%%%%%%%%%%%%%%%%%%%%%%%%%%%%%%%%%%%%%%%
%	Integral impr�pria do primeiro tipo				%
%%%%%%%%%%%%%%%%%%%%%%%%%%%%%%%%%%%%%%%%%%%%%%%%%%%%%%%%%%%%%%%%%%%%%%%%%
\newcommand{\intii}{\int_{-\infty}^{\infty}}

%%%%%%%%%%%%%%%%%%%%%%%%%%%%%%%%%%%%%%%%%%%%%%%%%%%%%%%%%%%%%%%%%%%%%%%%%
%	Elemento qu�mico. Em ordem dos argumentos, o n�mero de massa,	%
% o n�mero at�mico, o s�mbolo do elemento, a val�ncia e o n�mero   de	%
% n�utrons.								%
%%%%%%%%%%%%%%%%%%%%%%%%%%%%%%%%%%%%%%%%%%%%%%%%%%%%%%%%%%%%%%%%%%%%%%%%%
\newcommand{\el}[5]{\ensuremath{^{#1}_{#2}\mathrm{#3}^{#4}_{#5}}}

%%%%%%%%%%%%%%%%%%%%%%%%%%%%%%%%%%%%%%%%%%%%%%%%%%%%%%%%%%%%%%%%%%%%%%%%%
%	Environment para enunciado de exer�cios 			%
%%%%%%%%%%%%%%%%%%%%%%%%%%%%%%%%%%%%%%%%%%%%%%%%%%%%%%%%%%%%%%%%%%%%%%%%%
\newcounter{exercicio} 
\newenvironment{exercicio}[1][black]
	{\vspace{5mm}
	 \stepcounter{exercicio}
	 \begin{quotation}
	 {\color{#1}{\bf Exerc�cio \arabic{exercicio}.}}
	 \itshape}     
        {\end{quotation}
 	 \vspace{5mm}}




	\addtocounter{exercicio}{4}
	\begin{exercicio}[blue]
		Uma suprf�cie esf�rica condutora de raio $a$ � constitu�da por dois hemisf�rios isolados por um anel isolante colocado em $\theta=\pi/2$. O hemisf�rio superior � mantido num potencial $+\Phi_0$ enquanto o inferior � fixo em $-\Phi_0$. Calcule o potencial em todos os pontos do espa�o. Use as express�es $P_n(\cos\theta)$ e calcule os tr�s primeiros termos da expans�o.
	\end{exercicio}

	A solu��o da equa��o de Laplace em coordenadas esf�ricas com simetria azimutal �
	$$
	\phi(r,\theta) = \sum_{l=0}^\infty (A_lr^l + {B_l\over r^{l+1}})P_l(\cos\theta)
	$$
	onde $P_l(\cos\theta)$ � o polin�mio de Legendre de ordem $l$.
	
	Como n�o existe campo externo aplicado, o campo existente deve-se exclusivamente � esfera. Assim, $\phi(r,\theta)$
deve tender a zero quando $r\to\infty$. Logo, $A_l=0$ para $l\ge1$. Fazemos $A_0=0$ por simplicidade e, ent�o,
	$$
	\phi(r,\theta) = \sum_{l=0}^\infty {B_l\over r^{l+1}}P_l(\cos\theta)
	$$
	
	Como a esfera � condutora, ent�o o campo el�trico para $r\le a$ � nulo e, consequentemente, o potencial � constante 
e igual a $\Phi_0$ se $\theta\in[0,\pi/2[$ e igual a $-\Phi_0$ se $\theta\in]\pi/2,\pi]$. Convenientemente escrevemos
que o potencial na superf�cie --- que � igual ao potencial no interior da esfera --- � $\phi(a,\theta)$:
	$$
	\phi(a,\theta) = \sum_{l=0}^\infty {B_l\over a^{l+1}}P_l(\cos\theta)
	$$
	
	Multiplicando ambos os lados da equa��o por $P_m(\cos\theta)\sin\theta$ e integrando em $\theta$ de $0$ a $\pi$ encontramos
	\begin{eqnarray*}
	\int_0^\pi\phi(a,\theta)P_m(\cos\theta)\sin\theta\,d\theta  
			&=& \sum_{l=0}^\infty {B_l\over a^{l+1}}\int_0^\pi P_l(\cos\theta)P_m(\cos\theta)\sin\theta\, d\theta \\
			&=& {B_m\over a^{m+1}}\cdot{2\over2m + 1} 
	\end{eqnarray*}
	
	Ou seja,
	\begin{eqnarray*}
	B_m &=& {1\over2}(2m+1)a^{m+1}\int_0^\pi\phi(a,\theta)P_m(\cos\theta)\sin\theta\,d\theta	\\
		  &=& {1\over2}(2m+1)a^{m+1}\Phi_0\Big[\int_0^{\pi/2}P_m(\cos\theta)\sin\theta\,d\theta -
		  			\int_{\pi/2}^\pi P_m(\cos\theta)\sin\theta\,d\theta\Big] \\
		  &=& {1\over2}(2m+1)a^{m+1}\Phi_0\underbrace{\Big[\int_0^1P_m(x)dx + \int_0^{-1}P_m(x)dx\Big]}_{b_m} \\
		  &=& {1\over2}(2m+1)a^{m+1}\Phi_0 b_m
	\end{eqnarray*}
	
	A express�o geral para o potencial $\phi(r,\theta)$ ($r\ge a$) �, ent�o,
	\begin{equation}\label{eq:1}
	\phi(r,\theta) = {1\over2}\Phi_0\sum_{l=0}^\infty (2l+1){b_l\over(r/a)^{l+1}}P_l(\cos\theta),\qquad
	         b_l = \int_0^1P_l(x)dx + \int_0^{-1}P_l(x)dx
	\end{equation}
	
	Para $r\le a$ t�m-se simplesmente $\phi(a,\theta)$.
	
	Com um pouco de trabalho podemos mostrar que os tr�s primeiros $b_m$ n�o nulos s�o $b_1=1$, $b_3=-1/4$ e $b_5=1/8$.
Portanto temos
	$$
	{\phi(r,\theta)\over\Phi_0} = {3\over2}{1\over(r/a)^2}\cos\theta + 
	                              {7\over4}{1\over(r/a)^4}(5\cos^3\theta-3\cos\theta) +
	                              {11\over16}{1\over(r/a)^6}(63\cos^5\theta-70\cos^3\theta+15\cos\theta) + \cdots
	$$
	
	\begin{figure}[ht]
		\centering
	  \includegraphics[scale = 0.5]{Lista3Ex5Graf.pdf}
	  \caption{Aproxima��o de $\phi(a,\theta)$ para os primeiros vinte termos ($l\le20$ na equa��o (\ref{eq:1})). Na 				abscissa temos $\phi(a,\theta)/\Phi_0$ e na ordenada, $\theta\in[0,\pi]$.}
	\end{figure}
	
\end{document}
 


