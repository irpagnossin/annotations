\documentclass[a4paper,10pt]{article}

\usepackage[latin1]{inputenc}
\usepackage[portuges]{babel}
\usepackage[pdftex]{graphics,color}
\usepackage{graphicx}
\usepackage{wrapfig} 
\usepackage{epsfig} 
\usepackage{url}
\usepackage{ae,algorithm,alltt}
\usepackage{amsfonts,amstext,enumerate}
\usepackage{float,fancyvrb,fontenc}
\usepackage{geometry,hyperref,ifthen}
\usepackage{indentfirst,lastpage,longtable}
\usepackage{lscape,makeidx,mathrsfs}
\usepackage{multicol,pifont,psfrag}
\usepackage{setspace,showidx,subfigure}
\usepackage{texnames,textcomp,ulem}
\usepackage{url,varioref,version}
\usepackage{wasysym}

\begin{document}

\pagestyle{myheadings}
\markboth{}{\rm I.R.Pagnossin}
\renewcommand{\thefootnote}{\fnsymbol{footnote}}	

\title {Terceira lista de F�sica At�mica e Molecular}  
\author {I.R.Pagnossin \\ irpagnossin@hotmail.com}
\date {\today}
\maketitle

\newcommand{\op}[1]{\mathcal{#1}}
\newcommand{\bra}[1]{\ensuremath{\langle#1|}}% O VETOR BRA DO ESPA�O DE HILBERT.
\newcommand{\ket}[1]{\ensuremath{|#1\rangle}}% O VETOR KET DO ESPA�O DE HILBERT.
\newcommand{\iprod}[2]{\ensuremath{\langle#1|#2\rangle}}% O PRODUTO INTERNO NO ESPA�O DE HILBERT.
\newcommand{\hope}[3]{\ensuremath{\langle#1|#2|#3\rangle}}% O VALOR ESPERADO DO OPERADOR #2.
\newcommand{\intii}{\int_{-\infty}^{\infty}}
\newcommand{\med}[1]{\langle #1 \rangle}


\newcounter{exercicio} 
\newenvironment{exercicio}
	{\vspace{5mm}\stepcounter{exercicio}\begin{quotation}{\bf Exerc�cio \arabic{exercicio}.}\itshape} 
        {\end{quotation}\vspace{5mm}}

\begin{exercicio}[blue]
	Uma part�cula sem spin move-se num potencial central $V(r)=0$ para $r<a$ e $V(r)=\infty$ para $r>a$. Determine os n�veis de energia e as fun��es de onda normalizadas correspondentes aos estados $s$, isto �, estados com momento angular orbital nulo ($l=0$).
\end{exercicio}


	Devido ao fato de, da mesma forma que para o �tomo de hidrog�nio, $V(r)$ n�o depender de $\theta$ e
$\phi$, a equa��o de Schr�dinger em coordenadas esf�ricas pode ser separada como de costume:
	$$
	-{1\over\sin\theta}{d\over d\theta}(\sin\theta{d\Theta\over d\theta})+{m_l^2\Theta\over\sin^2\theta} = l(l+1)\Theta
	$$
	
	$$
	{d^2\Phi\over d\phi^2}=-m_l^2
	$$
	
	$$
	{1\over r^2}{d\over dr}(r^2{dR\over dr})+{2\mu\over\hbar^2}[E-V(r)]R=l(l+1){R\over r^2}
	$$
	
	Com $\mu$ a massa reduzida do el�tron.
 	A solu��o da parte angular � simplesmente os harm�nicos esf�ricos $Y_{lm}(\theta,\phi)$. O que
difere esta situa��o do �tomo de hidrog�nio � que devemos fazer $V(r)=0$ na equa��o acima, obtendo
	$$
	{1\over r^2}{d\over dr}(r^2{dR\over dr})+{2\mu\over\hbar^2}ER=l(l+1){R\over r^2}
	$$

	ou ainda,
	$$
	{d\over dr}(r^2{dR\over dr})+[k^2r^2-l(l+1)]R=0
	$$
	
	com $k^2=2\mu E/\hbar^2$.
	As solu��es desta equa��o s�o da forma
	$$
	R_l(r) = A_lj_l(kr) + B_ln_l(kr)
	$$
	
	onde $j_l(kr)$ e $n_l(kr)$ s�o, respectivamente, as fun��es esf�ricas de Bessel e Neumann. Como 
estamos trabalhando numa regi�o finita nas proximidades da origem, conclu�mos que $B_l=0$ pois 
$n_l(kr)$ diverge para $r\rightarrow0$. Ent�o
	$$
	R_l(r)=A_lj_l(kr)
	$$
	
	Como o potencial fora de uma esfera de raio $a$ � infinito, segue que $\psi_{nlm}$ tem de ser zero
nesta regi�o e, pora manter a continuidade, em $r=a$ tamb�m. Esta � a condi��o de contorno que devemos aplicar:
	$$
	R_l(a)=A_lj_l(ka)=0
	$$
	
	cuja solu��o � obviamente $\alpha_{nl}$, a n--�sima raiz da fun��o $j_l(kr)$. Ou seja, encontramos
	$$
	ka = \alpha_{nl} \Rightarrow {\sqrt{2\mu E_n}\over\hbar}=\alpha_{nl} \Rightarrow 
		E_n={\alpha_{nl}^2\hbar^2\over2\mu a^2}
	$$
	
	No caso de $l=0$, $j_0(kr)=\sin(kr)/(kr)$. Em outras palavras, a raiz $\alpha_{n0}=n\pi$, 
$n = 1,2,3,...$ e, portanto,
	$$
	E_n = {n^2\pi^2\hbar^2\over2\mu a^2}
	$$
	
	que s�o os n�veis de energia do po�o de potencial tridimensional.
\vspace{10mm} 
\begin{exercicio}[blue]
	Um el�tron num �tomo de hidrog�nio encontra-se no estado 
	$$
	\ket\psi = N(2\ket{100}-3\ket{200}+\ket{322})
	$$
	
	onde $\ket{nlm}$ representam os autoestados do el�tron ignorando-se o spin.
	
	{\tt (a)} Determine a constante $N$ de modo que o auto--estado seja normalizado.
	
	{\tt (b)} A fun��o de onda $\psi(\mathbf{r})=\iprod{\mathbf{r}}{\psi}$ tem paridade definida?
	
	{\tt (c)} Qual � a probabilidade de que na medida de energia se encontre 
$-m/8\hbar^2(e^2/4\pi\epsilon_0)^2$?
	
	{\tt (d)} Qual � a probabilidade de que na medida de $\opv{L}^2$ se encontre $6\hbar^2$?
	
	{\tt (e)} Qual � a probabilidade de que na medida de $\op L_z$ se encontre valor nulo?
	
	{\tt (f)} Quais s�o os valores esperados das medidas de energia, $\opv{L}^2$ e $\op L_z$?

\end{exercicio}

	{\tt (a)} 
	\begin{eqnarray*}
	\iprod{\psi}{\psi}	&=& N^*(2\bra{100}-3\bra{200}+\bra{322})\cdot N(2\ket{100}-3\ket{200}+\ket{322})	\\
	1										&=& |N|^2(4\iprod{100}{100}+9\iprod{200}{200}+\iprod{322}{322})	\\
	1										&=& |N|^2(4+9+1)
	\end{eqnarray*}
	
	ou seja, $|N|^2=1/14$.
	
	{\tt (b)}Uma fun��o de onda $\psi_{nlm}(r,\theta,\phi)$ � dita par se $l$ for par; e �mpar se $l$ for �mpar. Como $\ket{100}$, $\ket{200}$ e $\ket{322}$ s�o todos pares e a soma de fun��es pares �
uma fun��o par, temos que $\psi_{nlm}(r,\theta,\phi)$ � par. Isto �, tem paridade definida.

	{\tt (c)} Sabemos que 
	$$
	E_n=-{mZ^2\over2n^2\hbar^2}({e^2\over4\pi\epsilon_0})^2
	$$
	
	Deste modo, para $Z=1$ (�tomo de hidrog�nio), a energia $E=-(m/8\hbar^2)(e^2/4\pi\epsilon_0)^2$ 
corresponde a $n=2$. Logo, como a energia depende apenas de $n$, a probabilidade de medirmos tal
energia � igual � soma dos quadrados dos m�dulos dos coeficientes que multiplicam os kets que cont�m $n=2$. 
No nosso caso � apenas o ket $\ket{200}$ com coeficiente $3/\sqrt{14}$. A probabilidade procurada �,
ent�o, $P(n=2)=9/14=64,28\%$.

	{\tt (d)} $\op L^2\ket{nlm}=l(l+1)\ket{nlm}$ e, portanto, no nosso caso procuramos a situa��o em que $l(l+1)=6 \Rightarrow l=2$. A probabilidade � $1/14=7,14\%$, o quadrado do coeficiente do ket $\ket{322}$, o �nico com $l=2$.
	
	{\tt (e)} $\op L_z\ket{nlm}=m\hbar\ket{nlm}$ e, no nosso caso, procuramos a situa��o em que $m=0$. Isto � verificado para os kets $\ket{100}$ e $\ket{200}$. A probabilidade �, ent�o
	$$
	|2N|^2+|-3N|^2=(4+9)|N|^2={13\over14}\cong92,86\%
	$$
	
	{\tt (f)} 
	\begin{eqnarray*}
	E = \hope{\psi}{\mathcal H}{\psi} &=& N\bra\psi(2\mathcal H\ket{100}-3\mathcal H\ket{200}+
		\mathcal H\ket{322})		\\
		&=& -{mNZ^2\over2\hbar^2}({e^2\over4\pi\epsilon_0})^2\bra\psi({2\over1^2}\ket{100}-
			{3\over2^2}\ket{200}+{1\over3^2}\ket{322})	\\
		&=& -{139\over504}{mZ^2\over2\hbar^2}({e^2\over4\pi\epsilon_0})^2
	\end{eqnarray*} 

	\begin{eqnarray*}
	\med{\op L^2}	&=& \hope{\psi}{\op L^2}{\psi} = \bra{\psi}\op L^2[N(2\ket{100}-3\ket{200}
																									+\ket{322})]											\\
								&=& \bra{\psi}{[N(2\op L^2\ket{100}-3\op L^2\ket{200}+\op L^2\ket{322}]}	\\
								&=& 6N\hbar^2\iprod{\psi}{322}=6|N|^2\hbar^2={6\over14}\hbar^2			\\
								&=& {3\over7}\hbar^2
	\end{eqnarray*}
	
	\begin{eqnarray*}
	\med{\op L_z}	&=& \hope{\psi}{\op L_z}{\psi}=\bra{\psi}\op L_z[N(2\ket{100}-3\ket{200}
																									+\ket{322})]											\\
								&=& \bra{\psi}(2N\op L_z\ket{100}-3N\ket{200}+N\ket{322})						\\
								&=& 2N\hbar\iprod{\psi}{322}																				\\
								&=& 2|N|^2\hbar																											\\
								&=& {1\over7}\hbar
	\end{eqnarray*}
	
\vspace{10mm}
\newcommand{\der}[2][]{{\partial#1\over\partial#2}}
\newcommand{\Der}[2][]{{\partial^2#1\over\partial#2^2}}
\newcommand{\cossec}{\mathrm{cossec}} 

\begin{exercicio}[blue]

	Na representa��o das coordenadas, a a��o do observ�vel momento angular sobre um estado 
$\ket\psi$ � dada por
	$$
	\hope{\mathbf{r}}{\opv L}{\psi}=\opv L\iprod{\mathbf{r}}{\psi}
	$$

	onde $\opv L$ � o operador diferencial cujas componentes expressas em coordenadas esf�ricas s�o
	\begin{eqnarray*}
	\op L_x	&=& i\hbar(\sin\phi{\partial\over\partial\theta}+
				\cot\theta\cos\phi{\partial\over\partial\phi}),		\\
	\op L_y &=& i\hbar(-\cos\phi{\partial\over\partial\theta}+
				\cot\theta\sin\phi{\partial\over\partial\phi}),		\\
	\op L_z &=& -i\hbar{\partial\over\partial\phi}
	\end{eqnarray*} 

	{\tt (a)} Mostre que
	$$
	\opv L^2 = -\hbar^2[{1\over\sin\theta}{\partial\over\partial\theta}
			(\sin\theta{\partial\over\partial\theta})+
				{1\over\sin^2\theta}{\partial^2\over\partial\phi^2}]
	$$

	{\tt (b)} Mostre que
	$$
	\op L_\pm = \hbar e^{\pm i\phi}(\pm{\partial\over\partial\theta}+
			i\cot\theta{\partial\over\partial\phi})
	$$

	{\tt (c)} Uma part�cula est� num estado descrito pela fun��o de onda
	$$
	\psi(\mathbf{r})=f(r)\sin^3\theta e^{-3i\phi}
	$$

	Pela aplica��o dos operadores diferenciais $\opv L^2$ e $L_z$ determine os n�meros qu�nticos
$l$ e $m$.
\end{exercicio}


	Item {\tt (a)}: Sabemos que $\opv L^2=\op L_x^2+\op L_y^2+\op L_z^2$. Ent�o fa�amos por partes:
	\begin{eqnarray*} 
	\op L_x^2	&=& \hbar^2(\sin\phi\der\theta+\cot\theta\cos\phi\der\theta)
				(\sin\phi\der\theta+\cot\theta\cos\phi\der\phi)		\\
			&=& -\hbar^2[\sin^2\phi\Der\theta+\sin\phi\cos\phi\der\theta(\cot\theta\der\phi)+
				\cot\theta\cos\phi\der\phi(\sin\phi\der\theta)+
				\cot^2\theta\cos\phi\der\phi(\cos\phi\der\phi)]		\\
			&=& -\hbar^2[\sin^2\phi\Der\theta-\sin\phi\cos\phi\cossec\theta\der\phi+
				\sin\phi\cos\phi\cot\theta{\partial^2\over\partial\theta\partial\phi}+\\
			& & +\cot\theta\cos^2\phi\der\theta+\cot\theta\cos\phi\sin\phi
				{\partial^2\over\partial\phi\partial\theta}-
					\cot^2\theta\sin\phi\cos\phi\der\phi+
						\cot^2\theta\cos^2\phi\Der\phi]		\\
	\op L_y^2	&=& -\hbar^2(-\cos\phi\der\theta+\cot\theta\sin\phi\der\phi)
				(-\cos\phi\der\theta+\cot\theta\sin\phi\der\phi)	\\
			&=& -\hbar^2[\cos^2\phi\Der\theta+\sin\phi\cos\phi\cossec\theta\der\phi-
				\sin\phi\cos\phi\cot\theta{\partial^2\over\partial\theta\partial\phi}+\\
			& &	+\sin^2\phi\cot\theta\der\theta-\cot\theta\sin\phi\cos\phi
					{\partial^2\over\partial\theta\partial\phi}+
				\cot^2\theta\sin\phi\cos\phi\der\phi+\cot^2\theta\sin^2\phi\Der\phi]	\\
	\op L_z^2	&=& -\hbar^2\Der\phi
	\end{eqnarray*} 

	Somando tudo concluiremos que
 	$$ 
	\opv L^2 = -\hbar^2[{1\over\sin\theta}{\partial\over\partial\theta}
			(\sin\theta{\partial\over\partial\theta})+
				{1\over\sin^2\theta}{\partial^2\over\partial\phi^2}]
	$$

	Item {\tt (b)}:
	\begin{eqnarray*} 
	\op L_\pm 	&=& \op L_x\pm i\op L_y								\\
			&=& i\hbar(\sin\phi\der\theta+\cot\theta\cos\phi\der\phi)\pm
				i\cdot i\hbar(-\cos\phi\der\theta+\cot\theta\sin\phi\der\phi)		\\
			&=& i\hbar\sin\phi\der\theta-i\hbar\cot\theta\cos\phi\der\phi\pm
				\hbar\cos\phi\der\theta\mp\hbar\cot\theta\sin\phi\der\phi		\\
			&=& \pm\hbar(\cos\phi\pm i\sin\phi)\der\theta+\hbar\cot\theta
				(\mp\sin\phi+i\cos\phi)\der\phi						\\
			&=& \pm\hbar e^{\pm i\phi}\der\theta+i\hbar\cot\theta
				(\cos\theta\pm i\sin\phi)\der\phi					\\
			&=& \pm\hbar e^{\pm i\phi}\der\theta+i\hbar\cot\theta e^{\pm i\phi}\der\phi	\\
			&=& \hbar e^{\pm i\phi}(\pm\der\theta+i\cot\theta\der\phi)			
	\end{eqnarray*} 

	Item {\tt (c)}: Para determinarmos $m$ aplicamos o operador $\op L_z$ sobre a fun��o de onda
$\psi$:
	\begin{eqnarray*}
	\op L_z\ket\psi	&=& -i\hbar\der\phi\psi(\mathbf{r})						\\
			&=& -i\hbar\der\phi[f(r)\sin^3\theta e^{-3i\phi}]				\\
			&=& -3\hbar f(r)\sin^3\theta e^{-3i\phi}					\\
			&=& -3\hbar\psi(\mathbf{r})
	\end{eqnarray*}

	Logo, pela equa��o de autovalor do operador $\op L_z$,
	$$
	\op L_z\ket\psi=m\hbar\ket\psi
	$$

	temos $m=-3$. De forma an�loga procedemos para encontrar $l$:
	\begin{eqnarray*}
	\opv L^2\ket\psi&=& \hbar^2[{1\over\sin\theta}{\partial\over\partial\theta}
				(\sin\theta{\partial\over\partial\theta})+
				{1\over\sin^2\theta}{\partial^2\over\partial\phi^2}]\psi(\mathbf{r})	\\
		 	&=& \hbar^2[{1\over\sin\theta}{\partial\psi\over\partial\theta}
				(\sin\theta{\partial\psi\over\partial\theta})+
				{1\over\sin^2\theta}{\partial^2\psi\over\partial\phi^2}]		\\
			&\vdots&									\\
			&=& 12\hbar^2\psi(\mathbf{r}) 
	\end{eqnarray*}  

	E, como a equa��o de autovalor do operador $\opv L^2$ �
	$$
	\opv L^2\ket\psi=l(l+1)\hbar^2\ket\psi
	$$

	conclu�mos que $l(l+1)=12 \Rightarrow l=3$\footnote{Na solu��o desta pequena equa��o encontramos
duas ra�zes para $l(l+1)=12$: Uma positiva ($+3$) e a outra negativa ($-4$). O valor negativo � 
simplesmente ignorado pois $l=0,1,2,...$.}. 

	
	





\vspace{10mm} 
\begin{exercicio}[blue]
	Um el�tron no �tomo de hidrog�nio encontra-se no estado $\ket{21-1}=\ket{2p_-}$. 
	
	{\tt (a)} Escreva a fun��o de onda $\psi_{2p_-}(\mathbf r)$.
	
	{\tt (b)} Determine a distribui��o de probabilidades radia $D_{2p}(r)$ de4 se encontrar o el�tron a uma dist�ncia $r$ do n�cleo.
	
	{\tt (c)} Calcule $\med{r}$, $\med{r^2}$ e $\med{{1\over r}}$. Verifique o �ltimo resultado comparando com aquele encontrado no item {\tt (b)} do exerc�cio anterior.

\end{exercicio}

	{\tt (a)} A express�o geral para as fun��es de onda do �tomo de hidrog�nio �
	$$
	\psi_{nlm}(r,\theta,\phi) = R_{nl}(r)Y_{lm}(\theta,\phi)
	$$
	
	Para $\ket{21-1}$ temos, obviamente, $n=2$, $l=1$ e $m=-1$ e disto imediatamente encontramos\footnote{Basta buscar o problema de Sturm--Liouville em coordenadas esf�ricas em algum livro de f�sica--matem�tica ou resolver a equa��o.}
	\begin{eqnarray*}
	R_{21}(r) 						&=& {1\over\sqrt{3}}({Z\over2a_0})^{3/2}({Zr\over a_0})e^{-Zr/2a_0}	\\
	Y_{1-1}(\theta,\phi)	&=& ({3\over8\pi})^{1/2}e^{-i\phi}\sin\theta
	\end{eqnarray*}
	
	O resto das contas, que n�o passa de substitui��o, deixo ao leitor interessado.
	
	{\tt (b)} A distribui��o de probabilidade $D_{nl}(r)$ de encontrarmos o el�tron a uma dist�ncia $r$
do centro �, por defini��o,
	$$
	D_{nl}(r) dr = dr\int_{\theta = 0}^{\pi}\int_{\phi = 0}^{2\pi}\psi^*_{nlm}(r,\theta,\phi)
				\psi_{nlm}(r,\theta,\phi) r^2\sin\theta\mathrm{d}\phi\mathrm{d}\theta
	$$
	
	Como $|\psi_{nlm}(r,\theta,\phi)|^2=|R_{nl}(r)|^2|Y_{lm}(\theta,\phi)|^2$ e a parte angular por si
s�, quando integrada na superf�cie de uma esfera de raio $r$, resulta $1$, fica claro que a equa��o
acima � equivalente a 
	$$
	D_{nl}(r) = r^2|R_{nl}(r)|^2
	$$
	
	No nosso caso, como $n=2$ e $l=1$, 
	$$
	D_{21}(r)dr = {1\over\sqrt{3}}({Z\over2a_0})^{3/2}({Zr^3\over a_0})e^{-Zr/2a_0}dr
	
	{\tt (c)} Ao inv�s de calcularmos tr�s integrais, vale mais a pena utilizarmos um expoente gen�rico $\mu$:
	\begin{eqnarray*}
	\med{r^\mu}	&=& \int_0^\infty r^\mu D_{nl}(r)dr = \int_0^\infty r^{\mu+2}|R_{nl}(r)|^2dr = 	\\
							&=& \int_0^\infty r^{\mu+2}|R_{21}|^2 = \int_0^\infty r^{\mu+2}
										{1\over3}{Z^3\over8a_0^3}{Z^2r^2\over a_0^2}e^{-Zr/a_0}dr									\\
							&=& {Z^5\over24a_0^5}\int_0^\infty r^{\mu+4}e^{-Zr/a_0}dr 											\\
							&=& {Z^5\over24a_0^5}({a_0\over A})^{\mu+5}\int_0^\infty \lambda^{\mu+4}
										e^{-\lambda}\mathrm{d}\lambda																							\\
							&=& {1\over24}({a_0\over Z})^\mu\Gamma(\mu+5)																		\\
							&=& {(n+4)!\over24}({a_0\over Z})^\mu
	\end{eqnarray*}
	
	com $\mu\ge-4$ e $\mu\in\mathsymbol{Z}$. Ent�o quando $\mu=1$ teremos $\med{r} = 5a_0$; quando
$\mu=2$ encontramos $\med{r^2}=30a_0^2$. E, finalmente, para $\mu=-1$, $\med{1/r}=1/4a_0$. 

	Neste ponto � interessante comparar os resultados para $\med{1/r}$ obtido aqui e no item {\tt (b)} do exerc�cio anterior: Aqui, como $n=2$, vemos que podemos expressar $\med{1/r}$ como $1/2^2a_0$, que � o mesmo resultado obtido anteriormente.
\vspace{10mm} 
\begin{exercicio}[blue]
\end{exercicio}

\begin{figure}[htbp]
  \centering
  \includegraphics[scale=0.5]{ex5.pdf} 
  \caption{\footnotesize} 
\end{figure} 


\end{document}
 


