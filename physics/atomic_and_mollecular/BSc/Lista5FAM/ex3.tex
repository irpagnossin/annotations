\newcommand{\der}[2][]{{\partial#1\over\partial#2}}
\newcommand{\Der}[2][]{{\partial^2#1\over\partial#2^2}}
\newcommand{\cossec}{\mathrm{cossec}} 

\begin{exercicio}[blue]

	Na representa��o das coordenadas, a a��o do observ�vel momento angular sobre um estado 
$\ket\psi$ � dada por
	$$
	\hope{\mathbf{r}}{\opv L}{\psi}=\opv L\iprod{\mathbf{r}}{\psi}
	$$

	onde $\opv L$ � o operador diferencial cujas componentes expressas em coordenadas esf�ricas s�o
	\begin{eqnarray*}
	\op L_x	&=& i\hbar(\sin\phi{\partial\over\partial\theta}+
				\cot\theta\cos\phi{\partial\over\partial\phi}),		\\
	\op L_y &=& i\hbar(-\cos\phi{\partial\over\partial\theta}+
				\cot\theta\sin\phi{\partial\over\partial\phi}),		\\
	\op L_z &=& -i\hbar{\partial\over\partial\phi}
	\end{eqnarray*} 

	{\tt (a)} Mostre que
	$$
	\opv L^2 = -\hbar^2[{1\over\sin\theta}{\partial\over\partial\theta}
			(\sin\theta{\partial\over\partial\theta})+
				{1\over\sin^2\theta}{\partial^2\over\partial\phi^2}]
	$$

	{\tt (b)} Mostre que
	$$
	\op L_\pm = \hbar e^{\pm i\phi}(\pm{\partial\over\partial\theta}+
			i\cot\theta{\partial\over\partial\phi})
	$$

	{\tt (c)} Uma part�cula est� num estado descrito pela fun��o de onda
	$$
	\psi(\mathbf{r})=f(r)\sin^3\theta e^{-3i\phi}
	$$

	Pela aplica��o dos operadores diferenciais $\opv L^2$ e $L_z$ determine os n�meros qu�nticos
$l$ e $m$.
\end{exercicio}


	Item {\tt (a)}: Sabemos que $\opv L^2=\op L_x^2+\op L_y^2+\op L_z^2$. Ent�o fa�amos por partes:
	\begin{eqnarray*} 
	\op L_x^2	&=& \hbar^2(\sin\phi\der\theta+\cot\theta\cos\phi\der\theta)
				(\sin\phi\der\theta+\cot\theta\cos\phi\der\phi)		\\
			&=& -\hbar^2[\sin^2\phi\Der\theta+\sin\phi\cos\phi\der\theta(\cot\theta\der\phi)+
				\cot\theta\cos\phi\der\phi(\sin\phi\der\theta)+
				\cot^2\theta\cos\phi\der\phi(\cos\phi\der\phi)]		\\
			&=& -\hbar^2[\sin^2\phi\Der\theta-\sin\phi\cos\phi\cossec\theta\der\phi+
				\sin\phi\cos\phi\cot\theta{\partial^2\over\partial\theta\partial\phi}+\\
			& & +\cot\theta\cos^2\phi\der\theta+\cot\theta\cos\phi\sin\phi
				{\partial^2\over\partial\phi\partial\theta}-
					\cot^2\theta\sin\phi\cos\phi\der\phi+
						\cot^2\theta\cos^2\phi\Der\phi]		\\
	\op L_y^2	&=& -\hbar^2(-\cos\phi\der\theta+\cot\theta\sin\phi\der\phi)
				(-\cos\phi\der\theta+\cot\theta\sin\phi\der\phi)	\\
			&=& -\hbar^2[\cos^2\phi\Der\theta+\sin\phi\cos\phi\cossec\theta\der\phi-
				\sin\phi\cos\phi\cot\theta{\partial^2\over\partial\theta\partial\phi}+\\
			& &	+\sin^2\phi\cot\theta\der\theta-\cot\theta\sin\phi\cos\phi
					{\partial^2\over\partial\theta\partial\phi}+
				\cot^2\theta\sin\phi\cos\phi\der\phi+\cot^2\theta\sin^2\phi\Der\phi]	\\
	\op L_z^2	&=& -\hbar^2\Der\phi
	\end{eqnarray*} 

	Somando tudo concluiremos que
 	$$ 
	\opv L^2 = -\hbar^2[{1\over\sin\theta}{\partial\over\partial\theta}
			(\sin\theta{\partial\over\partial\theta})+
				{1\over\sin^2\theta}{\partial^2\over\partial\phi^2}]
	$$

	Item {\tt (b)}:
	\begin{eqnarray*} 
	\op L_\pm 	&=& \op L_x\pm i\op L_y								\\
			&=& i\hbar(\sin\phi\der\theta+\cot\theta\cos\phi\der\phi)\pm
				i\cdot i\hbar(-\cos\phi\der\theta+\cot\theta\sin\phi\der\phi)		\\
			&=& i\hbar\sin\phi\der\theta-i\hbar\cot\theta\cos\phi\der\phi\pm
				\hbar\cos\phi\der\theta\mp\hbar\cot\theta\sin\phi\der\phi		\\
			&=& \pm\hbar(\cos\phi\pm i\sin\phi)\der\theta+\hbar\cot\theta
				(\mp\sin\phi+i\cos\phi)\der\phi						\\
			&=& \pm\hbar e^{\pm i\phi}\der\theta+i\hbar\cot\theta
				(\cos\theta\pm i\sin\phi)\der\phi					\\
			&=& \pm\hbar e^{\pm i\phi}\der\theta+i\hbar\cot\theta e^{\pm i\phi}\der\phi	\\
			&=& \hbar e^{\pm i\phi}(\pm\der\theta+i\cot\theta\der\phi)			
	\end{eqnarray*} 

	Item {\tt (c)}: Para determinarmos $m$ aplicamos o operador $\op L_z$ sobre a fun��o de onda
$\psi$:
	\begin{eqnarray*}
	\op L_z\ket\psi	&=& -i\hbar\der\phi\psi(\mathbf{r})						\\
			&=& -i\hbar\der\phi[f(r)\sin^3\theta e^{-3i\phi}]				\\
			&=& -3\hbar f(r)\sin^3\theta e^{-3i\phi}					\\
			&=& -3\hbar\psi(\mathbf{r})
	\end{eqnarray*}

	Logo, pela equa��o de autovalor do operador $\op L_z$,
	$$
	\op L_z\ket\psi=m\hbar\ket\psi
	$$

	temos $m=-3$. De forma an�loga procedemos para encontrar $l$:
	\begin{eqnarray*}
	\opv L^2\ket\psi&=& \hbar^2[{1\over\sin\theta}{\partial\over\partial\theta}
				(\sin\theta{\partial\over\partial\theta})+
				{1\over\sin^2\theta}{\partial^2\over\partial\phi^2}]\psi(\mathbf{r})	\\
		 	&=& \hbar^2[{1\over\sin\theta}{\partial\psi\over\partial\theta}
				(\sin\theta{\partial\psi\over\partial\theta})+
				{1\over\sin^2\theta}{\partial^2\psi\over\partial\phi^2}]		\\
			&\vdots&									\\
			&=& 12\hbar^2\psi(\mathbf{r}) 
	\end{eqnarray*}  

	E, como a equa��o de autovalor do operador $\opv L^2$ �
	$$
	\opv L^2\ket\psi=l(l+1)\hbar^2\ket\psi
	$$

	conclu�mos que $l(l+1)=12 \Rightarrow l=3$\footnote{Na solu��o desta pequena equa��o encontramos
duas ra�zes para $l(l+1)=12$: Uma positiva ($+3$) e a outra negativa ($-4$). O valor negativo � 
simplesmente ignorado pois $l=0,1,2,...$.}. 

	
	




