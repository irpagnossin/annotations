\section{Camada semi-redutora do alum�nio bombardeado por raios X e filtrado com $0,5\unit{mmCu}$}
\label{sec:csrAlRX05mmCu}

	\begin{figure}[htb]
	 \centering
	 \includegraphics[scale=0.35]{figuras/csrAlRX05mmCu.pdf}
	 \caption{\footnotesize\label{fig:csrAlRX05mmCu}
	   A exposi��o em fun��o da espessura da barreira de alum�nio. Aqui, percebe-se que o resultado aproxima-se
	   melhor de uma reta que o obtido na se��o~\ref{sec:csrAlRX2mmAl}.}
  \end{figure}
  
  Ajustando uma reta ao gr�fico~\ref{fig:csrAlRX05mmCu} encontramos $\ln X=0,688(26)-0,0876(25)e$, ou seja,
$\mu=0,876\unit{cm^{-1}}$ e $\mu/\rho=0,324(9)\unit{cm^2/g}$. Portanto $e_{1/2}=0,791(22)\unit{cm}$.\par
	A energia efetiva encontra-se entre $40\unit{keV}$ e $50\unit{keV}$. Interpolando encontramos $45,49\unit{keV}$
(figura~\ref{fig:csrAlRX05mmCu}).
