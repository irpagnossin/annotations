
\font\ttsete=cmtt10 at 7pt
\setlength{\unitlength}{1cm}
\linethickness{0.05pt}

\begin{figure}[h]
\begin{picture}(10,2.5)(-3,0)

\put(0.000,0.7){\framebox(2.231,1){$k=0$}}				% Retangulo 0
\put(2.231,0.7){\framebox(3.347,1){1}}					% Retangulo 1
\put(5.578,0.7){\framebox(2.510,1){2}}					% Retangulo 2
\put(8.088,0.7){\framebox(1.255,1){3}}					% Retangulo 3
\put(9.343,0.7){\framebox(0.471,1){4}}					% Retangulo 4
\put(9.814,0.7){\framebox(0.141,1){}}					% Retangulo 5
\put(9.955,0.7){\framebox(0.045,1){}}					% Retangulo 6

\put(2.231,2){\vector(0,-1){0.2}}                                    	% Vetor 1
\put(5.578,2){\vector(0,-1){0.2}}					% Vetor 2
\put(8.088,2){\vector(0,-1){0.2}}					% Vetor 3

\put(1.431,2.1){\makebox(0.2,0.3)[bl]{\ttsete pLimites[0][n]}}		% pLimites[0][n]
\put(4.778,2.1){\makebox(0.2,0.3)[bl]{\ttsete pLimites[1][n]}}		% pLimites[1][n]
\put(7.288,2.1){\makebox(0.2,0.3)[bl]{\ttsete pLimites[2][n]}}		% pLimites[2][n]

\put(0.000,0){\line(1,0){10}}                                           % Base da escala
\multiput(0.000,0)(1,0){11}{\line(0,1){0.2}}    			% Escala de unidade
\multiput(0.100,0)(0.1,0){100}{\line(0,1){0.1}} 			% Escala de sub-unidade
\multiput(0.500,0)(1,0){10}{\line(0,1){0.15}}				% Meia-escala

\put(-0.2,0.3){\makebox(0.2,0.3)[bl]{\footnotesize 0\%}}    		% 0%
\put(+9.7,0.3){\makebox(0.2,0.3)[bl]{\footnotesize 100\%}} 		% 100%

\multiput(4.2,0)(0,0.2){12}{\line(0,1){0.1}}				% Linha tracejada

\end{picture}
\caption{\footnotesize\label{fig:regua}
	Representa��o das probabilidades de ocorr�ncias encontradas na 
tabela~\ref{tab:probabilidades}. Os n�meros indicam o n�mero de ocorr�ncias enquanto a largura de cada
ret�ngulo d� a probabilidade. }
\end{figure}
