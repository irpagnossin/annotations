%IVAN RAMOS PAGNOSSIN 11/03/2001 irpagnossin@hotmail.com

\begin{quotation}
{\color{blue}{\bf 3.}}
{\it	Encontre as posi��es relativas dos centros de massa dos sistemas {\bf (a)} Sol-J�piter e {\bf (b)} Terra-Lua.}
\end{quotation}

\begin{figure}[htb]
\begin{center} 
\includegraphics[scale=0.6]{cg.pdf}  
\caption{\footnotesize Um sistema bin�rio de corpos interagindo gravitacionalmente giram simultanamente em redor do centro de gravidade CG.}   
\end{center} 
\end{figure} 

	Os dois componentes de todo sistema bin�rio de massas interagindo gravitacionalmente movem-se em redor do centro de gravidade do sistema, 
conforme ilustrado abaixo. 

	A for�a entre eles, de natureza gravitacional, pode ser igualada a uma for�a centr�peta, respons�vel pelo deslocamento em circunfer�ncia:
	$$
	F_{12}=m_1{v_1^2\over r_1}={m_1\over r_1}\cdot({2\pi r_1\over P})^2={4\pi^2m_1r_1\over P^2}
	$$
	$$ 
	F_{21}=m_2{v_2^2\over r_2}={m_2\over r_2}\cdot({2\pi r_2\over P})^2={4\pi^2m_2r_2\over P^2}
	$$

	Note que $P$ � o mesmo para os dois corpos. Mas $F_{12}=F_{21}$, portanto $m_1r_1=m_2r_2$. Al�m disso, $r_2=a-r_1$, dando 
	$$ 
	m_1r_1=m_2(a-r_1)\Rightarrow(m_1+m_2)r_1=m_2a\Rightarrow
	$$
	\begin{equation} 
	\label{eq:posicao do corpo 1}
	\Rightarrow r_1={m_2a\over m_1+m_2}	
	\end{equation}
 
	E ent�o, 
	$$
	r_2={m_1a\over m_1+m_2}	
	$$ 

	{\tt Item (a)}: Suponha o sistema Sol-J�piter: $a=7,409\times10^{11}\,\mathrm{m}$ (que pode ser determinado como no exerc�cio 1). A massa de J�piter � 
$m_2=318\cdot M_{\bigoplus}\approx318\cdot6,03\times10^{24}\approx1,918\times10^{27}\,\mathrm{m}$, que pode ser determinada como no exerc�cio anterior,
uma vez conhecidos o per�odo e raio de revolu��o de seus sat�lites. Estes, por sua vez, podem ser obtidos relativamente ao di�metro do planeta, que tamb�m
pode ser determinado, uma vez que conhecemos a dist�ncia at� o Sol. A massa do Sol � $2,00\times10^{30}\,\mathrm{m}$\footnote{Note que estamos utilizando os
valores por n�s encontrados para as massas da Terra e Sol.}. Teremos
	$$
	r_1=r_{\bigodot}={1,918\times10^{27}\cdot7,409\times10^{11}\over1,918\times10^{27}+2,00\times10^{30}}\approx7,098\times10^{8}\,\mathrm{m}
	$$
	
	E o raio para J�piter �
	$$ 
	r_2=r_J={2,00\times10^{30}\cdot7,409\times10^{11}\over1,918\times10^{27}+2,00\times10^{30}}\approx7,402\times10^{11}\,\mathrm{m}  
	$$

	{\tt Item (b)}: Tomemos agora o sistema Terra-Lua: $a=4\times10^{8}\,\mathrm{m}$ (estimado no exerc�cio 1). A massa da Lua �
$m_L=0,0123M_{\bigoplus}$ (Perceba que n�o ser� necess�rio escrever a massa explicitamente). Neste caso a equa��o~(\ref{eq:posicao do corpo 1}) fica
	$$
	r_{\bigoplus}={m_La\over1,0123m_{\bigoplus}}={0,0123\over1,0123}a\approx4,860\times10^{6}\,\mathrm{m}
	$$

	J� $r_L$ �
	$$
	r_L={m_{\bigoplus}a\over1,0123m_{\bigoplus}}={a\over1,0123}\approx3,951\times10^{8}\,\mathrm{m} 
	$$

\vspace{10mm}	
	 
	

	