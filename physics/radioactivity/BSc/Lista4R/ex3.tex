\begin{exercicio}[blue]
	A t�cnica de fluoroscopia com raios X pode empregar o uso cont�nuo de um tubo de raios X com potencial
	de $80\unit{kV}$ e corrente de $3\unit{mA}$.
	\begin{itemize}
	\item	Qual a pot�ncia empregada neste equipamento?
	\item	Supondo que esse tubo possui alvo de tungst�nio, calcule o valor percentual aproximada para a
				energia dissipada em forma de raios X.
	\item	Com base nessas informa��es, discuta a produ��o de calor no alvo.
	\end{itemize}
\end{exercicio}

	\begin{itemize}
	\item	A pot�ncia aplicada ao circuito � simplesmente o produto da tens�o pela corrente, desde que a
				corrente seja cont�nua ou o circuito seja puramente resistivo:
				$$
				P = V\cdot I = 240\unit{W}
				$$
				
	\item	A energia dissipada por irradia��o X � igual � �rea sob o espectro, desde que se tenha corrigido
				os efeitos geom�tricos (O detector n�o mede todos os $4\pi\unit{str}$ existentes ao redor da fonte)
				e de atenua��o (Absor��o e/ou filtra��o pela massa de ar entre a fonte e o detector, por exemplo.),
				entre outros que n�o me lembro. Por triangula��o � poss�vel aproximar esta �rea:
				$$
				P_r \cong [{1\over2}(30-10)\cdot22,5\times10^3+{1\over2}(80-30)\cdot22,5\times10^3]\cdot1,602\times
						10^{-19}\unit[W] \cong 1\times10^{-10}\unit{W}
				$$
				
				Bem, essa quantidade de energia � ridiculamente menor que aquela aplicada ao circuito. Fica at� sem
				sentido escrev�-la em forma de percentual.
	\item	Como a energia dissipada na forma de raios X � muito pequena, t�m-se que praticamente toda a pot�ncia
				aplicada ao circuito � dissipada por efeito Joule, ou aquecimento. Este aquecimento � tal que seria
				capaz de elevar de um grau Celsius $57\unit{g}$ de �gua por segundo!
	\end{itemize}
