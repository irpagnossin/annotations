\section{Camada semi-redutora do acr�lico irradiado por $\mathrm{^{137}Cs}$.}
\label{sec:csr Acrilico 137Cs}

	A an�lise � id�ntica �quela feita na se��o~\ref{sec:csr Al 137Cs}:\par
	O gr�fico $\dot{X}\times e$ (fig.~\ref{fig:csr Acrilico 137Cs}) fornece-nos a lei de decaimento
exponencial $\ln\dot{X}=3,553(3)-0,00926(5)e$ a partir da qual deduzimos o coeficiente de decaimento
$\mu=0,0926(5)\unit{cm^{-1}}$ e $\mu/\rho=0,0661(3)\unit{cm^2/g}$, sendo $\rho=1400\unit{kg/m^3}$.\par
	A camada semi-redutora tamb�m � simples de obter: $e_{1/2}=\ln2/\mu=7,48(4)\unit{cm}$.\par
	\begin{figure}[htb]
	 \centering
	 \includegraphics[scale=0.35]{figuras/csrAcrilico137Cs.pdf}
	 \caption{\footnotesize\label{fig:csr Acrilico 137Cs}
	   A exposi��o como fun��o da espessura de acr�lico interposta � fonte e detector.
	   O ponto $(113,85;10,88)$ foi exclu�do do conjunto pois notava-se visualmente que se tratava de
	   algum erro durante a medida, dada sua discrep�ncia.}
	\end{figure}
	
	Procurando nas tabelas da refer�ncia~\cite{johns} o {\sl Baquelite} ($\mathrm{C_{43}H_{38}O_7}$), um nome mais ``douto''
para acr�lico localizamos $\mu/\rho$ entre $1\unit{MeV}$ e $1,25\unit{MeV}$:
 	\begin{table}[htb]
 	 \centering
 	 \begin{tabular}{cc}
 	 $h\nu$ ($\unit{keV}$) & $\mu/\rho$ ($\unit{cm^2/g}$) \\
 	 \hline
 	  800 & 0,0764 \\
 	 1000 & 0,0687 \\
 	 1250 & 0,0614 \\
 	 1500 & 0,0559 \\
 	 \hline
 	 \end{tabular}
 	 \caption{\footnotesize\label{tab:mu(E)}
 	   Valores de $\mu/\rho$ para o baquelite.}
 	\end{table}
 	
 	Plotando o gr�fico $\mu/\rho\times E$ para este intervalo\footnote{Escolhe-se um intevalo maior que
$[1;1,25]$ para melhorar o comportamento da interpola��o.},
	\begin{eqnarray*}
	{\mu\over\rho} &=& 0,1179(23) - 6,7(4)\times10^{-5}E + 1,69(18)\times10^{-8} \\
	0,0661         &=& \\
	\end{eqnarray*}
	resolvemos para $E$ e encontramos a energia efetiva do feixe: $E_{\mathrm{ef}}=1052,61\unit{keV}$.