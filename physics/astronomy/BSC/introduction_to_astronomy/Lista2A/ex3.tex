\begin{quotation}
{\bf 3.}
{\it	A velocidade orbital da Terra � de aproximadamente $v_{\bigoplus}=30\,\,\mathrm{km/s}$. 
Uma estrela localizada no plano da ecl�ptica emite uma linha espectral com comprimento de onda 
$\lambda=5173\,\,$\mbox{\AA}. Calcule a amplitude da oscila��o do comprimento de onda ao longo do ano.}
\end{quotation}

	Trata-se de um problema de efeito Doppler-Fuseau n�o relativ�stico para ondas 
eletromagn�ticas, cuja rela��o que nos interessa �
	$$
	{\lambda\over\lambda_0}=1+{v\over c}
	$$

	onde $\lambda_0$ � o comprimento de onda emitido pela fonte em quest�o (Estrela, no caso) e
$\lambda$ � o comprimento de onda medido por um observador viajando a uma velocidade $v$.

	Assim, o maior comprimento de onda medido ocorre quando a Terra se afasta com velocidade
$v_{\bigoplus}$:
	$$
	\lambda=(1+{v_{\bigoplus}\over c})\lambda_0 \approx (1+{3\times10^4\over3\times10^8})
			\lambda_0 \approx 5178\,\,\mbox{\AA}
	$$

	O menor comprimento verificado ocorre, por sua vez, quando a Terra se aproxima do astro:
	$$
	\lambda=(1-{v_{\bigoplus}\over c})\lambda_0 \approx (1-{3\times10^4\over3\times10^8})
			\lambda_0 \approx 5167\,\,\mbox{\AA}
	$$

	Portanto, a varia��o do comprimento de onda ao longo do ano � 
$\Delta\lambda\approx11\,\,$\AA.