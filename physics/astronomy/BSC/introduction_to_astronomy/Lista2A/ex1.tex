\begin{quotation}
{\bf 1.}
{\it A roata��o di�ria da Terra tamb�m produz aberra��o. {\bf (a)} Qual � o m�ximo valor desta
aberra��o diurna? {\bf (b)} Em que lugar da Terra este efeito � m�ximo?
}
\end{quotation}

	{\tt (a)} O valor desta aberra��o, para uma dada latitude $\phi$, ser� m�xima para astros
que estejam localizados ao longo da linha imagin�ria que liga o Norte e o Sul celestes, passando
pelo z�nite. Isto porque nesta situa��o o vetor velocidade de rota��o do ponto onde se encontra o
observador assume $90^{\mathrm{o}}$ com rela��o ao feixe de luz proveniente do astro. O valor desta 
aberra��o � aproximadamente
	\begin{eqnarray*}
	\theta&\approx&\tan\theta={v\over c}={r(\phi)\omega\over c}={R\omega\over c}\cos\phi\approx
		{6,350\times10^{6}\cdot7,27\times10^{-5}\over3\times10^8}\cos\phi	\\
					  &\approx&0,3''\cos\phi
	\end{eqnarray*}

	onde $r(\phi)$ � a dist�ncia do ponto onde se encontra o observador ao eixo de rota��o
da Terra.

	{\tt (b)} Ou seja, o maior valor da aberra��o diurna ocorre no equador ($\phi = 0$) e vale 
aproximadamente $0,3$ segundos de arco.

	

	