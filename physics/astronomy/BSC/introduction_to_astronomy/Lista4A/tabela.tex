\begin{landscape} 
\begin{table}[htbp]
  \centering
  \begin{tabular}{ccccccccc|cccccc|ccc|ccccccc} 
  NGC
&$[Fe/H]$ 
&$l$ 
&$\sigma_l(^{\mathrm{0}})$
&$\sigma_l(\mathrm{rad})$
&$b(^{\mathrm{0}})$
&$\sigma_b(^{\mathrm{0}})$
&$\sigma_b(\mathrm{rad})$
&$D (\mathrm{kpc})$
&$\sigma_D (kpc)$
&$X (kpc)$
&$\sigma_X (kpc)$
&$Y (kpc)$
&$\sigma_Y (kpc)$
&$Z (kpc)$
&$\sigma_Z (kpc)$
&$E(B-V)$
&$V(HB)$
&Classe
\\
\hline
\hline 
6235	&-1,40 &-1,1	&0,3	&0,01	&13,5	&0,3	&0,01	&11,7	&0,3	&11,4	&0,29	&-0,2	&0,06	&2,7	&0,09	&0,38	&15,3	&(F+)	\\
6266	&-1,29 &-6,4	&0,3	&0,01	&7,3	&0,3	&0,01	&5,9	&0,3	&5,8	&0,30	&-0,7	&0,05	&0,7	&0,05	&0,45	&14,72	&F8	\\
6273	&-1,68 &-3,1	&0,3	&0,01	&9,4	&0,3	&0,01	&10,5	&0,3	&10,3	&0,30	&-0,6	&0,06	&1,7	&0,07	&0,38	&14,03	&F4   \\
6284	&-1,24 &-1,7	&0,3	&0,01	&9,9	&0,3	&0,01	&10,5	&0,3	&10,3	&0,30	&-0,3	&0,05	&1,8	&0,07	&0,28	&15,98	&F9	\\
6287	&-2,05 &0,1	&0,3	&0,01	&11,0	&0,3	&0,01	&8,4	&0,3	&8,2	&0,29	&0,0	&0,04	&1,6	&0,07	&0,51	&14,85	&G4	\\
6293	&-1,95 &-2,4	&0,3	&0,01	&7,8	&0,3	&0,01	&7,5	&0,3	&7,4	&0,30	&-0,3	&0,04	&1,0	&0,06	&0,35	&15,02	&F3	\\
6304	&-0,59 &-4,2	&0,3	&0,01	&5,3	&0,3	&0,01	&5,2	&0,3	&5,2	&0,30	&-0,4	&0,03	&0,5	&0,04	&0,52	&14,2	&G2	\\
6316	&-0,55 &-2,8	&0,3	&0,01	&5,8	&0,3	&0,01	&10,6	&0,3	&10,5	&0,30	&-0,5	&0,06	&1,1	&0,06	&0,48	&16,01	&G4	\\
6325	&-1,44 &1,0	&0,3	&0,01	&8,0	&0,3	&0,01	&6,5	&0,3	&6,4	&0,30	&0,1	&0,03	&0,9	&0,05	&0,86	&14,37	&(F+)	\\
\hline
6333	&-1,78 &5,5	&0,3	&0,01	&10,7	&0,3	&0,01	&8,0	&0,3	&7,8	&0,29	&0,8	&0,05	&1,5	&0,07	&0,35	&15,11	&F3-F4	\\
6342	&-0,66 &4,9	&0,3	&0,01	&9,7	&0,3	&0,01	&15,0	&0,3	&14,7	&0,29	&1,3	&0,08	&2,5	&0,09	&0,4	&15,39	&G4	\\
6356*	&-0,54 &6,7	&0,3	&0,01	&10,2	&0,3	&0,01	&18,9	&0,3	&18,5	&0,29	&2,2	&0,10	&3,3	&0,11	&0,3	&16,49	&G5	\\
6440	&-0,34 &7,7	&0,3	&0,01	&3,8	&0,3	&0,01	&4,1	&0,3	&4,1	&0,30	&0,5	&0,05	&0,3	&0,03	&1,11	&14,45	&G5	\\
6441	&-0,53 &-6,5	&0,3	&0,01	&-5,0	&0,3	&0,01	&10,1	&0,3	&10,0	&0,30	&-1,1	&0,06	&-0,9	&0,06	&0,46	&15,38	&G4	\\
6522	&-1,44 &1,0	&0,3	&0,01	&-3,9	&0,3	&0,01	&6,3	&0,3	&6,3	&0,30	&0,1	&0,03	&-0,4	&0,04	&0,53	&15,17	&F8	\\
6528	&-0,23 &1,1	&0,3	&0,01	&-4,2	&0,3	&0,01	&7,3	&0,3	&7,3	&0,30	&0,1	&0,04	&-0,5	&0,04	&0,62	&14,47	&(G)	\\
6544*	&-1,56 &5,8	&0,3	&0,01	&-2,2	&0,3	&0,01	&4,2	&0,3	&4,2	&0,30	&0,4	&0,04	&-0,2	&0,02	&0,73	&12,44	&F9	 \\	
6553	&-0,29 &5,3	&0,3	&0,01	&-3,0	&0,3	&0,01	&5,6	&0,3	&5,6	&0,30	&0,5	&0,04	&-0,3	&0,03	&0,8	&14,11	&G4	\\
6558	&-1,44 &0,2	&0,3	&0,01	&-6,0	&0,3	&0,01	&9,0	&0,3	&9,0	&0,30	&0,0	&0,05	&-0,9	&0,06	&0,43	&15,23	&(F+)	\\
6569	&-0,86 &0,5	&0,3	&0,01	&-6,7	&0,3	&0,01	&7,7	&0,3	&7,6	&0,30	&0,1	&0,04	&-0,9	&0,05	&0,55	&15,09	&G4	\\
6624	&-0,37 &2,8	&0,3	&0,01	&-7,9	&0,3	&0,01	&7,5	&0,3	&7,4	&0,30	&0,4	&0,04	&-1,0	&0,06	&0,29	&14,88	&G4	\\
6626	&-1,44 &7,8	&0,3	&0,01	&-5,6	&0,3	&0,01	&5,8	&0,3	&5,7	&0,30	&0,8	&0,05	&-0,6	&0,04	&0,37	&14,33	&F8	\\
6638	&-1,15 &7,9	&0,3	&0,01	&-7,2	&0,3	&0,01	&13,3	&0,3	&13,1	&0,30	&1,8	&0,08	&-1,7	&0,08	&0,37	&15,16	&G2	\\
6637	&-0,59 &1,7	&0,3	&0,01	&-10,3	&0,3	&0,01	&10,4	&0,3	&10,2	&0,30	&0,3	&0,05	&-1,9	&0,08	&0,17	&15,45	&G5	\\
6642	&-1,29 &9,8	&0,3	&0,01	&-6,4	&0,3	&0,01	&5,3	&0,3	&5,2	&0,29	&0,9	&0,06	&-0,6	&0,04	&0,37	&14,18	&(F+)	\\
6656*	&-1,75 &9,9	&0,3	&0,01	&-7,6	&0,3	&0,01	&3,0	&0,3	&2,9	&0,29	&0,5	&0,05	&-0,4	&0,04	&0,35	&12,96	&F5	\\
6681	&-1,51 &2,9	&0,3	&0,01	&-12,5	&0,3	&0,01	&10,8	&0,3	&10,5	&0,29	&0,5	&0,06	&-2,3	&0,09	&0,05	&15,77	&G0-G1	\\
6355	&-1,50 &-0,4	&0,3	&0,01	&5,4	&0,3	&0,01	&6,6	&0,3	&6,6	&0,30	&0,0	&0,03	&0,6	&0,04	&0,73	&14,73	&G	\\
6401	&-1,13 &3,5	&0,3	&0,01	&4,0	&0,3	&0,01	&6,3	&0,3	&6,3	&0,30	&0,4	&0,04	&0,4	&0,04	&0,8	&14,54	&(F+)	\\
Pal 6	&-0,74 &2,1	&0,3	&0,01	&1,8	&0,3	&0,01	&2,6	&0,3	&2,6	&0,30	&0,1	&0,02	&0,1	&0,02	&1,45	&14,06	&(F-)	\\
6652*	&-0,99 &1,5	&0,3	&0,01	&-11,4	&0,3	&0,01	&21,3	&0,3	&20,9	&0,29	&0,5	&0,11	&-4,2	&0,12	&0,1	&16,23	&G2	\\
\hline

									

  \end{tabular}
  \caption{\footnotesize Informa��es sobre 31 aglomerados globulares da Via-L�ctea, utilizados para 
	determinar a dist�ncia do Sol ao centro da Gal�xia. $l$ e $b$ s�o as longitudes e latitudes 
	gal�ctica e $D_{\bigodot}$ � a dist�ncia entre o Sol e o aglomerado correspondente. $E(B-V)$ �
	o avermelhamento e $V(HB)$ a magnitude aparente corrigida pelo avermelhamento e normalizada 
	pela metalicidade $[Fe/H]$.}
  \label{tab:dados} 
\end{table} 
\end{landscape} 

