\begin{exercicio}[blue]
	Qualquer regi�o do espa�o em que a energia cin�tica da part�cula torna-se negativa � classicamente proibida. Para o �tomo de hidrog�nio no estado fundamental $1s$,
	
	{\tt (a)} Determine a regi�o classicamente proibida.
	
	{\tt (b)} Usando a fun��o de onda do estado fundamental $\psi_{1s}(\mathbf r)$, calcule a probabilidade de se encontrar o el�tron nessa regi�o.

\end{exercicio}

	{\tt (a)} Se a energia se conserva identificamos o limite cl�ssico (ponto de retorno) como aquele em que a energia mec�nica total � integralmente potencial: A energia total de um �tomo de hidrog�nio num estado $n$ � $-e^2Z^2/4\pi\epsilon_0a_02n^2$ e a energia potencial $U(r)$ � dada por $-e^2/4\pi\epsilon_0r^2$. Ent�o o maior $r = r_0$ para o qual a exist�ncia do el�tron � permitida classicamente � encontrado por
	$$
	E_n = U(r) \Rightarrow -{1\over4\pi\epsilon_0}{e^2\over r_0^2} = -{e^2Z^2\over4\pi\epsilon_0a_02n^2} \Rightarrow r_0 = {2a_0n^2\over Z^2}
	$$
	
	Assim, para o estado $1s$ ($n=1$ e $l=0$) temos $r_0=2a_0$.
	
	{\tt (b)} A probabilidade de encontrarmos o el�tron nesta regi�o classicamente proibida, baseando-nos na teoria qu�ntica, � o complementar da probabilidade de encontrarmos o el�tron na regi�o classicamente poss�vel. Ou seja, em algum lugar com $r<2a_0$:
	\begin{eqnarray*}
	P	&=& \int_{\phi=0}^{2\pi}\int_{\theta=0}^\pi\int_{r=0}^{2a_0}\psi_{100}^*(r,\theta,\phi)
					\psi_{100}(r,\theta,\phi) r^2\sin\theta\mathrm{dr\,d}\theta\mathrm{d}\phi		\\
		&=& \int_0^{2\pi}\mathrm{d}\phi\int_0^\pi\sin\theta\mathrm{d}\theta\int_0^{2a_0}
					|\psi_{100}(r,\theta,\phi)|^2 r^2dr																					\\
		&=& 4\pi\int_0^{2a_0} r^2[4({Z\over a_0})^3e^{-2Zr/a_0}{1\over4\pi}dr							\\
		&=& {1\over2}\int_0^4\lambda^2e^{-\lambda}\mathrm{d}\lambda												\\
		&=& 76,19\%
	\end{eqnarray*}
	
	Portanto a probabilidade de encontrarmos o el�tron na regi�o classicamente proibida � $23,81\%$.