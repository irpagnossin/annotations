\documentclass[a4paper,12pt]{scrartcl}
\usepackage[utf8]{inputenc}
\usepackage[T1]{fontenc}
\usepackage[brazil]{babel}

\title{Fônons}
\author{I. R. Pagnossin}

\newcommand\foreign[1]{\textsl{#1}}

\begin{document}

\maketitle

O termo ``fônons'' refere-se às vibrações de um material. Qualquer material vibra. Por exemplo, quando batemos no prato de uma bateria, ele vibra (e por isso emite som). Essas vibrações podem ser modeladas de pelo menos três formas diferentes, sendo a primeira delas clássica (ie, por meio das leis da mecânica clássica) e as outras duas, quântica (ie, por meio das leis da mecânica quântica).

Na primeira abordagem, interpretamos o material como um conjunto de átomos ligados uns aos outros e que oscilam. Eles são “osciladores acoplados”. A segunda maneira de analisar esse problema é considerá-los excitações quânticas de um estado fundamental, no qual não existe vibração. Finalmente, a terceira maneira envolve interpretar as vibrações como partículas (na verdade, “quasi-partículas”), chamadas fônons, que são criadas e destruídas (teoria de campo quântico). Assim, quando tocamos no prato da bateria, estamos criando e/ou destruindo milhares de partículas: os fônons.


\section{Teoria}

\subsection{Abordagem clássica: modos normais}

Vamos nos limitar aos materiais que apresentam-se na forma de estruturas cristalinas, isto é, na forma de um arranjo ordenado e repetitivo de átomos ligados uns aos outros.
EXTRA: fônons em materiais amorfos?

\subsection{Abordagem quântica \#1: osciladores quânticos}

Nesta abordagem, vamos considerar que cada átomo da rede é um oscilador harmônico quântico e, por isso, cada um deles tem energia $E_n = (n+1/2)\hbar \omega$.

\subsection{Abordagem quântica \#2: fônons}

Finalmente, os fônons propriamente ditos: eles são quasi-partículas criadas e destruídas no mar de Fermi.

\section{Fenômenos físicos}

Os fônons ou, equivalentemente, as vibrações dos átomos de um material, contribuem para vários fenômenos físicos, tais como a condutividade térmica e elétrica, o calor específico, a propagação do som [lista completa]. Aliás, o nome ``fônon'', que vem do grego \foreign{phonon}, significa som.

\subsection{Calor específico}
	
A capacidade térmica é a grandeza que relaciona energia, ou calor, com temperatura. Por exemplo, quando acendemos o fogo sob uma chaleira, fornecemos calor (energia) para a água. Como consequência, ela aquece. Se fizermos o mesmo experimento com álcool, veremos que sua temperatura varia muito mais rapidamente. Isso acontece porque a capacidade térmica do álcool é menor que a da água, isto é, o álcool tem menos capacidade de armazenar energia (calor) que a água. Realmente, nós aprendemos isso na escola por meio da equação $Q = C \Delta T$, onde $Q$ é o calor fornecido, $C$ é a capacidade térmica do material e $\Delta T$ é a variação na temperatura. Note que $C$ é uma quantidade extensiva, isto é, ela depende do ``tamanho'' do material. Realmente, é costume escrever $C = mc$, onde $c$ é o calor específico, que nada mais é que a capacidade térmica de uma quantidade padronizada de matéria.

Microscopicamente, essa energia é armazenada na forma de energia cinética (dos átomos, moléculas, elétrons etc) ou potencial. Realmente, o \emph{princípio da equipartição de energia} diz que cada grau de liberdade do material é capaz de armazenar $\frac{1}{2}k_BT$ de energia, onde $k_B$ é a constante de Boltzmann e $T$ é a temperatura (em kelvin). Por exemplo, o gás hélio é monoatômico (ie, é composto por átomos isolados de hélio). Imagine um átomo dentro de uma câmara. Ele pode mover-se em três direções: $x$, $y$ e $z$. Ou seja, ele tem três graus de liberdade. Isso significa que cada átomo de hélio tem a capacidade de armazenar $\frac{3}{2}k_BT$. É muito pouca energia, mas lembre-se de que, quando falamos sobre gases e materiais diversos, a quantidade de átomos é da ordem de $10^{23}$. Mais especificamente, cada mol desse gás contém $N_A \approx 6 \times 10^{23}$ átomos de hélio (número de Avogadro). Assim, um mol de gás hélio é capaz de armazenar $\frac{3}{2}N_Ak_BT = \frac{3}{2}RT$, onde $R = N_Ak_B$ é a \emph{constante universal dos gases}.

...

Mas a lei de Dulong-Petit falha quando a temperatura aproxima-se de zero (tipicamente, \underline{algumas dezenas de kelvin}). 

	


\end{document}
