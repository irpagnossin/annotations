\documentclass[a4paper,12pt]{scrartcl}
\usepackage[utf8]{inputenc}
\usepackage[T1]{fontenc}
\usepackage[brazil]{babel}
\usepackage{amsmath}
\usepackage{paralist}
\usepackage{indentfirst}

%opening
\title{Primeira lei da Termodinâmica}
\subtitle{Energia interna $U$}
\author{I. R. Pagnossin}

\begin{document}

\maketitle

\begin{abstract}
  Neste trabalho, exploramos a primeira lei da Termodinâmica utilizando como equação de estado a energia interna $U$, que pode ser função de quaisquer dois pares das variáveis de estado: $T$, $V$, $P$ ($H$, $S$, $G$...?).
\end{abstract}

\newcommand\D[3]{\ensuremath{\left(\frac{\partial #1}{\partial #2}\right)_{#3}}}

\section{$U \equiv U(T,V)$}

  Nesta primeira seção, utilizaremos as \emph{variáveis de estado} $T$ (temperatura) e $V$ (volume), com o intuito de associar os fenômenos termodinâmicos com aqueles da mecânica clássica.
  Particularmente, sabemos que, na mecânica clássica, a energia mecânica pode assumir duas formas: energia cinética, associada ao movimento, e energia potencial, associada a algum tipo de interação presente (gravitacional, elétrica \etc).
  Sabemos ainda que, segundo XXXX, podemos associar a temperatura de um sistema à agitação das moléculas e átomos que o compõe, e que as interações entre as moléculas é dominada pelo eletromagnetismo (a interação gravitacional é desprezível [FAZER CONTAS]).
  
  Deste modo, quando dizemos que a temperatura de um gás aumenta, entendemos intuitivamente que a energia cinética dele aumenta.
  Por outro lado, quando esse sistema expande (contrai), a distância entre as moléculas aumenta (diminui), causando uma redução (aumento) na energia potencial.
  Utilizando essas associações, fica mais fácil de entender alguns resultados da termodinâmica. Particularmente, a primeira lei da termodinâmica.
  

  Primeira lei da Termodinâmica:
  \begin{equation}\label{eq:1st-law}
  U = Q + W. 
  \end{equation}

  Ou seja, a energia (dita ``interna'') $U$ de um material é dada pela soma do calor $Q$ que lhe foi fornecido com o trabalho que o meio externo exerceu sobre esse material.
  
  Da mecânica clássica, $dW = -P\, dV$, onde $P$ é a \emph{pressão externa} e $V$ é o volume do material.
  
  Do Cálculo Diferencial e Integral, como $U \equiv U(T,V)$ é uma função contínua de $T$ (temperatura) e $V$, vale:
  \begin{equation*}
  dU = \D{U}{T}{V}\, dT + \D{U}{V}{T}\, dV.
  \end{equation*}
  
  Usando a primeira lei, podemos escrever de maneira genérica (para qualquer material):
  \begin{align}
  dQ &= dU - dW \nonumber\\
     &= \D{U}{T}{V}\, dT+ \left[P + \D{U}{V}{T}\right]\, dV.\label{eq:dQ}
  \end{align}

  O calor (eneregia) fornecido ao material tem três destinações possíveis:
  \begin{compactenum}
  \item Ele pode ser armazenado na forma de energia cinética dos átomos e moléculas que compõem o material.
  Essa forma de energia corresponde à temperatura (segundo a teoria cinética dos gases):
  \begin{equation*}
  dQ_1 = \D{U}{T}{V}\, dT.
  \end{equation*}

  \item O calor também pode ser armazenado na forma de energia potencial, que em geral pode ser associado à distância entre as moléculas/átomos que compõem o material. Ou seja, essa parcela do calor altera o volume ocupado pelo material:
  \begin{equation*}
  dQ_2 = \D{U}{V}{T}\, dV.
  \end{equation*}
  
  \item Mas para expandir, o material precisa ``abrir espaço'' no meio externo, trabalhando contra ele. Portanto, parte do calor fornecido ao material deve ser utilizado para realizar trabalho:
  \begin{equation*}
  dQ_3 = -P\, dV,
  \end{equation*}
  lembrando que $P$ é a pressão que o meio externo ao material exerce sobre o material.
  
  Note que o sinal de menos dá conta do fato de que essa parcela do calor de fato não é transferida ao material, mas sim utilizada para realizar trabalho contra o meio externo.
  \end{compactenum}

  \subsection{Processo isotérmico: $dT = 0$}
    
  Neste caso, o calor fornecido pelo material é (i) armazenado na forma de energia potencial e (ii) usado para ``abrir caminho'' no meio externo:
  \begin{equation*}
  dQ = \left[P + \D{U}{V}{T}\right]\, dV.
  \end{equation*}
  
  Se o processo ocorre no vácuo, $P = 0$ e, por conseguinte, todo o calor fornecido é armazenado no sistema.
  
  Inversamente, se $dQ$ é removido do sistema (diz-se que o sistema ``rejeita o calor $dQ$''), ele é obtido aos custos da (i) redução da energia potencial nas ligações químicas e (ii) da energia extra que o sistema ganhou ao ser comprimido pelo meio externo.
  
  \subsection{Processo isocórico: $dV = 0$}
  
  Para um processo termodinâmico em que não há variação de volume (chamado ``isocórico''), isto é, quando $dV = 0$, o calor fornecido (removido) é todo empregado no aumento (redução) da energia cinética das moléculas e átomos do material. Ou seja, há apenas variação de temperatura. Isso pode ser deduzido imediatamente a partir de \eqref{eq:dQ} fazendo $dV = 0$:
  \begin{align*}
  dQ \equiv dQ_1 &= \D{U}{T}{V}\, dT\\
                 &:= C_V\, dT,
  \end{align*}
  onde $C_V = \D{U}{T}{V}$ é a \emph{capacidade térmica} (a volume constante) do material. Esse parâmetro, então, relaciona a energia fornecida (ou removida) com a variação da temperatura do material.
  
  Ou seja, $C_V$ é a taxa com que a energia (calor) é armazenada (ou removida) das vibrações do material, com relação à temperatura.
  
  Se usarmos essa definição em \eqref{eq:dQ}, obtemos uma forma alternativa para a primeira lei da Termodinâmica, tão genérica quanto \eqref{eq:1st-law}:
  \begin{equation}
  dQ = C_V\, dT + \left[P + \D{U}{V}{T}\right]\, dV.\qquad U\equiv U(T,V)
  \end{equation}
  
  \subsection*{Processo adiabático: $dQ = 0$}
  
  Por outro lado, quando $dQ = 0$,
  \begin{equation*}
  0 = C_V\, dT + P\left(dV\right)_T + \D{U}{V}{T}\left(dV\right)_T,
  \end{equation*}
  ou seja, a toda variação na energia cinética ocorre uma variação oposta na energia potencial, além do trabalho. É o que vemos em sistemas mecânicos onde a energia mecânica é conservada. Isso fica mais claro se $P = 0$ (vácuo no meio externo). Neste caso, $C_V\, dT = - \D{U}{V}{T}\left(dV\right)_T$.

  

  
\section{$U \equiv U(P,T)$}
  
  Se escolhermos usar a pressão $P$ e a temperatura $T$ como \emph{variáveis de estado} independentes, podemos afirmar que:
  \begin{equation}\label{eq:dU(P,T)}
  dU = \D{U}{P}{T}\, dP + \D{U}{T}{P}\, dT,
  \end{equation}
  já que $U(P,T)$ é uma função (de estado) contínua. Note que essa afirmação vem do Cálculo Diferencial e Integral, não da Física.
  
  Agora, lembrando que $dW = -P\, dV$ e usando novamente a expressão diferencial da primeira lei da Termodinâmica, $dU = dQ + dW$, podemos escrever:
  \begin{equation*}
  dQ = \D{U}{P}{T}\, dP + \D{U}{T}{P}\, dT + P\, dV
  \end{equation*}
  
  Mas $V$ desta vez não é uma variável de estado. Ao invés disso, $V equiv V(P,T)$, de modo que
  \begin{equation}
  dV = \D{V}{P}{T} dP + \D{V}{T}{P} dT.
  \end{equation}
  
  Assim,
  \begin{equation*}
  dQ = \left[\D{U}{P}{T} + P\, \D{V}{P}{T}\right] dP + \left[\D{U}{T}{P} + P \D{V}{T}{P}\right] dT. ???
  \end{equation*}

  

  \subsection*{Processo isobárico: $dP = 0$}
  
  Analogamente a $C_V$, definimos a \emph{capacidade térmica a pressão constante} de modo que ela relacione o calor fornecido com uma variação na temperatura (energia cinética):
  \begin{equation}
  \left(dQ\right)_P = C_P\left(dT\right)_P.
  \end{equation}
  
  Note, entretanto, que não relacionamos $C_P$ com uma variação da energia interna, isto é, não afirmamos que $C_V = \D{U}{T}{P}$. \emph{Isso é incorreto!}
  
  De \eqref{eq:dU(P,T)} e da forma fundamental da primeira lei da Termodinâmica, 

  
  
  
  
  
  
  \subsection*{Processo isotérmico}
  
  De $dT = 0$ decorre que:
  \begin{equation*}
  \left(dQ\right)_T = P\left(dV\right)_T + \D{U}{V}{T}\left(dV\right)_T.
  \end{equation*}
  
  Ou seja, quando não há variação de temperatura, o calor fornecido (removido) ao material é (i) armazenado na forma energia potencial (variação do volume) \emph{e} (ii) é usado para ``abrir caminho'' no meio externo. Note ainda que, se esse processo ocorresse no vácuo, $P = 0$ e todo o calor fornecido seria armazenado na forma de energia potencial.
  

  

  
  
  


\end{document}
