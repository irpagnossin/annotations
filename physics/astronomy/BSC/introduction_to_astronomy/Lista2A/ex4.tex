\begin{quotation}
{\bf 4.}
{\it	Compare a insola��o (fluxo de energia solar ao n�vel do solo) em S�o Paulo nos solst�cios
de ver�o e inverno.
}
\end{quotation} 

	Dado que a cidade de S�o Paulo est� sob o tr�pico de capric�rnio, ent�o no solst�cio de
ver�o do hemisf�rio sul, ao meio dia, o Sol est� muito pr�ximo do z�nite, de forma que, para uma
�rea $A$ qualquer no solo, incide um fluxo de pot�ncia $P$. Por outro lado, no solst�cio
de inverno (do hemisf�rio sul), os raios solares formam um �ngulo de $46^0$ com a perpendicular
ao solo, de tal sorte que o mesmo fluxo $P$\footnote{Desconsiderando
o fato de agora o hemisf�rio sul estar um pouco mais longe do Sol que o hemisf�rio norte. Esta
diferen�a � muito pequena.} agora incide numa �rea $B=A/\sin(46^o)\approx A/0,6946$. Ou seja, se
no solst�cio de ver�o t�nhamos uma insola��o $I_v=\Phi/A$, agora temos 
$I_i=P/B=0,6496P/A\approx 0,65\,I_v$. Em palavras: A insola��o no inverno � $65\%$ menor que
no ver�o.

	%\begin{figure}[htb]\label{fig:ex4}
		%\centering
		%\includegraphics[scale=0.5]{ex10.pdf}
		%\caption{\footnotesize O mesmo fluxo $P$ proveniente do Sol atinge o solo em �ngulos 
%diferentes ao longo do ano. Esta inclina��o altera a insola��o e, consequentemente, a temperatura
%do solo, dando origem �s esta��es do ano.}
	%\end{figure}

