\begin{quotation}
{\bf 6.}
{\it 	Uma nave espacial viaja at� a Lua. {\tt (a)} Em que ponto ela experimenta gravidade zero?
{\tt (b)} Que tempo levar� a nave para circum-navegar a Lua numa �rbita a $70\,\,\mathrm{km}$ de
altura?}
\end{quotation}

	{\tt (a)} Ignorando a influ�ncia de outros corpos celestes, a posi��o de gravidade zero 
est� contida na linha imagin�ria que liga a Terra � Lua, a uma dist�ncia $r$ da Terra tal que a
for�a gravitacional entre a nave e a Terra seja igual � for�a gravitacional exercida pela Lua
sobre a nave:
	\begin{eqnarray*}
	F_{\bigoplus N}	&=&	F_{LN}	\\
	G{mM_{\bigoplus}\over r^2}	&=&	G{mM_L\over (d-r)^2}	\\
	{M_{\bigoplus}\over r^2}	&=&	{M_L\over (d-r)^2}	\\
	\end{eqnarray*}

	Daqui decorre imediatamente que 
	$$
	r={d\over 1+\sqrt{M_L/M_{\bigoplus}}}\cong 0,9\, d
	$$

	sendo $d$ a dist�ncia Terra-Lua. Ou seja, a nave deve estar nove vezes mais perto da Lua que da 
Terra para experimentar a gravidade zero.
	
	{\tt (b)} Pela terceira Lei de Kepler, 
	$$
	T^2={4\pi^2\over GM_L}(h+R_L)^3={4\pi^2\over 6,67\times10^{-11}\cdot0,0123M_{\bigoplus}}
		(70000+1738000)^3=1\mathrm{h}55\mathrm{m}2\mathrm{s}
	$$

	onde $h$ � a altura da �rbita da nave sobre a superf�cie da Lua.