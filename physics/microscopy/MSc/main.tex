%%%%%%%%%%%%%%%%%%%%%%%%%%%%%%%%%%%%%%%%%%%%%%%%%%%%%%%%%%%%
\documentclass[a4paper,10pt]{book}[1995/12/01]

\usepackage[latin1]{inputenc}
\usepackage[brazil]{babel}
\usepackage{graphics,graphicx}
\usepackage{fancyhdr}
\usepackage{geometry}
\usepackage{irpackage}
\usepackage{makeidx}
\usepackage{pifont}
\usepackage{wrapfig}
\usepackage{hyperref}
\usepackage{amstext}

\pagestyle{fancy}
\fancyhead{}
\fancyhead[LE,RO]{\slshape Microscopia de varredura por sonda}
\fancyhead[LO,RE]{\slshape\bfseries I. R. Pagnossin}
\fancyfoot[C]{\thepage}
\renewcommand{\headrulewidth}{0.4pt}
\renewcommand{\footrulewidth}{0.4pt}

\newcommand{\scanner}{{\slshape scanner}}
\newcommand{\creep}  {{\slshape creep}}
\newcommand{\xcoupling}{{\slshape cross-coupling}}
\newcommand{\tensor}[1]{\ensuremath{\mbox{\sffamily\bfseries #1}}}
\newcommand{\vetor}[1]{\ensuremath{\mbox{\bfseries #1}}}
\renewcommand{\footnoterule}{}


\makeindex
%%%%%%%%%%%%%%%%%%%%%%%%%%%%%%%%%%%%%%%%%%%%%%%%%%%%%%%%%%%%
\begin{document}

\title{\LARGE 
Microscopia de varredura por sonda\\Scanning probe microscopy}
\author{Ivan Ramos Pagnossin\thanks{irpagnossin@hotmail.com}}
\maketitle

\tableofcontents

\chapter{Scanning probe microscopy}
  \label{cap:introduction}
  %	Projeto: 	Scanning probe microscopy
%	Autor:	 	Ivan Ramos Pagnossin
%	Data:		 	07/06/2002
%	Arquivo: 	introduction.tex
%	Cap�tulo:	Introcu��o � microscopia de varredura por sonda.


	Diversas tecnologias de microscopia v�m sendo desenvolvidas desde que Robert Hooke\index{Robert Hooke} primeiro observou, em 1665, a estrutura de uma corti�a com o aux�lio de um microsc�pio �ptico\index{Microsc�pio �ptico}. Uma dessas tecnologias � a de varredura\index{Varredura}, onde o sensor (que pode ser �ptico, el�trico, magn�tico, etc.) varre a amostra, registrando, ponto a ponto, algum tipo de intera��o pertinente � observa��o. A id�ia � formar um conjunto de ternos ordenados $(x,y,z)$, onde $(x,y)$ � a posi��o da sonda numa amostra mapeada por uma matriz quadrada de pontos igualmente distanciados (figura~\ref{fig:varredura}) e $z$ � algum registro de intera��o entre sonda e amostra que depende do tipo de microscopia que se pretende fazer.\par
	
	Uma caracter�stica peculiar � t�cnica de varredura � o fato de que a sonda n�o registra a intera��o com a amostra de forma cont�nua, como ocorre num microsc�pio �ptico usual, por exemplo; mas sim em intervalos denominados {\slshape tamanho de pixel}\index{Tamanho de pixel} ($\Delta$) que, por simplicidade, � constante durante toda uma medida, tanto na varredura lenta\index{Varredura!lenta} quanto na r�pida\index{Varredura!r�pida} (veja a seguir). Al�m disso, a �rea varrida pela sonda �, em geral, quadrada, tamb�m apenas para simplificar a administra��o dos resultados.\par
	
	A varredura de uma �nica linha na figura~\ref{fig:varredura} leva $T=1/\nu$ segundos (SI), onde $\nu$ � a taxa de leitura\index{Varredura!taxa de leitura} de linhas (ajust�vel pelo usu�rio). Finalizada a leitura de uma linha, a sonda passa para a seguinte, levando outros $T$ segundos para complet�-la. Se a linha possui $N$ pontos, ent�o teremos $N$ linhas, pois a matriz � quadrada. Isto significa que a varredura de toda a matriz leva $NT$ segundos, obviamente maior que o tempo gasto por linha. Por este motivo nomeamos a leitura de uma linha por {\bfseries varredura r�pida} e a varredura na dire��o perpendicular de {\bfseries varredura lenta}.\par	
		
\begin{figure*}[htb]
  \setlength{\unitlength}{0.7mm}
  \begin{minipage}[t]{0.48\textwidth}\centering	  
		\begin{picture}(100,100)(-7,-7)
	  	\multiput(0,0)(0,10){10}{\multiput(0,0)(10,0){10}{\color{blue}\circle*{1.5}}}
	  	\multiput(0,8)(0,10){9}{\multiput(0,0)(10,0){10}{\color{red}\circle*{1.5}}}
	  	\multiput(0,0)(0,10){10}{\vector(1,0){90}}
	  	\multiput(90,8)(0,10){9}{\vector(-1,0){90}}
	  	\multiput(90,0)(0,10){9}{\line(0,1){8}}
	  	\multiput(0,8)(0,10){9}{\line(0,1){2}}
	  	\put(-5,-5){\vector(1,0){15}}\put(-5,13){\makebox(0,0){$y$}}
	  	\put(-5,-5){\vector(0,1){15}}\put(13,-5){\makebox(0,0){$x$}}
	  	\put(40,-7.5){\line(0,1){5}}\put(35,-5){\vector(1,0){5}}
	  	\put(50,-7.5){\line(0,1){5}}\put(55,-5){\vector(-1,0){5}}
	  	\put(57,-5){\makebox(0,0)[cl]{\scriptsize Tamanho de pixel}}
	  	\put(35,95){\vector(1,0){20}}
	  	\put(95,35){\vector(0,1){20}}
		\end{picture}
		\caption{\label{fig:varredura}\footnotesize As linhas em preto indicam o sentido da varredura. Os pontos azuis representam as medidas enquanto a sonda\index{Sonda} move-se da esquerda para a direita ({\slshape trace})\index{Trace}; os vermelhos representam as medidas no sentido contr�rio ({\slshape retrace})\index{Retrace}. Embora na representa��o acima os pontos azuis e vermelhos esteja em posi��es diferente, em realidade eles superp�e-se, ao menos idealmente.}
	\end{minipage}\hfill
	\begin{minipage}[t]{0.48\textwidth}
		\begin{picture}(100,100)(-7,-7)
  		\multiput(0,0)(0,10){10}{\multiput(0,0)(10,0){10}{\color{blue}\circle*{1.5}}}
  		\multiput(0,8)(0,10){9}{\multiput(0,0)(10,0){10}{\color{red}\circle*{1.5}}}
  		\multiput(90,0)(0,10){10}{\vector(-1,0){90}}
  		\multiput(0,8)(0,10){9}{\vector(1,0){90}}
	  	\multiput(90,0)(0,10){9}{\line(0,1){8}}
	  	\multiput(0,8)(0,10){9}{\line(0,1){2}}
	  	\put(-5,-5){\vector(1,0){15}}\put(-5,13){\makebox(0,0){$y$}}
	  	\put(-5,-5){\vector(0,1){15}}\put(13,-5){\makebox(0,0){$x$}}
	  	\put(55,95){\vector(-1,0){20}}
	  	\put(95,55){\vector(0,-1){20}}
		\end{picture}
		\caption{\footnotesize A figura ao lado representa o movimento de baixo para cima da varredura lenta; As no��es de {\slshape trace} e {\slshape retrace} s�o absolutamente relativas, exatamente como o que � ``de baixo para cima'' e ``de cima para baixo'': Podemos sem problema algum trocar os nomes. O importante � o conceito de que cada um varre a amostra num sentido diferente.}			
	\end{minipage}
\end{figure*}

	J� que os registros das intera��es n�o � feito de forma cont�nua, qualquer estrutura menor que $\Delta$ passa desapercebida pela sonda\footnote{Exceto quanto esta estrutura ocorre de forma peri�dica.}. Ou seja, este m�todo � veross�mil apenas para estruturas maiores que o tamanho de pixel.\par
	
	O n�mero de medidas por linha (par�metro {\slshape number of samples})\index{Number of samples}, que � igual ao n�mero de linhas varridas, por vezes � tomado como sin�nimo de resolu��o. Este valor � de crucial import�ncia pois a partir dele podemos determinar o tamanho de pixel, um dos principais fatores a se considerar tanto na an�lise de uma medida quanto no planejamento de um experimento. Por exemplo, suponha que num determinado experimento $\Delta=1\unit{\mu m}$, ent�o qualquer estrutura inferior a $1\unit{\mu m}$ n�o poderia ser associada, com seguran�a, � amostra. Ou seja, o tamanho de pixel � de sum�ria import�ncia para nos dar crit�rios de avalia��o do que se obt�m.\par
	
	Podemos determinar o tamanho de pixel facilmente uma vez conhecidas a resolu��o e a �rea varrida pela sonda (definida pelo usu�rio): Por exemplo, digamos que se queira medir uma �rea de $200\unit{nm}\times200\unit{nm}$ com uma resolu��o de $256$ pontos por linha. Ent�o � claro que $\Delta=200\unit{nm}/256\cong8\unit{\AA}$ (veja a 
figura~\ref{fig:varredura}).\label{ex:pixel}\par

	Quando utilizamos como sonda uma diminuta ponta\index{Ponta}\index{Sonda!Ponta} que interage localmente com a amostra, especificamos a t�cnica n�o s� como microscopia de varredura\index{Microscopia!de varredura}, mas sim {\slshape microscopia de varredura por sonda}\index{Microscopia!de varredura!por sonda} ou, em ingl�s, {\bf scanning probe microscopy (SPM)}. S�o v�rios o modos (ou modalidades de SPM\index{Modalidades de SPM}, distinguindo-se cada qual pela intera��o entre ponta e amostra (veja lista a seguir): Na modalidade {\slshape microscopia de for�a magn�tica}\index{Microscopia!de for�a magn�tica}, e.g., a ponta � imantada e interage magneticamente com a amostra, identificando os dom�nios magn�ticos desta. Por outro lado, a {\slshape microscopia de tunelamento}\index{Microscopia!de tunelamento} baseia-se no conhecido fen�meno qu�ntico de tunelamento, medindo, ent�o, a distribui��o eletr�nica dos �tomos da superf�cie da amostra.

	Um outro ponto que caracteriza a t�cnica SPM � o fato de a sonda (ponta) interagir com a {\bfseries superf�cie} da amostra\footnote{Ou, mais rigorosamente, com os planos at�micos mais superficiais.}, sendo incapaz de avaliar o interior da amostra como ocorre com a microscopia de varredura eletr�nica, por exemplo\index{Microscopia!de varredura!eletr�nica}. Al�m disso, todas as t�cnicas SPM podem ser utilizadas em press�o ambiente, desde que tomados alguns cuidados, principalmente na interpreta��o dos resultados. Isto por que nessas condi��es toda e qualquer amostra apresenta �gua adsorvida que, dependendo da t�cnica utilizada, pode ou n�o modificar os resultados. No modo AFM de contato\index{AFM!de contato}\index{AFM!contact}, por exemplo, esta preocupa��o � desnecess�ria; N�o no AFM de contato intermitente\index{AFM!de contato intermitente}\index{AFM!tapping mode}, entretanto (veja cap�tulo~\ref{cap:afm}). Outro artefato que geralmente se apresenta na superf�cie das amostras s�o �xidos, de preocupa��o fundamental para a microscopia de tunelamento, que depende da condutividade\index{Condutividade} da superf�cie (veja cap�tulo~\ref{cap:stm})
	
	Abaixo segue uma pequena rela��o dos tipos de microscopia SPM\index{Microscopia!SPM} existentes.
	
\begin{dingautolist}{202}
\item AFM --- Atomic force microscope (Microscopia de for�a at�mica)
	\begin{dingautolist}{202}
		\item Contact AFM (AFM de contato)
		\item Tapping mode AFM (AFM de contato intermitente)
		\item Non-contact AFM (AFM de n�o-contato)
	\end{dingautolist}
\item STM --- Scanning tunneling microscopy (Microscopia de tunelamento)	
\item MFM --- Magnetic force microscopy (Microscopia de for�a magn�tica)
\item EFM --- Eletric force microscopy (Microscopia de for�a el�trica)
\item LFM --- Lateral force microscopy (Microscopia de for�a lateral)
\item FMM --- Force modulation microscopy (Microscopia de for�a modulada)
\item SCM --- Scanning capacitance microscopy (Microscopia de capacit�ncia)
\item TSM --- Thermal scanning microscopy (Microscopia t�rmica)
\item NSOM --- Near-file scanning optical microscopy (Microscopia de campo pr�ximo)
\end{dingautolist}

\chapter{Scanners}  
  \label{cap:scanners}\index{Scanner}
  %	Projeto:  Scanning probe microscopy
%	Autor:	  Ivan Ramos Pagnossin
%	Data:		  07/06/2002
%	Arquivo:  scanner.tex
% Cap�tulo: Scanner
	
\section{Princ�pios}
	
	Os deslocamentos nanom�tricos exigidos pela microscopia de varredura atualmente s�o feitos baseados no fen�meno da piezoeletricidade\index{Piezoeletricidade}. Dispositivos com essa fun��o s�o genericamente conhecidos como {\slshape atuadores piezoel�tricos}\index{Atuador piezoel�trico} ou,
especificamente na microscopia de varredura, {\slshape scanners} (figura~\ref{fig:scanner1})\index{Scanner}. No que segue, trataremos ambos como sin�nimos.\par

	Existem dois tipos b�sicos de atuadores piezoel�tricos: Os realimentados ({\slshape closed-loop}) \index{Atuador piezoel�trico!realimentado} e os n�o-realimentados ({\slshape opened-loop})\index{Atuador piezoel�trico!n�o realimentado}. O primeiro, de maior custo, conta com um sistema de realimenta��o eletr�nica que verifica continuamente os deslocamentos produzidos; o segundo n�o. O material piezoel�trico mais comum no mercado � o titanato de zirc�nio chumbo (PZT)\index{PZT}.
		
\begin{figure}[htb]
  \begin{minipage}[b]{0.46\textwidth}
	  \centering\includegraphics[scale=0.5]{figuras/scanner/scanner1.jpg}
	  \caption{\label{fig:scanner1}\footnotesize Um \scanner\ t�pico e seu esquema el�trico simplificado: Um cil�ndro 		met�lico � recoberto interna e externamente por PZT.}
	\end{minipage}\hfill
	\begin{minipage}[b]{0.46\textwidth}
	  \centering\includegraphics[scale=0.5]{figuras/scanner/scanner2.jpg} 
	  \caption{\label{fig:scanner2}\footnotesize Da esquerda para a direita, o material antes do tratamento apresenta seus dip�los el�tricos \index{Dip�lo el�trico} distribuidos aleatoriamente. Durante o aquecimento e polariza��o todos os dip�los s�o alinhados ao campo el�trico aplicado. No final, ao retornar � temperatura ambiente, o PZT mant�m todos os dipolos praticamente alinhados, gerando coer�ncia de resposta a campos el�tricos, mesmo que pequenos.}	  
	\end{minipage}
\end{figure}

	A fabrica��o de cer�micas piezoel�tricas constitui-se dos seguintes passos (conforme~\cite{pi}): Primeiro os elementos b�sicos (como o tit�nio no PZT) s�o triturados e misturados. A mistura � ent�o aquecida a $75\%$ da temperatura de sinteriza��o para acelerar as rea��es qu�micas. O material � novamente triturado e re-misturado para aumentar sua reatividade. Em seguida � cozido � temperatura de $750^{\mathrm{o}}$, seguido de nova sinteriza��o entre $1250^{\mathrm{o}}$ e
$1350^{\mathrm{o}}$. Neste ponto a cer�mica pode ser cortada e lapidada na forma desejada. � ainda nesta fase que se fixam os eletrodos. O procedimento final � a polariza��o, que ocorre num banho aquecido em �leo e sujeito a campos el�tricos de v�rios kilovolts por mil�metro. Com isso o material expande-se ao longo do campo el�trico e contrai-se no plano perpendicular, alinhando todos os dipolos el�tricos moleculares (dom�nios de Weiss)\index{Weis, dom�nios}. A situa��o � mantida por algumas horas e em seguida resfria-se a cer�mica, ainda com o campo aplicado. Ao final, o material apresenta uma polariza��o residual (que pode, entretanto, ser destru�da se os limites el�trico, mec�nico ou t�rmico forem superados).
Como resultado temos uma cer�mica que expande-se (ou contrai-se) ao longo do campo aplicado, e proporcionalmente � sua intesidade. A figura~\ref{fig:scanner2} ilustra o processo.

	Naturalmente, com o tempo os dip�los el�tricos moleculares movem-se de forma a desalinharem-se em busca de uma configura��o de menor energia, causando a perda da propriedade piezoel�trica. Para evitar este processo � preciso manter o \scanner\ em freq��nte uso, aplicando-lhe sempre um campo el�trico. Este desalinhamento � imensamente acelerado para temperaturas superiores a aproximadamente $150\unit{^\mathrm{o}C}$, a temperatura Curie\index{Temperatura Curie} do material. Isto se chama {\slshape aging} (ou {\slshape envelhecimento}) e ser� melhor comentado na se��o~\ref{sec:aging}.\par

	Em primeira aproxima��o, o deslocamento relativo $\vetor s$ total do material piezoel�trico � linear com o campo el�trico aplicado, segundo $\vetor s = \tensor d\vetor E$, sendo $\tensor d$ o tensor de acoplamento piezoel�trico, \index{Acoplamento piezoel�trico} dado em $\unit{C/N}$ (SI). Para o PZT-5A, e.g., $\tensor d$ � da ordem de $10^{-12}\unit{C/N}$. O car�cter tensorial dessa express�o � �til para representar o fato experimental de que $\vetor s$ n�o � paralelo a $\vetor E$. Contudo, a rela��o anterior n�o �, a rigor, verdadeira, pois os deslocamentos $\vetor s$ n�o s�o lineares com $\vetor E$, dependendo de muitas outras propriedades do material e do pr�prio campo. A este fen�meno damos o nome de {\bfseries n�o-linearidade intr�nseca} (veja se��o~\ref{sec:nli}).
	
	Um atuador piezoel�trico �, como dissemos, um dispositivo que utiliza materiais piezoel�tricos para produzir nano-deslocamentos controlados. Mas como ser� poss�vel controlar tais deslocamentos se n�o conhecemos o tensor $\tensor d$? O primeiro m�todo, usado nos atuadores n�o-realimentados, � calibr�-los freq�entemente utilizando uma amostra padr�o cujas dimens�es s�o conhecidas (veja, e.g., a figura~\ref{fig:padrao}). Com isso o software respons�vel por controlar o \scanner\ pode manter atualizados arquivos de configura��es que refletem o comportamento do material piezoel�trico utilizado mediante aos est�mulos externos. Lembre-se que qualquer material piezoel�trico envelhece (se��o~\ref{sec:aging}), alterando suas propriedades. Isto explica a necessidade da calibra��o \index{Calibra��o} rotineira. Este procedimento, entretanto, n�o elimina completamente os efeitos da n�o linearidade do material piezoel�trico. De fato, {\slshape scanners} n�o-realimentados apresentam erros aproximadamente $10$ vezes maiores que os realimentados: Neste tipo de atuador, n�o h� a necessidade de calibra��es\footnote{Veja se��o~\ref{sec:corrections}.} pois o dispositivo conta com sistemas �pticos ou capacitivos para medir, continuamente, o deslocamento $\vetor s$. Assim, as a��es do software respons�vel pelos deslocamentos n�o levam em considera��o as tens�es aplicadas --- exceto para evitar romper os limites de usabilidade da cer�mica ---, mas somente o efeito delas.\par
	
	Consideremos um pequeno exemplo da situa��o acima para clarear a diferen�a entre os dois tipos de {\slshape scanners}:
Suponha que tenhamos duas cer�micas piezoel�tricas id�nticas, uma sendo utilizada por um atuador realimentado e outro por um n�o-realimentado (calibrado {\slshape hoje}). Tome ainda como verdade que {\slshape hoje} uma diferen�a de potencial $V$ � capaz de produzir um deslocamento $\vetor s$ no material (representemos por $V\to\vetor s$), e que {\slshape amanh�} esta mesma tens�o corresponde a um deslocamento $\vetor s+\Delta\vetor s$ ($V\to\vetor s+\Delta\vetor s$). Ent�o, {\slshape hoje} ambos os {\slshape scanners}, ao serem solicitados pelo usu�rio a produzir o deslocamento $\vetor s$, acertam igualmente. {\slshape Amanh�}, por�m, o atuador n�o-realimentado, utilizando ainda os arquivos de configura��es de {\slshape hoje}, ``acredita'' que a diferen�a de potencial $V$ ainda � capaz de produzir o deslocamento solicitado e, por isso, aplica $V$ na cer�mica, errando, portanto, de $\Delta\vetor s$ neste valor. O atuador realimentado, por outro lado, n�o sabe --- e nunca soube, na verdade --- que {\slshape hoje} $V$ � capaz de produzir $\vetor s$; nem que {\slshape amanh�} � necess�rio $V+\Delta V$ para produzir o mesmo efeito. O que ele faz � simplesmente aplicar uma tens�o na piezo e medir o deslocamento. Ent�o, independentemente se � {\slshape hoje} ou {\slshape amanh�}, o atuador realimentado sempre atinge $\vetor s$. Em termos esquem�ticos, para o \scanner\ realimentado teremos $V+\Delta V\to\vetor s$ e, para o n�o-realimentado, $V\to\vetor s+\Delta\vetor s$.\par

	Existem outros efeitos n�o lineares que s�o tratados, apenas por did�tica, separadamente. Em todos eles teremos sempre a situa��o acima: Os {\bfseries atuadores}\footnote{Note que n�o estamos nos referindo �s cer�micas piezoel�tricas, que sempre apresentam tais alinearidades; mas sim dos {\bfseries atuadores} (ou {\slshape scanners}), que podem sempre contar com outros dispositivos para evitar tais problemas.} realimentados n�o sofrem, na pr�tica, tais efeitos (exceto em $z$, como ser� visto na se��o~\ref{sec:corrections}.
	
\newcommand{\dimensao}{\unit{\mu m}}
\begin{figure}[htb]
	\centering
	\includegraphics[scale=0.5]{figuras/scanner/scanner3.jpg}
	
	\begin{tabular}{ccc}
	  \hline
	  {\bfseries Model} & {\bfseries Scan size} & {\bfseries Vertical Range} \\
	  \hline
	  AS-0.5 (A)	&	$0,4\dimensao\times0,4\dimensao$	& $0,4\dimensao$ \\
	  AS-12 (E)   & $10\dimensao\times10\dimensao$ & $2,5\dimensao$ \\
	  AS-130 (J)  & $125\dimensao\times125\dimensao$ & $5,0\dimensao$ \\
	  AS-130V (J vertical) & $125\dimensao\times125\dimensao$ & $5,0\dimensao$ \\
	  AS-200 & $200\dimensao\times200\dimensao$ & $8,0\dimensao$ \\
	  \hline
	\end{tabular}
		
	\caption{\label{fig:scanner3}\footnotesize Alguns {\slshape scanners} n�o-realimentados fabricados pela Digital Instruments para microsc�pios {\slshape multimode} SPM. Da esquerda para a direita: AS-130V, AS-200, AS-130 e AS-0.5.}
\end{figure}

	A figura~\ref{fig:scanner1} ilustra a utiliza��o pr�tica de materiais piezoel�tricos como {\slshape scanners} (ou atuadores piezoel�tricos): Ao acionarmos o gerador $\pm Z$, um campo el�trico na dire��o radial � gerado, causando uma contra��o (distens�o) do PZT nesta dire��o; Resulta disto uma distens�o (contra��o) na dire��o perpendicular ($z$). Os movimentos em $x$ e $y$ s�o um pouco mais complicados: Primeiramente, note que diferentemente da parte superior do cilindro, a parte inferior possui pequenas frestas entre os materiais: Essas frestas tornam independentes\label{scanner:explanation} os movimentos produzidos pelos geradores $\pm X$ e $\pm Y$. Al�m disso, perceba que as partes em vermelho e amarelo est�o sempre em oposi��o de fase. Isto faz com que se o material contrair-se � esquerda, ele estica-se � direita, produzindo um movimento para a esquerda no extremo inferior do cil�ndro e vice-versa. Processo id�ntico ocorre para o gerador $\pm Y$. Assim, consegue-se garar movimentos em todas as tr�s dire��es.\par
	
	Diferentes modalidades de SPM utilizam diferentes tipos de sonda. Com isso, muitas vezes torna-se mais simples acoplar n�o a sonda ao \scanner\ --- que � a imagem que primeiro nos v�m a mente --- mas a {\bfseries amostra} ao \scanner. Assim, ao inv�s de movimentarmos a sonda {\bfseries sobre} a amostra, movimentamos a amostra {\bfseries sob} a sonda. O efeito � obviamente id�ntico e, de fato, � o esquema adotado por microsc�pios do tipo {\slshape Multimode} da digital Instruments como, por exemplo, o NanoScope$^{\mbox{\copyright}}$IIIA \index{NanoScope} do laborat�rio de filmes finos do Instituto de F�sica da Universidade de S�o Paulo.

	A seguir discutimos os efeitos n�o lineares sofridos por todos os tipos de materiais piezoel�tricos. Obviamente, como procuramos deixar claro anteriormente, tais fen�menos, embora sempre presentes, atormentam somente os atuadores n�o-realimentados.

%%%%%%%%%%%%%%%%%%%%%%%%%%%%%%%%%%%%%%%%%%%%%%%%%%%%%%%%%%%%%%%%%%%%%%%%%%%%%%%%%%%%%%%%%%%%%%%%%%%%%
% Se��o: N�o linearidade intr�nseca - Cap�tulo: Scanners
%%%%%%%%%%%%%%%%%%%%%%%%%%%%%%%%%%%%%%%%%%%%%%%%%%%%%%%%%%%%%%%%%%%%%%%%%%%%%%%%%%%%%%%%%%%%%%%%%%%%%
\section{N�o linearidade intr�nseca}
	\label{sec:nli}\index{N�o linearidade intr�nseca}

	Conforme comentamos h� pouco, o deslocamento $\vetor s$ do material piezoel�trico n�o � linear com $\vetor E$. Em termos pr�ticos, esta n�o linearidade apresenta-se como uma distor��o da matriz de varredura (fig.~\ref{fig:varredura}), conforme ilustra a figura~\ref{fig:nli}. \index{Varredura!Matriz de}\par
	
	A forma atualmente utilizada pelos fabricantes de {\slshape scanners} (ou, no nosso caso, de microsc�pios SPM) para quantificar a n�o linearidade intr�nseca � express�-la como o quociente do m�ximo desvio da linearidade $\Delta x$ (veja figura~\ref{fig:nli}) pelo deslocamento linear esperado $x$ correspondente � tens�o aplicada:
	$$
	\mbox{N�o linearidade intr�nseca} = {\Delta x\over x}
	$$
	
	Valore t�picos giram em torno de $2\%$ a $25\%$.\par
	
	A partir daqui lembraremos que $\vetor s = x\hat i+y\hat j+z\hat k$, com $\hat i$ e $\hat j$ os versores que definem o plano da varredura \index{Varredura!Plano de} (figura~\ref{fig:varredura}) e $\hat k$ o versor na perpendicular deste plano. O terno $(x,y,z)$, com $x$ e $y$ determinados pelo \scanner\ e $z$ medido conforme o que veremos nos cap�tulos seguintes, determina univocamente o resultado em cada ponto da malha de varredura. \index{Varredura!Malha de} Pelo exposto pode-se dizer que o que vale $x$ vale para $y$ e $z$, exceto quanto dito o contr�rio.
	
\setlength{\unitlength}{2mm}
\begin{figure}
  \centering
  \begin{minipage}[c]{0.46\textwidth}
  	\centering\begin{picture}(30,30)\put(0,0){\circle*{0.001}}
\put(0,1){\circle*{0.001}}
\put(0,2){\circle*{0.001}}
\put(0,3){\circle*{0.001}}
\put(0,4){\circle*{0.001}}
\put(0,5){\circle*{0.001}}
\put(0,6){\circle*{0.001}}
\put(0,7){\circle*{0.001}}
\put(0,8){\circle*{0.001}}
\put(0,9){\circle*{0.001}}
\put(0,10){\circle*{0.001}}
\put(0,11){\circle*{0.001}}
\put(0,12){\circle*{0.001}}
\put(0,13){\circle*{0.001}}
\put(0,14){\circle*{0.001}}
\put(0,15){\circle*{0.001}}
\put(0,16){\circle*{0.001}}
\put(0,17){\circle*{0.001}}
\put(0,18){\circle*{0.001}}
\put(0,19){\circle*{0.001}}
\put(0,20){\circle*{0.001}}
\put(0,21){\circle*{0.001}}
\put(0,22){\circle*{0.001}}
\put(0,23){\circle*{0.001}}
\put(0,24){\circle*{0.001}}
\put(0,25){\circle*{0.001}}
\put(0,26){\circle*{0.001}}
\put(0,27){\circle*{0.001}}
\put(0,28){\circle*{0.001}}
\put(0,29){\circle*{0.001}}
\put(0,30){\circle*{0.001}}
\put(1,0){\circle*{0.001}}
\put(1,1){\circle*{0.001}}
\put(1,2){\circle*{0.001}}
\put(1,3){\circle*{0.001}}
\put(1,4){\circle*{0.001}}
\put(1,5){\circle*{0.001}}
\put(1,6){\circle*{0.001}}
\put(1,7){\circle*{0.001}}
\put(1,8){\circle*{0.001}}
\put(1,9){\circle*{0.001}}
\put(1,10){\circle*{0.001}}
\put(1,11){\circle*{0.001}}
\put(1,12){\circle*{0.001}}
\put(1,13){\circle*{0.001}}
\put(1,14){\circle*{0.001}}
\put(1,15){\circle*{0.001}}
\put(1,16){\circle*{0.001}}
\put(1,17){\circle*{0.001}}
\put(1,18){\circle*{0.001}}
\put(1,19){\circle*{0.001}}
\put(1,20){\circle*{0.001}}
\put(1,21){\circle*{0.001}}
\put(1,22){\circle*{0.001}}
\put(1,23){\circle*{0.001}}
\put(1,24){\circle*{0.001}}
\put(1,25){\circle*{0.001}}
\put(1,26){\circle*{0.001}}
\put(1,27){\circle*{0.001}}
\put(1,28){\circle*{0.001}}
\put(1,29){\circle*{0.001}}
\put(1,30){\circle*{0.001}}
\put(2,0){\circle*{0.001}}
\put(2,1){\circle*{0.001}}
\put(2,2){\circle*{0.001}}
\put(2,3){\circle*{0.001}}
\put(2,4){\circle*{0.001}}
\put(2,5){\circle*{0.001}}
\put(2,6){\circle*{0.001}}
\put(2,7){\circle*{0.001}}
\put(2,8){\circle*{0.001}}
\put(2,9){\circle*{0.001}}
\put(2,10){\circle*{0.001}}
\put(2,11){\circle*{0.001}}
\put(2,12){\circle*{0.001}}
\put(2,13){\circle*{0.001}}
\put(2,14){\circle*{0.001}}
\put(2,15){\circle*{0.001}}
\put(2,16){\circle*{0.001}}
\put(2,17){\circle*{0.001}}
\put(2,18){\circle*{0.001}}
\put(2,19){\circle*{0.001}}
\put(2,20){\circle*{0.001}}
\put(2,21){\circle*{0.001}}
\put(2,22){\circle*{0.001}}
\put(2,23){\circle*{0.001}}
\put(2,24){\circle*{0.001}}
\put(2,25){\circle*{0.001}}
\put(2,26){\circle*{0.001}}
\put(2,27){\circle*{0.001}}
\put(2,28){\circle*{0.001}}
\put(2,29){\circle*{0.001}}
\put(2,30){\circle*{0.001}}
\put(3,0){\circle*{0.001}}
\put(3,1){\circle*{0.001}}
\put(3,2){\circle*{0.001}}
\put(3,3){\circle*{0.001}}
\put(3,4){\circle*{0.001}}
\put(3,5){\circle*{0.001}}
\put(3,6){\circle*{0.001}}
\put(3,7){\circle*{0.001}}
\put(3,8){\circle*{0.001}}
\put(3,9){\circle*{0.001}}
\put(3,10){\circle*{0.001}}
\put(3,11){\circle*{0.001}}
\put(3,12){\circle*{0.001}}
\put(3,13){\circle*{0.001}}
\put(3,14){\circle*{0.001}}
\put(3,15){\circle*{0.001}}
\put(3,16){\circle*{0.001}}
\put(3,17){\circle*{0.001}}
\put(3,18){\circle*{0.001}}
\put(3,19){\circle*{0.001}}
\put(3,20){\circle*{0.001}}
\put(3,21){\circle*{0.001}}
\put(3,22){\circle*{0.001}}
\put(3,23){\circle*{0.001}}
\put(3,24){\circle*{0.001}}
\put(3,25){\circle*{0.001}}
\put(3,26){\circle*{0.001}}
\put(3,27){\circle*{0.001}}
\put(3,28){\circle*{0.001}}
\put(3,29){\circle*{0.001}}
\put(3,30){\circle*{0.001}}
\put(4,0){\circle*{0.001}}
\put(4,1){\circle*{0.001}}
\put(4,2){\circle*{0.001}}
\put(4,3){\circle*{0.001}}
\put(4,4){\circle*{0.001}}
\put(4,5){\circle*{0.001}}
\put(4,6){\circle*{0.001}}
\put(4,7){\circle*{0.001}}
\put(4,8){\circle*{0.001}}
\put(4,9){\circle*{0.001}}
\put(4,10){\circle*{0.001}}
\put(4,11){\circle*{0.001}}
\put(4,12){\circle*{0.001}}
\put(4,13){\circle*{0.001}}
\put(4,14){\circle*{0.001}}
\put(4,15){\circle*{0.001}}
\put(4,16){\circle*{0.001}}
\put(4,17){\circle*{0.001}}
\put(4,18){\circle*{0.001}}
\put(4,19){\circle*{0.001}}
\put(4,20){\circle*{0.001}}
\put(4,21){\circle*{0.001}}
\put(4,22){\circle*{0.001}}
\put(4,23){\circle*{0.001}}
\put(4,24){\circle*{0.001}}
\put(4,25){\circle*{0.001}}
\put(4,26){\circle*{0.001}}
\put(4,27){\circle*{0.001}}
\put(4,28){\circle*{0.001}}
\put(4,29){\circle*{0.001}}
\put(4,30){\circle*{0.001}}
\put(5,0){\circle*{0.001}}
\put(5,1){\circle*{0.001}}
\put(5,2){\circle*{0.001}}
\put(5,3){\circle*{0.001}}
\put(5,4){\circle*{0.001}}
\put(5,5){\circle*{0.001}}
\put(5,6){\circle*{0.001}}
\put(5,7){\circle*{0.001}}
\put(5,8){\circle*{0.001}}
\put(5,9){\circle*{0.001}}
\put(5,10){\circle*{0.001}}
\put(5,11){\circle*{0.001}}
\put(5,12){\circle*{0.001}}
\put(5,13){\circle*{0.001}}
\put(5,14){\circle*{0.001}}
\put(5,15){\circle*{0.001}}
\put(5,16){\circle*{0.001}}
\put(5,17){\circle*{0.001}}
\put(5,18){\circle*{0.001}}
\put(5,19){\circle*{0.001}}
\put(5,20){\circle*{0.001}}
\put(5,21){\circle*{0.001}}
\put(5,22){\circle*{0.001}}
\put(5,23){\circle*{0.001}}
\put(5,24){\circle*{0.001}}
\put(5,25){\circle*{0.001}}
\put(5,26){\circle*{0.001}}
\put(5,27){\circle*{0.001}}
\put(5,28){\circle*{0.001}}
\put(5,29){\circle*{0.001}}
\put(5,30){\circle*{0.001}}
\put(6,0){\circle*{0.001}}
\put(6,1){\circle*{0.001}}
\put(6,2){\circle*{0.001}}
\put(6,3){\circle*{0.001}}
\put(6,4){\circle*{0.001}}
\put(6,5){\circle*{0.001}}
\put(6,6){\circle*{0.001}}
\put(6,7){\circle*{0.001}}
\put(6,8){\circle*{0.001}}
\put(6,9){\circle*{0.001}}
\put(6,10){\circle*{0.001}}
\put(6,11){\circle*{0.001}}
\put(6,12){\circle*{0.001}}
\put(6,13){\circle*{0.001}}
\put(6,14){\circle*{0.001}}
\put(6,15){\circle*{0.001}}
\put(6,16){\circle*{0.001}}
\put(6,17){\circle*{0.001}}
\put(6,18){\circle*{0.001}}
\put(6,19){\circle*{0.001}}
\put(6,20){\circle*{0.001}}
\put(6,21){\circle*{0.001}}
\put(6,22){\circle*{0.001}}
\put(6,23){\circle*{0.001}}
\put(6,24){\circle*{0.001}}
\put(6,25){\circle*{0.001}}
\put(6,26){\circle*{0.001}}
\put(6,27){\circle*{0.001}}
\put(6,28){\circle*{0.001}}
\put(6,29){\circle*{0.001}}
\put(6,30){\circle*{0.001}}
\put(7,0){\circle*{0.001}}
\put(7,1){\circle*{0.001}}
\put(7,2){\circle*{0.001}}
\put(7,3){\circle*{0.001}}
\put(7,4){\circle*{0.001}}
\put(7,5){\circle*{0.001}}
\put(7,6){\circle*{0.001}}
\put(7,7){\circle*{0.001}}
\put(7,8){\circle*{0.001}}
\put(7,9){\circle*{0.001}}
\put(7,10){\circle*{0.001}}
\put(7,11){\circle*{0.001}}
\put(7,12){\circle*{0.001}}
\put(7,13){\circle*{0.001}}
\put(7,14){\circle*{0.001}}
\put(7,15){\circle*{0.001}}
\put(7,16){\circle*{0.001}}
\put(7,17){\circle*{0.001}}
\put(7,18){\circle*{0.001}}
\put(7,19){\circle*{0.001}}
\put(7,20){\circle*{0.001}}
\put(7,21){\circle*{0.001}}
\put(7,22){\circle*{0.001}}
\put(7,23){\circle*{0.001}}
\put(7,24){\circle*{0.001}}
\put(7,25){\circle*{0.001}}
\put(7,26){\circle*{0.001}}
\put(7,27){\circle*{0.001}}
\put(7,28){\circle*{0.001}}
\put(7,29){\circle*{0.001}}
\put(7,30){\circle*{0.001}}
\put(8,0){\circle*{0.001}}
\put(8,1){\circle*{0.001}}
\put(8,2){\circle*{0.001}}
\put(8,3){\circle*{0.001}}
\put(8,4){\circle*{0.001}}
\put(8,5){\circle*{0.001}}
\put(8,6){\circle*{0.001}}
\put(8,7){\circle*{0.001}}
\put(8,8){\circle*{0.001}}
\put(8,9){\circle*{0.001}}
\put(8,10){\circle*{0.001}}
\put(8,11){\circle*{0.001}}
\put(8,12){\circle*{0.001}}
\put(8,13){\circle*{0.001}}
\put(8,14){\circle*{0.001}}
\put(8,15){\circle*{0.001}}
\put(8,16){\circle*{0.001}}
\put(8,17){\circle*{0.001}}
\put(8,18){\circle*{0.001}}
\put(8,19){\circle*{0.001}}
\put(8,20){\circle*{0.001}}
\put(8,21){\circle*{0.001}}
\put(8,22){\circle*{0.001}}
\put(8,23){\circle*{0.001}}
\put(8,24){\circle*{0.001}}
\put(8,25){\circle*{0.001}}
\put(8,26){\circle*{0.001}}
\put(8,27){\circle*{0.001}}
\put(8,28){\circle*{0.001}}
\put(8,29){\circle*{0.001}}
\put(8,30){\circle*{0.001}}
\put(9,0){\circle*{0.001}}
\put(9,1){\circle*{0.001}}
\put(9,2){\circle*{0.001}}
\put(9,3){\circle*{0.001}}
\put(9,4){\circle*{0.001}}
\put(9,5){\circle*{0.001}}
\put(9,6){\circle*{0.001}}
\put(9,7){\circle*{0.001}}
\put(9,8){\circle*{0.001}}
\put(9,9){\circle*{0.001}}
\put(9,10){\circle*{0.001}}
\put(9,11){\circle*{0.001}}
\put(9,12){\circle*{0.001}}
\put(9,13){\circle*{0.001}}
\put(9,14){\circle*{0.001}}
\put(9,15){\circle*{0.001}}
\put(9,16){\circle*{0.001}}
\put(9,17){\circle*{0.001}}
\put(9,18){\circle*{0.001}}
\put(9,19){\circle*{0.001}}
\put(9,20){\circle*{0.001}}
\put(9,21){\circle*{0.001}}
\put(9,22){\circle*{0.001}}
\put(9,23){\circle*{0.001}}
\put(9,24){\circle*{0.001}}
\put(9,25){\circle*{0.001}}
\put(9,26){\circle*{0.001}}
\put(9,27){\circle*{0.001}}
\put(9,28){\circle*{0.001}}
\put(9,29){\circle*{0.001}}
\put(9,30){\circle*{0.001}}
\put(10,0){\circle*{0.001}}
\put(10,1){\circle*{0.001}}
\put(10,2){\circle*{0.001}}
\put(10,3){\circle*{0.001}}
\put(10,4){\circle*{0.001}}
\put(10,5){\circle*{0.001}}
\put(10,6){\circle*{0.001}}
\put(10,7){\circle*{0.001}}
\put(10,8){\circle*{0.001}}
\put(10,9){\circle*{0.001}}
\put(10,10){\circle*{0.001}}
\put(10,11){\circle*{0.001}}
\put(10,12){\circle*{0.001}}
\put(10,13){\circle*{0.001}}
\put(10,14){\circle*{0.001}}
\put(10,15){\circle*{0.001}}
\put(10,16){\circle*{0.001}}
\put(10,17){\circle*{0.001}}
\put(10,18){\circle*{0.001}}
\put(10,19){\circle*{0.001}}
\put(10,20){\circle*{0.001}}
\put(10,21){\circle*{0.001}}
\put(10,22){\circle*{0.001}}
\put(10,23){\circle*{0.001}}
\put(10,24){\circle*{0.001}}
\put(10,25){\circle*{0.001}}
\put(10,26){\circle*{0.001}}
\put(10,27){\circle*{0.001}}
\put(10,28){\circle*{0.001}}
\put(10,29){\circle*{0.001}}
\put(10,30){\circle*{0.001}}
\put(11,0){\circle*{0.001}}
\put(11,1){\circle*{0.001}}
\put(11,2){\circle*{0.001}}
\put(11,3){\circle*{0.001}}
\put(11,4){\circle*{0.001}}
\put(11,5){\circle*{0.001}}
\put(11,6){\circle*{0.001}}
\put(11,7){\circle*{0.001}}
\put(11,8){\circle*{0.001}}
\put(11,9){\circle*{0.001}}
\put(11,10){\circle*{0.001}}
\put(11,11){\circle*{0.001}}
\put(11,12){\circle*{0.001}}
\put(11,13){\circle*{0.001}}
\put(11,14){\circle*{0.001}}
\put(11,15){\circle*{0.001}}
\put(11,16){\circle*{0.001}}
\put(11,17){\circle*{0.001}}
\put(11,18){\circle*{0.001}}
\put(11,19){\circle*{0.001}}
\put(11,20){\circle*{0.001}}
\put(11,21){\circle*{0.001}}
\put(11,22){\circle*{0.001}}
\put(11,23){\circle*{0.001}}
\put(11,24){\circle*{0.001}}
\put(11,25){\circle*{0.001}}
\put(11,26){\circle*{0.001}}
\put(11,27){\circle*{0.001}}
\put(11,28){\circle*{0.001}}
\put(11,29){\circle*{0.001}}
\put(11,30){\circle*{0.001}}
\put(12,0){\circle*{0.001}}
\put(12,1){\circle*{0.001}}
\put(12,2){\circle*{0.001}}
\put(12,3){\circle*{0.001}}
\put(12,4){\circle*{0.001}}
\put(12,5){\circle*{0.001}}
\put(12,6){\circle*{0.001}}
\put(12,7){\circle*{0.001}}
\put(12,8){\circle*{0.001}}
\put(12,9){\circle*{0.001}}
\put(12,10){\circle*{0.001}}
\put(12,11){\circle*{0.001}}
\put(12,12){\circle*{0.001}}
\put(12,13){\circle*{0.001}}
\put(12,14){\circle*{0.001}}
\put(12,15){\circle*{0.001}}
\put(12,16){\circle*{0.001}}
\put(12,17){\circle*{0.001}}
\put(12,18){\circle*{0.001}}
\put(12,19){\circle*{0.001}}
\put(12,20){\circle*{0.001}}
\put(12,21){\circle*{0.001}}
\put(12,22){\circle*{0.001}}
\put(12,23){\circle*{0.001}}
\put(12,24){\circle*{0.001}}
\put(12,25){\circle*{0.001}}
\put(12,26){\circle*{0.001}}
\put(12,27){\circle*{0.001}}
\put(12,28){\circle*{0.001}}
\put(12,29){\circle*{0.001}}
\put(12,30){\circle*{0.001}}
\put(13,0){\circle*{0.001}}
\put(13,1){\circle*{0.001}}
\put(13,2){\circle*{0.001}}
\put(13,3){\circle*{0.001}}
\put(13,4){\circle*{0.001}}
\put(13,5){\circle*{0.001}}
\put(13,6){\circle*{0.001}}
\put(13,7){\circle*{0.001}}
\put(13,8){\circle*{0.001}}
\put(13,9){\circle*{0.001}}
\put(13,10){\circle*{0.001}}
\put(13,11){\circle*{0.001}}
\put(13,12){\circle*{0.001}}
\put(13,13){\circle*{0.001}}
\put(13,14){\circle*{0.001}}
\put(13,15){\circle*{0.001}}
\put(13,16){\circle*{0.001}}
\put(13,17){\circle*{0.001}}
\put(13,18){\circle*{0.001}}
\put(13,19){\circle*{0.001}}
\put(13,20){\circle*{0.001}}
\put(13,21){\circle*{0.001}}
\put(13,22){\circle*{0.001}}
\put(13,23){\circle*{0.001}}
\put(13,24){\circle*{0.001}}
\put(13,25){\circle*{0.001}}
\put(13,26){\circle*{0.001}}
\put(13,27){\circle*{0.001}}
\put(13,28){\circle*{0.001}}
\put(13,29){\circle*{0.001}}
\put(13,30){\circle*{0.001}}
\put(14,0){\circle*{0.001}}
\put(14,1){\circle*{0.001}}
\put(14,2){\circle*{0.001}}
\put(14,3){\circle*{0.001}}
\put(14,4){\circle*{0.001}}
\put(14,5){\circle*{0.001}}
\put(14,6){\circle*{0.001}}
\put(14,7){\circle*{0.001}}
\put(14,8){\circle*{0.001}}
\put(14,9){\circle*{0.001}}
\put(14,10){\circle*{0.001}}
\put(14,11){\circle*{0.001}}
\put(14,12){\circle*{0.001}}
\put(14,13){\circle*{0.001}}
\put(14,14){\circle*{0.001}}
\put(14,15){\circle*{0.001}}
\put(14,16){\circle*{0.001}}
\put(14,17){\circle*{0.001}}
\put(14,18){\circle*{0.001}}
\put(14,19){\circle*{0.001}}
\put(14,20){\circle*{0.001}}
\put(14,21){\circle*{0.001}}
\put(14,22){\circle*{0.001}}
\put(14,23){\circle*{0.001}}
\put(14,24){\circle*{0.001}}
\put(14,25){\circle*{0.001}}
\put(14,26){\circle*{0.001}}
\put(14,27){\circle*{0.001}}
\put(14,28){\circle*{0.001}}
\put(14,29){\circle*{0.001}}
\put(14,30){\circle*{0.001}}
\put(15,0){\circle*{0.001}}
\put(15,1){\circle*{0.001}}
\put(15,2){\circle*{0.001}}
\put(15,3){\circle*{0.001}}
\put(15,4){\circle*{0.001}}
\put(15,5){\circle*{0.001}}
\put(15,6){\circle*{0.001}}
\put(15,7){\circle*{0.001}}
\put(15,8){\circle*{0.001}}
\put(15,9){\circle*{0.001}}
\put(15,10){\circle*{0.001}}
\put(15,11){\circle*{0.001}}
\put(15,12){\circle*{0.001}}
\put(15,13){\circle*{0.001}}
\put(15,14){\circle*{0.001}}
\put(15,15){\circle*{0.001}}
\put(15,16){\circle*{0.001}}
\put(15,17){\circle*{0.001}}
\put(15,18){\circle*{0.001}}
\put(15,19){\circle*{0.001}}
\put(15,20){\circle*{0.001}}
\put(15,21){\circle*{0.001}}
\put(15,22){\circle*{0.001}}
\put(15,23){\circle*{0.001}}
\put(15,24){\circle*{0.001}}
\put(15,25){\circle*{0.001}}
\put(15,26){\circle*{0.001}}
\put(15,27){\circle*{0.001}}
\put(15,28){\circle*{0.001}}
\put(15,29){\circle*{0.001}}
\put(15,30){\circle*{0.001}}
\put(16,0){\circle*{0.001}}
\put(16,1){\circle*{0.001}}
\put(16,2){\circle*{0.001}}
\put(16,3){\circle*{0.001}}
\put(16,4){\circle*{0.001}}
\put(16,5){\circle*{0.001}}
\put(16,6){\circle*{0.001}}
\put(16,7){\circle*{0.001}}
\put(16,8){\circle*{0.001}}
\put(16,9){\circle*{0.001}}
\put(16,10){\circle*{0.001}}
\put(16,11){\circle*{0.001}}
\put(16,12){\circle*{0.001}}
\put(16,13){\circle*{0.001}}
\put(16,14){\circle*{0.001}}
\put(16,15){\circle*{0.001}}
\put(16,16){\circle*{0.001}}
\put(16,17){\circle*{0.001}}
\put(16,18){\circle*{0.001}}
\put(16,19){\circle*{0.001}}
\put(16,20){\circle*{0.001}}
\put(16,21){\circle*{0.001}}
\put(16,22){\circle*{0.001}}
\put(16,23){\circle*{0.001}}
\put(16,24){\circle*{0.001}}
\put(16,25){\circle*{0.001}}
\put(16,26){\circle*{0.001}}
\put(16,27){\circle*{0.001}}
\put(16,28){\circle*{0.001}}
\put(16,29){\circle*{0.001}}
\put(16,30){\circle*{0.001}}
\put(17,0){\circle*{0.001}}
\put(17,1){\circle*{0.001}}
\put(17,2){\circle*{0.001}}
\put(17,3){\circle*{0.001}}
\put(17,4){\circle*{0.001}}
\put(17,5){\circle*{0.001}}
\put(17,6){\circle*{0.001}}
\put(17,7){\circle*{0.001}}
\put(17,8){\circle*{0.001}}
\put(17,9){\circle*{0.001}}
\put(17,10){\circle*{0.001}}
\put(17,11){\circle*{0.001}}
\put(17,12){\circle*{0.001}}
\put(17,13){\circle*{0.001}}
\put(17,14){\circle*{0.001}}
\put(17,15){\circle*{0.001}}
\put(17,16){\circle*{0.001}}
\put(17,17){\circle*{0.001}}
\put(17,18){\circle*{0.001}}
\put(17,19){\circle*{0.001}}
\put(17,20){\circle*{0.001}}
\put(17,21){\circle*{0.001}}
\put(17,22){\circle*{0.001}}
\put(17,23){\circle*{0.001}}
\put(17,24){\circle*{0.001}}
\put(17,25){\circle*{0.001}}
\put(17,26){\circle*{0.001}}
\put(17,27){\circle*{0.001}}
\put(17,28){\circle*{0.001}}
\put(17,29){\circle*{0.001}}
\put(17,30){\circle*{0.001}}
\put(18,0){\circle*{0.001}}
\put(18,1){\circle*{0.001}}
\put(18,2){\circle*{0.001}}
\put(18,3){\circle*{0.001}}
\put(18,4){\circle*{0.001}}
\put(18,5){\circle*{0.001}}
\put(18,6){\circle*{0.001}}
\put(18,7){\circle*{0.001}}
\put(18,8){\circle*{0.001}}
\put(18,9){\circle*{0.001}}
\put(18,10){\circle*{0.001}}
\put(18,11){\circle*{0.001}}
\put(18,12){\circle*{0.001}}
\put(18,13){\circle*{0.001}}
\put(18,14){\circle*{0.001}}
\put(18,15){\circle*{0.001}}
\put(18,16){\circle*{0.001}}
\put(18,17){\circle*{0.001}}
\put(18,18){\circle*{0.001}}
\put(18,19){\circle*{0.001}}
\put(18,20){\circle*{0.001}}
\put(18,21){\circle*{0.001}}
\put(18,22){\circle*{0.001}}
\put(18,23){\circle*{0.001}}
\put(18,24){\circle*{0.001}}
\put(18,25){\circle*{0.001}}
\put(18,26){\circle*{0.001}}
\put(18,27){\circle*{0.001}}
\put(18,28){\circle*{0.001}}
\put(18,29){\circle*{0.001}}
\put(18,30){\circle*{0.001}}
\put(19,0){\circle*{0.001}}
\put(19,1){\circle*{0.001}}
\put(19,2){\circle*{0.001}}
\put(19,3){\circle*{0.001}}
\put(19,4){\circle*{0.001}}
\put(19,5){\circle*{0.001}}
\put(19,6){\circle*{0.001}}
\put(19,7){\circle*{0.001}}
\put(19,8){\circle*{0.001}}
\put(19,9){\circle*{0.001}}
\put(19,10){\circle*{0.001}}
\put(19,11){\circle*{0.001}}
\put(19,12){\circle*{0.001}}
\put(19,13){\circle*{0.001}}
\put(19,14){\circle*{0.001}}
\put(19,15){\circle*{0.001}}
\put(19,16){\circle*{0.001}}
\put(19,17){\circle*{0.001}}
\put(19,18){\circle*{0.001}}
\put(19,19){\circle*{0.001}}
\put(19,20){\circle*{0.001}}
\put(19,21){\circle*{0.001}}
\put(19,22){\circle*{0.001}}
\put(19,23){\circle*{0.001}}
\put(19,24){\circle*{0.001}}
\put(19,25){\circle*{0.001}}
\put(19,26){\circle*{0.001}}
\put(19,27){\circle*{0.001}}
\put(19,28){\circle*{0.001}}
\put(19,29){\circle*{0.001}}
\put(19,30){\circle*{0.001}}
\put(20,0){\circle*{0.001}}
\put(20,1){\circle*{0.001}}
\put(20,2){\circle*{0.001}}
\put(20,3){\circle*{0.001}}
\put(20,4){\circle*{0.001}}
\put(20,5){\circle*{0.001}}
\put(20,6){\circle*{0.001}}
\put(20,7){\circle*{0.001}}
\put(20,8){\circle*{0.001}}
\put(20,9){\circle*{0.001}}
\put(20,10){\circle*{0.001}}
\put(20,11){\circle*{0.001}}
\put(20,12){\circle*{0.001}}
\put(20,13){\circle*{0.001}}
\put(20,14){\circle*{0.001}}
\put(20,15){\circle*{0.001}}
\put(20,16){\circle*{0.001}}
\put(20,17){\circle*{0.001}}
\put(20,18){\circle*{0.001}}
\put(20,19){\circle*{0.001}}
\put(20,20){\circle*{0.001}}
\put(20,21){\circle*{0.001}}
\put(20,22){\circle*{0.001}}
\put(20,23){\circle*{0.001}}
\put(20,24){\circle*{0.001}}
\put(20,25){\circle*{0.001}}
\put(20,26){\circle*{0.001}}
\put(20,27){\circle*{0.001}}
\put(20,28){\circle*{0.001}}
\put(20,29){\circle*{0.001}}
\put(20,30){\circle*{0.001}}
\put(21,0){\circle*{0.001}}
\put(21,1){\circle*{0.001}}
\put(21,2){\circle*{0.001}}
\put(21,3){\circle*{0.001}}
\put(21,4){\circle*{0.001}}
\put(21,5){\circle*{0.001}}
\put(21,6){\circle*{0.001}}
\put(21,7){\circle*{0.001}}
\put(21,8){\circle*{0.001}}
\put(21,9){\circle*{0.001}}
\put(21,10){\circle*{0.001}}
\put(21,11){\circle*{0.001}}
\put(21,12){\circle*{0.001}}
\put(21,13){\circle*{0.001}}
\put(21,14){\circle*{0.001}}
\put(21,15){\circle*{0.001}}
\put(21,16){\circle*{0.001}}
\put(21,17){\circle*{0.001}}
\put(21,18){\circle*{0.001}}
\put(21,19){\circle*{0.001}}
\put(21,20){\circle*{0.001}}
\put(21,21){\circle*{0.001}}
\put(21,22){\circle*{0.001}}
\put(21,23){\circle*{0.001}}
\put(21,24){\circle*{0.001}}
\put(21,25){\circle*{0.001}}
\put(21,26){\circle*{0.001}}
\put(21,27){\circle*{0.001}}
\put(21,28){\circle*{0.001}}
\put(21,29){\circle*{0.001}}
\put(21,30){\circle*{0.001}}
\put(22,0){\circle*{0.001}}
\put(22,1){\circle*{0.001}}
\put(22,2){\circle*{0.001}}
\put(22,3){\circle*{0.001}}
\put(22,4){\circle*{0.001}}
\put(22,5){\circle*{0.001}}
\put(22,6){\circle*{0.001}}
\put(22,7){\circle*{0.001}}
\put(22,8){\circle*{0.001}}
\put(22,9){\circle*{0.001}}
\put(22,10){\circle*{0.001}}
\put(22,11){\circle*{0.001}}
\put(22,12){\circle*{0.001}}
\put(22,13){\circle*{0.001}}
\put(22,14){\circle*{0.001}}
\put(22,15){\circle*{0.001}}
\put(22,16){\circle*{0.001}}
\put(22,17){\circle*{0.001}}
\put(22,18){\circle*{0.001}}
\put(22,19){\circle*{0.001}}
\put(22,20){\circle*{0.001}}
\put(22,21){\circle*{0.001}}
\put(22,22){\circle*{0.001}}
\put(22,23){\circle*{0.001}}
\put(22,24){\circle*{0.001}}
\put(22,25){\circle*{0.001}}
\put(22,26){\circle*{0.001}}
\put(22,27){\circle*{0.001}}
\put(22,28){\circle*{0.001}}
\put(22,29){\circle*{0.001}}
\put(22,30){\circle*{0.001}}
\put(23,0){\circle*{0.001}}
\put(23,1){\circle*{0.001}}
\put(23,2){\circle*{0.001}}
\put(23,3){\circle*{0.001}}
\put(23,4){\circle*{0.001}}
\put(23,5){\circle*{0.001}}
\put(23,6){\circle*{0.001}}
\put(23,7){\circle*{0.001}}
\put(23,8){\circle*{0.001}}
\put(23,9){\circle*{0.001}}
\put(23,10){\circle*{0.001}}
\put(23,11){\circle*{0.001}}
\put(23,12){\circle*{0.001}}
\put(23,13){\circle*{0.001}}
\put(23,14){\circle*{0.001}}
\put(23,15){\circle*{0.001}}
\put(23,16){\circle*{0.001}}
\put(23,17){\circle*{0.001}}
\put(23,18){\circle*{0.001}}
\put(23,19){\circle*{0.001}}
\put(23,20){\circle*{0.001}}
\put(23,21){\circle*{0.001}}
\put(23,22){\circle*{0.001}}
\put(23,23){\circle*{0.001}}
\put(23,24){\circle*{0.001}}
\put(23,25){\circle*{0.001}}
\put(23,26){\circle*{0.001}}
\put(23,27){\circle*{0.001}}
\put(23,28){\circle*{0.001}}
\put(23,29){\circle*{0.001}}
\put(23,30){\circle*{0.001}}
\put(24,0){\circle*{0.001}}
\put(24,1){\circle*{0.001}}
\put(24,2){\circle*{0.001}}
\put(24,3){\circle*{0.001}}
\put(24,4){\circle*{0.001}}
\put(24,5){\circle*{0.001}}
\put(24,6){\circle*{0.001}}
\put(24,7){\circle*{0.001}}
\put(24,8){\circle*{0.001}}
\put(24,9){\circle*{0.001}}
\put(24,10){\circle*{0.001}}
\put(24,11){\circle*{0.001}}
\put(24,12){\circle*{0.001}}
\put(24,13){\circle*{0.001}}
\put(24,14){\circle*{0.001}}
\put(24,15){\circle*{0.001}}
\put(24,16){\circle*{0.001}}
\put(24,17){\circle*{0.001}}
\put(24,18){\circle*{0.001}}
\put(24,19){\circle*{0.001}}
\put(24,20){\circle*{0.001}}
\put(24,21){\circle*{0.001}}
\put(24,22){\circle*{0.001}}
\put(24,23){\circle*{0.001}}
\put(24,24){\circle*{0.001}}
\put(24,25){\circle*{0.001}}
\put(24,26){\circle*{0.001}}
\put(24,27){\circle*{0.001}}
\put(24,28){\circle*{0.001}}
\put(24,29){\circle*{0.001}}
\put(24,30){\circle*{0.001}}
\put(25,0){\circle*{0.001}}
\put(25,1){\circle*{0.001}}
\put(25,2){\circle*{0.001}}
\put(25,3){\circle*{0.001}}
\put(25,4){\circle*{0.001}}
\put(25,5){\circle*{0.001}}
\put(25,6){\circle*{0.001}}
\put(25,7){\circle*{0.001}}
\put(25,8){\circle*{0.001}}
\put(25,9){\circle*{0.001}}
\put(25,10){\circle*{0.001}}
\put(25,11){\circle*{0.001}}
\put(25,12){\circle*{0.001}}
\put(25,13){\circle*{0.001}}
\put(25,14){\circle*{0.001}}
\put(25,15){\circle*{0.001}}
\put(25,16){\circle*{0.001}}
\put(25,17){\circle*{0.001}}
\put(25,18){\circle*{0.001}}
\put(25,19){\circle*{0.001}}
\put(25,20){\circle*{0.001}}
\put(25,21){\circle*{0.001}}
\put(25,22){\circle*{0.001}}
\put(25,23){\circle*{0.001}}
\put(25,24){\circle*{0.001}}
\put(25,25){\circle*{0.001}}
\put(25,26){\circle*{0.001}}
\put(25,27){\circle*{0.001}}
\put(25,28){\circle*{0.001}}
\put(25,29){\circle*{0.001}}
\put(25,30){\circle*{0.001}}
\put(26,0){\circle*{0.001}}
\put(26,1){\circle*{0.001}}
\put(26,2){\circle*{0.001}}
\put(26,3){\circle*{0.001}}
\put(26,4){\circle*{0.001}}
\put(26,5){\circle*{0.001}}
\put(26,6){\circle*{0.001}}
\put(26,7){\circle*{0.001}}
\put(26,8){\circle*{0.001}}
\put(26,9){\circle*{0.001}}
\put(26,10){\circle*{0.001}}
\put(26,11){\circle*{0.001}}
\put(26,12){\circle*{0.001}}
\put(26,13){\circle*{0.001}}
\put(26,14){\circle*{0.001}}
\put(26,15){\circle*{0.001}}
\put(26,16){\circle*{0.001}}
\put(26,17){\circle*{0.001}}
\put(26,18){\circle*{0.001}}
\put(26,19){\circle*{0.001}}
\put(26,20){\circle*{0.001}}
\put(26,21){\circle*{0.001}}
\put(26,22){\circle*{0.001}}
\put(26,23){\circle*{0.001}}
\put(26,24){\circle*{0.001}}
\put(26,25){\circle*{0.001}}
\put(26,26){\circle*{0.001}}
\put(26,27){\circle*{0.001}}
\put(26,28){\circle*{0.001}}
\put(26,29){\circle*{0.001}}
\put(26,30){\circle*{0.001}}
\put(27,0){\circle*{0.001}}
\put(27,1){\circle*{0.001}}
\put(27,2){\circle*{0.001}}
\put(27,3){\circle*{0.001}}
\put(27,4){\circle*{0.001}}
\put(27,5){\circle*{0.001}}
\put(27,6){\circle*{0.001}}
\put(27,7){\circle*{0.001}}
\put(27,8){\circle*{0.001}}
\put(27,9){\circle*{0.001}}
\put(27,10){\circle*{0.001}}
\put(27,11){\circle*{0.001}}
\put(27,12){\circle*{0.001}}
\put(27,13){\circle*{0.001}}
\put(27,14){\circle*{0.001}}
\put(27,15){\circle*{0.001}}
\put(27,16){\circle*{0.001}}
\put(27,17){\circle*{0.001}}
\put(27,18){\circle*{0.001}}
\put(27,19){\circle*{0.001}}
\put(27,20){\circle*{0.001}}
\put(27,21){\circle*{0.001}}
\put(27,22){\circle*{0.001}}
\put(27,23){\circle*{0.001}}
\put(27,24){\circle*{0.001}}
\put(27,25){\circle*{0.001}}
\put(27,26){\circle*{0.001}}
\put(27,27){\circle*{0.001}}
\put(27,28){\circle*{0.001}}
\put(27,29){\circle*{0.001}}
\put(27,30){\circle*{0.001}}
\put(28,0){\circle*{0.001}}
\put(28,1){\circle*{0.001}}
\put(28,2){\circle*{0.001}}
\put(28,3){\circle*{0.001}}
\put(28,4){\circle*{0.001}}
\put(28,5){\circle*{0.001}}
\put(28,6){\circle*{0.001}}
\put(28,7){\circle*{0.001}}
\put(28,8){\circle*{0.001}}
\put(28,9){\circle*{0.001}}
\put(28,10){\circle*{0.001}}
\put(28,11){\circle*{0.001}}
\put(28,12){\circle*{0.001}}
\put(28,13){\circle*{0.001}}
\put(28,14){\circle*{0.001}}
\put(28,15){\circle*{0.001}}
\put(28,16){\circle*{0.001}}
\put(28,17){\circle*{0.001}}
\put(28,18){\circle*{0.001}}
\put(28,19){\circle*{0.001}}
\put(28,20){\circle*{0.001}}
\put(28,21){\circle*{0.001}}
\put(28,22){\circle*{0.001}}
\put(28,23){\circle*{0.001}}
\put(28,24){\circle*{0.001}}
\put(28,25){\circle*{0.001}}
\put(28,26){\circle*{0.001}}
\put(28,27){\circle*{0.001}}
\put(28,28){\circle*{0.001}}
\put(28,29){\circle*{0.001}}
\put(28,30){\circle*{0.001}}
\put(29,0){\circle*{0.001}}
\put(29,1){\circle*{0.001}}
\put(29,2){\circle*{0.001}}
\put(29,3){\circle*{0.001}}
\put(29,4){\circle*{0.001}}
\put(29,5){\circle*{0.001}}
\put(29,6){\circle*{0.001}}
\put(29,7){\circle*{0.001}}
\put(29,8){\circle*{0.001}}
\put(29,9){\circle*{0.001}}
\put(29,10){\circle*{0.001}}
\put(29,11){\circle*{0.001}}
\put(29,12){\circle*{0.001}}
\put(29,13){\circle*{0.001}}
\put(29,14){\circle*{0.001}}
\put(29,15){\circle*{0.001}}
\put(29,16){\circle*{0.001}}
\put(29,17){\circle*{0.001}}
\put(29,18){\circle*{0.001}}
\put(29,19){\circle*{0.001}}
\put(29,20){\circle*{0.001}}
\put(29,21){\circle*{0.001}}
\put(29,22){\circle*{0.001}}
\put(29,23){\circle*{0.001}}
\put(29,24){\circle*{0.001}}
\put(29,25){\circle*{0.001}}
\put(29,26){\circle*{0.001}}
\put(29,27){\circle*{0.001}}
\put(29,28){\circle*{0.001}}
\put(29,29){\circle*{0.001}}
\put(29,30){\circle*{0.001}}
\put(30,0){\circle*{0.001}}
\put(30,1){\circle*{0.001}}
\put(30,2){\circle*{0.001}}
\put(30,3){\circle*{0.001}}
\put(30,4){\circle*{0.001}}
\put(30,5){\circle*{0.001}}
\put(30,6){\circle*{0.001}}
\put(30,7){\circle*{0.001}}
\put(30,8){\circle*{0.001}}
\put(30,9){\circle*{0.001}}
\put(30,10){\circle*{0.001}}
\put(30,11){\circle*{0.001}}
\put(30,12){\circle*{0.001}}
\put(30,13){\circle*{0.001}}
\put(30,14){\circle*{0.001}}
\put(30,15){\circle*{0.001}}
\put(30,16){\circle*{0.001}}
\put(30,17){\circle*{0.001}}
\put(30,18){\circle*{0.001}}
\put(30,19){\circle*{0.001}}
\put(30,20){\circle*{0.001}}
\put(30,21){\circle*{0.001}}
\put(30,22){\circle*{0.001}}
\put(30,23){\circle*{0.001}}
\put(30,24){\circle*{0.001}}
\put(30,25){\circle*{0.001}}
\put(30,26){\circle*{0.001}}
\put(30,27){\circle*{0.001}}
\put(30,28){\circle*{0.001}}
\put(30,29){\circle*{0.001}}
\put(30,30){\circle*{0.001}}
\end{picture}
  \end{minipage}\hfill
  \begin{minipage}[c]{0.46\textwidth}
    \centering\begin{picture}(30,30)\put( 0.000, 0.000){\circle*{0.001}}
\put( 0.000, 1.416){\circle*{0.001}}
\put( 0.000, 2.813){\circle*{0.001}}
\put( 0.000, 4.176){\circle*{0.001}}
\put( 0.000, 5.486){\circle*{0.001}}
\put( 0.000, 6.732){\circle*{0.001}}
\put( 0.000, 7.902){\circle*{0.001}}
\put( 0.000, 8.989){\circle*{0.001}}
\put( 0.000, 9.989){\circle*{0.001}}
\put( 0.000,10.902){\circle*{0.001}}
\put( 0.000,11.732){\circle*{0.001}}
\put( 0.000,12.486){\circle*{0.001}}
\put( 0.000,13.176){\circle*{0.001}}
\put( 0.000,13.813){\circle*{0.001}}
\put( 0.000,14.416){\circle*{0.001}}
\put( 0.000,15.000){\circle*{0.001}}
\put( 0.000,15.584){\circle*{0.001}}
\put( 0.000,16.187){\circle*{0.001}}
\put( 0.000,16.824){\circle*{0.001}}
\put( 0.000,17.514){\circle*{0.001}}
\put( 0.000,18.268){\circle*{0.001}}
\put( 0.000,19.098){\circle*{0.001}}
\put( 0.000,20.011){\circle*{0.001}}
\put( 0.000,21.011){\circle*{0.001}}
\put( 0.000,22.098){\circle*{0.001}}
\put( 0.000,23.268){\circle*{0.001}}
\put( 0.000,24.514){\circle*{0.001}}
\put( 0.000,25.824){\circle*{0.001}}
\put( 0.000,27.187){\circle*{0.001}}
\put( 0.000,28.584){\circle*{0.001}}
\put( 0.000,30.000){\circle*{0.001}}
\put( 1.416, 0.000){\circle*{0.001}}
\put( 1.416, 1.416){\circle*{0.001}}
\put( 1.416, 2.813){\circle*{0.001}}
\put( 1.416, 4.176){\circle*{0.001}}
\put( 1.416, 5.486){\circle*{0.001}}
\put( 1.416, 6.732){\circle*{0.001}}
\put( 1.416, 7.902){\circle*{0.001}}
\put( 1.416, 8.989){\circle*{0.001}}
\put( 1.416, 9.989){\circle*{0.001}}
\put( 1.416,10.902){\circle*{0.001}}
\put( 1.416,11.732){\circle*{0.001}}
\put( 1.416,12.486){\circle*{0.001}}
\put( 1.416,13.176){\circle*{0.001}}
\put( 1.416,13.813){\circle*{0.001}}
\put( 1.416,14.416){\circle*{0.001}}
\put( 1.416,15.000){\circle*{0.001}}
\put( 1.416,15.584){\circle*{0.001}}
\put( 1.416,16.187){\circle*{0.001}}
\put( 1.416,16.824){\circle*{0.001}}
\put( 1.416,17.514){\circle*{0.001}}
\put( 1.416,18.268){\circle*{0.001}}
\put( 1.416,19.098){\circle*{0.001}}
\put( 1.416,20.011){\circle*{0.001}}
\put( 1.416,21.011){\circle*{0.001}}
\put( 1.416,22.098){\circle*{0.001}}
\put( 1.416,23.268){\circle*{0.001}}
\put( 1.416,24.514){\circle*{0.001}}
\put( 1.416,25.824){\circle*{0.001}}
\put( 1.416,27.187){\circle*{0.001}}
\put( 1.416,28.584){\circle*{0.001}}
\put( 1.416,30.000){\circle*{0.001}}
\put( 2.813, 0.000){\circle*{0.001}}
\put( 2.813, 1.416){\circle*{0.001}}
\put( 2.813, 2.813){\circle*{0.001}}
\put( 2.813, 4.176){\circle*{0.001}}
\put( 2.813, 5.486){\circle*{0.001}}
\put( 2.813, 6.732){\circle*{0.001}}
\put( 2.813, 7.902){\circle*{0.001}}
\put( 2.813, 8.989){\circle*{0.001}}
\put( 2.813, 9.989){\circle*{0.001}}
\put( 2.813,10.902){\circle*{0.001}}
\put( 2.813,11.732){\circle*{0.001}}
\put( 2.813,12.486){\circle*{0.001}}
\put( 2.813,13.176){\circle*{0.001}}
\put( 2.813,13.813){\circle*{0.001}}
\put( 2.813,14.416){\circle*{0.001}}
\put( 2.813,15.000){\circle*{0.001}}
\put( 2.813,15.584){\circle*{0.001}}
\put( 2.813,16.187){\circle*{0.001}}
\put( 2.813,16.824){\circle*{0.001}}
\put( 2.813,17.514){\circle*{0.001}}
\put( 2.813,18.268){\circle*{0.001}}
\put( 2.813,19.098){\circle*{0.001}}
\put( 2.813,20.011){\circle*{0.001}}
\put( 2.813,21.011){\circle*{0.001}}
\put( 2.813,22.098){\circle*{0.001}}
\put( 2.813,23.268){\circle*{0.001}}
\put( 2.813,24.514){\circle*{0.001}}
\put( 2.813,25.824){\circle*{0.001}}
\put( 2.813,27.187){\circle*{0.001}}
\put( 2.813,28.584){\circle*{0.001}}
\put( 2.813,30.000){\circle*{0.001}}
\put( 4.176, 0.000){\circle*{0.001}}
\put( 4.176, 1.416){\circle*{0.001}}
\put( 4.176, 2.813){\circle*{0.001}}
\put( 4.176, 4.176){\circle*{0.001}}
\put( 4.176, 5.486){\circle*{0.001}}
\put( 4.176, 6.732){\circle*{0.001}}
\put( 4.176, 7.902){\circle*{0.001}}
\put( 4.176, 8.989){\circle*{0.001}}
\put( 4.176, 9.989){\circle*{0.001}}
\put( 4.176,10.902){\circle*{0.001}}
\put( 4.176,11.732){\circle*{0.001}}
\put( 4.176,12.486){\circle*{0.001}}
\put( 4.176,13.176){\circle*{0.001}}
\put( 4.176,13.813){\circle*{0.001}}
\put( 4.176,14.416){\circle*{0.001}}
\put( 4.176,15.000){\circle*{0.001}}
\put( 4.176,15.584){\circle*{0.001}}
\put( 4.176,16.187){\circle*{0.001}}
\put( 4.176,16.824){\circle*{0.001}}
\put( 4.176,17.514){\circle*{0.001}}
\put( 4.176,18.268){\circle*{0.001}}
\put( 4.176,19.098){\circle*{0.001}}
\put( 4.176,20.011){\circle*{0.001}}
\put( 4.176,21.011){\circle*{0.001}}
\put( 4.176,22.098){\circle*{0.001}}
\put( 4.176,23.268){\circle*{0.001}}
\put( 4.176,24.514){\circle*{0.001}}
\put( 4.176,25.824){\circle*{0.001}}
\put( 4.176,27.187){\circle*{0.001}}
\put( 4.176,28.584){\circle*{0.001}}
\put( 4.176,30.000){\circle*{0.001}}
\put( 5.486, 0.000){\circle*{0.001}}
\put( 5.486, 1.416){\circle*{0.001}}
\put( 5.486, 2.813){\circle*{0.001}}
\put( 5.486, 4.176){\circle*{0.001}}
\put( 5.486, 5.486){\circle*{0.001}}
\put( 5.486, 6.732){\circle*{0.001}}
\put( 5.486, 7.902){\circle*{0.001}}
\put( 5.486, 8.989){\circle*{0.001}}
\put( 5.486, 9.989){\circle*{0.001}}
\put( 5.486,10.902){\circle*{0.001}}
\put( 5.486,11.732){\circle*{0.001}}
\put( 5.486,12.486){\circle*{0.001}}
\put( 5.486,13.176){\circle*{0.001}}
\put( 5.486,13.813){\circle*{0.001}}
\put( 5.486,14.416){\circle*{0.001}}
\put( 5.486,15.000){\circle*{0.001}}
\put( 5.486,15.584){\circle*{0.001}}
\put( 5.486,16.187){\circle*{0.001}}
\put( 5.486,16.824){\circle*{0.001}}
\put( 5.486,17.514){\circle*{0.001}}
\put( 5.486,18.268){\circle*{0.001}}
\put( 5.486,19.098){\circle*{0.001}}
\put( 5.486,20.011){\circle*{0.001}}
\put( 5.486,21.011){\circle*{0.001}}
\put( 5.486,22.098){\circle*{0.001}}
\put( 5.486,23.268){\circle*{0.001}}
\put( 5.486,24.514){\circle*{0.001}}
\put( 5.486,25.824){\circle*{0.001}}
\put( 5.486,27.187){\circle*{0.001}}
\put( 5.486,28.584){\circle*{0.001}}
\put( 5.486,30.000){\circle*{0.001}}
\put( 6.732, 0.000){\circle*{0.001}}
\put( 6.732, 1.416){\circle*{0.001}}
\put( 6.732, 2.813){\circle*{0.001}}
\put( 6.732, 4.176){\circle*{0.001}}
\put( 6.732, 5.486){\circle*{0.001}}
\put( 6.732, 6.732){\circle*{0.001}}
\put( 6.732, 7.902){\circle*{0.001}}
\put( 6.732, 8.989){\circle*{0.001}}
\put( 6.732, 9.989){\circle*{0.001}}
\put( 6.732,10.902){\circle*{0.001}}
\put( 6.732,11.732){\circle*{0.001}}
\put( 6.732,12.486){\circle*{0.001}}
\put( 6.732,13.176){\circle*{0.001}}
\put( 6.732,13.813){\circle*{0.001}}
\put( 6.732,14.416){\circle*{0.001}}
\put( 6.732,15.000){\circle*{0.001}}
\put( 6.732,15.584){\circle*{0.001}}
\put( 6.732,16.187){\circle*{0.001}}
\put( 6.732,16.824){\circle*{0.001}}
\put( 6.732,17.514){\circle*{0.001}}
\put( 6.732,18.268){\circle*{0.001}}
\put( 6.732,19.098){\circle*{0.001}}
\put( 6.732,20.011){\circle*{0.001}}
\put( 6.732,21.011){\circle*{0.001}}
\put( 6.732,22.098){\circle*{0.001}}
\put( 6.732,23.268){\circle*{0.001}}
\put( 6.732,24.514){\circle*{0.001}}
\put( 6.732,25.824){\circle*{0.001}}
\put( 6.732,27.187){\circle*{0.001}}
\put( 6.732,28.584){\circle*{0.001}}
\put( 6.732,30.000){\circle*{0.001}}
\put( 7.902, 0.000){\circle*{0.001}}
\put( 7.902, 1.416){\circle*{0.001}}
\put( 7.902, 2.813){\circle*{0.001}}
\put( 7.902, 4.176){\circle*{0.001}}
\put( 7.902, 5.486){\circle*{0.001}}
\put( 7.902, 6.732){\circle*{0.001}}
\put( 7.902, 7.902){\circle*{0.001}}
\put( 7.902, 8.989){\circle*{0.001}}
\put( 7.902, 9.989){\circle*{0.001}}
\put( 7.902,10.902){\circle*{0.001}}
\put( 7.902,11.732){\circle*{0.001}}
\put( 7.902,12.486){\circle*{0.001}}
\put( 7.902,13.176){\circle*{0.001}}
\put( 7.902,13.813){\circle*{0.001}}
\put( 7.902,14.416){\circle*{0.001}}
\put( 7.902,15.000){\circle*{0.001}}
\put( 7.902,15.584){\circle*{0.001}}
\put( 7.902,16.187){\circle*{0.001}}
\put( 7.902,16.824){\circle*{0.001}}
\put( 7.902,17.514){\circle*{0.001}}
\put( 7.902,18.268){\circle*{0.001}}
\put( 7.902,19.098){\circle*{0.001}}
\put( 7.902,20.011){\circle*{0.001}}
\put( 7.902,21.011){\circle*{0.001}}
\put( 7.902,22.098){\circle*{0.001}}
\put( 7.902,23.268){\circle*{0.001}}
\put( 7.902,24.514){\circle*{0.001}}
\put( 7.902,25.824){\circle*{0.001}}
\put( 7.902,27.187){\circle*{0.001}}
\put( 7.902,28.584){\circle*{0.001}}
\put( 7.902,30.000){\circle*{0.001}}
\put( 8.989, 0.000){\circle*{0.001}}
\put( 8.989, 1.416){\circle*{0.001}}
\put( 8.989, 2.813){\circle*{0.001}}
\put( 8.989, 4.176){\circle*{0.001}}
\put( 8.989, 5.486){\circle*{0.001}}
\put( 8.989, 6.732){\circle*{0.001}}
\put( 8.989, 7.902){\circle*{0.001}}
\put( 8.989, 8.989){\circle*{0.001}}
\put( 8.989, 9.989){\circle*{0.001}}
\put( 8.989,10.902){\circle*{0.001}}
\put( 8.989,11.732){\circle*{0.001}}
\put( 8.989,12.486){\circle*{0.001}}
\put( 8.989,13.176){\circle*{0.001}}
\put( 8.989,13.813){\circle*{0.001}}
\put( 8.989,14.416){\circle*{0.001}}
\put( 8.989,15.000){\circle*{0.001}}
\put( 8.989,15.584){\circle*{0.001}}
\put( 8.989,16.187){\circle*{0.001}}
\put( 8.989,16.824){\circle*{0.001}}
\put( 8.989,17.514){\circle*{0.001}}
\put( 8.989,18.268){\circle*{0.001}}
\put( 8.989,19.098){\circle*{0.001}}
\put( 8.989,20.011){\circle*{0.001}}
\put( 8.989,21.011){\circle*{0.001}}
\put( 8.989,22.098){\circle*{0.001}}
\put( 8.989,23.268){\circle*{0.001}}
\put( 8.989,24.514){\circle*{0.001}}
\put( 8.989,25.824){\circle*{0.001}}
\put( 8.989,27.187){\circle*{0.001}}
\put( 8.989,28.584){\circle*{0.001}}
\put( 8.989,30.000){\circle*{0.001}}
\put( 9.989, 0.000){\circle*{0.001}}
\put( 9.989, 1.416){\circle*{0.001}}
\put( 9.989, 2.813){\circle*{0.001}}
\put( 9.989, 4.176){\circle*{0.001}}
\put( 9.989, 5.486){\circle*{0.001}}
\put( 9.989, 6.732){\circle*{0.001}}
\put( 9.989, 7.902){\circle*{0.001}}
\put( 9.989, 8.989){\circle*{0.001}}
\put( 9.989, 9.989){\circle*{0.001}}
\put( 9.989,10.902){\circle*{0.001}}
\put( 9.989,11.732){\circle*{0.001}}
\put( 9.989,12.486){\circle*{0.001}}
\put( 9.989,13.176){\circle*{0.001}}
\put( 9.989,13.813){\circle*{0.001}}
\put( 9.989,14.416){\circle*{0.001}}
\put( 9.989,15.000){\circle*{0.001}}
\put( 9.989,15.584){\circle*{0.001}}
\put( 9.989,16.187){\circle*{0.001}}
\put( 9.989,16.824){\circle*{0.001}}
\put( 9.989,17.514){\circle*{0.001}}
\put( 9.989,18.268){\circle*{0.001}}
\put( 9.989,19.098){\circle*{0.001}}
\put( 9.989,20.011){\circle*{0.001}}
\put( 9.989,21.011){\circle*{0.001}}
\put( 9.989,22.098){\circle*{0.001}}
\put( 9.989,23.268){\circle*{0.001}}
\put( 9.989,24.514){\circle*{0.001}}
\put( 9.989,25.824){\circle*{0.001}}
\put( 9.989,27.187){\circle*{0.001}}
\put( 9.989,28.584){\circle*{0.001}}
\put( 9.989,30.000){\circle*{0.001}}
\put(10.902, 0.000){\circle*{0.001}}
\put(10.902, 1.416){\circle*{0.001}}
\put(10.902, 2.813){\circle*{0.001}}
\put(10.902, 4.176){\circle*{0.001}}
\put(10.902, 5.486){\circle*{0.001}}
\put(10.902, 6.732){\circle*{0.001}}
\put(10.902, 7.902){\circle*{0.001}}
\put(10.902, 8.989){\circle*{0.001}}
\put(10.902, 9.989){\circle*{0.001}}
\put(10.902,10.902){\circle*{0.001}}
\put(10.902,11.732){\circle*{0.001}}
\put(10.902,12.486){\circle*{0.001}}
\put(10.902,13.176){\circle*{0.001}}
\put(10.902,13.813){\circle*{0.001}}
\put(10.902,14.416){\circle*{0.001}}
\put(10.902,15.000){\circle*{0.001}}
\put(10.902,15.584){\circle*{0.001}}
\put(10.902,16.187){\circle*{0.001}}
\put(10.902,16.824){\circle*{0.001}}
\put(10.902,17.514){\circle*{0.001}}
\put(10.902,18.268){\circle*{0.001}}
\put(10.902,19.098){\circle*{0.001}}
\put(10.902,20.011){\circle*{0.001}}
\put(10.902,21.011){\circle*{0.001}}
\put(10.902,22.098){\circle*{0.001}}
\put(10.902,23.268){\circle*{0.001}}
\put(10.902,24.514){\circle*{0.001}}
\put(10.902,25.824){\circle*{0.001}}
\put(10.902,27.187){\circle*{0.001}}
\put(10.902,28.584){\circle*{0.001}}
\put(10.902,30.000){\circle*{0.001}}
\put(11.732, 0.000){\circle*{0.001}}
\put(11.732, 1.416){\circle*{0.001}}
\put(11.732, 2.813){\circle*{0.001}}
\put(11.732, 4.176){\circle*{0.001}}
\put(11.732, 5.486){\circle*{0.001}}
\put(11.732, 6.732){\circle*{0.001}}
\put(11.732, 7.902){\circle*{0.001}}
\put(11.732, 8.989){\circle*{0.001}}
\put(11.732, 9.989){\circle*{0.001}}
\put(11.732,10.902){\circle*{0.001}}
\put(11.732,11.732){\circle*{0.001}}
\put(11.732,12.486){\circle*{0.001}}
\put(11.732,13.176){\circle*{0.001}}
\put(11.732,13.813){\circle*{0.001}}
\put(11.732,14.416){\circle*{0.001}}
\put(11.732,15.000){\circle*{0.001}}
\put(11.732,15.584){\circle*{0.001}}
\put(11.732,16.187){\circle*{0.001}}
\put(11.732,16.824){\circle*{0.001}}
\put(11.732,17.514){\circle*{0.001}}
\put(11.732,18.268){\circle*{0.001}}
\put(11.732,19.098){\circle*{0.001}}
\put(11.732,20.011){\circle*{0.001}}
\put(11.732,21.011){\circle*{0.001}}
\put(11.732,22.098){\circle*{0.001}}
\put(11.732,23.268){\circle*{0.001}}
\put(11.732,24.514){\circle*{0.001}}
\put(11.732,25.824){\circle*{0.001}}
\put(11.732,27.187){\circle*{0.001}}
\put(11.732,28.584){\circle*{0.001}}
\put(11.732,30.000){\circle*{0.001}}
\put(12.486, 0.000){\circle*{0.001}}
\put(12.486, 1.416){\circle*{0.001}}
\put(12.486, 2.813){\circle*{0.001}}
\put(12.486, 4.176){\circle*{0.001}}
\put(12.486, 5.486){\circle*{0.001}}
\put(12.486, 6.732){\circle*{0.001}}
\put(12.486, 7.902){\circle*{0.001}}
\put(12.486, 8.989){\circle*{0.001}}
\put(12.486, 9.989){\circle*{0.001}}
\put(12.486,10.902){\circle*{0.001}}
\put(12.486,11.732){\circle*{0.001}}
\put(12.486,12.486){\circle*{0.001}}
\put(12.486,13.176){\circle*{0.001}}
\put(12.486,13.813){\circle*{0.001}}
\put(12.486,14.416){\circle*{0.001}}
\put(12.486,15.000){\circle*{0.001}}
\put(12.486,15.584){\circle*{0.001}}
\put(12.486,16.187){\circle*{0.001}}
\put(12.486,16.824){\circle*{0.001}}
\put(12.486,17.514){\circle*{0.001}}
\put(12.486,18.268){\circle*{0.001}}
\put(12.486,19.098){\circle*{0.001}}
\put(12.486,20.011){\circle*{0.001}}
\put(12.486,21.011){\circle*{0.001}}
\put(12.486,22.098){\circle*{0.001}}
\put(12.486,23.268){\circle*{0.001}}
\put(12.486,24.514){\circle*{0.001}}
\put(12.486,25.824){\circle*{0.001}}
\put(12.486,27.187){\circle*{0.001}}
\put(12.486,28.584){\circle*{0.001}}
\put(12.486,30.000){\circle*{0.001}}
\put(13.176, 0.000){\circle*{0.001}}
\put(13.176, 1.416){\circle*{0.001}}
\put(13.176, 2.813){\circle*{0.001}}
\put(13.176, 4.176){\circle*{0.001}}
\put(13.176, 5.486){\circle*{0.001}}
\put(13.176, 6.732){\circle*{0.001}}
\put(13.176, 7.902){\circle*{0.001}}
\put(13.176, 8.989){\circle*{0.001}}
\put(13.176, 9.989){\circle*{0.001}}
\put(13.176,10.902){\circle*{0.001}}
\put(13.176,11.732){\circle*{0.001}}
\put(13.176,12.486){\circle*{0.001}}
\put(13.176,13.176){\circle*{0.001}}
\put(13.176,13.813){\circle*{0.001}}
\put(13.176,14.416){\circle*{0.001}}
\put(13.176,15.000){\circle*{0.001}}
\put(13.176,15.584){\circle*{0.001}}
\put(13.176,16.187){\circle*{0.001}}
\put(13.176,16.824){\circle*{0.001}}
\put(13.176,17.514){\circle*{0.001}}
\put(13.176,18.268){\circle*{0.001}}
\put(13.176,19.098){\circle*{0.001}}
\put(13.176,20.011){\circle*{0.001}}
\put(13.176,21.011){\circle*{0.001}}
\put(13.176,22.098){\circle*{0.001}}
\put(13.176,23.268){\circle*{0.001}}
\put(13.176,24.514){\circle*{0.001}}
\put(13.176,25.824){\circle*{0.001}}
\put(13.176,27.187){\circle*{0.001}}
\put(13.176,28.584){\circle*{0.001}}
\put(13.176,30.000){\circle*{0.001}}
\put(13.813, 0.000){\circle*{0.001}}
\put(13.813, 1.416){\circle*{0.001}}
\put(13.813, 2.813){\circle*{0.001}}
\put(13.813, 4.176){\circle*{0.001}}
\put(13.813, 5.486){\circle*{0.001}}
\put(13.813, 6.732){\circle*{0.001}}
\put(13.813, 7.902){\circle*{0.001}}
\put(13.813, 8.989){\circle*{0.001}}
\put(13.813, 9.989){\circle*{0.001}}
\put(13.813,10.902){\circle*{0.001}}
\put(13.813,11.732){\circle*{0.001}}
\put(13.813,12.486){\circle*{0.001}}
\put(13.813,13.176){\circle*{0.001}}
\put(13.813,13.813){\circle*{0.001}}
\put(13.813,14.416){\circle*{0.001}}
\put(13.813,15.000){\circle*{0.001}}
\put(13.813,15.584){\circle*{0.001}}
\put(13.813,16.187){\circle*{0.001}}
\put(13.813,16.824){\circle*{0.001}}
\put(13.813,17.514){\circle*{0.001}}
\put(13.813,18.268){\circle*{0.001}}
\put(13.813,19.098){\circle*{0.001}}
\put(13.813,20.011){\circle*{0.001}}
\put(13.813,21.011){\circle*{0.001}}
\put(13.813,22.098){\circle*{0.001}}
\put(13.813,23.268){\circle*{0.001}}
\put(13.813,24.514){\circle*{0.001}}
\put(13.813,25.824){\circle*{0.001}}
\put(13.813,27.187){\circle*{0.001}}
\put(13.813,28.584){\circle*{0.001}}
\put(13.813,30.000){\circle*{0.001}}
\put(14.416, 0.000){\circle*{0.001}}
\put(14.416, 1.416){\circle*{0.001}}
\put(14.416, 2.813){\circle*{0.001}}
\put(14.416, 4.176){\circle*{0.001}}
\put(14.416, 5.486){\circle*{0.001}}
\put(14.416, 6.732){\circle*{0.001}}
\put(14.416, 7.902){\circle*{0.001}}
\put(14.416, 8.989){\circle*{0.001}}
\put(14.416, 9.989){\circle*{0.001}}
\put(14.416,10.902){\circle*{0.001}}
\put(14.416,11.732){\circle*{0.001}}
\put(14.416,12.486){\circle*{0.001}}
\put(14.416,13.176){\circle*{0.001}}
\put(14.416,13.813){\circle*{0.001}}
\put(14.416,14.416){\circle*{0.001}}
\put(14.416,15.000){\circle*{0.001}}
\put(14.416,15.584){\circle*{0.001}}
\put(14.416,16.187){\circle*{0.001}}
\put(14.416,16.824){\circle*{0.001}}
\put(14.416,17.514){\circle*{0.001}}
\put(14.416,18.268){\circle*{0.001}}
\put(14.416,19.098){\circle*{0.001}}
\put(14.416,20.011){\circle*{0.001}}
\put(14.416,21.011){\circle*{0.001}}
\put(14.416,22.098){\circle*{0.001}}
\put(14.416,23.268){\circle*{0.001}}
\put(14.416,24.514){\circle*{0.001}}
\put(14.416,25.824){\circle*{0.001}}
\put(14.416,27.187){\circle*{0.001}}
\put(14.416,28.584){\circle*{0.001}}
\put(14.416,30.000){\circle*{0.001}}
\put(15.000, 0.000){\circle*{0.001}}
\put(15.000, 1.416){\circle*{0.001}}
\put(15.000, 2.813){\circle*{0.001}}
\put(15.000, 4.176){\circle*{0.001}}
\put(15.000, 5.486){\circle*{0.001}}
\put(15.000, 6.732){\circle*{0.001}}
\put(15.000, 7.902){\circle*{0.001}}
\put(15.000, 8.989){\circle*{0.001}}
\put(15.000, 9.989){\circle*{0.001}}
\put(15.000,10.902){\circle*{0.001}}
\put(15.000,11.732){\circle*{0.001}}
\put(15.000,12.486){\circle*{0.001}}
\put(15.000,13.176){\circle*{0.001}}
\put(15.000,13.813){\circle*{0.001}}
\put(15.000,14.416){\circle*{0.001}}
\put(15.000,15.000){\circle*{0.001}}
\put(15.000,15.584){\circle*{0.001}}
\put(15.000,16.187){\circle*{0.001}}
\put(15.000,16.824){\circle*{0.001}}
\put(15.000,17.514){\circle*{0.001}}
\put(15.000,18.268){\circle*{0.001}}
\put(15.000,19.098){\circle*{0.001}}
\put(15.000,20.011){\circle*{0.001}}
\put(15.000,21.011){\circle*{0.001}}
\put(15.000,22.098){\circle*{0.001}}
\put(15.000,23.268){\circle*{0.001}}
\put(15.000,24.514){\circle*{0.001}}
\put(15.000,25.824){\circle*{0.001}}
\put(15.000,27.187){\circle*{0.001}}
\put(15.000,28.584){\circle*{0.001}}
\put(15.000,30.000){\circle*{0.001}}
\put(15.584, 0.000){\circle*{0.001}}
\put(15.584, 1.416){\circle*{0.001}}
\put(15.584, 2.813){\circle*{0.001}}
\put(15.584, 4.176){\circle*{0.001}}
\put(15.584, 5.486){\circle*{0.001}}
\put(15.584, 6.732){\circle*{0.001}}
\put(15.584, 7.902){\circle*{0.001}}
\put(15.584, 8.989){\circle*{0.001}}
\put(15.584, 9.989){\circle*{0.001}}
\put(15.584,10.902){\circle*{0.001}}
\put(15.584,11.732){\circle*{0.001}}
\put(15.584,12.486){\circle*{0.001}}
\put(15.584,13.176){\circle*{0.001}}
\put(15.584,13.813){\circle*{0.001}}
\put(15.584,14.416){\circle*{0.001}}
\put(15.584,15.000){\circle*{0.001}}
\put(15.584,15.584){\circle*{0.001}}
\put(15.584,16.187){\circle*{0.001}}
\put(15.584,16.824){\circle*{0.001}}
\put(15.584,17.514){\circle*{0.001}}
\put(15.584,18.268){\circle*{0.001}}
\put(15.584,19.098){\circle*{0.001}}
\put(15.584,20.011){\circle*{0.001}}
\put(15.584,21.011){\circle*{0.001}}
\put(15.584,22.098){\circle*{0.001}}
\put(15.584,23.268){\circle*{0.001}}
\put(15.584,24.514){\circle*{0.001}}
\put(15.584,25.824){\circle*{0.001}}
\put(15.584,27.187){\circle*{0.001}}
\put(15.584,28.584){\circle*{0.001}}
\put(15.584,30.000){\circle*{0.001}}
\put(16.187, 0.000){\circle*{0.001}}
\put(16.187, 1.416){\circle*{0.001}}
\put(16.187, 2.813){\circle*{0.001}}
\put(16.187, 4.176){\circle*{0.001}}
\put(16.187, 5.486){\circle*{0.001}}
\put(16.187, 6.732){\circle*{0.001}}
\put(16.187, 7.902){\circle*{0.001}}
\put(16.187, 8.989){\circle*{0.001}}
\put(16.187, 9.989){\circle*{0.001}}
\put(16.187,10.902){\circle*{0.001}}
\put(16.187,11.732){\circle*{0.001}}
\put(16.187,12.486){\circle*{0.001}}
\put(16.187,13.176){\circle*{0.001}}
\put(16.187,13.813){\circle*{0.001}}
\put(16.187,14.416){\circle*{0.001}}
\put(16.187,15.000){\circle*{0.001}}
\put(16.187,15.584){\circle*{0.001}}
\put(16.187,16.187){\circle*{0.001}}
\put(16.187,16.824){\circle*{0.001}}
\put(16.187,17.514){\circle*{0.001}}
\put(16.187,18.268){\circle*{0.001}}
\put(16.187,19.098){\circle*{0.001}}
\put(16.187,20.011){\circle*{0.001}}
\put(16.187,21.011){\circle*{0.001}}
\put(16.187,22.098){\circle*{0.001}}
\put(16.187,23.268){\circle*{0.001}}
\put(16.187,24.514){\circle*{0.001}}
\put(16.187,25.824){\circle*{0.001}}
\put(16.187,27.187){\circle*{0.001}}
\put(16.187,28.584){\circle*{0.001}}
\put(16.187,30.000){\circle*{0.001}}
\put(16.824, 0.000){\circle*{0.001}}
\put(16.824, 1.416){\circle*{0.001}}
\put(16.824, 2.813){\circle*{0.001}}
\put(16.824, 4.176){\circle*{0.001}}
\put(16.824, 5.486){\circle*{0.001}}
\put(16.824, 6.732){\circle*{0.001}}
\put(16.824, 7.902){\circle*{0.001}}
\put(16.824, 8.989){\circle*{0.001}}
\put(16.824, 9.989){\circle*{0.001}}
\put(16.824,10.902){\circle*{0.001}}
\put(16.824,11.732){\circle*{0.001}}
\put(16.824,12.486){\circle*{0.001}}
\put(16.824,13.176){\circle*{0.001}}
\put(16.824,13.813){\circle*{0.001}}
\put(16.824,14.416){\circle*{0.001}}
\put(16.824,15.000){\circle*{0.001}}
\put(16.824,15.584){\circle*{0.001}}
\put(16.824,16.187){\circle*{0.001}}
\put(16.824,16.824){\circle*{0.001}}
\put(16.824,17.514){\circle*{0.001}}
\put(16.824,18.268){\circle*{0.001}}
\put(16.824,19.098){\circle*{0.001}}
\put(16.824,20.011){\circle*{0.001}}
\put(16.824,21.011){\circle*{0.001}}
\put(16.824,22.098){\circle*{0.001}}
\put(16.824,23.268){\circle*{0.001}}
\put(16.824,24.514){\circle*{0.001}}
\put(16.824,25.824){\circle*{0.001}}
\put(16.824,27.187){\circle*{0.001}}
\put(16.824,28.584){\circle*{0.001}}
\put(16.824,30.000){\circle*{0.001}}
\put(17.514, 0.000){\circle*{0.001}}
\put(17.514, 1.416){\circle*{0.001}}
\put(17.514, 2.813){\circle*{0.001}}
\put(17.514, 4.176){\circle*{0.001}}
\put(17.514, 5.486){\circle*{0.001}}
\put(17.514, 6.732){\circle*{0.001}}
\put(17.514, 7.902){\circle*{0.001}}
\put(17.514, 8.989){\circle*{0.001}}
\put(17.514, 9.989){\circle*{0.001}}
\put(17.514,10.902){\circle*{0.001}}
\put(17.514,11.732){\circle*{0.001}}
\put(17.514,12.486){\circle*{0.001}}
\put(17.514,13.176){\circle*{0.001}}
\put(17.514,13.813){\circle*{0.001}}
\put(17.514,14.416){\circle*{0.001}}
\put(17.514,15.000){\circle*{0.001}}
\put(17.514,15.584){\circle*{0.001}}
\put(17.514,16.187){\circle*{0.001}}
\put(17.514,16.824){\circle*{0.001}}
\put(17.514,17.514){\circle*{0.001}}
\put(17.514,18.268){\circle*{0.001}}
\put(17.514,19.098){\circle*{0.001}}
\put(17.514,20.011){\circle*{0.001}}
\put(17.514,21.011){\circle*{0.001}}
\put(17.514,22.098){\circle*{0.001}}
\put(17.514,23.268){\circle*{0.001}}
\put(17.514,24.514){\circle*{0.001}}
\put(17.514,25.824){\circle*{0.001}}
\put(17.514,27.187){\circle*{0.001}}
\put(17.514,28.584){\circle*{0.001}}
\put(17.514,30.000){\circle*{0.001}}
\put(18.268, 0.000){\circle*{0.001}}
\put(18.268, 1.416){\circle*{0.001}}
\put(18.268, 2.813){\circle*{0.001}}
\put(18.268, 4.176){\circle*{0.001}}
\put(18.268, 5.486){\circle*{0.001}}
\put(18.268, 6.732){\circle*{0.001}}
\put(18.268, 7.902){\circle*{0.001}}
\put(18.268, 8.989){\circle*{0.001}}
\put(18.268, 9.989){\circle*{0.001}}
\put(18.268,10.902){\circle*{0.001}}
\put(18.268,11.732){\circle*{0.001}}
\put(18.268,12.486){\circle*{0.001}}
\put(18.268,13.176){\circle*{0.001}}
\put(18.268,13.813){\circle*{0.001}}
\put(18.268,14.416){\circle*{0.001}}
\put(18.268,15.000){\circle*{0.001}}
\put(18.268,15.584){\circle*{0.001}}
\put(18.268,16.187){\circle*{0.001}}
\put(18.268,16.824){\circle*{0.001}}
\put(18.268,17.514){\circle*{0.001}}
\put(18.268,18.268){\circle*{0.001}}
\put(18.268,19.098){\circle*{0.001}}
\put(18.268,20.011){\circle*{0.001}}
\put(18.268,21.011){\circle*{0.001}}
\put(18.268,22.098){\circle*{0.001}}
\put(18.268,23.268){\circle*{0.001}}
\put(18.268,24.514){\circle*{0.001}}
\put(18.268,25.824){\circle*{0.001}}
\put(18.268,27.187){\circle*{0.001}}
\put(18.268,28.584){\circle*{0.001}}
\put(18.268,30.000){\circle*{0.001}}
\put(19.098, 0.000){\circle*{0.001}}
\put(19.098, 1.416){\circle*{0.001}}
\put(19.098, 2.813){\circle*{0.001}}
\put(19.098, 4.176){\circle*{0.001}}
\put(19.098, 5.486){\circle*{0.001}}
\put(19.098, 6.732){\circle*{0.001}}
\put(19.098, 7.902){\circle*{0.001}}
\put(19.098, 8.989){\circle*{0.001}}
\put(19.098, 9.989){\circle*{0.001}}
\put(19.098,10.902){\circle*{0.001}}
\put(19.098,11.732){\circle*{0.001}}
\put(19.098,12.486){\circle*{0.001}}
\put(19.098,13.176){\circle*{0.001}}
\put(19.098,13.813){\circle*{0.001}}
\put(19.098,14.416){\circle*{0.001}}
\put(19.098,15.000){\circle*{0.001}}
\put(19.098,15.584){\circle*{0.001}}
\put(19.098,16.187){\circle*{0.001}}
\put(19.098,16.824){\circle*{0.001}}
\put(19.098,17.514){\circle*{0.001}}
\put(19.098,18.268){\circle*{0.001}}
\put(19.098,19.098){\circle*{0.001}}
\put(19.098,20.011){\circle*{0.001}}
\put(19.098,21.011){\circle*{0.001}}
\put(19.098,22.098){\circle*{0.001}}
\put(19.098,23.268){\circle*{0.001}}
\put(19.098,24.514){\circle*{0.001}}
\put(19.098,25.824){\circle*{0.001}}
\put(19.098,27.187){\circle*{0.001}}
\put(19.098,28.584){\circle*{0.001}}
\put(19.098,30.000){\circle*{0.001}}
\put(20.011, 0.000){\circle*{0.001}}
\put(20.011, 1.416){\circle*{0.001}}
\put(20.011, 2.813){\circle*{0.001}}
\put(20.011, 4.176){\circle*{0.001}}
\put(20.011, 5.486){\circle*{0.001}}
\put(20.011, 6.732){\circle*{0.001}}
\put(20.011, 7.902){\circle*{0.001}}
\put(20.011, 8.989){\circle*{0.001}}
\put(20.011, 9.989){\circle*{0.001}}
\put(20.011,10.902){\circle*{0.001}}
\put(20.011,11.732){\circle*{0.001}}
\put(20.011,12.486){\circle*{0.001}}
\put(20.011,13.176){\circle*{0.001}}
\put(20.011,13.813){\circle*{0.001}}
\put(20.011,14.416){\circle*{0.001}}
\put(20.011,15.000){\circle*{0.001}}
\put(20.011,15.584){\circle*{0.001}}
\put(20.011,16.187){\circle*{0.001}}
\put(20.011,16.824){\circle*{0.001}}
\put(20.011,17.514){\circle*{0.001}}
\put(20.011,18.268){\circle*{0.001}}
\put(20.011,19.098){\circle*{0.001}}
\put(20.011,20.011){\circle*{0.001}}
\put(20.011,21.011){\circle*{0.001}}
\put(20.011,22.098){\circle*{0.001}}
\put(20.011,23.268){\circle*{0.001}}
\put(20.011,24.514){\circle*{0.001}}
\put(20.011,25.824){\circle*{0.001}}
\put(20.011,27.187){\circle*{0.001}}
\put(20.011,28.584){\circle*{0.001}}
\put(20.011,30.000){\circle*{0.001}}
\put(21.011, 0.000){\circle*{0.001}}
\put(21.011, 1.416){\circle*{0.001}}
\put(21.011, 2.813){\circle*{0.001}}
\put(21.011, 4.176){\circle*{0.001}}
\put(21.011, 5.486){\circle*{0.001}}
\put(21.011, 6.732){\circle*{0.001}}
\put(21.011, 7.902){\circle*{0.001}}
\put(21.011, 8.989){\circle*{0.001}}
\put(21.011, 9.989){\circle*{0.001}}
\put(21.011,10.902){\circle*{0.001}}
\put(21.011,11.732){\circle*{0.001}}
\put(21.011,12.486){\circle*{0.001}}
\put(21.011,13.176){\circle*{0.001}}
\put(21.011,13.813){\circle*{0.001}}
\put(21.011,14.416){\circle*{0.001}}
\put(21.011,15.000){\circle*{0.001}}
\put(21.011,15.584){\circle*{0.001}}
\put(21.011,16.187){\circle*{0.001}}
\put(21.011,16.824){\circle*{0.001}}
\put(21.011,17.514){\circle*{0.001}}
\put(21.011,18.268){\circle*{0.001}}
\put(21.011,19.098){\circle*{0.001}}
\put(21.011,20.011){\circle*{0.001}}
\put(21.011,21.011){\circle*{0.001}}
\put(21.011,22.098){\circle*{0.001}}
\put(21.011,23.268){\circle*{0.001}}
\put(21.011,24.514){\circle*{0.001}}
\put(21.011,25.824){\circle*{0.001}}
\put(21.011,27.187){\circle*{0.001}}
\put(21.011,28.584){\circle*{0.001}}
\put(21.011,30.000){\circle*{0.001}}
\put(22.098, 0.000){\circle*{0.001}}
\put(22.098, 1.416){\circle*{0.001}}
\put(22.098, 2.813){\circle*{0.001}}
\put(22.098, 4.176){\circle*{0.001}}
\put(22.098, 5.486){\circle*{0.001}}
\put(22.098, 6.732){\circle*{0.001}}
\put(22.098, 7.902){\circle*{0.001}}
\put(22.098, 8.989){\circle*{0.001}}
\put(22.098, 9.989){\circle*{0.001}}
\put(22.098,10.902){\circle*{0.001}}
\put(22.098,11.732){\circle*{0.001}}
\put(22.098,12.486){\circle*{0.001}}
\put(22.098,13.176){\circle*{0.001}}
\put(22.098,13.813){\circle*{0.001}}
\put(22.098,14.416){\circle*{0.001}}
\put(22.098,15.000){\circle*{0.001}}
\put(22.098,15.584){\circle*{0.001}}
\put(22.098,16.187){\circle*{0.001}}
\put(22.098,16.824){\circle*{0.001}}
\put(22.098,17.514){\circle*{0.001}}
\put(22.098,18.268){\circle*{0.001}}
\put(22.098,19.098){\circle*{0.001}}
\put(22.098,20.011){\circle*{0.001}}
\put(22.098,21.011){\circle*{0.001}}
\put(22.098,22.098){\circle*{0.001}}
\put(22.098,23.268){\circle*{0.001}}
\put(22.098,24.514){\circle*{0.001}}
\put(22.098,25.824){\circle*{0.001}}
\put(22.098,27.187){\circle*{0.001}}
\put(22.098,28.584){\circle*{0.001}}
\put(22.098,30.000){\circle*{0.001}}
\put(23.268, 0.000){\circle*{0.001}}
\put(23.268, 1.416){\circle*{0.001}}
\put(23.268, 2.813){\circle*{0.001}}
\put(23.268, 4.176){\circle*{0.001}}
\put(23.268, 5.486){\circle*{0.001}}
\put(23.268, 6.732){\circle*{0.001}}
\put(23.268, 7.902){\circle*{0.001}}
\put(23.268, 8.989){\circle*{0.001}}
\put(23.268, 9.989){\circle*{0.001}}
\put(23.268,10.902){\circle*{0.001}}
\put(23.268,11.732){\circle*{0.001}}
\put(23.268,12.486){\circle*{0.001}}
\put(23.268,13.176){\circle*{0.001}}
\put(23.268,13.813){\circle*{0.001}}
\put(23.268,14.416){\circle*{0.001}}
\put(23.268,15.000){\circle*{0.001}}
\put(23.268,15.584){\circle*{0.001}}
\put(23.268,16.187){\circle*{0.001}}
\put(23.268,16.824){\circle*{0.001}}
\put(23.268,17.514){\circle*{0.001}}
\put(23.268,18.268){\circle*{0.001}}
\put(23.268,19.098){\circle*{0.001}}
\put(23.268,20.011){\circle*{0.001}}
\put(23.268,21.011){\circle*{0.001}}
\put(23.268,22.098){\circle*{0.001}}
\put(23.268,23.268){\circle*{0.001}}
\put(23.268,24.514){\circle*{0.001}}
\put(23.268,25.824){\circle*{0.001}}
\put(23.268,27.187){\circle*{0.001}}
\put(23.268,28.584){\circle*{0.001}}
\put(23.268,30.000){\circle*{0.001}}
\put(24.514, 0.000){\circle*{0.001}}
\put(24.514, 1.416){\circle*{0.001}}
\put(24.514, 2.813){\circle*{0.001}}
\put(24.514, 4.176){\circle*{0.001}}
\put(24.514, 5.486){\circle*{0.001}}
\put(24.514, 6.732){\circle*{0.001}}
\put(24.514, 7.902){\circle*{0.001}}
\put(24.514, 8.989){\circle*{0.001}}
\put(24.514, 9.989){\circle*{0.001}}
\put(24.514,10.902){\circle*{0.001}}
\put(24.514,11.732){\circle*{0.001}}
\put(24.514,12.486){\circle*{0.001}}
\put(24.514,13.176){\circle*{0.001}}
\put(24.514,13.813){\circle*{0.001}}
\put(24.514,14.416){\circle*{0.001}}
\put(24.514,15.000){\circle*{0.001}}
\put(24.514,15.584){\circle*{0.001}}
\put(24.514,16.187){\circle*{0.001}}
\put(24.514,16.824){\circle*{0.001}}
\put(24.514,17.514){\circle*{0.001}}
\put(24.514,18.268){\circle*{0.001}}
\put(24.514,19.098){\circle*{0.001}}
\put(24.514,20.011){\circle*{0.001}}
\put(24.514,21.011){\circle*{0.001}}
\put(24.514,22.098){\circle*{0.001}}
\put(24.514,23.268){\circle*{0.001}}
\put(24.514,24.514){\circle*{0.001}}
\put(24.514,25.824){\circle*{0.001}}
\put(24.514,27.187){\circle*{0.001}}
\put(24.514,28.584){\circle*{0.001}}
\put(24.514,30.000){\circle*{0.001}}
\put(25.824, 0.000){\circle*{0.001}}
\put(25.824, 1.416){\circle*{0.001}}
\put(25.824, 2.813){\circle*{0.001}}
\put(25.824, 4.176){\circle*{0.001}}
\put(25.824, 5.486){\circle*{0.001}}
\put(25.824, 6.732){\circle*{0.001}}
\put(25.824, 7.902){\circle*{0.001}}
\put(25.824, 8.989){\circle*{0.001}}
\put(25.824, 9.989){\circle*{0.001}}
\put(25.824,10.902){\circle*{0.001}}
\put(25.824,11.732){\circle*{0.001}}
\put(25.824,12.486){\circle*{0.001}}
\put(25.824,13.176){\circle*{0.001}}
\put(25.824,13.813){\circle*{0.001}}
\put(25.824,14.416){\circle*{0.001}}
\put(25.824,15.000){\circle*{0.001}}
\put(25.824,15.584){\circle*{0.001}}
\put(25.824,16.187){\circle*{0.001}}
\put(25.824,16.824){\circle*{0.001}}
\put(25.824,17.514){\circle*{0.001}}
\put(25.824,18.268){\circle*{0.001}}
\put(25.824,19.098){\circle*{0.001}}
\put(25.824,20.011){\circle*{0.001}}
\put(25.824,21.011){\circle*{0.001}}
\put(25.824,22.098){\circle*{0.001}}
\put(25.824,23.268){\circle*{0.001}}
\put(25.824,24.514){\circle*{0.001}}
\put(25.824,25.824){\circle*{0.001}}
\put(25.824,27.187){\circle*{0.001}}
\put(25.824,28.584){\circle*{0.001}}
\put(25.824,30.000){\circle*{0.001}}
\put(27.187, 0.000){\circle*{0.001}}
\put(27.187, 1.416){\circle*{0.001}}
\put(27.187, 2.813){\circle*{0.001}}
\put(27.187, 4.176){\circle*{0.001}}
\put(27.187, 5.486){\circle*{0.001}}
\put(27.187, 6.732){\circle*{0.001}}
\put(27.187, 7.902){\circle*{0.001}}
\put(27.187, 8.989){\circle*{0.001}}
\put(27.187, 9.989){\circle*{0.001}}
\put(27.187,10.902){\circle*{0.001}}
\put(27.187,11.732){\circle*{0.001}}
\put(27.187,12.486){\circle*{0.001}}
\put(27.187,13.176){\circle*{0.001}}
\put(27.187,13.813){\circle*{0.001}}
\put(27.187,14.416){\circle*{0.001}}
\put(27.187,15.000){\circle*{0.001}}
\put(27.187,15.584){\circle*{0.001}}
\put(27.187,16.187){\circle*{0.001}}
\put(27.187,16.824){\circle*{0.001}}
\put(27.187,17.514){\circle*{0.001}}
\put(27.187,18.268){\circle*{0.001}}
\put(27.187,19.098){\circle*{0.001}}
\put(27.187,20.011){\circle*{0.001}}
\put(27.187,21.011){\circle*{0.001}}
\put(27.187,22.098){\circle*{0.001}}
\put(27.187,23.268){\circle*{0.001}}
\put(27.187,24.514){\circle*{0.001}}
\put(27.187,25.824){\circle*{0.001}}
\put(27.187,27.187){\circle*{0.001}}
\put(27.187,28.584){\circle*{0.001}}
\put(27.187,30.000){\circle*{0.001}}
\put(28.584, 0.000){\circle*{0.001}}
\put(28.584, 1.416){\circle*{0.001}}
\put(28.584, 2.813){\circle*{0.001}}
\put(28.584, 4.176){\circle*{0.001}}
\put(28.584, 5.486){\circle*{0.001}}
\put(28.584, 6.732){\circle*{0.001}}
\put(28.584, 7.902){\circle*{0.001}}
\put(28.584, 8.989){\circle*{0.001}}
\put(28.584, 9.989){\circle*{0.001}}
\put(28.584,10.902){\circle*{0.001}}
\put(28.584,11.732){\circle*{0.001}}
\put(28.584,12.486){\circle*{0.001}}
\put(28.584,13.176){\circle*{0.001}}
\put(28.584,13.813){\circle*{0.001}}
\put(28.584,14.416){\circle*{0.001}}
\put(28.584,15.000){\circle*{0.001}}
\put(28.584,15.584){\circle*{0.001}}
\put(28.584,16.187){\circle*{0.001}}
\put(28.584,16.824){\circle*{0.001}}
\put(28.584,17.514){\circle*{0.001}}
\put(28.584,18.268){\circle*{0.001}}
\put(28.584,19.098){\circle*{0.001}}
\put(28.584,20.011){\circle*{0.001}}
\put(28.584,21.011){\circle*{0.001}}
\put(28.584,22.098){\circle*{0.001}}
\put(28.584,23.268){\circle*{0.001}}
\put(28.584,24.514){\circle*{0.001}}
\put(28.584,25.824){\circle*{0.001}}
\put(28.584,27.187){\circle*{0.001}}
\put(28.584,28.584){\circle*{0.001}}
\put(28.584,30.000){\circle*{0.001}}
\put(30.000, 0.000){\circle*{0.001}}
\put(30.000, 1.416){\circle*{0.001}}
\put(30.000, 2.813){\circle*{0.001}}
\put(30.000, 4.176){\circle*{0.001}}
\put(30.000, 5.486){\circle*{0.001}}
\put(30.000, 6.732){\circle*{0.001}}
\put(30.000, 7.902){\circle*{0.001}}
\put(30.000, 8.989){\circle*{0.001}}
\put(30.000, 9.989){\circle*{0.001}}
\put(30.000,10.902){\circle*{0.001}}
\put(30.000,11.732){\circle*{0.001}}
\put(30.000,12.486){\circle*{0.001}}
\put(30.000,13.176){\circle*{0.001}}
\put(30.000,13.813){\circle*{0.001}}
\put(30.000,14.416){\circle*{0.001}}
\put(30.000,15.000){\circle*{0.001}}
\put(30.000,15.584){\circle*{0.001}}
\put(30.000,16.187){\circle*{0.001}}
\put(30.000,16.824){\circle*{0.001}}
\put(30.000,17.514){\circle*{0.001}}
\put(30.000,18.268){\circle*{0.001}}
\put(30.000,19.098){\circle*{0.001}}
\put(30.000,20.011){\circle*{0.001}}
\put(30.000,21.011){\circle*{0.001}}
\put(30.000,22.098){\circle*{0.001}}
\put(30.000,23.268){\circle*{0.001}}
\put(30.000,24.514){\circle*{0.001}}
\put(30.000,25.824){\circle*{0.001}}
\put(30.000,27.187){\circle*{0.001}}
\put(30.000,28.584){\circle*{0.001}}
\put(30.000,30.000){\circle*{0.001}}
\end{picture}
  \end{minipage}
  \caption{\label{fig:nli}\footnotesize Exemplo de varredura ($30\times30$ pontos) sem (� esquerda) e com (� direita) a influ�ncia da n�o linearidade intr�nsica (se��o~\ref{sec:nli}).}
\end{figure}

\begin{figure}[htb]
	\begin{minipage}[b]{0.46\textwidth}\centering
		\includegraphics[scale=0.25]{figuras/scanner/padrao.jpg}
		\caption{\label{fig:padrao}\footnotesize Exemplo de amostra ($80\unit{\mu m}\times80\unit{\mu m}$) utilizada para calibrar o \scanner. No caso, sabe-se que a periodicidade da amostra � de $10\unit{\mu m}$. Os orif�cios possuem $180\unit{\mu m}$ de profundidade.}
	\end{minipage}\hfill
	\begin{minipage}[b]{0.46\textwidth}\centering
		\includegraphics[scale=0.35]{figuras/scanner/histerese1.pdf}
		\caption{\label{fig:histerese1}\footnotesize Gr�fico ilustrando a histerese.}  
	\end{minipage}
\end{figure}


	{\slshape Scanners} n�o-realimentados geralmente possuem softwares capazes de tratar a imagem obtida de modo a reduzir os efeitos da n�o-linearidade intr�nseca (se��o~\ref{sec:corrections}).
	
%%%%%%%%%%%%%%%%%%%%%%%%%%%%%%%%%%%%%%%%%%%%%%%%%%%%%%%%%%%%%%%%%%%%%%%%%%%%%%%%%%%%%%%%%%%%%%%%%%%%%
% Se��o: Histerese - Cap�tulo: Scanners
%%%%%%%%%%%%%%%%%%%%%%%%%%%%%%%%%%%%%%%%%%%%%%%%%%%%%%%%%%%%%%%%%%%%%%%%%%%%%%%%%%%%%%%%%%%%%%%%%%%%%
\section{Histerese}
	\label{sec:histerese}\index{Histerese}
	
	A histerese tamb�m est� presente em materiais piezoel�tricos e tem origem fundamentalmente nos efeitos de polariza��o na rede cristalina \index{Rede cristalina} e em fric��es moleculares. Nesse caso, t�m-se que varia��es (na tens�o aplicada) de intensidades id�nticas, mas de sentidos opostos, resultam deslocamentos diferentes (veja gr�fico~\ref{fig:histerese1}). Matematicamente podemos escrever $|\vetor s(\Delta\vetor E)|\ne|\vetor s(-\Delta\vetor E)|$.\par
	
	A histerese � quantificada conforme o quociente da m�xima diferen�a entre os $\Delta x$ para $\Delta E$ positivo e negativo (Em termos do gr�figo~\ref{fig:histerese1} � a m�xima dist�ncia entre a ``ida'' e a ``volta'') pelo maior distens�o que � poss�vel produzir na cer�mica (Ponto de retorno do gr�fico~\ref{fig:histerese1}), podendo atingir valores de at� $20\%$:
	$$
	\mbox{Histerese} = {\Delta x\over x}
	$$
	
	Os efeitos da histerese s�o t�o grandes que s�o facilmente evidenci�veis: Na figura~\ref{fig:varredura} note que as medidas (representadas pelos pontos) s�o feitas somente em um dos sentidos da varredura horizontal (trace). Isto � feito justamente para evitar o efeito da histerese. Contudo, o usu�rio pode habilitar um segundo canal de obten��o de dados de modo a obter duas imagens simultaneamente, uma no sentido $\Delta x>0$ ({\slshape trace}) e outra em $\Delta x<0$ ({\slshape retrace}). Este procedimento nos permite ilustrar\footnote{Na pr�tica este mesmo procedimento nos permite determinar se a histerese est� ou n�o ocorrendo. Caso esteja --- e isto vale para qualquer outro dos efeitos descritos, exceto o {\slshape aging} (se��o~\ref{sec:aging}) ---, � preciso refazer as calibra��es.} o efeito da histerese na varredura r�pida  (figura~\ref{fig:histerese2}):
		
\begin{figure}[htb]
  \begin{minipage}[b]{.3\textwidth}\centering
    \includegraphics[scale=0.2]{figuras/scanner/1204t.jpg}
  \end{minipage}\hfill
  \begin{minipage}[b]{.3\textwidth}\centering
  	\includegraphics[scale=0.2]{figuras/scanner/1204r.jpg}
  \end{minipage}\hfill
  \begin{minipage}[b]{.3\textwidth}\centering
  	\includegraphics[scale=0.2]{figuras/scanner/1204h.jpg}
  \end{minipage}
  \caption{\label{fig:histerese2}\footnotesize Da esquerda para a direita, imagens da amostra padr�o obtidas simultaneamente em {\slshape trace} e {\slshape retrace}, ilustrando o efeito da histerese na varredura r�pida. Na extrema direita a subtra��o de uma pela outra. Note, nas bordas laterais, que as duas imagens n�o s�o iguais [AFM de contato ($80\unit{\mu m}\times80\unit{\mu m}$), $512$ pontos por linha].}
\end{figure}

	Na varredura lenta, a histerese pode ser evidenciada obtendo-se duas imagens consecutivas (fig.~\ref{fig:histerese3}).

\begin{figure}[htb]
  \begin{minipage}[b]{.3\textwidth}\centering
    \includegraphics[scale=0.2]{figuras/scanner/1212t.jpg}
  \end{minipage}\hfill
  \begin{minipage}[b]{.3\textwidth}\centering
  	\includegraphics[scale=0.2]{figuras/scanner/1212r.jpg}
  \end{minipage}\hfill
  \begin{minipage}[b]{.3\textwidth}\centering
  	\includegraphics[scale=0.2]{figuras/scanner/1212h.jpg}
  \end{minipage}
  \caption{\label{fig:histerese3}\footnotesize Da esquerda para a direita, imagens obtidas sequencialmente, ilustrando o efeito da histerese na varredura lenta. Na extrema direita a subtra��o de uma pela outra. Note que as duas imagens n�o s�o iguais [AFM de contato ($80\unit{\mu m}\times80\unit{\mu m}$), $512$ pontos por linha].}
\end{figure}
			
%%%%%%%%%%%%%%%%%%%%%%%%%%%%%%%%%%%%%%%%%%%%%%%%%%%%%%%%%%%%%%%%%%%%%%%%%%%%%%%%%%%%%%%%%%%%%%%%%%%%%
%	Se��o: Creep - Cap�tulo: Scanners
%%%%%%%%%%%%%%%%%%%%%%%%%%%%%%%%%%%%%%%%%%%%%%%%%%%%%%%%%%%%%%%%%%%%%%%%%%%%%%%%%%%%%%%%%%%%%%%%%%%%%
\section{Creep}
	\label{sec:creep}\index{Creep}
	
	Qualquer material piezoel�trico n�o se altera imediatamente ap�s a aplica��o de um campo el�trico. De fato, uma altera��o abrupta na tens�o aplicada faz com que a cer�mica oscile em torno de seu deslocamento nominal (figura~\ref{fig:creep1}), levando algo em torno de $\tau=3/\nu_0$ segundos ($[\nu_0]=s^{-1}$, SI) para atingir o deslocamento nominal, sendo $\nu_0$ a sua freq��ncia de resson�ncia\footnote{$\nu_0=2\pi\sqrt{(k/m_e)}$, sendo $k$ a constante el�stica e $m_e$ � a massa efetiva da piezoel�trica, aproximadamente um ter�o da massa total do material.}. \index{Resson�ncia} Ent�o, e.g., se $\nu_0=50\unit{kHz}$, ele oscilar� por aproximadamente $60\unit{\mu s}$ antes de atingir o valor nominal correspondente � tens�o aplicada.\par

\begin{wrapfigure}{r}{0.6\textwidth}
	\centering
  \includegraphics[scale=0.5]{figuras/scanner/creep1.pdf}
  \caption{\label{fig:creep1}\footnotesize O cristal piezoel�trico, quando bruscamente perturbado, oscila em torno do valor nominal durante aproximadamente $\tau=3/\nu_0$.}
\end{wrapfigure}
	
	Mas h� ainda outro efeito: Ap�s atingirem o valor nominal, os dip�los moleculares continuam a se alinharem. Este alinhamento prolonga-se por v�rios minutos e at� horas, segundo a express�o logar�tmica (fig.~\ref{fig:creep2})
	$$
	\Delta L(t) = \Delta L_0 \Big[1+\gamma\log\big({t\over t_0}\big)\Big]
	$$
	onde $\Delta L_0$ � o deslocamento ap�s decorrido $t_0$, contados a partir do final da aplica��o da tens�o; $\gamma$ � o {\slshape fator de creep}, \index{Creep!Fator de} que depende do material e assume valores da ordem de $0,01$ a $0,02$.	Obviamente, como todos os outros efeitos descritos, somente {\slshape scanners} sem realimenta��o sofrem desse problema e, do ponto de vista do usu�rio, este manifesta-se como uma cont�nua distor��o da imagem obtida para aquisi��es sucessivas. Na figura~\ref{fig:creep2}, e.g., fizemos quatro varreduras decentralizadas na amostra padr�o (figura~\ref{fig:padrao}), cada qual separada da seguinte por alguns minutos.
	
\begin{figure}[htb]
  \begin{minipage}[b]{0.2\textwidth}\centering
    \includegraphics[scale=0.2]{figuras/scanner/creep1255.jpg}
  \end{minipage}\hfill
  \begin{minipage}[b]{0.2\textwidth}\centering
  	\includegraphics[scale=0.2]{figuras/scanner/creep1258.jpg}
  \end{minipage}\hfill
  \begin{minipage}[b]{0.2\textwidth}\centering
  	\includegraphics[scale=0.2]{figuras/scanner/creep1303.jpg}
  \end{minipage}\hfill
  \begin{minipage}[b]{0.2\textwidth}\centering
  	\includegraphics[scale=0.2]{figuras/scanner/creep1306.jpg}
  \end{minipage}
  
  \caption{\label{fig:creep2}\footnotesize A seq��ncia de imagens acima foi obtida numa amplia��o da figura~\ref{fig:padrao}, com {\slshape offset} diferente de zero. Ou seja, decentralizada. � n�tido o efeito do \creep: Embora tenhamos fixado a �rea da nova varredura (a amplia��o), o piezoel�trico continua a modificar-se, causando distor��es nas imagens [AFM de contato, $16,91\unit{\mu m}\times 16,91\unit{\mu m}$, $512$ pontos por linha].}
\end{figure}
	
	Como o \creep\ � um efeito que ocorre para tempos longos, fica �bvio que a varredura r�pida n�o sofre este efeito, dado que � realizada na casa dos segundos. 

\begin{figure}[htb]
		\centering\includegraphics[scale=0.5]{figuras/scanner/creep2.jpg}
  	\caption{\label{fig:creep3}\footnotesize Mesmo quando finalizada a aplica��o do sinal el�trico, o piezoel�trico continua a expandir-se (ou contrair-se) por um longo tempo. � o efeito chamado {\slshape creep}.}
\end{figure}
	
	A figura~\ref{fig:creep2} efetivamente nos explica por que devemos evitar obter imagens decentralizadas (com {\slshape off set} diferente de zero) em microsc�pios que n�o contam com corre��es via hardware.	Mas o \creep\ aparece numa situa��o ainda mais cr�tica: Suponha que escolhamos obter uma imagem com $512$ pontos por linha a uma freq��ncia de $1\unit{Hz}$. Ent�o, como a matriz de aquisi��o de dados � quadrada (veja figura~\ref{fig:varredura}), levaremos $512$ segundos, ou oito minutos, para completar a varredura lenta. Isto significa que o \scanner\ permanece deslocado do centro durante um tempo significativamente longo (no m�nimo uns dois minutos) e, portanto, exposto ao \creep. Obviamente teremos uma imagem gradualmente distorcida nas bordas verticais (A varredura lenta, nos microsc�pios da Digital Instruments, s�o sempre colocados na vertical do monitor). A esse efeito some-se o {\slshape cross coupling} (se��o~\ref{sec:cross-coupling}) e teremos uma imagem nada confi�vel nas bordas. De fato, uma imagem s� ser� realmente confi�vel quando feita na mesma freq��ncia utilizada para calibrar o \scanner. Isto por que neste caso o arquivo de calibra��o reflete exatamente, ou quase, o comportamento do piezoel�trico para aquela freq��ncia. Se n�o for o caso, sempre haver� um erro maior na obten��o das imagens. No eixo $z$ o \creep\ apresenta uma ``borda'' nas varia��es de altura: Algo como o fen�meno de Gibbs observado em aproxima��es por s�rie de Fourrier em fun��es descont�nuas.
		
		Quantitativamente, o \creep\ � expresso como a raz�o $\Delta x_c/\Delta x$ e sempre relativamente a um tempo cr�tico $t_{cr}$ (observe a figura~\ref{fig:creep3}). Valores t�picos para o \creep\ s�o de $1\%$ a $20\%$ para tempos cr�ticos entre $10\unit{s}$ e $100\unit{s}$.
		

%%%%%%%%%%%%%%%%%%%%%%%%%%%%%%%%%%%%%%%%%%%%%%%%%%%%%%%%%%%%%%%%%%%%%%%%%%%%%%%%%%%%%%%%%%%%%%%%%%%%%
% Se��o: Aging - Cap�tulo: Scanners
%%%%%%%%%%%%%%%%%%%%%%%%%%%%%%%%%%%%%%%%%%%%%%%%%%%%%%%%%%%%%%%%%%%%%%%%%%%%%%%%%%%%%%%%%%%%%%%%%%%%%
\section{Aging}
	\label{sec:aging}\index{Aging}
	
	Se uma cer�mica piezoel�trica permanece muito tempo sem sofrer influ�ncia de campos el�tricos externos, os dipolos moleculares desalinham-se gradualmente com o tempo, resultando uma redu��o do ganho (ou deflex�o por tens�o aplicada). Em contrapartida, se um piezoel�trico � constantemente utilizado, os dipolos alinham-se cada vez mais, aumentando o ganho. Este comportamento pode ser observado na figura~\ref{fig:aging}. Vemos que a necessidade de recalibrarmos {\slshape scanners} n�o-realimentados aparece-nos insistentemente!\par
	
\begin{figure}[htb]
	\begin{minipage}[b]{0.46\textwidth}\centering
  	\includegraphics[scale=0.35]{figuras/scanner/aging.jpg}
  	\caption{\label{fig:aging}\footnotesize Um \scanner, se n�o utilizado com freq��ncia, perde sua polariza��o, reduzindo seu ganho. O oposto ocorre se ele � sempre utilizado.}	
  \end{minipage}\hfill
  \begin{minipage}[b]{0.46\textwidth}\centering
  	\includegraphics[scale=0.35]{figuras/scanner/cross-coupling.jpg}
  	\caption{\label{fig:cross-coupling1}\footnotesize Imagem em corte vertical da amostra padr�o de calibra��o (fig.~\ref{fig:padrao}). O \xcoupling\ altera o perfil da imagem, imprimindo-lhe uma curvatura n�o verdadeira.}
  \end{minipage}
\end{figure}

%%%%%%%%%%%%%%%%%%%%%%%%%%%%%%%%%%%%%%%%%%%%%%%%%%%%%%%%%%%%%%%%%%%%%%%%%%%%%%%%%%%%%%%%%%%%%%%%%%%%%
%	Se��o: Cross coupling - Cap�tulo: Scanners
%%%%%%%%%%%%%%%%%%%%%%%%%%%%%%%%%%%%%%%%%%%%%%%%%%%%%%%%%%%%%%%%%%%%%%%%%%%%%%%%%%%%%%%%%%%%%%%%%%%%%
\section{Cross coupling}
	\label{sec:cross-coupling}\index{Cross-coupling}

	Embora tenhamos dito, na p�gina~\pageref{scanner:explanation}, que os deslocamentos em $\hat i$, $\hat j$ e $\hat k$ s�o dois-a-dois independentes, isto n�o � bem verdade. Primeiro por que, mesmo na aproxima��o de primeira ordem, o pr�prio tensor de acoplamento piezoel�trico encarrega-se de tornar o vetor $\vetor s$ n�o paralelo a $\vetor E$. Al�m disso, � f�cil ver na figura~\ref{fig:scanner1} que como as cer�micas s�o acomodadas sobre o mesmo cil�ndro met�lico, ent�o � �bvio que elas n�o s�o independentes em seus deslocamentos. Essas duas verdades, contudo, pesam pouco quando comparadas com a geometria do sistema: De acordo com o que explicamos no in�cio do cap�tulo, quando desejamos que o atuador piezoel�trico produza um deslocamento em $x$, \eg, o cil�ndro met�lico curva-se neste sentido. Essa curvatura � acompanhada de um distanciamento da sonda � superf�cie, que o hardware obviamente compreende como uma altera��o em $z$, registrando esta altera��o, erroneamente, como parte do perfil da amostra.\par
	
	Uma forma de verificar se o efeito est� presente � utilizar uma amostra padr�o plana ou de curvatura conhecida (figura~\ref{fig:cross-coupling1}).\par
	
	Como sempre h� recursos em software capazes de minimizar este efeito. Contudo, caso a amostra analisada realmente possua uma curvatura, esta ser� removida.

%%%%%%%%%%%%%%%%%%%%%%%%%%%%%%%%%%%%%%%%%%%%%%%%%%%%%%%%%%%%%%%%%%%%%%%%%%%%%%%%%%%%%%%%%%%%%%%%%%%%%
% Se��o: Corre��es - Cap�tulo: Scanners
%%%%%%%%%%%%%%%%%%%%%%%%%%%%%%%%%%%%%%%%%%%%%%%%%%%%%%%%%%%%%%%%%%%%%%%%%%%%%%%%%%%%%%%%%%%%%%%%%%%%%
\section{Corre��es}
	\label{sec:corrections}\index{Corre��es}\index{Corre��es!via software}\index{Corre��es!via hardware}
		
	A ``corre��o'' desses erros pode ser feita via software, mas deve ser feita com muito crit�rio pois, em verdade, o que ele faz � modificar o resultado obtido baseando-se num comportamento padr�o (esperado) ou, em softwares mais desenvolvidos, em modelamentos matem�ticos de materiais piezoel�tricos. Obviamente, qualquer altera��o via software constitui uma {\bf modifica��o} do resultado da medida. De fato, � virtualmente poss�vel obter qualquer coisa com corre��es via software, independentemente do resultado bruto. Em outras palavras, utilizar corre��es via software sem um conhecimento bastante aprofundado do que cada op��o faz pode mascarar ou ainda criar objetos na amostra. Corre��es desse tipo s�o capazes de reduzir para at� $10\%$ os erros nas medidas.\par
	
	Um detalhe importante: No caso de atuadores piezoel�tricos utilizados na microscopia, mesmo os realimentados sofrem alguns dos efeitos acima comentados na dire��o $z$. Isto por que diferentemente das dire��es $x$ e $y$, esta � passiva, no sentido de que espelha uma intera��o desconhecida entre a ponta e a amostra. Em $x$ e $y$ � o usu�rio que determina os deslocamentos ($\vetor s\cdot\hat i$ e $\vetor s\cdot\hat j$); mas reside em $z$ a resposta da varredura. Portanto, nenhum \scanner, mesmo os realimentados, podem eliminar completamente os efeitos acima nesta dire��o. Isto leva-nos � conclus�o de que tamb�m para atuadores realimentados � preciso calibr�-los.
		
\chapter{STM --- Microscopia de tunelamento}
	\label{cap:stm}\index{STM}\index{Tunelamento}
	%	Projeto:  Scanning probe microscopy
%	Autor:	  Ivan Ramos Pagnossin
%	Data:		  13/06/2002
%	Arquivo:  cap2_STM.tex
% Cap�tulo: Microscopia de tunelamento

	Na microscopia de tunelamento a intera��o entre ponta e amostra � o efeito t�nel, \index{Efeito t�nel} explic�vel somente com a utiliza��o do conceito de dualidade onda-part�cula \index{Dualidade onda-part�cula} inerente � mec�nica qu�ntica nova, inating�vel atrav�s da mec�nica cl�ssica.\footnote{Atualmente, contudo, existe uma s�rie de cientistas procurando mostrar que mesmo o tunelamento pode ser explicado com base nos conceitos da mec�nica cl�ssica somente.} A primeira imagem por microscopia de tunelamento foi obtida em 1981 pelos cientistas Gerd Bining e Heinrich Rohrer, \index{Binning, Gerd}\index{Rohrer, Heinrich} rendendo-lhes, por isso, o pr�mio nobel de f�sica em 1986. Nesta t�cnica, uma ponta condutora muito fina (obtida geralmente por corros�o eletrol�tica: \index{Corros�o eletrol�tica} figura~\ref{fig:ponta-STM}) � mantida a uma diferen�a de potencial com rela��o � amostra tal que quando a sonda atinge aproximadamente $10\unit{\angstron}$ de dist�ncia uma corrente de tunelamento passa a ser detect�vel.
	
\begin{wrapfigure}{r}{0.6\textwidth}\centering
  \includegraphics[scale=1]{figuras/stm/ponta-stm.jpg}
  \caption{\label{fig:ponta-STM}\footnotesize Exemplo de ponta de tungst�nio utilizada em microscopia de tunelamento: � esquerda a ponta obtida ap�s a corros�o eletrol�tica; � direita, trabalhada com feixes de �ons.}
\end{wrapfigure}
		
	Sabe-se que dois metais diferentes ao serem postos em contato apresentam uma diferen�a de potencial natural devida � diferen�a entre os n�veis de Fermi \index{Fermi!N�veis de} desses materiais.\upcite[p�g.517]{eisberg} Se, ao inv�s disso, mantivermos esses dois materiais muito pr�ximos, ocorrer� tunelamento de um para o outro at� que as taxas de tunelamento igualem-se, atingindo um equil�brio din�mico. Se, no entanto, aplicarmos uma diferen�a de potencial entre esses dois metais e a mantivermos atrav�s de uma fonte externa (como uma bateria), estaremos dando prefer�ncia a um dos sentidos de tunelamento e o equil�brio n�o � atingido. A isto segue o aparecimento de uma {\slshape corrente de tunelamento} \index{Tunelamento!Corrente de} que varia exponencialmente com a dist�ncia:
	$$
	I(z)\cong Ve^{-2cz} \Rightarrow I(z+\delta)\cong Ve^{-2c(z+\delta)} =e^{-2c\delta} I(z)
	$$
	com $c$ uma constante e $V$ a tens�o aplicada. Isto significa que uma pequena varia��o na dist�ncia entre os metais altera consideravelmente a corrente de tunelamento. Assim, se um desses metais for a ponta (sonda) e o outro for a amostra, teremos um m�todo extremamente sens�vel de verificarmos altera��es na corrente de tunelamento. Mais que isso: Como a corrente de tunelamento depende crucialmente da distribui��o dos orbitais na superf�cie da amostra, a microscopia de tunelamento � capaz de analisar a distribui��o eletr�nica nessa superf�cie. Mas podemos ir mais longe: Se for poss�vel associar a distribui��o eletr�nica (orbitais) superficial � morfologia, ent�o teremos como enxergar essa morfologia na escala at�mica. De fato, verifica-se experimentalmente que essa t�cnica possui resolu��o de fra��es de �ngstrons da dire��o $z$ e de �ngstrons lateralmente (veja figura~\ref{fig:grafite1}). �, sem d�vida, a t�cnica que nos permite enxergar mais ``de perto'' uma amostra. Mas nem tudo s�o flores: Como a t�cnica necessita a circula��o de uma corrente (de tunelamento) � claro que ambos os metais envolvidos devem ser condutores: Quanto a ponta n�o h� problemas, pois s�o feitas de tungst�nio justamente por este motivo (entre outro); quanto � amostra, vemos que essa t�cnia limita-nos a amostras condutoras ou, em casos mais extremos, semi-condutoras. Um problema freq�ente na microscopia de tunelamento devido justamente a essa limita��o de condutividade das partes envolvidas s�o os �xidos que se depositam sobre a superf�cie da amostra. � comum precisarmos limpar a amostra antes de efetuarmos medidas com essa t�cnica.

\begin{figure}[htb]
	\begin{minipage}[c]{0.46\textwidth}\centering
		\includegraphics[scale=0.4]{figuras/stm/grafite1.jpg}		
	\end{minipage}\hfill
	\begin{minipage}[c]{0.46\textwidth}\centering
		\includegraphics[scale=0.4]{figuras/stm/grafite2.jpg}		
	\end{minipage}
	\caption{\label{fig:grafite1}\footnotesize Imagem em STM de grafite altamente orientado. � esquerda uma varredura de $5\unit{\mu m}\times5\unit{\mu m}$ evidencia os desn�veis entre os planos at�micos (linhas diagonais). � direita a mesma imagem, ampliada para $3\unit{nm}\times3\unit{nm}$, onde a distribui��o eletr�nica at�mica passa a aparecer. Note a estrutura em colm�ia, a mais comum alotropia do carbono. Embora as ``bolinhas'' vis�veis sejam de fato �tomos, a estrutura apresentada n�o � completa: Faltam alguns �tomos que a t�cnica n�o mostra. Isto ocorre basicamente porque os planos at�micos n�o s�o verticalmente alinhados (veja figura~\ref{fig:grafite2}).}
\end{figure}
	
	Existe duas formas de se varrer uma amostra: Fixando a altura da ponta ou mantendo a corrente de tunelamento constante. No primeiro a sonda fixa sua posi��o $z$ e varre a amostra registrando as varia��es na corrente de tunelamento (Note que manter a {\bfseries altura} constante n�o � o mesmo que manter a {\bfseries dist�ncia} ponta-amostra constante); No segundo a posi��o $z$ da ponta � modificada, ponto a ponto da malha de varredura, de forma a manter a corrente de tunelamento constante. Note que no primeiro caso, por n�o alteramos a posi��o $z$, teremos muito pouca influ�ncia dos ``males'' dos {\slshape scanners}: Quando muito poderemos ter apenas o \creep.\par

\begin{wrapfigure}{r}{0.5\textwidth}\centering
	\includegraphics[scale=0.65]{figuras/stm/grafite3.jpg}
	\caption{\label{fig:grafite2}\footnotesize A forma alotr�pica mais comum do carbono � aquela que se organiza hexagonalmente (em forma de colm�ia). Numa imagem STM de tal estrutura (fig.~\ref{fig:grafite1}), entretanto, o que se v� na realidade s�o os hex�gonos hachurados. Ou seja, o STM n�o enxerga todos os �tomos. Isto se deve basicamente devido a um efeito da superposi��o das monocamadas. Para uma explica��o aprofundada veja~\cite[p�g.388]{roland}. Na figura acima, $A=2,546\unit{\angstron}$ e $C=6,67\unit{\angstron}$.}
\end{wrapfigure}

	Al�m do exposto no par�grafo anterior, o m�todo altura fixa � mais r�pido, mas n�o nos permite analisar superf�cies com irregularidades superiores a $10\angstron$, o que n�o ocorre no segundo. Note mais essa sutileza: Ao optarmos pelo m�todo de corrente de tunelamento constante, n�o necessariamente estamos mantendo a dist�ncia ponta-amostra constante pois como j� foi mencionado, a microscopia de tunelamento, na realidade, mede a distribui��o eletr�nica da superf�cie. Ent�o, e.g., numa regi�o oxidada da superf�cie a corrente de tunelamento cair� abruptamente e o microsc�pio acreditar� na necessidade de aproximar a ponta da amostra, interpretando esse deslocamento como um vale na morfologia da amostra, o que n�o � verdade (De fato teremos justamente o oposto: Uma protuber�ncia). Fica claro que a analogia entre morfologia e distribui��o eletr�nica, \index{Distribui��o eletr�nica} que � o que realmente mede o microsc�pio de tunelamento, deve ser feita com muito cuidado.\par
	
	Como exemplo das ludibria��es causadas pelos microsc�pios de tunelamento quando da analogia entre distribui��o eletr�nica e morfologia pode ser observada numa superf�cie de grafite (figura~\ref{fig:grafite1}): A estrutura do grafite mais comum na natureza � aquela que se organiza hexagonalmente (figura~\ref{fig:grafite2}) e embora o primeiro vizinho de qualquer �tomo de carbono esteja a $1,42\unit{\angstron}$, de fato o que se enxerga s�o protuber�ncias distanciadas de $2,456\unit{\angstron}$. Este efeito decorre do fato de os planos at�micos do carbono estarem deslocados um em rela��o ao outro (figura~\ref{fig:grafite2}). Para uma explica��o mais extensa consulte a refer�ncia~\cite[p�g. 388]{roland}.

\subsection{STS --- Espectroscopia de tunelamento}
	\label{sec:EspectroscopiaDeTunelamento}\index{STS}\index{Espectroscopia!de tunelamento}
	
	A espectroscopia de tunelamento limita-se � interpreta��o dos resultados da microscopia de tunelamento como propredades eletr�nicas e qu�micas (liga��es, vizinhan�as, etc.). Existem v�rias t�cnicas de proced�-la, algumas das quais resumimos abaixo:
	
	\begin{dingautolist}{202}
		\item Fixando a corrente de tunelamento, mede-se as corre��es em $z$ para diferentes tens�es aplicadas entre ponta e amostra ({\slshape bias}), analisando os efeitos do {\slshape bias} \index{Bias} nas respostas.
		
		\item Analogamente, podemos fixar a altura e registrar as varia��es na corrente para diferentes {\slshape bias} aplicado.
		
		\item Outro tipo de estudo � fixar o \scanner\ num ponto da malha de varredura (obviamente de prefer�ncia o centro) e variar a diferen�a de potencial entre ponta e amostra ({\slshape bias}). 
	
		\item Utilizando o esquema do item anterior para todos os pontos da malha de varredura podemos mapear a estrutura eletr�nica da regi�o varrida.
	\end{dingautolist}
	
	Phillipe Sautet \index{Sautet, Phillipe} tem utilizado a microscopia de tunelamento para analisar a adsor��o de diversos elementos na $\mathrm{Pt}(1,1,1)$. Como resultado, ele tem observado que nas imagens STM tais �tomos podem causar protuber�ncias de at� $2\angstron$ e depress�es de at� $0,35\angstron$, dependendo do �tomo adsorvido. � um exemplo claro de como as imagens STM n�o podem ser associadas ao bel prazer com a morfologia da amostra. 
	
		

	
		
\chapter{AFM --- Microscopia de for�a at�mica}
	\label{cap:afm}\index{AFM}
	%	Projeto:  Scanning probe microscopy
%	Autor:	  Ivan Ramos Pagnossin
%	Data:		  15/06/2002
%	Arquivo:  afm.tex
% Cap�tulo: Microscopia de for�a at�mica

	S�o tr�s os modos de microscopia de for�a at�mica: AFM de contato, \index{AFM!de contato} AFM de n�o contato \index{AFM!de n�o contato} e AFM de contato intermitente. \index{AFM!de contato intermitente} Em todos eles a intera��o sonda-amostra � a for�a de Wan der Waals, \index{Van der Waals, For�a de} cuja depend�ncia com a dist�ncia � ilustrada na figura~\ref{fig:van-der-waals}. Dissemos que se trata de AFM de contato quando a sonda entra em contato com a amostra\footnote{Este ponto � caracterizado pela invers�o no sinal da for�a de Van der Waals no gr�fico da figura~\ref{fig:van-der-waals}.}) ou, identicamente, quando a for�a de Wan der Waals torna-se repulsiva. Nas outras duas modalidades a for�a � atrativa e a sonda oscila sobre a amostra. Quando esta oscila��o toca a amostra intermitentemente damos o nome de contato intermitente (ou, mais comumente, {\slshape tapping mode}); Quando n�o toca em momento algum, chamamos AFM de n�o contato. Note que o AFM de contato {\bfseries n�o oscila}, permanecendo constantemente em contato com a superf�cie.

\begin{figure}[htb]
	\begin{minipage}[c]{0.46\textwidth}\centering
		\includegraphics[scale=0.5]{figuras/afm/van-der-waals.jpg}
		\caption{\label{fig:van-der-waals}\footnotesize Depend�ncia com a dist�ncia da for�a de Van der Waals.}
	\end{minipage}\hfill
	\begin{minipage}[c]{0.46\textwidth}\centering
		\includegraphics[scale=0.5]{figuras/afm/sonda-afm.jpg}
		\caption{\label{fig:sonda-afm-contato}\footnotesize Esquema de organiza��o da sonda para a microscopia de for�a at�mica.}
	\end{minipage}
\end{figure}

	A ordem de grandeza da for�a de Van der Waals est� entre $10^{-6}\unit{N}$ e $10^{-7}\unit{N}$ para a AFM de contato e por volta de $10^{-12}\unit{N}$ para a AFM de n�o contato. Para a AFM de contato intermitente temos valores intermedi�rios (veja figura~\ref{fig:van-der-waals}). Note que, diferentemente da microscopia de tunelamento, a de for�a at�mica n�o limita-se a amostras condutoras pois n�o mede corrente alguma. Esta � uma vantagem sobre a STM. No entanto, a AFM n�o atinge a mesma resolu��o que a STM, mantendo-se na ordem de alguns �ngstrons tanto lateral quanto verticalmente para o AFM de contato.  (Para as outras t�cnicas a resolu��o � ainda menor).	
		
\section{A sonda}
	\label{sec:sonda-afm}\index{Sonda}
	
	Na microscopia de for�a at�mica a sonda � um sistema um pouco mais complicado que na de tunelamento: Trata-se de um conjunto de uma ponta muito pequena montada sobre um {\slshape cantilever} \index{Cantilever} (Uma pequena haste) que, por sua vez, est� acoplada ao \scanner; um feixe de laser e um fotodetector, \index{Fotodetector} montados conforme a figura~\ref{fig:sonda-afm-contato}: Ao aproximarmos a ponta da amostra, o cantilever come�a a defletir-se conforme sua constante el�stica $k$, no intuito de sempre equilibrar a for�a de Van der Waals (fig.~\ref{fig:van-der-waals}). A medida dessa deflex�o � feita atrav�s do laser, que atinge o cantilever em sua extremidade (sobre a ponta) e � refletido at� um fotodetector dividido em quatro se��es (fig.~\ref{fig:fotodetector}). Conforme o cantilever � mais ou menos defletido, o feixe refletido atinge o fotodetector mais para cima ou mais para baixo. Deflex�es laterais no cantilever aparecem como deslocamentos laterais do feixe refletido, o que d� origem � microscopia de for�a lateral (cap�tulo~\ref{cap:lfm}).\par

	A rigor existe ainda uma terceira for�a envolvida: A de capilaridade, atrativa e da ordem de $10^{-8}\unit{N}$. Pela ordem de grandeza fica claro que no modo AFM de contato tal for�a � irrelevante, pois contribui no m�ximo com $1\%$. Mas para os outros, de forma alguma. Verdadeiramente t�m-se que o AFM de contato intermitente � pouco (ou n�o �) suscept�vel � capilaridade; O AFM de contato � muito suscept�vel. Com o exposto vemos que os tr�s modos se completam, no sentido de que eles s�o gradativamente mais delicados com a amostra na mesma medida que s�o mais suscept�veis ao men�sco de �gua \index{Men�sco de �gua} adsorvido na amostra (e outros artefatos). Neste aspecto, o AFM de n�o contato oferece a vantagem de se poder analisar a topografia de uma amostra sem que haja contato real entre ponta e amostra, enquanto o AFM de contato n�o se deixa levar por esse artefato sempre presente. O AFM de contato intermitente faz a ponte entre uma modalidade e outra, juntando vantagens e desvantagens de ambas.\par
	
	A deflex�o do cantilever ao encostar na amostra n�o precisa ser zero e, de fato, pode ser qualquer. Essa for�a com que a ponta pressiona a amostra, em termos da deflex�o observada pelo detector, recebe o nome de {\slshape setpoint}. Quanto maior o {\slshape setpoint}, \index{Setpoint} maior � a press�o sobre a amostra. Ent�o parece �bvio que devemos ser cautelosos ao escolh�-lo pois, caso contr�rio, fatalmente destruiremos a amostra.
	
\section{AFM de contato}
	\label{sec:AFM-de-contato}\index{AFM!de contato}
	
	No caso do AFM de contato (cantilever n�o oscilante) os deslocamentos em $z$ s�o feitos sempre com o intuito de recuperar o setpoint: Suponha, e.g., que a sonda foi encostada na superf�cie com setpoint de $0,5\unit{V}$ (A tens�o aplicada � piezoel�trica para produzir tal press�o) e, ao deslocar-se para o ponto vizinho na matriz de varredura, encontrou uma protuber�ncia. Ent�o a proximidade ponta-amostra diminuiu, aumentando a for�a de Van der Waals e conseq�entemente aumentando a deflex�o do cantilever. Essa altera��o na deflex�o aparece como um deslocamento vertical para cima do laser refletido, alterando a leitura do fotodiodo (veja figura~\ref{fig:fotodetector}). O software do microsc�pio percebe esta altera��o e retrai a piezoel�trica na dire��o $z$ at� que a deflex�o original seja atingida. Neste momento o microsc�pio registra a corre��o em $z$ necess�ria para restaurar a situa��o original. Este esquema de acompanhamento �ptico das defelx�es � capaz de captar varia��es de at� fra��es de �ngstron.\par
	
	Um ponto importante a se notar � que a constante el�stica do cantilever deve ser menor que a da estrutura at�mica que se pretende observar. Caso contr�rio n�o ser� o cantilever que ir� defletir, mas sim a amostra. Isto obviamente leva a uma altera��o da amostra e, muitas vezes, � sua destrui��o. Mais especificamente, como $k_{at}=\omega^2_{at}\,m_{at}$, sendo $k_{at}$, $\omega_{at}$ e $m_{at}$ a constante el�stica, a freq��ncia de resson�ncia e a massa at�mica da amostra, temos que a constante el�stica do cantilever deve ser inferior a aproximadamente $10^{13}\unit{Hz}\cdot10^{-24}\unit{N}=10\unit{N/m}$ --- utilizando valores t�picos.
	
	H� ainda uma outra forma de varrermos a amostra: Ao inv�s de mantermos constante a deflex�o do cantilever, mantemos a altura $z$ constante, registrando as varia��es na deflex�o. � uma forma an�loga �quela discutida na microscopia de tunelamento: � mais r�pida mas limita-se a amostras com pouca rugosidade. � especialmente utilizada em estudos de forma��o de superf�cie, onde � interessante uma varredura r�pida; em ``tempo real''.
	
\section{AFM de n�o contato e {\slshape tapping mode}}
	\label{sec:AFM-de-contato-intermitente}\index{AFM!de contato intermitente}\index{AFM!de n�o-contato}
	
	No caso do AFM de n�o contato a id�ia b�sica � parecida, exceto que como o cantilever oscila, tamb�m o setpoint, a rigor. Ent�o o que chamamos de setpoint toma outra forma: Ele pode ser a amplitude da oscila��o, a freq��ncia ou a defasagem entre as oscila��es do cantilever e a fonte osciladora: A piezoel�trica. Assim, ponto a ponto na malha de varredura o microsc�pio restaura um desses valores, registrando a corre��o em $z$ necess�ria para retornar � situa��o inicial. Verifica-se experimentalmente que a qualidade das imagens obtidas utilizando-se a freq��ncia como setpoint � maior, seguida da defasagem.\par
	
	Para o AFM de contato intermitente somente o modo de amplitude de oscila��o est� dispon�vel, j� que neste caso a ponta toca a amostra.\par
	
	Uma quest�o freq�ente � o porque da oscila��o do cantilever. Isto explica-se pelo fato de que a for�a de Van der Waals n�o seria detect�vel se assim n�o o f�sse: A algumas dezenas ou centenas de �ngstrons acima da amostra, que � a regi�o onde atua o AFM de n�o contato, a for�a de Van der Waals � da ordem de $10^{-12}\unit{N}$, o que causa uma deflex�o da ordem de um cent�simo de �ngstron, incomensur�vel pelas t�cnicas atuais.\par
	
	Para funcionar corretamente, tanto o {\slshape tapping mode} quanto o AFM de n�o contato devem oscilar na freq��ncia de resson�ncia do cantilever, em geral por volta de $100\unit{kHz}$ a $400\unit{kHz}$. A amplitude, tamb�m de resson�ncia, � da ordem de $10\unit{\angstron}$ a $100\unit{\angstron}$. O pr�prio microsc�pio possui um sistema espec�fico para determinar a freq��ncia de resson�ncia do cantilever, mas existe uma sutileza: Determina-se a freq��ncia de resson�ncia com a ponta bastante longe da amostra para evitar qualquer toque acidental. Com isso, o amortecimento devido � for�a de Van der Waals n�o � levado em considera��o, o que significa que ao colocarmos a ponta para oscilar pr�ximo � superf�cie, a freq��ncia de resson�ncia ter� sido sutilmente alterada, prejudicando as medidas. Vejamos: $F=-k\Delta z$ � a for�a de restaura��o. Tome $F_w(z)$ como a for�a de Van der Waals, dependente da dist�ncia ponta-amostra. Ent�o, para um pequeno deslocamento $\Delta z$ teremos 
	$$
	F_w(z+\Delta z)\cong{\partial F_w(z)\over\partial z}\Delta z
	$$
	que pode ser combinada com a for�a de restaura��o da seguinte forma:
	$$
	F_{\mbox{total}}=F+F_w =-k\Delta z + {\partial F_w(z)\over\partial z}\Delta z =
		-\left[k - {\partial F_w(z)\over\partial z}\right]\Delta z = -k_{ef}\Delta z
	$$
	
	Pelo gr�fico da figura~\ref{fig:van-der-waals} vemos que $\partial F_w(z)/\partial z>0$, o que leva a $k_{ef}<k$, reduzindo a freq��ncia de resson�ncia:
	$$
	\omega^2 = {k_{ef}\over m} = {1\over m}\left({k-{\partial F_w\over\partial z}}\right) = 
		\omega_0^2\left(1-{1\over k}{\partial F_w\over\partial z}\right) \Rightarrow 
			\omega = \omega_0 \left(1-{1\over k}{\partial F_w\over\partial z}\right)^{1/2}
	$$
	que pode ser aproximado por
	$$
	\omega = \omega_0 \left(1-{1\over 2k}{\partial F_w\over\partial z}\right)
	$$
	considerando a derivada da for�a de Van der Waals pequena quando comparada � unidade. Experimentalmente sabemos que $\Delta\omega\cong\omega_0/10$. Isto nos leva a uma regra pr�tica e simples: Ap�s medirmos a freq��ncia de resson�ncia pelo microsc�pio, utilizamos esta freq��ncia acrescida de $10\%$ como um dos par�metros exigidos. Isto nos assegura que ao iniarmos as medidas, com a ponta pr�xima � amostra, estaremos bem pr�ximos da freq��ncia e resson�ncia.
	
	Nos modos AFM de n�o contato e tapping mode h� uma limita��o a mais quanto � constante el�stica do cantilever: Sabemos que ela n�o pode ser superior � constante el�stica da amostra, mas agora tamb�m devemos impor que a constante n�o seja muito pequena para que a varia��o $\Delta\omega$ seja pequena. Outro aspecto a se levar em considera��o nesses dois modos � o fator de qualidade $Q$ do cantilever:
	$$
	Q = 2\pi\left({\mbox{Energia armazenada por ciclo}\over\mbox{Energia dissipada por ciclo}}\right)={\omega_0\over\gamma}
	$$
	sendo $\gamma$ a constante de amortecimento, inerente ao cantilever. Quanto maior este n�mero, menor ser� a dissipa��o de energia no sistema.\par
	
	Vale a pena lembrar que como a for�a de Van der Waals, por ser essencialmente coulombiana, depende das constantes diel�tricas da ponta, amostra e meio em que est�o imersas. Em l�quidos polares, e.g., a for�a de Van der Waals � bastante reduzida, podendo at� causar a invers�o do sentido da intera��o. Ou seja, a regra passada acima, quanto � corre��o da freq��ncia de resson�ncia, embora coerente, n�o � sempre correta. O que nos leva novamente a lembrarmos que o julgamento do usu�rio est� acima de tudo.\par
	
	Finalmente, medidas nesses dois modos devem, sempre que poss�vel, serem feitas a baixas temperaturas pois a anergia vibracional da amostra a temperatura ambiente pode estar relacionada a amplitudes de vibra��o consider�veis. De fato, tomando a temperatura ambiente ($k_BT\cong 0,0258\unit{eV}$) com $k=1\unit{N/m}$ e teremos vibra��es de aproximadamente $0,6\unit{\angstron}$, que est� na ordem de grandeza mensur�vel pela t�cnica de n�o contato.\par
	
	Exemplos de imagens em AFM podem ser observados nos capitulos que seguem.
	
\chapter{MFM --- Microscopia de for�a magn�tica}
	\label{cap:mfm}\index{MFM}
	%	Projeto:  Scanning probe microscopy
%	Autor:	  Ivan Ramos Pagnossin
%	Data:		  16/06/2002
%	Arquivo:  mfm.tex
% Cap�tulo: Microscopia de for�a magn�tica

	Na microscopia de for�a magn�tica, como j� deve ter ficado claro pelo nome, a intera��o ponta-amostra � a magn�tica. Os princ�pios de funcionamento s�o todos id�nticos � microscopia de for�a at�mica de n�o contato, exceto que agora a intera��o deixa de ser a de Van der Waals (� claro que esta intera��o n�o deixa de existir; apenas torna-se de menor magnitude quando comparada � magn�tica) para ser a magn�tica: A ponta � recoberta por um filme ferromagn�tico imantado permanentemente.\par
	
	Como deduzimos no cap�tulo anterior, a freq��ncia de resson�ncia $\omega_0$ altera-se para
	$$
	\omega = \omega_0\,\left(1-{1\over2k}{\partial F^\perp\over\partial z}\right)
	$$
	onde modificamos um pouco a simbologia apenas para torn�-la mais corerente com a situa��o atual: $F^\perp=\vetor F\cdot\hat k$ � a componente perpendicular � superf�cie da for�a magn�tica entre ponta e amostra. Pela express�o acima fica claro que qualquer for�a paralela ao plano da amostra n�o influi sobre $\omega_0$, o que indica que a microscopia de for�a magn�tica s� � capaz de captar dom�nios magn�ticos perpendiculares � superf�cie da amostra.\par
	
	Algo t�o importante quanto � perceber que, diferentemente da for�a de Van der Waals, a for�a magn�tica pode ser tanto atrativa quanto repulsiva, o que significa que n�o tem sentido colocar como freq��ncia de resson�ncia $1,1\omega_0$ nem tampouco $0,9\omega_0$. No caso simplesmente encontramos a freq��ncia de resson�ncia e a utilizamos diretamente sem 
tratamento algum (A n�o ser, � claro, que se conhe�a a amostra).\par

\begin{figure}[htb]
	\begin{minipage}[c]{0.46\textwidth}\centering
		\includegraphics[scale=0.4]{figuras/mfm/mfm1.jpg}		
	\end{minipage}\hfill
	\begin{minipage}[c]{0.46\textwidth}\centering
		\includegraphics[scale=0.4]{figuras/mfm/mfm2.jpg}		
	\end{minipage}
	\caption{\label{fig:mfm}\footnotesize Imagem MFM de $25\unit{\mu m}\times25\unit{\mu m}$ de uma fita magn�tica comercial. � esquerda temos a medida em AFM de contato que permitiu isolar, � direita, os dom�nios magn�ticos na {\bfseries mesma} regi�o.}
\end{figure}
	
	Mas, durante a varredura, como poder� o microsc�pio distinguir morfologia da amostra e campo magn�tico? Para isso utiliza-se uma t�cnica chamada de {\slshape interleave}. \index{Interleave} Genericamente, o conceito de interleave em SPM indica a possibilidade de se fazer duas medidas simult�neas com t�cnicas diferentes. No caso, a sonda varre uma linha utilizando o AFM de contato (ou de contato intermitente), retorna e varre novamente a {\bfseries mesma linha} utilizando MFM. Retorna e passa para a linha seguinte, repetindo o processo at� o fim da medida (Note a diferen�a entre interleave e retrace). A vantagem � que na segunda leitura (da mesma linha) a sonda j� conhece a topografia, o que lhe permite manter-se a uma dist�ncia constante da amostra (que pode ser definida pelo usu�rio), eliminando quase que por completo a ambiguidade entre morfologia e campo magn�tico h� pouco discutida. Dito de outra forma, na segunda passagem da sonda o microsc�pio consulta em seu banco de dados a posi��o $(x,y,z)$ que ele acabou de obter por AFM de contato (ou de contato intermitente), posiciona-se ali e executa, em seguida, uma medida em MFM. Com isso ele (microsc�pio) consegue uma nova corre��o em $z$ que diz respeito somente � intera��o magn�tica entre a ponta e a amostra.\par
		
	Na figura~\ref{fig:mfm} temos uma medida MFM de uma fita magn�tica comercial. � esquerda a imagem em AFM de contato e � direita, MFM.
		
\chapter{LFM --- Microscopia de for�a lateral}
	\label{cap:lfm}\index{LFM}
	%	Projeto:  Scanning probe microscopy
%	Autor:	  Ivan Ramos Pagnossin
%	Data:		  16/06/2002
%	Arquivo:  lfm.tex
% Cap�tulo: Microscopia de for�a lateral

	A microscopia de for�a lateral � muito parecida � microscopia de for�a at�mica de contato. De fato, qualquer aparelho apto a fazer medidas AFM de contato pode executar LFM, incluse em modo interleave. Isto acontece por que a �nica diferen�a pr�tica entre um e outro � que no LFM ao inv�s de o fotodetector se preocupar com as deflex�es verticais do cantilever, ele mede as deflex�es laterais. Na verdade o fotodetector (fig.~\ref{fig:fotodetector}) mede ambas as deflex�es simultaneamente, no sentido de que quanto mais deslocado do centro estiver o laser, maior ser� a tens�o (em m�dulo) observada.

\begin{wrapfigure}{r}{0.4\textwidth}
	\centering
	\setlength{\unitlength}{.5cm}
	\begin{picture}(10,10)(-5,-5)
		\put(-5,+5){\line(1,0){10}}
		\put(-5,+0){\line(1,0){10}}
		\put(-5,-5){\line(1,0){10}}
		\put(-5,-5){\line(0,1){10}}
		\put(+0,-5){\line(0,1){10}}
		\put(+5,-5){\line(0,1){10}}
		\put(-2.5,-2.5){\makebox(0,0){B}}
		\put(-2.5,+2.5){\makebox(0,0){A}}
		\put(+2.5,-2.5){\makebox(0,0){D}}
		\put(+2.5,+2.5){\makebox(0,0){C}}
		\put(1,.75){\line(0,1){0.5}}
		\put(.75,1){\line(1,0){0.5}}
	\end{picture}
	\caption{\label{fig:fotodetector}\footnotesize Esquema da setoriza��o do fotodetector. A cruz deslocada do centro indica onde o laser est� atingindo o detector. Quanto mais deslocado do centro, maior a tens�o. Em LFM as tens�es $V_C+V_D$ e $V_A+V_B$ indicam a deflex�o. Nas outras modalidades, � $V_A+V_C$ e $V_B+V_D$.}
\end{wrapfigure}
				
	A deflex�o lateral do cantilever deve-se �s for�as de atrito exis-tentes entre a ponta e a amostra que, como aprendemos desde cedo no prim�rio, depende --- macroscopicamente falando --- da for�a normal exercida pela amostra sobre a ponta e pelo coeficiente de atrito $\mu$: $|\vetor F_{\text{at}}| = \mu |\vetor N|$. Essa for�a de atrito gera um torque $\vec\tau = \vetor r\times\vetor F_{\text{at}}$ que torce o cantilever, deslocando late-ralmente o feixe de laser refletido. Mantendo fixa a normal, que est� diretamente ligada ao setpoint escolhido, as medidas dar�o uma representa��o de como $\mu$ varia ponto a ponto na amostra. Contudo (e h� sempre um contudo), tamb�m � poss�vel defletirmos lateralmente o cantilever sem que haja mudan�a em $\mu$, sobretudo nas bordas entre regi�es mais baixas e mais altas (pois nesses pontos a sonda ``enrosca'' nessas bordas. Para evitar isso utilizamos novamente o conceito de interleave, varrendo a amostra com AFM de contato e re-varrendo a mesma linha com o conhecimento adquirido na varredura anterior. Mas diferentemente do MFM, que sempre trabalha em interleave, o LFM permite-nos n�o utiliz�-lo. Com isso podemos, ao inv�s de medirmos as varia��es regionais de $\mu$, destacar as bordas da amostra. Ou seja, o que era um problema acabou se tornando um aliado pois temos agora duas microscopias diferentes com exatamente o mesmo sistema.\par
	
	Algo que precisa ser lembrado sempre ao se utilizar este m�todo � que a varredura r�pida, neste caso, deve ser posicionada de forma perpendicular ao eixo maior do cantilever. Caso contr�rio as deflex�es por for�as laterais sobrepor-se-�o �quelas devidas � topografia (na linha vertical do fotodetector, fig.~\ref{fig:fotodetector}). O par�metro, no software da Digital Instruments, \index{Digital Instruments}\index{DI} e.g., que nos permite ajustar tal �ngulo � o {\slshape scan angle}. \index{Scan angle}\par
	
	Como j� deve ter ocorrido ao leitor, uma vez que o torque (e, claro, a tor��o do cantilever) depende da normal, quanto mais fortemente pressionarmos a ponta sobre a amostra, maior ser� o sinal recebido. Isso � de fato verdade mas deve ser utilizado com cautela pois na mesma medida podemos danificar a amostra.
	
\begin{figure}[htb]
	\begin{minipage}[c]{0.46\textwidth}\centering
		\includegraphics[scale=0.4]{figuras/lfm/lfm1.jpg}		
	\end{minipage}\hfill
	\begin{minipage}[c]{0.46\textwidth}\centering
		\includegraphics[scale=0.4]{figuras/lfm/lfm2.jpg}		
	\end{minipage}
	\caption{\label{fig:lfm}\footnotesize Imagem de $15\unit{\mu m}\times 15\unit{\mu m}$ de uma borracha escolar. � esquerda temos a medida em AFM de contato e � direita LFM. Nestas imagens n�o utilizamos o interleave. Ao inv�s disso, ilustramos as deflex�es (direita) devido tanto �s varia��es de $\mu$ (regi�es planas � esquerda mas n�o � direita) como �quelas devidas � pr�pria topologia (destaque).}
\end{figure}
	
	Segue agora uma breve explica��o sobre o funcionamento do fotodetector que � v�lida para todas as t�cnicas de SPM que o utilizam.\par
	
	Na figura~\ref{fig:fotodetector} temos uma ilustra��o simplificada da setoriza��o que � feita sobre o fotodetector. A pequena cruz indica o ponto onde o laser refletido pelo cantilever atinge-o. Durante uma medida AFM as deflex�es do cantilever fazem o laser (cruz) variar sua posi��o sobre o fotodetector somente na dire��o vertical, enquanto numa medida LFM o laser varia horizontalmente sua posi��o. Fica ent�o claro como � poss�vel realizarmos medidas de AFM e LFM simultaneamente. Em termos pr�ticos, o microsc�pio, ao avaliar o curso vertical do laser (e, portanto, a deflex�o relativa ao AFM), mede a tens�o $(V_A+V_C)-(V_B+V_D)$. Analogamente, durante o curso horizontal do laser (relativa ao LFM), a tens�o medida � a $(V_A+V_B)-(V_C+V_D)$. A tens�o oposta � medida � a que deve ser aplicada ao atuador piezoel�trico para restaurar o setpoint.\par
	
	Aproveitando o ensejo, toda vez antes de efetuarmos uma medida, precisamos ``alinhar o lase''. Ou seja, precisamos fazer com que a ponta, pouco antes de tocar a amostra, reflita o laser sobre o centro do fotodetector. Isto � feito analisando justamente as tens�es comentadas acima. Existe tr�s displays num microsc�pio AFM/LFM: O primeiro refere-se � tens�o $(V_A+V_C)-(V_B+V_D)$, o segundo a $(V_A+V_B)-(V_C+V_D)$ e o terceiro � soma total: $V_A+V_B+V_C+V_D$. Os dois primeiros devem ser manualmente zerados (atrav�s de bot�es de ajustes), enquanto o �ltimo deve ser maximizado. A �ltima leitura d� conta de garantir que o laser n�o est� incidindo fora do fotodetector, enquanto as outras, como acabamos de ver, responsabilizam-se por centralizar o laser. Isto feito o microsc�pio est� pronto para medir.
	
\chapter{Outros modos de SPM}
	\label{cap:others-modes}
	%	Projeto:  Scanning probe microscopy
%	Autor:	  Ivan Ramos Pagnossin
%	Data:		  17/06/2002
%	Arquivo:  others.tex
% Cap�tulo: Outros modos de SPM

	\begin{description}
		\item{\bfseries FMM --- Microscopia de for�a modulada:}\index{FMM}\index{Microscopia!de for�a modulada}
		Esta t�cnica de microscopia nos permite avaliar as varia��es de rigidez \index{Rigidez} da amostra. Para isso faz-se oscilar a piezoel�trica --- em sua freq��ncia de resson�ncia --- utilizando como ponto de apoio o contato ponta-amostra (que � feito no sentido do AFM de contato). Se o material analisado for r�gido o ponto de apoio mant�m-se fixo e a amplitude de oscila��o aumenta. Se, por outro lado, a rigidez n�o for das maiores, este ponto oscila (pois a ponta deforma a amostra naquele ponto) juntamente com a piezoel�trica (mas n�o em fase), reduzindo a amplitude. Deste modo, vemos que grandes amplitudes referem-se a regi�es mais r�gidas, enquanto menores amplitudes dizem respeito a regi�es mais ``macias''.\par
		\item{\bfseries EFM --- Microscopia de for�a el�trica:}\index{EFM}\index{Microscopia!de for�a el�trica}
		Id�ntica ao MFM, exceto por utilizar a for�a el�trica (gerada por uma diferen�a de potencial aplicada entre ponta e amostra) como intera��o ponta-amostra.
		\item{\bfseries SCM --- Microscopia de capacit�ncia:}\index{SCM}\index{Microscopia!de capacit�ncia}
		Esta modalidade de microscopia mede a varia��o da capacit�ncia \index{Capacit�ncia} ao longo da amostra: Aplica-se uma tens�o cont�nua entre amostra e ponta, que oscila sem encostar na superf�cie (AFM de n�o-contato). Pode-se utilizar o interleave para evitar ambiguidades com a topografia. Este tipo de microscopia nos permite observar a distribui��o de portadores \index{portadores} pouco abaixo da superf�cie ou ainda analisar uma amostra diel�trica \index{Diel�trico} depositada sobre um substrato condutor.
		\item{\bfseries TSM --- Microscopia t�rmica:}\index{TSM}\index{Microscopia!t�rmica}
		Avalia a condutividade t�rmica \index{Condutividade t�rmica} da superf�cie da amostra ou, em alguns casos, em planos at�micos abaixo da superf�cie. A sonda neste tipo de microscopia � um pequeno termopar, \index{Termopar} aquecido por uma fonte externa: Ao se aproximar da amostra, calor � transferido para ela e, obviamente, a velocidade de varia��o da temperatura da ponta (termopar) � proporcional � condutividade.\par
		� claro que o meio no qual est�o imersos a ponta e a amostra influi nos resultados. Para medidas feitas em ambiente aberto, o caminho livre m�dio das mol�culas \index{Caminho livre m�dio} que comp�em o ar � de aproximadamente $l=66\unit{nm}$. Portanto, para dist�ncias ponta-amostra maiores que $l$ a condu��o por correntes de convec��o \index{Corrente de convec��o} predomina. Para dist�ncias inferiores teremos dom�nio das transfer�ncias por radia��o, \index{Radia��o} principalmente infravermelho. \index{Infravermelho} Para evitar influ�ncias do meio, em geral faz-se medidas TSM (n�o confunda com STM) com o termopar acoplado a um cantilever oscilante.\par
		A resolu��o desta t�cnica � da ordem de $40\unit{nm}$ a $100\unit{nm}$, variando com a dist�ncia ponta-amostra e a dimens�o da ponta. A resolu��o vertical � por volta de $3\unit{nm}$ e em temperatura � da ordem de $3\unit{mK}$ (correspondente a alguns nanowatts).
		\item{\bfseries NSOM --- Microscopia de campo pr�ximo:}\index{NSOM}\index{Microscopia!de campo pr�ximo}
		NSOM � o acr�nimo de {\slshape near-field scanning optical microscopy} e trata-se de uma t�cnica �ptica recente que consegue, com radia��o de comprimento $\lambda$, resolver detalhes de at� aproximadamente $\lambda/20$. A t�cnica � interessante pois consegue burlar o limite de resolu��o \index{Resolu��o} dos microsc�pios �pticos tradicionais. Para isso, a extremidade da amostra � fabricada com uma pequena abertura muito menor que $\lambda$ e iluminada por laser. 
		\item{\bfseries Nanolitografia:}\index{Nanolitografia}
		Em geral utilizamos as t�cnicas de SPM para obter informa��es a respeito de uma superf�cie. As vezes, entretanto, t�m-se o interesse de modific�-las intensionalmente de forma a produzir marcas espec�ficas na amostra. Isto � feito utilizando-se as t�cnicas de AFM de contato com setpoint bastante algo. Ou seja, com alta press�o sobre a amostra. A este tipo de atividade chamamos nanolitografia.
\end{description}

\begin{figure}[htb]
	\begin{minipage}[c]{0.46\textwidth}\centering
		\includegraphics[scale=0.4]{figuras/others/fmm2.jpg}		
	\end{minipage}\hfill
	\begin{minipage}[c]{0.46\textwidth}\centering
		\includegraphics[scale=0.4]{figuras/others/fmm3.jpg}		
	\end{minipage}
	\caption{\footnotesize Imagem FMM ($20\unit{\mu m}\times 20\unit{\mu m}$) de uma borracha escolar. � esquerda a imagem em AFM de contato; � direita em FMM.}
\end{figure}

\begin{figure}\centering
	\includegraphics[scale=0.5]{figuras/others/fmm1.jpg}
	\caption{\footnotesize Esquema de funcionamento do FMM.}
\end{figure}
\chapter{Exemplo de aplica��o: Pontos qu�nticos}
	\label{cap:quantum-dots}\index{QD}\index{Quantum-dots}
	\section{Introdu��o}

	Bloch \index{Bloch} introduziu, no final da d�cada de 20, o conceito de estrutura de bandas \index{Bandas} para s�lidos cristalinos, causando uma
verdadeira revolu��o no ent�o mundo f�sico preocupado com os estudos dos �tomos por meios da tamb�m recente {\itshape teoria qu�ntica nova}: Nos �tomos, as energias dos el�trons confinados s�o discretas e, at� onde permite o princ�pio da incerteza, bem determinadas. Em contrapartida, nos s�lidos ({\itshape bulks}) \index{Bulks} a energia de um el�tron � uma fun��o n�o un�voca do momento, resultando nas bandas cont�nuas, {\itshape gaps} \index{Gap} e densidades cont�nuas de estado. \index{Densidade de estado} E a despeito da aproxima��o para uma rede infinita, nenhum desacordo experimental foi encontrado.
	
	Se um portador num s�lido for confinado em alguma dimens�o da ordem de seu comprimento de onda de {\itshape de Broglie} ($\lambda = h/p$) \index{de Broglie!Comprimento de onda de} ent�o � poss�vel verificar fen�menos qu�nticos. Mas devido � organiza��o espacial desse material, a massa a ser utilizada no c�lculo �, na realidade, a
massa efetiva do portador, em geral muito menor que a sua massa de repouso. Com isso a escala na qual o fen�meno 
qu�ntico � observ�vel torna-se maior; tipicamente v�rias dezenas de vezes. Esta descoberta encorajou muitos pesquisadores a construir materiais que confinassem portadores em camadas muito finas: Entre as d�cadas de 50 e 60
isto era feito com filmes de semi-metais depositados sobre substratos de mica. Mas evoluiu-se pouco neste sentido.

	Um efeito importante observado neste contexto foi o aumento do {\itshape gap} de energia com a redu��o da espessura da camada confinadora.
	
	Este tipo de pesquisa permaneceu tecnologicamente bastante limitada at� o final da d�cada de 60 com o advento do crescimento molecular epitaxial, \index{MBE} uma t�cnica capaz de crescer cristais a uma taxa de poucas camadas monoat�micas (monocamadas). \index{Monocamada} Foi a gota d��gua. A partir da� tornou-se poss�vel confinar bidimensionamente um g�s de portadores e heteroestruturas cada vez mais complexas ({\itshape superlattices}) \index{Superlattice} foram sendo propostas; Em 1974 {\itshape Chang et al.} observou o efeito de tunelamento ressonante, provando a validade da teoria qu�ntica para explicar fen�menos de transporte em camadas muito finas de heteroestruturas.

\section{Pontos qu�nticos}

	Ao final da d�cada de 80 as principais propriedades das {\itshape superlattices} e po�os qu�nticos \index{Po�os qu�nticos} j� eram conhecidas e o interesse voltou-se para estruturas ainda mais confinadas: Os fios e os pontos qu�nticos (por volta de 1989). \index{Pontos qu�nticos}\index{Fios qu�nticos} O completo confinamento dos portadores levou � localiza��o espacial desses portadores e � quebra da estrutura de bandas do s�lido. Estudos da �poca previram a utiliza��o dos pontos qu�nticos em novos dispositivos opto-eletr�nicos tais como lasers\cite{harris}, mem�rias\cite{yusa} e detectores de infravermelho\cite{maimon}, entre outras.
	
\section{SIQD - Stress induced quantum dots}\index{SIQD}

	Existem v�rias formas de se construir pontos qu�nticos (veja \cite[p�g. 5]{bimberg}) mas atualmente a mais promissora
� aquela que se aproveita do processo de auto-organiza��o descoberto no in�cio da d�cada de 90: Crescimento molecular
epitaxial ({\itshape Molecular beam epitaxy} --- MBE): Basicamente o que se faz � depositar, monocamada por monocamada, sobre um substrato de par�metro de rede $a$ uma estrutura com par�metro de rede $b>a$. \index{Par�metro de rede} Com isso a estrutura depositada tende a se ajustar ao par�metro de rede anterior, ao custo de um ac�lumo de energia potencial na forma de tens�es mec�nicas. Este processo sustenta-se at� aproximadamente $1,5$ monocamadas. Al�m disso a energia armazenada � tamanha que a rede depositada se quebra e os �tomos passam a se agrupar em "montes": Os pontos qu�nticos. 

	Um ponto qu�ntico t�pico possui um di�metro da ordem de $10\unit{nm}$ e cont�m aproximadamente $10^4$ �tomos.
	
\section{T�cnica a ser utilizada}

	O que pretendemos observar � justamente o tamanho e a forma desses pontos qu�nticos como fun��o da quantidade de material depositado. Em princ�pio tanto STM como AFM s�o t�cnicas aceit�veis. Entrementes, o modo STM apresenta uma desvantagem com rela��o ao AFM: A freq�ente deposi��o de �xido na superf�cie do material isola-o eletricamente tornando necess�rio um pr�-tratamento da superf�cie, o que n�o seria necess�rio no caso do AFM.
	
	Entre os modos AFM dispon�veis, escolhemos o de contato por ser o mais simples. Contudo, havendo tempo dispon�vel,
outra inte��o deste trabalho � explorar as diferen�as entre os tr�s modos de AFM numa mesma imagem.

\vfill
\pagebreak

\section{Pontos qu�nticos de InAs crescidos sobre InGaAs, $2,4\unit{ML}$}

\begin{figure}[htb]
	\centering
	\includegraphics[scale=0.3]{figuras/quantum-dots/InGaAs-InAs-2-40ML.jpg}
	\caption{\footnotesize A imagem acima � resultado de uma medida em AFM de contato intermitente ($1\unit{\mu m}\times 1\unit{\mu m}$) e v�-se aproximadamente $900$ pontos qu�nticos (bolinhas em branco) com raio e altura m�dias de $13(3)\unit{nm}$ e $5(1)\unit{nm}$, respectivamente.}
\end{figure}

\begin{figure}[htb]
	\begin{minipage}[c]{0.46\textwidth}
		\centering
		\includegraphics[scale=0.75]{figuras/quantum-dots/InGaAs-InAs-h.jpg}		
		\vspace{5mm}		
		\includegraphics[scale=0.75]{figuras/quantum-dots/InGaAs-InAs-hr.jpg}		
	\end{minipage}\hfill
	\begin{minipage}[c]{0.46\textwidth}
		\centering
		\includegraphics[scale=0.75]{figuras/quantum-dots/InGaAs-InAs-r.jpg}		
		\vspace{5mm}		
		\includegraphics[scale=0.75]{figuras/quantum-dots/InGaAs-InAs-av.jpg}		
	\end{minipage}
\end{figure}

\vfill\pagebreak
\section{Pontos qu�nticos de InAs crescidos sobre GaAs, $1,75\unit{ML}$}

\begin{figure}[htb]
	\centering
	\includegraphics[scale=0.3]{figuras/quantum-dots/GaAs-InAs-1-75ML.jpg}
	\caption{\footnotesize A imagem acima � resultado de uma medida em AFM de contato ($1\unit{\mu m}\times 1\unit{\mu m}$) e v�-se aproximadamente $160$ pontos qu�nticos (bolinhas em branco) com raio e altura m�dias de $14(2)\unit{nm}$ e $5,2(9)\unit{nm}$, respectivamente.}
\end{figure}

\begin{figure}[htb]
	\begin{minipage}[c]{0.46\textwidth}
		\centering
		\includegraphics[scale=0.75]{figuras/quantum-dots/GaAs-InAs-h.jpg}
		\vspace{5mm}		
		\includegraphics[scale=0.75]{figuras/quantum-dots/GaAs-InAs-hr.jpg}		
	\end{minipage}\hfill
	\begin{minipage}[c]{0.46\textwidth}
		\centering
		\includegraphics[scale=0.75]{figuras/quantum-dots/GaAs-InAs-r.jpg}		
		\vspace{5mm}		
		\includegraphics[scale=0.75]{figuras/quantum-dots/GaAs-InAs-av.jpg}		
	\end{minipage}
\end{figure}


\begin{thebibliography}{9}
	\bibitem{eisberg}		R. Eisberg, R. Resnick, {\itshape F�sica qu�ntica}, ed. Campus, 1979
	\bibitem{roland}		R. Wiesendanger, {\itshape Scanning probe microscopy and spectroscopy}, ed. Cambridge, 1994
	\bibitem{salvadori}	M. C. Salvadori, {\itshape Notas de aula do curso de microscopia de for�a at�mica e tunelamento}
	\bibitem{bimberg}	D. Bimberg, M. Grundmann, N. N. Ledentsov, {\itshape Quantum dot heterostructures}, ed. Wiley,                         1999.
	\bibitem{mike}		M. J. Silva, {\itshape Crescimento e caracteriza��o de pontos qu�nticos de InAs auto-formados},
										Disserta��o de mestrado, 1999.
	\bibitem{maimon}	M. Maimon, E. Finkman, G. Bahir, S. E. Schacham, J. M. Garcia e P. M. Petroff, Appl. Phys. Lett.
                    {\bfseries 73}, 2003 (1998).
	\bibitem{harris}	L. Harris, D. J. Mawbray, M. S. Skolnick, M. Hopkinson e G. Hill, Appl. Phys. Lett. {\bfseries 73},
                    969 (1998).
  \bibitem{yusa}		G. Yusa e H. Sakaki, Appl. Phys. Lett. {\bfseries 70}, 345 (1997).
  \bibitem{arakawa}	Y. Arakawa, H. Sakaki, Appl. Phys. Lett., {\bfseries 40}, 939 (1982)
  \bibitem{pi}	http://www.physikinstrumente.com/tutorial/
\end{thebibliography}

\printindex
\end{document}
 


