\begin{exercicio}[blue]
	Obtenha a energia m�nima que deve ter um f�ton para que seja poss�vel arrancar um n�utron do
n�cleo do deut�rio:
	$$
	\el{2}{1}{H}{}{}\,+\,\gamma\,\to\,\mathrm{n}\,+\,\el{1}{1}{H}{}{}
	$$

	A rea��o inversa � chamada rea��o de captura de n�utrons pelo hidrog�nio, que ocorre em 
blindagens, juntamente com a termaliza��o dos n�utrons.
\end{exercicio}

	As massas dos reagentes e produtos na equa��o acima s�o
	\begin{eqnarray*}
	\mathrm{m(^2H)}	&=& 2,014\,102\,\mathrm{u}	\\
	\mathrm{m(n)}	&=& 1,008\,665\,\mathrm{u}	\\
	\mathrm{m(^1H)}	&=& 1,007\,825\,\mathrm{u}
	\end{eqnarray*}

	Assim, como a energia dos reagentes deve ser maior que a dos produtos 
($1878,227\,992\,\mathrm{MeV}$), devemos ter um f�ton com energia de maior que 
$1878,227\,992 - 1876,003\,723 = 2,224\,265\,\mathrm{MeV}$ para que a re��o possa ocorrer.
