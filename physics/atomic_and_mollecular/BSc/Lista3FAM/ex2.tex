\begin{quotation}
{\bf 2.}
{\it	Seja $\psi(x)$ uma fun��o de onda real com paridade definida, isto �, $\psi(-x)=\psi(x)$ (par) ou
	$\psi(-x)=-\psi(x)$ (�mpar). Mostre que nesse caso tanto o valor esperado da posi��o $\med{\op X}$ 
	quanto do momento $\med{\op P}$ s�o nulos.}
\end{quotation}

	O valor esperado da posi��o � simplesmente
	\begin{eqnarray*}
	\med{\op X}	&=& \hope{\psi}{\op X}{\psi}							\\
			&=& \intii \psi^*(x)x\psi(x)dx							\\
			&=& \int_{-\infty}^0 \psi^*(x)x\psi(x)dx+\int_0^{\infty} \psi^*(x)x\psi(x)dx	\\
			&=& -\int_0^{-\infty} \psi^*(x)x\psi(x)dx+\int_0^{\infty} \psi^*(x)x\psi(x)dx	 
	\end{eqnarray*}

	Fazendo a substitui��o $x=-u$ na primeira integral ficamos com
	$$
	\med{\op X}=-\int_0^{\infty} \psi^*(-u)u\psi(-u)du+\int_0^{\infty} \psi^*(x)x\psi(x)dx
	$$

	Agora note que, seja $\psi(x)$ para ou �mpar, a identidade anterior n�o se altera pois
	\begin{eqnarray*}
	\mathrm{(Par)}&:&\,\,\,\, \psi^*(-u)\psi(-u)=[+\psi^*(u)][+\psi(u)]=\psi^*(u)\psi(u) 	\\
	\mathrm{(\mbox{�}mpar)}&:&\,\,\,\, \psi^*(-u)\psi(-u)=[-\psi^*(u)][-\psi(u)]=\psi^*(u)\psi(u)
	\end{eqnarray*}

	Ent�o fica claro, lembrando-se que $u$ e $x$ s�o apenas r�tulos para o argumento da fun��o de onda, 
que $\med{\op X}$ deve se anular. Caso bastante semelhante ocorre para o operador momento:
	\begin{eqnarray*}
	\med{\op P}	&=& \hope{\psi}{\op P}{\psi}							\\
			&=& -i\hbar\intii dx\,\,\psi^*(x){d\over dx}\psi(x)				\\
			&=& -i\hbar\int_{-\infty}^0 dx\,\,\psi^*(x){d\over dx}\psi(x)
				-i\hbar\int_o^{\infty} dx\,\,\psi^*(x){d\over dx}\psi(x)		\\
			&=& i\hbar\int_0^{-\infty} dx\,\,\psi^*(x){d\over dx}\psi(x)
				-i\hbar\int_0^{\infty} dx\,\,\psi^*(x){d\over dx}\psi(x)		
	\end{eqnarray*}

	Utilizando novamente a mudan�a de vari�vel $x=-u$ na primeira integral e verificando que
	$$
	{d\over dx}=-{d\over du}
	$$

	chegamos a
	$$
	\med{\op P}=i\hbar\int_0^{\infty} du\,\,\psi^*(-u){d\over du}\psi(-u)
				-i\hbar\int_0^{\infty} dx\,\,\psi^*(x){d\over dx}\psi(x)		
	$$   
	
	e outra vez, pelos mesmos argumentos, conclu�mos que $\med{\op P}=0$
			
			
	