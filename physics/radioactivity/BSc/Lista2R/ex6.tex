\begin{exercicio}[blue]
	Um feixe de f�tons com energia de $10,22\unit{keV}$ sofre um espalhamento Compton.
Calcule a m�nima energia da radia��o espalhada e a m�xima energia do el�tron de recuo.
Repita o exerc�cio supondo que os f�tons incidentes t�m enegia de $1,022\unit{MeV}$.
\end{exercicio}

	O comprimento de onda do f�ton espalhado difere da do f�ton incidente por uma quantidade 
$\Delta\lambda$ dada por
	$$
	\Delta\lambda=\lambda_c(1-\cos\theta),\qquad \lambda_c={h\over m_e c}
	$$

	O feixe espalhado com menor energia e, consequentemente, o el�tron de recuo com maior energia
ocorre para $\theta=\pi$, que corresponde a uma diferen�a $\Delta\lambda=2\lambda_c$.\par
	A energia de um f�ton incidente � dada por $E_0=hc/\lambda_0$, onde $\lambda_0$ � o 
comprimento de onda deste f�ton. Inversamente, $\lambda_0=hc/E_0$. Isto posto, e incluindo a varia��o
conhecida $\Delta\lambda$, notamos que o f�ton espalhado possui comprimento de onda 
$\lambda=\lambda_0+\Delta\lambda=\lambda_0+2\lambda_c$ e, consequentemente, energia
	$$
	E={hc\over\lambda_0+2\lambda_c}={hc\over{hc\over E_0}+2\lambda_c}={E_0\over1+{2E_0\over\mu_e}}
	$$
	onde $\mu_e$ � a massa de repouso do el�tron em MeV. Ou seja,
	$$
	E={E_0\over1+3,91E_0},\qquad [E_0]=\mathrm{MeV}
	$$

	A energia do el�tron de recuo �, pela conserva��o de energia, $E_e=E_0-E$. Ent�o, para
$E_0=10,22\unit{keV}$ temos $E=9,826\unit{keV}$ ($96,15\,\%$) e $E_e=0,393\unit{keV}$ ($3,85\,\%$); E,
para $E_0=1,022\unit{MeV}$, $E=204,399\unit{keV}$ ($20,00\,\%$) e $E_e=817,600\unit{keV}$ ($80,00\,\%$). Os 
valores entre par�nteses indicam a porcentagem de ocorr�ncia daquele efeito.
