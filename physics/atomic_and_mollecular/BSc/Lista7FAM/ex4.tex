\begin{exercicio}[blue]
	Um el�tron no �tomo de hidrog�nio encontra-se no estado $\ket{21-1}=\ket{2p_-}$. 
	
	{\tt (a)} Escreva a fun��o de onda $\psi_{2p_-}(\mathbf r)$.
	
	{\tt (b)} Determine a distribui��o de probabilidades radia $D_{2p}(r)$ de4 se encontrar o el�tron a uma dist�ncia $r$ do n�cleo.
	
	{\tt (c)} Calcule $\med{r}$, $\med{r^2}$ e $\med{{1\over r}}$. Verifique o �ltimo resultado comparando com aquele encontrado no item {\tt (b)} do exerc�cio anterior.

\end{exercicio}

	{\tt (a)} A express�o geral para as fun��es de onda do �tomo de hidrog�nio �
	$$
	\psi_{nlm}(r,\theta,\phi) = R_{nl}(r)Y_{lm}(\theta,\phi)
	$$
	
	Para $\ket{21-1}$ temos, obviamente, $n=2$, $l=1$ e $m=-1$ e disto imediatamente encontramos\footnote{Basta buscar o problema de Sturm--Liouville em coordenadas esf�ricas em algum livro de f�sica--matem�tica ou resolver a equa��o.}
	\begin{eqnarray*}
	R_{21}(r) 						&=& {1\over\sqrt{3}}({Z\over2a_0})^{3/2}({Zr\over a_0})e^{-Zr/2a_0}	\\
	Y_{1-1}(\theta,\phi)	&=& ({3\over8\pi})^{1/2}e^{-i\phi}\sin\theta
	\end{eqnarray*}
	
	O resto das contas, que n�o passa de substitui��o, deixo ao leitor interessado.
	
	{\tt (b)} A distribui��o de probabilidade $D_{nl}(r)$ de encontrarmos o el�tron a uma dist�ncia $r$
do centro �, por defini��o,
	$$
	D_{nl}(r) dr = dr\int_{\theta = 0}^{\pi}\int_{\phi = 0}^{2\pi}\psi^*_{nlm}(r,\theta,\phi)
				\psi_{nlm}(r,\theta,\phi) r^2\sin\theta\mathrm{d}\phi\mathrm{d}\theta
	$$
	
	Como $|\psi_{nlm}(r,\theta,\phi)|^2=|R_{nl}(r)|^2|Y_{lm}(\theta,\phi)|^2$ e a parte angular por si
s�, quando integrada na superf�cie de uma esfera de raio $r$, resulta $1$, fica claro que a equa��o
acima � equivalente a 
	$$
	D_{nl}(r) = r^2|R_{nl}(r)|^2
	$$
	
	No nosso caso, como $n=2$ e $l=1$, 
	$$
	D_{21}(r)dr = {1\over\sqrt{3}}({Z\over2a_0})^{3/2}({Zr^3\over a_0})e^{-Zr/2a_0}dr
	
	{\tt (c)} Ao inv�s de calcularmos tr�s integrais, vale mais a pena utilizarmos um expoente gen�rico $\mu$:
	\begin{eqnarray*}
	\med{r^\mu}	&=& \int_0^\infty r^\mu D_{nl}(r)dr = \int_0^\infty r^{\mu+2}|R_{nl}(r)|^2dr = 	\\
							&=& \int_0^\infty r^{\mu+2}|R_{21}|^2 = \int_0^\infty r^{\mu+2}
										{1\over3}{Z^3\over8a_0^3}{Z^2r^2\over a_0^2}e^{-Zr/a_0}dr									\\
							&=& {Z^5\over24a_0^5}\int_0^\infty r^{\mu+4}e^{-Zr/a_0}dr 											\\
							&=& {Z^5\over24a_0^5}({a_0\over A})^{\mu+5}\int_0^\infty \lambda^{\mu+4}
										e^{-\lambda}\mathrm{d}\lambda																							\\
							&=& {1\over24}({a_0\over Z})^\mu\Gamma(\mu+5)																		\\
							&=& {(n+4)!\over24}({a_0\over Z})^\mu
	\end{eqnarray*}
	
	com $\mu\ge-4$ e $\mu\in\mathsymbol{Z}$. Ent�o quando $\mu=1$ teremos $\med{r} = 5a_0$; quando
$\mu=2$ encontramos $\med{r^2}=30a_0^2$. E, finalmente, para $\mu=-1$, $\med{1/r}=1/4a_0$. 

	Neste ponto � interessante comparar os resultados para $\med{1/r}$ obtido aqui e no item {\tt (b)} do exerc�cio anterior: Aqui, como $n=2$, vemos que podemos expressar $\med{1/r}$ como $1/2^2a_0$, que � o mesmo resultado obtido anteriormente.