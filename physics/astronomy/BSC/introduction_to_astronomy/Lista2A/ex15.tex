\begin{quotation}
{\bf 15.}
{\it	Considere uma bin�ria visual cuja �rbita � circular e de inclina��o zero. Foram
	determinados: (i) Seu per�odo, de $8$ anos, (ii) sua m�xima separa��o angular de
	$3''$ e (iii) sua dist�ncia de $2\,\,\mathrm{pc}$. {\tt (a)} Calcule a soma das
	massas das estrelas dessa bin�ria. {\tt (b)} Sabendo que, com rela��o ao centro 
	de massa, a secund�ria est� a uma dist�ncia $2$ vezes maior que a dist�ncia da
	prim�ria, determine a massa de cada estrela}
\end{quotation}

	{\tt (a)} O valor da maior separa��o, em unidades astron�micas, � dado simplesmente 
pela raz�o da maior separa��o em segundos de arco e da paralaxe trigonom�trica $\Pi$:
	$$
	a(\mathrm{UA}) = {a('')\over\Pi}
	$$

	Mas sabemos que a paralaxe trigonom�trica, quando dada em segundos de arco, 
relaciona-se com a dist�ncia, em parsecs, por
	$$
	\Pi('') = {1\over d(\mathrm{pc})}
	$$

	Logo, levando esta �ltima na primeira,
	$$
	a(\mathrm{UA}) = a('')\cdot d(\mathrm{pc}) = 3\cdot 2 = 6\,\,\mathrm{UA}
	$$

	Mas, pela terceira Lei de Kepler,
	$$
	m_1+m_2 = {a^3\over P^2}
	$$

	desde que $[m_i]=M_{\bigodot}$, $[a]=\mathrm{UA}$ e $[P]=\mathrm{anos}$. Ent�o
	\begin{equation}
	m_1+m_2 = {6\,\,\mathrm{UA}\over8\,\,\mathrm{anos}} = 3,375\,\,\mathrm{M_{\bigodot}}
	\label{eq:massa total}
	\end{equation} 

	{\tt (b)} A posi��o relativa de cada componente do sistema com rela��o ao centro de
massa j� nos � conhecida desde a primeira lista:
	$$
	m_1r_1=m_2r_2
	$$

	Ent�o, como $r_2=2r_1$, $m_1=2m_2$. Isto, aplicado na equa��o (\ref{eq:massa total}),
d� que $m_2={3,375/3}=1,125\,\,\mathrm{M_{\bigodot}}$ e, consequentemente, 
$m_1=2,25\,\,\mathrm{M_{\bigodot}}$.  






