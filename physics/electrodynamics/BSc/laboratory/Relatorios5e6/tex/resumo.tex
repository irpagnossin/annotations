	No presente experimento procuramos evidenciar v�rias leis utilizadas na �ptica geom�trica e
�ptica f�sica. Come�amos por verificar a {\it lei dos pontos conjugados} determinando, para isso,
a dist�ncia focal de uma lente convergente. Em seguida comparamos duas formas diferentes de se calcular 
a {\it amplifica��o linear} desta mesma lente: $m=-i/o$ e $m=h_i/h_o$. A lente divergente, que n�o forma 
imagem
projet�vel, tornou-se um bom exerc�cio no que tange as conven��es ``estranhas'' utilizadas pela �ptica
geom�trica durante a determina��o indireta de sua dist�ncia focal. A lupa, outro instrumento bastante
�til e conhecido foi tamb�m avaliado. Mais interessante ainda foi o estudo do ``efeito lupa'' que um orif�cio
pequeno assume. Em seguida, e j� passando para a �ptica f�sica, estudamos a polariza��o de um laser de
He--Ne, o {\it �ngulo de Brewster} e sua rela��o com o {\it �ndice de refra��o}, 
{\it coeficientes de reflex�o} e a {\it Lei de Malus}. Finalmente,
divertimo-nos com o interessante efeito da {\it birrefring�ncia} e {\it fotoelasticidade}.
