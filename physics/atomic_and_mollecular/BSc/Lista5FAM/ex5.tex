\begin{exercicio}[blue]
	
	Partindo do harm�nico esf�rico
	$$
	Y_{20}(\theta,\phi)=\sqrt{5\over16\pi}(3\cos^2\theta-1)	
	$$

	determine o restante dos harm�nicos esf�ricos $Y_{2m}(\theta,\phi)$ pela aplica��o dos 
operadores diferenciais $\op L_\pm$.
\end{exercicio}

	O esquema para se encontrar todos as autofun��es �, graficamente,
	$$
	Y_{2-2}\stackrel{\op L_-}{\longleftarrow}Y_{2-1}\stackrel{\op L_-}{\longleftarrow}Y_{20}
	\stackrel{\op L_+}{\longrightarrow}Y_{21}\stackrel{\op L_+}{\longrightarrow}Y_{22}
	$$

	Antes de tudo lembremos que
	$$
	\op L_+\ket{lm}=\sqrt{l(l+1)-m(m+1)}\hbar\ket{lm+1}
	$$
	$$
	\op L_-\ket{lm}=\sqrt{l(l+1)+m(m+1)}\hbar\ket{lm-1}
	$$

	e comecemos com o primeiro passo:
	\begin{eqnarray*} 
	\op L_+\ket{20}	&=&\sqrt{2(2+1)-0(0+1)}\hbar\ket{21}=\sqrt{6}\hbar\ket{21}\Rightarrow
			\ket{21}={1\over\sqrt{6}\hbar}\ket{20}					\\ 
			&=& \hbar e^{i\phi}(\der\theta+i\cot\theta\der\phi)
					\sqrt{5\over16\pi}(3\cos^2\theta-1)		\\
			&=& \sqrt{5\over16\pi}\hbar e^{i\phi}\der\theta(3\cos^2\theta-1)	\\
			&\vdots&\\
			&=& {3\over2}\sqrt{5\over\pi}\hbar e^{i\phi}\cos\theta\sin\theta
	\end{eqnarray*}

	Ent�o, 
	$$
	\ket{21}={3\over2}\sqrt{5\over6\pi} e^{i\phi}\sin\theta\cos\theta	
	$$

	Novamente,
	\begin{eqnarray*} 
	\op L_+\ket{21}	&=&\sqrt{2(2+1)-1(1+1)}\hbar\ket{22}=2\hbar\ket{21}\Rightarrow
			\ket{22}={1\over2\hbar}\ket{21} 					\\
			&=& \hbar e^{i\phi}(\der\theta+i\cot\theta\der\phi)
				{3\over2}\sqrt{5\over6\pi} e^{i\phi}\cos\theta\sin\theta	\\
			&=& {3\over2}\sqrt{5\over6\pi}\hbar e^{i\phi}[e^{i\phi}\der\theta
				(\cos\theta\sin\theta)+i\cot\theta\cos\theta\sin\theta\der\phi e^{i\phi}]\\
			&\vdots&\\
			&=& -{3\over2}\sqrt{5\over6\pi}\hbar e^{2i\phi}\sin^2\theta
	\end{eqnarray*}

	Isto �,
	$$
	\ket{22}= -{3\over4}\sqrt{5\over6\pi}e^{2i\phi}\sin^2\theta
	$$

	Agora fa�amos o caminho inverso, a partir de $\ket{20}$, obviamente:
	\begin{eqnarray*}
	\op L_-\ket{20}	&=&\sqrt{2(2+1)+0(0+1)}\hbar\ket{2-1}\Rightarrow
				\ket{2-1}={1\over\sqrt{6}\hbar}\op L_-\ket{20}			\\
			&=& \hbar e^{-i\phi}(-\der\theta+i\cot\theta\der\phi)
					\sqrt{5\over16\pi}(3\cos^2\theta-1)			\\
			&=& \sqrt{5\over16\pi}\hbar e^{-i\phi}[-\der\theta(3\cos^2\theta-1)]	\\
			&\vdots&\\
			&=& -{3\over2}\sqrt{5\over\pi}\hbar e^{-i\phi}\sin\theta\cos\theta
	\end{eqnarray*}

	ent�o,
	$$
	\ket{2-1}=-{3\over2}\sqrt{5\over6\pi} e^{-i\phi}\sin\theta\cos\theta
	$$

	Novamente,	
	\begin{eqnarray*} 
	\op L_-\ket{2-1}&=&\sqrt{2(2+1)-1(-1+1)}\hbar\ket{2-2} \Rightarrow 
				\ket{2-2}={1\over\sqrt{6}\hbar}\ket{2-1}				\\
			&=& \hbar e^{-i\phi}(-\der\theta+i\cot\theta\der\phi)
				(-{3\over2}\sqrt{5\over6\pi} e^{-i\phi}\sin\theta\cos\theta)	\\
			&=& -{3\over2}\sqrt{5\over6\pi}\hbar e^{-i\phi}[-e^{-i\phi}(\cos^2\theta
				-\sin^2\theta)+\cos^2\theta e^{-i\phi}]					\\
			&\vdots&\\
			&=& -{3\over2}\sqrt{5\over6\pi}\hbar e^{-2i\phi}\sin^2\theta 
	\end{eqnarray*}

	E, finalmente,
	$$
	\ket{2-2}=-{1\over4}\sqrt{5\over\pi} e^{-2i\phi}\sin^2\theta
	$$

