\begin{exercicio}[blue]
	Um el�tron num �tomo de hidrog�nio encontra-se no estado 
	$$
	\ket\psi = N(2\ket{100}-3\ket{200}+\ket{322})
	$$
	
	onde $\ket{nlm}$ representam os autoestados do el�tron ignorando-se o spin.
	
	{\tt (a)} Determine a constante $N$ de modo que o auto--estado seja normalizado.
	
	{\tt (b)} A fun��o de onda $\psi(\mathbf{r})=\iprod{\mathbf{r}}{\psi}$ tem paridade definida?
	
	{\tt (c)} Qual � a probabilidade de que na medida de energia se encontre 
$-m/8\hbar^2(e^2/4\pi\epsilon_0)^2$?
	
	{\tt (d)} Qual � a probabilidade de que na medida de $\opv{L}^2$ se encontre $6\hbar^2$?
	
	{\tt (e)} Qual � a probabilidade de que na medida de $\op L_z$ se encontre valor nulo?
	
	{\tt (f)} Quais s�o os valores esperados das medidas de energia, $\opv{L}^2$ e $\op L_z$?

\end{exercicio}

	{\tt (a)} 
	\begin{eqnarray*}
	\iprod{\psi}{\psi}	&=& N^*(2\bra{100}-3\bra{200}+\bra{322})\cdot N(2\ket{100}-3\ket{200}+\ket{322})	\\
	1										&=& |N|^2(4\iprod{100}{100}+9\iprod{200}{200}+\iprod{322}{322})	\\
	1										&=& |N|^2(4+9+1)
	\end{eqnarray*}
	
	ou seja, $|N|^2=1/14$.
	
	{\tt (b)}Uma fun��o de onda $\psi_{nlm}(r,\theta,\phi)$ � dita par se $l$ for par; e �mpar se $l$ for �mpar. Como $\ket{100}$, $\ket{200}$ e $\ket{322}$ s�o todos pares e a soma de fun��es pares �
uma fun��o par, temos que $\psi_{nlm}(r,\theta,\phi)$ � par. Isto �, tem paridade definida.

	{\tt (c)} Sabemos que 
	$$
	E_n=-{mZ^2\over2n^2\hbar^2}({e^2\over4\pi\epsilon_0})^2
	$$
	
	Deste modo, para $Z=1$ (�tomo de hidrog�nio), a energia $E=-(m/8\hbar^2)(e^2/4\pi\epsilon_0)^2$ 
corresponde a $n=2$. Logo, como a energia depende apenas de $n$, a probabilidade de medirmos tal
energia � igual � soma dos quadrados dos m�dulos dos coeficientes que multiplicam os kets que cont�m $n=2$. 
No nosso caso � apenas o ket $\ket{200}$ com coeficiente $3/\sqrt{14}$. A probabilidade procurada �,
ent�o, $P(n=2)=9/14=64,28\%$.

	{\tt (d)} $\op L^2\ket{nlm}=l(l+1)\ket{nlm}$ e, portanto, no nosso caso procuramos a situa��o em que $l(l+1)=6 \Rightarrow l=2$. A probabilidade � $1/14=7,14\%$, o quadrado do coeficiente do ket $\ket{322}$, o �nico com $l=2$.
	
	{\tt (e)} $\op L_z\ket{nlm}=m\hbar\ket{nlm}$ e, no nosso caso, procuramos a situa��o em que $m=0$. Isto � verificado para os kets $\ket{100}$ e $\ket{200}$. A probabilidade �, ent�o
	$$
	|2N|^2+|-3N|^2=(4+9)|N|^2={13\over14}\cong92,86\%
	$$
	
	{\tt (f)} 
	\begin{eqnarray*}
	E = \hope{\psi}{\mathcal H}{\psi} &=& N\bra\psi(2\mathcal H\ket{100}-3\mathcal H\ket{200}+
		\mathcal H\ket{322})		\\
		&=& -{mNZ^2\over2\hbar^2}({e^2\over4\pi\epsilon_0})^2\bra\psi({2\over1^2}\ket{100}-
			{3\over2^2}\ket{200}+{1\over3^2}\ket{322})	\\
		&=& -{139\over504}{mZ^2\over2\hbar^2}({e^2\over4\pi\epsilon_0})^2
	\end{eqnarray*} 

	\begin{eqnarray*}
	\med{\op L^2}	&=& \hope{\psi}{\op L^2}{\psi} = \bra{\psi}\op L^2[N(2\ket{100}-3\ket{200}
																									+\ket{322})]											\\
								&=& \bra{\psi}{[N(2\op L^2\ket{100}-3\op L^2\ket{200}+\op L^2\ket{322}]}	\\
								&=& 6N\hbar^2\iprod{\psi}{322}=6|N|^2\hbar^2={6\over14}\hbar^2			\\
								&=& {3\over7}\hbar^2
	\end{eqnarray*}
	
	\begin{eqnarray*}
	\med{\op L_z}	&=& \hope{\psi}{\op L_z}{\psi}=\bra{\psi}\op L_z[N(2\ket{100}-3\ket{200}
																									+\ket{322})]											\\
								&=& \bra{\psi}(2N\op L_z\ket{100}-3N\ket{200}+N\ket{322})						\\
								&=& 2N\hbar\iprod{\psi}{322}																				\\
								&=& 2|N|^2\hbar																											\\
								&=& {1\over7}\hbar
	\end{eqnarray*}
	