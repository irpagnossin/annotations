%IVAN RAMOS PAGNOSSIN 11/02/2001 irpagnossin@hotmail.com

\begin{quotation}
{\color{blue}{\bf 4.}}
{\it	Calcule a dist�ncia desde a superf�cie da Terra at� um sat�lite em �rbita geoestacion�ria. Em que plano deve-se encontrar a �rbita do sat�lite?}
\end{quotation}

\vspace{5mm}

	Para um sat�lite ser geoestacion�rio, seu per�odo de transla��o ao redor da Terra deve ser de um dia, aproximadamente 24 horas, e, al�m disso,
deve estar orbitando num plano perpendicular ao vetor momento angular da Terra. Em outras palavras, ele deve permanecer em algum plano paralelo ao equador.

	Pela {\it 3$^{\underline{a}}$ Lei de Kepler}, e levando em conta que a massa do sat�lite � absurdamente menor que a da Terra ($m_{\bigoplus}\gg m_s$),
	$$
	a^3={Gm_{\bigoplus}P^2\over4\pi^2}={6,67\times10^{-11}\cdot6,03\times10^{24}\cdot(24\cdot3600)^2\over4\pi^2}\approx4,236\times10^{7}\,\mathrm{m}
	$$

	Mas o raio da Terra � $6378\,\mathrm{km}$ (Obtido primeiramente por Erat�stenes). Ent�o a dist�ncia deste sat�lite at� a superf�cie da Terra � 
$d=4,236\times10^{7}-0,6378\times10^{7}\approx3,599\times10^{7}\,\mathrm{m}$. 

\vspace{10mm}
