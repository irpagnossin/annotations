\begin{exercicio}[blue]
	Calcule a energia efetiva ou equivalente de um feixe de raios X cuja camada semi-redutora ({\sc csr})
	vale $3,2\unit{mm}$ de Al. Qual a {\sc csr} esperadada para esse feixe se medida no cobre?
\end{exercicio}

	A partir da lei de decaimento exponencial encontramos $\mu=\ln2/x_{1/2}$, com $x_{1/2}$ sendo a {\sc csr}
do material --- Alum�nio, no caso. Ent�o � claro que $\mu_{\mathrm{Al}}=216,61\unit{m^{-1}}$. O coeficiente
de atenua��o m�ssico �, ent�o\footnote{A densidade do alum�nio � $2699\unit{kg/m^3}$.}, 
$\mu/\rho=0,0802\unit{m^2/kg}$. Procurando nas tabelas do alum�nio (Utilizamos a do Johns) verificamos que
a energia efetiva est� entre $500$ e $662\unit{keV}$. Para encontrarmos um valor mais preciso colhemos alguns
pontos nesta regi�o, plotamos e ajustamos uma reta\footnote{Nesta regi�o uma reta foi suficiente para 
atingirmos um $\chi^2_{\mathrm{red}}=0,99$.}:
	$$
	{\mu\over\rho}=-7\times10^{-5}\cdot E+0,1198
	$$
	
	Determinamos assim a energia efetiva: $E_{ef}=565\unit{keV}$. Agora, para determinar o {\sc csr} no cobre
procedemos de forma inversa: Buscamos na tabela do cobre os valores dos coeficientes de atenua��o na mesma
regi�o acima\footnote{Utilizamos o intervalo $[400--626\unit{keV}]$.} e novamente ajustamos uma reta:
	$$
	{\mu\over\rho}=-8\times10^{-5}\cdot E+0,1264,
	$$
	de onde conclu�mos imediatamente que a camada semi-redutora deste feixe para o cobre � $0,95\unit{mm}$,
lembrando que a densidade do cobre � $8960\unit{kg/m^3}$.
	
	