	Todos os experimentos tiveram concord�ncia muito boa com a teoria, exceto talvez o caso da
lente divergente que, como comentamos, apresentou um erro sistem�tico bastante elevado em compara��o
com todos os outros resultados obtidos. 
	Mas existe um ponto que n�o foi sequer inclu�do no seu respectivo local pois apresentou 
resultados absurdos. Trata-se da determina��o do �ndice de refra��o da lente a partir da f�rmula do
fabricante de lentes:
	$$
	{1\over f}=(n-1)[{1\over R_2}-{1\over R_1}]+{(n-1)^2\over n}{t\over R_1R_2}
	$$

	em primeira aproxima��o desconsideramos a segunda parcela e utilizamos
	$$
	{1\over f}=(n-1)[{1\over R_2}-{1\over R_1}]
	$$

	A dist�ncia focal, da lente convergente $C36$ por exemplo, foi determinada na sess�o \ref{sec:lc}:
$f=25,14(3)\,\mathrm{cm}$ e os raios $R_1$ e $R_2$ medidos atrav�s de um pequeno aparelho chamado 
rel�gio, que relacionava sua escala $x$ com o raio de curvatura da lente atrav�s da express�o
	$$
	{1\over R}=1,92\cdot x+0,4\,\,\,\,\,\,\mathrm{[R]=m}
	$$

	Para este caso espec�fico medimos $x=1,75$ e $x=1,875$, que correspondem respectivamente a
$R_1=26,60\,\mathrm{cm}$ e $R_2=25,00\,\mathrm{cm}$. Mas estes valores s�o muito pr�ximos da dist�ncia
focal encontrada. Este � o primeiro ponto estranho. Se insistirmos nestes valores e procurarmos calcular
os �ndices de refra��o, encontramos $n=17,53$, muito alto!\footnote{O valor mais pr�ximo que pudemos obter
atrav�s de \cite[p�g.12-42]{handbook} foi o do Tantalato de pot�ssio ($\mathrm{KTaO_3}$), e subst�ncias
com este �ndice eram muito raras!} Este � o segundo ponto estranho. Se tentarmos
resolver o problema lembrando dos velhos tempos de col�gio onde $R=2f$ e recalcularmos o �ndice, ainda
assim encontramos $n=34,06$, ainda mais absurdo! Segue abaixo a tabela de �ndices de refra��o encontrados
para cada uma das duas lentes convergentes:

\begin{table}[htbp]
  \centering
  \begin{tabular}{cc}
  \hline
  Lente		&	$n$			 	\\
  \hline						
  C36		&	17,53				\\
  C5		&	58,28				\\
  \hline
  \end{tabular}
\end{table}    

	O �ndice de refra��o da lente divergente n�o foi poss�vel medir pois $R_1=R_2=8,22\,\mathrm{cm}$ e,
consequentemente, o primeiro termo anula-se, exigindo a inclus�o do segundo que, por sua vez, exigia a
espessura da lente, que n�o foi medida, por motivos desconhecidos.

	Finalmente repetimos: A despeito deste ponto, todos os outros experimentos foram bem sucedidos.


