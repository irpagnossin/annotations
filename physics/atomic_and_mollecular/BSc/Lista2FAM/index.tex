\documentclass[a4paper,10pt]{article}

\usepackage[latin1]{inputenc}
\usepackage[portuges]{babel}
\usepackage[pdftex]{graphics,color}
\usepackage{graphicx}
\usepackage{wrapfig} 
\usepackage{epsfig} 
\usepackage{url}
\usepackage{ae,algorithm,alltt}
\usepackage{amsfonts,amstext,enumerate}
\usepackage{float,fancyvrb,fontenc}
\usepackage{geometry,hyperref,ifthen}
\usepackage{indentfirst,lastpage,longtable}
\usepackage{lscape,makeidx,mathrsfs}
\usepackage{multicol,pifont,psfrag}
\usepackage{setspace,showidx,subfigure}
\usepackage{texnames,textcomp,ulem}
\usepackage{url,varioref,version}
\usepackage{wasysym}

\begin{document}

\pagestyle{myheadings}
\markboth{}{\rm I.R.Pagnossin}
\renewcommand{\thefootnote}{\fnsymbol{footnote}}	

\title {Segunda lista de F�sica At�mica e Molecular}  
\author {I.R.Pagnossin \\ irpagnossin@hotmail.com}
\date {\today}
\maketitle

\newcommand{\op}[1]{\mathcal{#1}}
\newcommand{\bra}[1]{\ensuremath{\langle#1|}}% O VETOR BRA DO ESPA�O DE HILBERT.
\newcommand{\ket}[1]{\ensuremath{|#1\rangle}}% O VETOR KET DO ESPA�O DE HILBERT.
\newcommand{\iprod}[2]{\ensuremath{\langle#1|#2\rangle}}% O PRODUTO INTERNO NO ESPA�O DE HILBERT.
\newcommand{\hope}[3]{\ensuremath{\langle#1|#2|#3\rangle}}% O VALOR ESPERADO DO OPERADOR #2.
\newcommand{\intii}{\int_{-\infty}^{\infty}}

\begin{exercicio}[blue]

	A partir exclusivamente das rela��es de comuta��o can�nicas satisfeitas pelos obser�veis de
posi��o e momento, mostre que o operador momento angular orbital $\opv L$ com componentes

\hspace{-10mm}
\begin{minipage}[c]{0.3\textwidth}
 $$
 \op L_x = \op Y\op P_z - \op Z\op P_y
 $$
\end{minipage}
\begin{minipage}[c]{0.3\textwidth}
 $$
 \op L_y = \op Z\op P_x - \op X\op P_z
 $$
\end{minipage}
%\hfill
\begin{minipage}[c]{0.3\textwidth}
 $$
 \op L_z = \op X\op P_y - \op Y\op P_x
 $$
\end{minipage}

	satisfaz as rela��es

\begin{minipage}[c]{0.46\textwidth}
 {\tt (a)} $\opv L^{\dagger}=\opv L$
 
 {\tt (b)} $\opv L \times \opv L = i\hbar\opv L$

 {\tt (c)} $[\opv L^2,\opv L] = \mathbf{0}$ 
\end{minipage} 
\hfill
\begin{minipage}[c]{0.46\textwidth}
 {\tt (d)} $[\opv L^2,\op L_{\pm}]=0\,\,\,\, \mathrm{onde}\,\, \op L_{\pm}=L_x\pm i\op L_y$

 {\tt (e)} $[\op L_z,\op L_{\pm}] = \pm \hbar L_{\pm}$

 {\tt (f)} $\op L_{\pm}\op L_{\mp} = \opv L^2-L_z^2\pm\hbar L_z$
\end{minipage} 

\end{exercicio} 

	item {\tt (a)}: 
	\begin{eqnarray*}
	\opv L^{\dagger}&=& (\op L_x\hat i +\op L_y\hat j +\op L_z\hat k)^{\dagger}	\\
			&=& \op L_x^{\dagger}\hat i +\op L_y^{\dagger}\hat j +\op L_z^{\dagger}\hat k	
	\end{eqnarray*} 

	N�o h� necessidade de calcularmos cada uma das componentes pois elas s�o sim�tricas por uma 
rota��o do sistema $\{\hat i,\hat j,\hat k\}$ em torno do vetor $\vec u=(1,1,1)$. Ou seja, uma vez 
provada a equival�ncia $\op L_x^{\dagger}=\op L_x$, a mesma equival�ncia para $\op L_y$ e $\op L_z$ 
pode ser obtida por simples rota��o do sistema de coordenadas utilizado. Deste modo: Uma primeira
rota��o transformaria $x$ em $y$, $y$ em $z$ e $z$ em $x$. Ent�o, neste caso, a partir de
$\op L_x^{\dagger}=\op L_x$ concluir�amos que $\op L_y^{\dagger}=\op L_y$. Uma outra rota��o de simetria
daria $\op L_z^{\dagger}=\op L_z$. Ent�o, basta demonstrarmos $\op L_x^{\dagger}=\op L_x$:
	\begin{eqnarray*}
 	\op L_x^{\dagger}	&=& (\op Y\op P_z - \op Z\op P_y)^{\dagger}	\\
				&=& (\op Y\op P_z)^{\dagger} - 
					(\op Z\op P_y)^{\dagger}		\\
				&=& \op P_z^{\dagger}\op Y^{\dagger} - 
					\op P_y^{\dagger}\op Z^{\dagger}	\\
				&=& \op P_z\op Y - \op P_y\op Z			\\
				&=& \op Y\op P_z - \op Z\op P_y
	\end{eqnarray*}

	A �ltima passagem foi poss�vel pois $[\op Y,\op P_z]=0$ e $[\op Z,\op P_y]=0$, j� que atuam em
espa�os diferentes. Isto prova a igualdade para $L_x$ e, portanto, para $L_y$ e $L_z$. Consequentemente,
provamos que $\opv L^{\dagger}=\opv L$. Em palavras, provamos que $\opv L$ � hermiteano.

	Item {\tt (b)}: 
	\begin{eqnarray*}
	\opv L\times \opv L = \left |\begin{array}{ccc}
				\hat i	&	\hat j	&	\hat k	\\
				\op L_x	&	\op L_y	&	\op L_z	\\
				\op L_x	&	\op L_y	&	\op L_z	
				\end{array} \right | 
			&=& (\op L_y\op L_z-\op L_z\op L_y)\hat i+
				(\op L_z\op L_x-\op L_x\op L_z)\hat j+
					(\op L_x\op L_y-\op L_y\op L_x)\hat k				\\
			&=& [\op L_y,\op L_z]\hat i+[\op L_z,\op L_x]\hat j+[\op L_x,\op L_y]\hat k	\\ 
			&=& i\hbar\op L_x\hat i+i\hbar\op L_y\hat j+i\hbar\op L_z\hat k			\\
			&=& i\hbar\opv L
	\end{eqnarray*} 

	Item {\tt (c)}:
	\begin{eqnarray*} 
	[\opv L^2,\opv L]&=& [\op L_x^2+\op L_y^2+\op L_z^2,\opv L] 			\\
			&=& [\op L_x^2,\opv L]+[\op L_y^2,\opv L]+[\op L_z^2,\opv L]
	\end{eqnarray*}

	Vejamos cada uma das parcelas separadamente:
	\begin{eqnarray*} 
	[\op L_x^2,\opv L]&=& [\op L_x^2,\op L_x\hat i+\op L_y\hat j+\op L_z\hat k]			   \\
			  &=& [\op L_x^2,\op L_x]\hat i+[\op L_x^2,\op L_y]\hat j+[\op L_x^2,\op L_z]\hat k\\
			  &=& -[\op L_y,\op L_x^2]\hat j-[\op L_z,\op L_x^2]				\\
			  &=& -([\op L_y,\op L_x]\op L_x+L_x[\op L_y,\op L_x])\hat i-
				([\op L_z,L_x]\op L_x+\op L_x[\op L_z,\op L_x])\hat k			\\
			  &=& i\hbar\{L_x,\op L_z\}\hat j-i\hbar\{\op L_x,\op L_y\}\hat k		\\
	\end{eqnarray*}

	Analogamente,
	\begin{eqnarray*} 
[\op L_y^2,\opv L]&=& -i\hbar\{\op L_y,\op L_z\}\hat i+i\hbar\{\op L_x,\op L_y\}\hat k\\               
\rule{0mm}{0mm}[\op L_z^2,\opv L]&=& i\hbar\{\op L_y,\op L_z\}\hat i-i\hbar\{\op L_x,\op L_z\}\hat j
	\end{eqnarray*} 

	e, portanto, $[\opv L^2,\opv L]=\mathbf{0}$\footnote{Note que o resultado � vetorial.}. 

	Item {\tt (d)}:
	\begin{eqnarray*} 
	[\opv L^2,\op L_{\pm}]	&=& [\op L_x^2+\op L_y^2+\op L_z^2,\op L_\pm]	\\
				&=& [\op L_x^2,\op L_{\pm}]+[\op L_y^2,\op L_{\pm}]+[\op L_z^2,\op L_{\pm}]\\
				&=& [\op L_x^2,\pm i\op L_y]+[\op L_y^2,\op L_x]+[\op L_z^2,\op L_x]\pm
					i[\op L_z^2,\op L_y]			\\
				&=& \mp i[\op L_y,\op L_x^2]-[\op L_x,\op L_y^2]-[\op L_x,\op L_z^2]
					\mp i[\op L_y,\op L_z^2]	
	\end{eqnarray*}

	Vejamos cada uma das parcelas individualmente:
	\begin{eqnarray*} 
 	[\op L_y,\op L_x^2]	&=& [\op L_y,\op L_x]\op L_x+\op L_x[\op L_y,\op L_x]	\\
				&=& -i\hbar\op L_z\op L_x-i\hbar\op L_x\op L_z		\\
				&=& -i(\op L_z\op L_x+\op L_x\op L_z)			\\
				&=& -i\hbar\{\op L_z,\op L_x\}				\\
	\rule{0mm}{0mm}[\op L_x,\op L_y^2]	&=& [\op L_x,\op L_y]\op L_y+\op L_y[\op L_x,\op L_y]	\\
				&=& i\hbar\op L_z\op L_y+i\hbar\op L_y\op L_z		\\
				&=& i\hbar\{\op L_z,\op L_y\}				\\
	\rule{0mm}{0mm}[\op L_x,\op L_z^2]	&=& [\op L_x,\op L_z]\op L_z+\op L_z[\op L_x,\op L_z]	\\
				&=& -i\hbar\op L_y\op L_z-i\hbar\op L_z\op L_y		\\
				&=& -i\hbar\{\op L_y,\op L_z\}				\\
	\rule{0mm}{0mm}[\op L_y,\op L_z^2]	&=& [\op L_y,\op L_z]\op L_z+\op L_z[\op L_y,\op L_z]	\\
				&=& i\hbar\op L_x\op L_z+i\hbar\op L_z\op L_x		\\
				&=& i\hbar\{\op L_x,\op L_z\}		
	\end{eqnarray*} 

	Ent�o
	\begin{eqnarray*} 
	[\opv L^2,\op L_{\pm}]	&=& \mp i(-i\hbar)\{\op L_z,\op L_x\}\mp i(i\hbar)\{\op L_x,\op L_z\}	\\
				&=& \mp \hbar\{\op L_x,\op L_z\}\pm\hbar\{\op L_x,\op L_z\}		\\
				&=& 0
	\end{eqnarray*} 

	Item {\tt (e)}:
	\begin{eqnarray*} 
	[\op L_z,\op L_{\pm}]	&=& [\op L_z,\op L_x\pm i\op L_y]		\\
				&=& [\op L_z,\op L_x]\pm i[\op L_z,\op L_y]	\\
				&=& i\hbar\op L_y\pm i(-i\hbar\op L_x)		\\
				&=& i\hbar\op L_y\pm\hbar\op L_x		\\
				&=& \hbar(\pm \op L_x+i\op L_y)			\\
				&=& \pm\hbar(\op L_x\pm i\op L_y)		\\
				&=& \pm\hbar\op L_{\pm}
	\end{eqnarray*} 

	Item {\tt (f)}:
	\begin{eqnarray*} 
	\op L_\pm\op L_\mp	&=& (\op L_x\pm i\op L_y)(\op L_x\mp i\op L_y)			\\
				&=& \op L_x^2\mp i\op L_x\op L_y\pm i\op L_y\op L_x+\op L_y^2	\\
				&=& \op L_x^2+\op L_y^2\pm i(\op L_y\op L_x\mp \op L_x\op L_y)	\\
				&=& \opv L^2-\op L_z^2\pm i[\op L_y,\op L_x]			\\
				&=& \opv L^2-\op L_z^2\pm i(-i\hbar \op L_z)			\\
				&=& \opv L^2-\op L_z^2\pm\hbar\op L_z			
	\end{eqnarray*} 

	








\vspace{10mm} 
\begin{exercicio}[blue]
	O $\el{210}{}{Po}{}{}$ com atividade $2\,\mathrm{MBq}$ decai por desintegra��o alfa  (meia-vida de
$138,4\,\mathrm{dias}$).

	a) Mostre que esse decaimento � poss�vel e que decaimentos com emiss�o de n�utrons ou pr�tons n�o
	s�o poss�veis para esse nucl�deo.

	b) Qual a atividade em $\mu Ci$ para essa fonte?

	c) Qual a massa da amostra de Pol�nio-210?
\end{exercicio}

	{\tt (a)} Para mostrarmos que uma rea��o de desintegra��o � poss�vel devemos ter a energia dos
	reagentes maior que a dos produtos. Ou, em outras palavras, devemos ter uma energia de desintegra��o
	positiva. Vejamos para cada uma das rea��es:
	$$
	\el{210}{84}{Po}{}{}\,\to\,\el{4}{2}{He}{2+}{}\,+\,\el{206}{82}{Pb}{}{}\,+\,Q
	$$

	Das tabelas (refer�ncia \cite{serway}), para o decaimento alfa
	\begin{eqnarray*} 
	\mathrm{m(Po)}	&=& 209,982\,848\,\mathrm{u}	\\
	\mathrm{m(Pb)} 	&=& 205,974\,440\,\mathrm{u}	\\
	\mathrm{m(He)}	&=&   4,002\,602\,\mathrm{u}
	\end{eqnarray*}

	Ent�o $Q/c^2 = \mathrm{m(Po)-m(He)-m(Pb)}=0,005\,806\,\mathrm{u} > 0$ e, portanto esta rea��o � 
poss�vel\footnote{Isto n�o garante que ela ocorra, mas lhe d� a possibilidade.}. 

	J� para o decaimento com emiss�o de n�utron
	$$
	\el{210}{84}{Po}{}{}\,\to\,\el{1}{0}{n}{0}{}\,+\,\el{209}{84}{Po}{}{}\,+\,Q
	$$

	\begin{eqnarray*}
	\mathrm{m(n)}		&=&   1,008\,665\,\mathrm{u} 	\\
	\mathrm{m(^{209}Po}	&=& 208,982\,405\,\mathrm{u}  
	\end{eqnarray*}

	$$
	{Q\over c^2} = \mathrm{m(^{210}Po)-m(n)-m(^{209}Po)} = -0,008\,222 < 0
	$$

	E a rea��o n�o � poss�vel. Equivalentemente, 

	$$
	\el{210}{84}{Po}{}{}\,\to\,\el{1}{1}{p}{+}{}\,+\,\el{209}{83}{Bi}{}{}\,+\,\mathrm{u}
	$$

	\begin{eqnarray*}
	\mathrm{m(p^+)}	&=&   1,007\,276\,\mathrm{u}	\\
	\mathrm{m(Bi)}	&=& 208,980\,374\,\mathrm{u}	
	\end{eqnarray*}

	$$
	{Q\over c^2} = \mathrm{m(Po)-m(p)-m(Bi)} =  -0,004\,802\,\mathrm{u} < 0 
	$$

	A equa��o, mais uma vez, n�o � poss�vel.

	{\tt (b)} Sabemos que $1\,\mathrm{Ci} = 1,7\times10^{10}\,\mathrm{Bq}$ e, assim, 
$2\times10^6\,\mathrm{Bq} = 54,054\,\mathrm{\mu Ci}$.

	{\tt (c)} A atividade do n�cleo e a quantidade em que ocorre relacionam-se por $A=\lambda N$,
sendo $A$ a atividade, $\lambda$ a constante de decaimento e $N$ o n�mero de n�cleos. Assim, no in�cio
da contagem do tempo, quando a atividade do Pol�nio-210 era $2\,\mathrm{MBa}$, o n�mero de �tomos (e,
consequentemente, de n�cleos) era
	$$
	N = {A\over\lambda} = {A T_{1/2}\over\ln 2} = 3,4502\times10^{13}
	$$

	Assim, como em um mol ($6,022\times10^{23}$) temos M\footnote{A massa at�mica.} gramas do 
elemento, conclu�mos que a massa de Pol�nio-210 no �nicio de nosso experimento era de
	$$
	\mathrm{m(Po)} = 3,4502\times10^{13}\cdot{\mathrm{1\,mol(Po)}\over6,022\times10^{23}}\cdot 
		{209,982\,848\,\mathrm{g}\over 1\,mol(Po)} = 0,012\,031\,\mathrm{\mu g}
	$$
	

	

\vspace{10mm}
\begin{exercicio}[blue]
	Considere uma part�cula movendo-se na presen�a de um potencial central proporcional a $r^\mu$. Ent�o pode-se mostrar que os valores esperados da energia cin�tica e potencial num estado estacion�rio satisfazem a rela��o $2\med{\op T}=\mu\med{\op V}$, conhecido como {\it Teorema do Virial}.
	
	{\tt (a)} Usando o teorema do virial no caso de um �tomo com um el�tron e carga nuclear $Z$, mostre que
		$$
		\hope{nlm}{\op V}{nlm} = \med{\op V}_{nlm}=2E_n,\,\,\,\,\,\,\op T_{nlm}=-E_n
		$$
		
		onde
		$$
		E_n=-{e^2\over4\pi\epsilon_0}a_0}{Z^2\over n^2}
		$$
		
		{\tt (b)} usando o resultado do item {\tt (a)} mostre que o valor esperado do inverso da dist�ncia ao n~ucleo � dado por
		$$
		\med{{1\over r}}_{nlm}={Z\over a_0n^2}
		$$
		
\end{exercicio}


	{\tt (a)} No caso do hidrog�nio, sabemos que $\mu=-1$ e, portanto, o teorema do Virial toma a forma $2\med{\op T}=-\med{\op V}$. Mas a energia mec�nica total do �tomo � simplesmente
		$$
		E_n = \med{E_n}= \med{\op T} + \med{\op V} = -{1\over2}\med{\op V} + \med{\op V}
		$$	

	que obviamente resulta em $\med{\op V}=2E_n$ e, por conseguinte, $\med{\op T}=E_n$.
	
	{\tt (b)} 
	\begin{eqnarray*}
	\med{\op V} &=& \hope{nlm}{\op V}{nlm} = \med{{1\over4\pi\epsilon_0}\cdot{Ze^2\over r}}		\\
	2E_n				&=& {Ze^2\over4\pi\epsilon_0}\med{{1\over r}}_{nlm}
	\end{eqnarray*}
	
	Segue que 
	$$
	\med{{1\over r}}_{nlm} = {4\pi\epsilon_0\over Ze^2}2E_n = {Z\over a_0n^2} = {1\over a_0n^2}
	$$
	
	lembrando que $Z=1$.	
\vspace{10mm} 
\begin{exercicio}[blue]
	Uma fonte de $^{210}\mathrm{Po}$ (emissor alfa, meia-vida de 138,4 dias), com $10\,\mathrm{\mu Ci}$
� mantida dentro de um tubo selado, inicialmente em v�cuo. Ap�s 1 ano desse encapsulamento, que massa de
�tomos de h�lio ter� sido formada dentro do tubo?
\end{exercicio}

	A equa��o de decaimento do Pol�nio-210 �
	$$
	\el{210}{84}{Po}{}{}\,\to\,\el{4}{2}{He}{2+}{}\,+\,\el{206}{82}{Pb}{}{}
	$$

	Pela estequiometria, vemos que em qualquer tempo $t$ teremos um n�mero igual de n�cleos de h�lio e
chumbo. Este �ltimo, por outro lado, pode ser relacionado com o n�mero de n�cleos de Pol�nio atrav�s da
lei de decaimento:
	$$
	\mathrm{N(Pb) = N(He) = N_0(1-e^{-\lambda t}) = {A_0T_{1/2}\over\ln 2}(1-e^{-\lambda t})}
	$$

	onde $N_0$ � o n�mero inicial de n�cleos de Pol�nio, $\lambda$ sua constante de decaimento, 
$T_{1/2}$ sua meia-vida e $A_0$ sua atividade inicial. Assim, ap�s um ano ($31\,536\,000\,\mathrm{s}$) de 
confinamento, o n�mero de part�culas alfa (N�cleos de h�lio) ser�
	$$
	\mathrm{N(1\,ano)} = 5,357\,460\times10^{12}\cdot{\mathrm{1\,mol(^4He)\over 6,022\,137%
\times10^{23}}}\cdot{4,002\,602\,\mathrm{g}\over\mathrm{1\,mol(^4He)}} = 3,560\,907\times10^{-11}%
\,\mathrm{g}
	$$

	

\vspace{10mm} 
\begin{exercicio}
	Considere uma part�cula de massa $m$ sujeita a um potencial unidimensional da forma
	\[V(x)=\left\{ \begin{array}{ll}
		{1\over2}kx^2 	& \mathrm{para}\,\,\,\, x>0	\\	
		\infty		& \mathrm{para}\,\,\,\, x<0
			\end{array}\right .\]

	{\tt (a)} Escreva a equa��o de Shrodinger para $x>0$ e a condi��o que a fun��o de onda
deve satisfazer em $x=0$ e $x\rightarrow\infty$. Verifique quais das fun��es de onda do oscilador
harm�nico com os mesmos valores de $m$ e $k$ podem representar as fun��es de onda permitidas e as
fun��es de onda correspondentes.

	{\tt (b)} Calcule os valores esperados $\med{\op X}$ e $\med{\op P}$ para o estado 
fundamental.
\end{exercicio}

	{\tt (a)} Conquanto $x$ permane�a positivo, o Hamiltoniano do problema � id�ntico ao
do oscilador harm�nico:
	$$
	\op H = {1\over2m}\op P^2+{1\over2}k\op X^2
	$$

	que, na representa��o das coordenadas, escreve-se
	$$
	-{\hbar^2\over2m}{d^2\over dx^2}+{1\over2}kx^2
	$$

	ent�o a equa��o de Schrodinger �
	\begin{eqnarray*}
	\op H\ket\psi		&=&	E\ket\psi			\\
	\hope{x}{\op H}{\psi}	&=&	E\iprod{x}{\psi}		\\
	\hope{x}{-{\hbar^2\over2m}{d^2\over dx^2}+{1\over2}kx^2}{\psi}
				&=&	E\psi(x)			\\
	-{\hbar^2\over2m}{d^2\psi(x)\over dx^2}+{1\over2}kx^2\psi(x)	
				&=&	E\psi(x)
	\end{eqnarray*}

	As condi��es de contorno para $\psi(x)$ s�o $\psi(0)=0$ e $\psi(\infty)=0$.

	Depois de algum tempo analisando a situa��o conclui-se que qualquer solu��o do oscilador
harm�nico unidimensional que possua $x=0$ como raiz � tamb�m {\bf uma solu��o} desse problema. De 
fato, como estas solu��es satisfazem a equa��o de Schrodinger e �s condi��es de contorno 
-- Lembre-se que as solu��es do oscilador tendem a zero no infinito, � claro que se trata de uma 
solu��o. Mais que isso: Como � sempre poss�vel fazer um mapeamento contrativo do espa�o de Hilbert, segue que
a solu��o � �nica. Ent�o esta � {\bf a solu��o}. Mas quais s�o elas? O que vai caracterizar a raiz
que queremos s�o os polin�mios de Hermite pois a exponencial nunca se anula. Vejamos:
	\begin{eqnarray*}
	H_0(0) &=& a_0		\\
	H_1(0) &=& a_0y=0	\\
	H_2(0) &=& a_0		\\
	H_3(0) &=& 0		\\
	H_4(0) &=& 12a_0	\\
	H_5(0) &=& 0		\\
	       &\vdots&		
	\end{eqnarray*}

	Assim fica claro que as solu��es do nosso problema s�o aquelas com �ndice $n$ �mpar do
problema do oscilador harm�nico. Matematicamente escrevemos
	$$
	\psi_n(x)=N_nH_n(\alpha x)e^{-\alpha^2x^2/2},\,\,\,\,n=1,3,5,7,...
	$$

	sendo $N_n$ a constante de normaliza��o que, obviamente � diferente da do oscilador 
harm�nico. � imediato tamb�m que a equa��o da energia permanece
	$$
	E_n=(n+{1\over2}\hbar\omega,\,\,\,\,\,n=1,3,5,7,...
	$$

	{\tt (b)} A fun��o de onda para o estado fundamental ($n=1$) �
	$$
	\psi_1(x)=N_1\alpha xe^{-\alpha^2x^2/2},\,\,\,\,\,E_1={3\over2}\hbar\omega
	$$

	Vamos, em primeiro lugar, normaliz�-la:
	\begin{eqnarray*}
	\iprod{\psi}{\psi}	&=& |N_1|^2\alpha^2\int_0^{\infty}x^2e^{-\alpha^2x^2}dx		\\
	1			&=& |N_1|^2\alpha^2{1\over2}\Gamma({3\over2})\alpha^{-3}	\\
				&=& {|N_1|^2\over4\alpha}\Gamma({1\over2})			\\
				&=& {|N_1|^2\sqrt{\pi}\over4\alpha}	
	\end{eqnarray*}

	Levando-nos a $|N_1|^2=4\alpha\over\sqrt{\pi}$. Agora podemos calcular o valor m�dio de
$\op X$:
	\begin{eqnarray*}
	\hope{\psi}{\op X}{\psi}	&=& |N_1|^2\alpha^2\int_0^\infty x^3e^{-\alpha^2x^2}dx	\\
					&=& |N_1|^2\alpha^2{1\over2}\Gamma(2)\alpha^{-4}	\\
					&=& {|N_1|^2\over2\alpha^2}				\\
					&=& {2\over\alpha\sqrt{\pi}}		
	\end{eqnarray*}

	e o valor m�dio de $\op P$:
	\begin{eqnarray*} 
	\hope{\psi}{\op P}{\psi}	&=& -i\hbar|N_1|^2\alpha^2\int_0^\infty xe^{-\alpha^2x^2/2}
						{d\over dx}[xe^{-\alpha^2x^2/2}]dx		\\
					&=& -i\hbar|N_1|^2\alpha^2
						[\int_0^\infty xe^{-\alpha^2x^2}dx-
						\alpha^2\int_0^\infty x^3e^{-\alpha^2x^2}dx]	\\
					&=& -i\hbar|N_1|^2\alpha^2
						[{1\over2}\Gamma(1)\alpha^{-2}-
						\alpha^2{1\over2}\Gamma(2)\alpha^{-4}]		\\
					&=& -i\hbar|N_1|^2\alpha^2[{1\over2\alpha}-
								{1\over2\alpha}]		\\
					&=& 0
	\end{eqnarray*} 
 	







\end{document}
 


