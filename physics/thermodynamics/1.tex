\documentclass[a4paper,12pt]{article}
\usepackage[utf8]{inputenc}
\usepackage[T1]{fontenc}
\usepackage[brazil]{babel}
\usepackage{amsmath}
\usepackage{siunitx}\sisetup{locale=FR}
\usepackage{booktabs}
\newcommand\foreign[1]{\textsl{#1}}
\newcommand\ie{\foreign{ie}}
\newcommand\TODO{\textbf{==TODO TODO==}}

\title{Nível de água num copo com gelo}
\author{I. R. Pagnossin}

\begin{document}

\maketitle

\begin{abstract}
  Um cubo de gelo num copo de água, ao liquefazer-se, não eleva o nível da água, apesar de parte do gelo ($8\%$ de seu volume) ficar acima da superfície.
\end{abstract}

  Considere um cubo de gelo de volume $V_{s}$ (o subscrito ``$s$'' significa ``fase sólida'').
  Sua densidade é $\rho_{s}$, enquanto a densidade da água na qual ele flutua é $\rho_{l}$ (``$l$'' de ``fase líquida'').
  
  \paragraph{Argumento \#1:} na fase sólida, podemos escrever: $m = \rho_{s} V_{s}$.
  Como a transição de fase conserva a massa $m$ do gelo, também é verdade que $m = \rho_{l} V_{l}$, desta vez para a fase líquida.
  Nesta última expressão, $V_{l}$ é o volume ocupado pelo gelo após liquefazer-se, \ie, seu volume na fase líquida.
  Note que o único parâmetro comum é a massa, mas daí podemos concluir que:
  \begin{equation}\label{eq:relacao-1}
  \frac{\rho_{l}}{\rho_{s}} = \frac{V_{s}}{V_{l}}.
  \end{equation}
  
  Apenas como curiosidade: $\rho_{s} = \SI{920}{kg/m^3}$ e $\rho_{l} = \SI{1000}{kg/m^3}$, de modo que o volume ocupado pela água na fase sólida (gelo) é $9\%$ maior que na fase líquida. Não é por outro motivo que o gelo flutua!

  \paragraph{Argumento \#2:} na fase sólida, podemos argumentar que o cubo encontra-se em equilíbrio estático.
  Deste modo, sua força-peso $P$ é balanceada pelo empuxo $E$, ou seja: $E = P$.
  A primeira é dada por:
  \begin{equation*}
  P = mg = \rho_{s}V_{s}g,
  \end{equation*}
  onde $g$ é a aceleração da gravidade.
  
  Já a força de empuxo é igual à força-peso associada à massa de água deslocada, $m_{d}$:
  \begin{equation*}
  E = m_{d}g = \rho_{l} V_{d} g,
  \end{equation*}
  onde $V_{d}$ é o volume de água deslocado pelo gelo.
  Perceba que, em princípio, ele não tem nada a ver com $V_l$ ou $V_s$.
  De fato, $V_d$ é o espaço que o gelo abriu na água para acomodar-se ali.
  Apesar disso, como $E = P$, podemos afirmar que:
  \begin{equation}\label{eq:relacao-2}
  \frac{\rho_{l}}{\rho_{s}} = \frac{V_{s}}{V_{d}}.
  \end{equation}
  
  Comparando as relações \eqref{eq:relacao-1} e \eqref{eq:relacao-2}, obtemos o resultado desejado:
  \begin{equation*}
    V_{d} = V_{l}.
  \end{equation*}

  Assim, ao colocarmos um cubo de gelo num copo d'água, ele desloca um volume de água $V_{d}$ que é equivalente ao volume que ele \emph{ocupará} ao liquefazer-se: $V_{l}$.
  É por isso que o nível da água no copo não é alterado.
  
  Na verdade, esse resultado é verdadeiro para qualquer objeto imerso num líquido do mesmo material (para que valha o argumento \#1) e que seja menos denso em sua fase sólida (argumento \#2).
  O argumento de que a massa é conservada é verdadeiro sempre que não houver reações nucleares envolvidas.
  
  Há vários compostos químicos que apresentam a propriedade de ter a fase sólida menos densa que a líquida, mas apenas cinco elementos químicos têm esse comportamento. São eles:
  
  \begin{center}
  \begin{tabular}{lccc}
  \toprule
  Elemento & $\rho_s$ (\si{g/cc}) & $\rho_l$ (\si{g/cc}) & Fusão (\si{\degreeCelsius})\\
  \midrule
  Arsênio & 4,70 & 5,22 & 817 \\
  Bismuto & 9,78 & 10,07 & 271\\
  Gálio   & 5,90 & 6,09 & 30 \\
  Germânio & 5,32 & 5,60 & 940 \\
  Silício & 2,33 & 2,51 & 1410\\
  \bottomrule
  \end{tabular}%
  \end{center}
  
  Assim, um recipiente contendo um bloco de gálio sólido e totalmente preenchido com gálio líquido (a \SI{50}{\degreeCelsius}, por exemplo) não sofre alteração no seu nível após o gálio liquefazer-se.
  
  
  
  
\end{document}
