\begin{quotation}
{\bf 3.}
{\it	O operador hamiltoniano de uma part�cula em uma dimens�o sujeita a uma potencial $V(x)$
	�
	$$
	\op H = {\op P^2\over 2m}+V(\op X)
	$$

	{\tt (a)} Sabendo-se que $\op X$ e $\op P$ satisfazem a rela��o de comuta��o can�nica
	$[\op X,\op P]=i\hbar$, mostre que $\op P=(im/\hbar)[\op H,\op X]$.

	{\tt (b)} Usando o resultado do item anterior, mostre que para qualquer autofun��o do
	operador $\op H$ correspondendo a um estado ligado, o valor esperado do momento �
	nulo: $\med{\op P}=0$.}
\end{quotation}

	{\tt (a)}
	\begin{eqnarray*}
	[\op H,\op X]	&=& [{\op P^2\over 2m}+V(\op X),\op X]			\\
			&=& {1\over2m}[\op P^2,\op X]+[V(\op X),\op X]		\\
			&=& -{1\over2m}(\op P[\op X,\op P]+[\op X,\op P]\op P)
						+[V(\op X),\op X]		\\
			&=& -{1\over2m}(\op P i\hbar + i\hbar \op P)
						+[V(\op X),\op X]		\\
			&=& -{i\hbar\over m}\op P+[V(\op X),\op X]		\\
			&=& -{i\hbar\over m}\op P	
	\end{eqnarray*}

	O comutador de $V(\op X)$ e $\op X$ � zero pois o potencial s� depende de $\op X$.
Podemos chegar a esta conclus�o de forma mais clara assumindo a representa��o das coordenadas,
onde $\op X=x$:
	$$
	[V(\op X),\op X]=[V(x),x]=V(x)x-xV(x)=0
	$$
 
	{\tt (b)}
	\begin{eqnarray}
	\med{\op P}	&=& \hope{\psi}{\op P}{\psi}				\nonumber\\
			&=& {im\over\hbar}\hope{\psi}{[\op H,\op X]}{\psi}	\nonumber\\	
	\label{eq:ex3}	&=& {im\over\hbar}\hope{\psi}{\op H\op X}{\psi}-
				{im\over\hbar}\hope{\psi}{\op X\op H}{\psi}	\\
			&=& {im\over\hbar}x\hope{\psi}{\op H}{\psi}-
				{im\over\hbar}E\hope{\psi}{\op X}{\psi}		\nonumber\\
			&=& {im\over\hbar}xE\iprod{\psi}{\psi}-
				{im\over\hbar}Ex\iprod{\psi}{\psi}		\nonumber\\
			&=& {im\over\hbar}xE-{im\over\hbar}Ex			\nonumber\\
			&=& 0							\nonumber
	\end{eqnarray}

	Onde em (\ref{eq:ex3}) utilizamos $\op X\ket\psi=x\ket\psi$ e $\op H\ket\psi=E\ket\psi$,
o problema de Sturm-Liouville para os operadores posi��o e hamiltoniano, respectivamente.
	