\documentclass[a4paper,12pt]{article}
\usepackage[utf8]{inputenc}
\usepackage[T1]{fontenc}
\usepackage[brazil]{babel}
\usepackage{amsmath}

\let\vec=\mathbf
\begin{document}

  Considere um operador linear $A$ que, num sistema de coordenadas qualquer, é expresso por:
  \begin{equation*}
  A = \begin{pmatrix}
    1 & 3 \\
    3 & 1
  \end{pmatrix}.
  \end{equation*}
  
  Seus autovalores (ou valores próprios), são números reais (ou complexos) $\lambda$ que satisfazem a equação:
  \begin{equation*}
  A \vec u = \lambda \vec u,
  \end{equation*}
  sendo $\vec u = \begin{pmatrix}x\\y\end{pmatrix}$ um vetor desse espaço.
  Então, a equação anterior fica assim:
  \begin{equation*}
  \begin{pmatrix}
    A_{xx} & A_{xy} \\
    A_{yx} & A_{yy}
  \end{pmatrix}
  \begin{pmatrix}
    x \\
    y
  \end{pmatrix}
  =
  \lambda
  \begin{pmatrix}
    x \\
    y
  \end{pmatrix}
  \Rightarrow
  \begin{pmatrix}
    A_{xx} - \lambda & A_{xy} \\
    A_{yx}           & A_{yy} - \lambda    
  \end{pmatrix}
  \cdot
  \begin{pmatrix}    
    x \\
    y
  \end{pmatrix}  
  =
  \begin{pmatrix}    
    0 \\
    0
  \end{pmatrix}
  \end{equation*}

  Essa equação tem solução apenas se:
  \begin{equation}\label{eq:ker}
  \left|\begin{matrix}
    A_{xx} - \lambda & A_{xy} \\
    A_{yx}           & A_{yy} - \lambda    
  \end{matrix}\right|
  = 0,
  \end{equation}
  que é a equação que determina os autovalores de $A$.
  
  Considerando o exemplo dado, \eqref{eq:ker} fica:
  \begin{equation}
  \left|\begin{matrix}
    1 - \lambda & 3 \\
    3           & 1 - \lambda    
  \end{matrix}\right|
  = (1-\lambda)^2-9 = 0
  \Rightarrow
  1-\lambda = \pm 3
  \Rightarrow
  \lambda = 1 \mp 3.
  \end{equation}
  
  Ou seja, os autovalores de $A$ são $\lambda_1 = -2$ e $\lambda_2 = 4$.




\end{document}
