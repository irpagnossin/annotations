\begin{exercicio}[blue]
	Uma fonte de $^{210}\mathrm{Po}$ (emissor alfa, meia-vida de 138,4 dias), com $10\,\mathrm{\mu Ci}$
� mantida dentro de um tubo selado, inicialmente em v�cuo. Ap�s 1 ano desse encapsulamento, que massa de
�tomos de h�lio ter� sido formada dentro do tubo?
\end{exercicio}

	A equa��o de decaimento do Pol�nio-210 �
	$$
	\el{210}{84}{Po}{}{}\,\to\,\el{4}{2}{He}{2+}{}\,+\,\el{206}{82}{Pb}{}{}
	$$

	Pela estequiometria, vemos que em qualquer tempo $t$ teremos um n�mero igual de n�cleos de h�lio e
chumbo. Este �ltimo, por outro lado, pode ser relacionado com o n�mero de n�cleos de Pol�nio atrav�s da
lei de decaimento:
	$$
	\mathrm{N(Pb) = N(He) = N_0(1-e^{-\lambda t}) = {A_0T_{1/2}\over\ln 2}(1-e^{-\lambda t})}
	$$

	onde $N_0$ � o n�mero inicial de n�cleos de Pol�nio, $\lambda$ sua constante de decaimento, 
$T_{1/2}$ sua meia-vida e $A_0$ sua atividade inicial. Assim, ap�s um ano ($31\,536\,000\,\mathrm{s}$) de 
confinamento, o n�mero de part�culas alfa (N�cleos de h�lio) ser�
	$$
	\mathrm{N(1\,ano)} = 5,357\,460\times10^{12}\cdot{\mathrm{1\,mol(^4He)\over 6,022\,137%
\times10^{23}}}\cdot{4,002\,602\,\mathrm{g}\over\mathrm{1\,mol(^4He)}} = 3,560\,907\times10^{-11}%
\,\mathrm{g}
	$$

	
